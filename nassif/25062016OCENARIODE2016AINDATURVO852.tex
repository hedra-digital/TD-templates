\chapterspecial{25/\allowbreak{}06/\allowbreak{}2016 O cenário de 2016 ainda turvo}{}{}
 

O governo continua tendo de matar um leão por dia. Mas de leão em leão
começa a tornar o zoológico mais habitável.

A edição da Lei da Repatriação de Ativos, o Regime Especial de
Regularização Cambial e Tributária (\versal{RERCT}), abre possibilidade de
recursos brasileiros no exterior. A~lei prevê anistia para crimes
fiscais -- caixa 2, sonegação. Mas impedirá lavagem de dinheiro ou
dinheiro provindo de origem incerta e não sabida.

Seu sucesso dependerá, em grande parte, da reconquista da credibilidade
por parte da política econômica.

\asterisc{}~

Por outro lado, a decisão de quitar de vez as chamadas ``pedaladas
fiscais'' permitiu injetar recursos nos bancos públicos. O~desafio será
reativar o crédito tendo como piso os 14,25\% da taxa básica de juros,
sujeito a aumento.

Uma das grandes inconsistências da política pública brasileira continua
sendo o papel do Banco Central.

Há necessidade premente de deter a escalada da dívida bruta, alimentada
pela Selic. E~a necessidade urgente de reativar a economia, não apenas
pela questão do emprego, mas para impedir o colapso fiscal.

Há evidências de que o nível de atividade não justifica taxas nessas
alturas. A~simples análise dos dados de 2015 comprova que a Selic, pelo
canal de transmissão do crédito, não tem nenhuma eficácia sobre a
inflação, já que inexiste qualquer pressão de demanda.

Esta semana, o Copom respondeu às críticas com uma falsa relação de
causa"-efeito. Sustentou que a maior prova de que o sistema é eficaz é
que, de acordo com a pesquisa semanal, a maioria dos economistas
consultados endossa a política, porque aposta em um aperto da Selic.

Ora, a lógica dos economistas consultados não é apontar o que consideram
certo ou errado no Copom, mas acertar o que o Copom irá fazer. O~Copom é
o agente coordenador das expectativas.

\asterisc{}

A incapacidade do governo de conseguir do \versal{BC} um mínimo de racionalidade
é um dos grandes empecilhos para a retomada da economia.

Outro ponto é a excessiva cautela do setor público depois do
estardalhaço da Lava Jato. Hoje em dia, qualquer decisão, por mais óbvia
que seja, é escandalizável.

Por todos esses fatores, e pelos desastres cometidos pelo voluntarismo
de 2013 e 2014, dificilmente se sairá do arroz com feijão.

\asterisc{}

De positivo, tem"-se o relativo arrefecimento da campanha pelo
impeachment.

Do lado político, a ameaça do impeachment abriu espaço para jogadas
oportunistas que terminaram por desgastar os principais beneficiários.

A reputação de alguns deles foi construída sem exposição, sem testá"-los
em momentos de alta visibilidade. Com a luz batendo de frente,
desmancharam"-se como gelo ao sol. É~o caso do vice"-presidente Michel
Temer, que construiu uma liderança acomodatícia no \versal{PMDB} e, agora, purga
os pecados para preservar a presidência do partido.

\asterisc{}

Do lado jurídico, o Ministério Público Federal começa a atuar com mais
comedimento, provavelmente pelos sinais emitidos pelo \versal{STF} (Supremo
Tribunal Federal) e por uma posição mais firme da Procuradoria Geral da
República contra os ímpetos dos jovens turcos.

Um dos sinais foi a decisão da subprocuradora Ela Viecko, barrando a
sofreguidão com que procuradores e Policiais Federais de Minas avançavam
sobre o governador Fernando Pimentel.

Outro, a divulgação da íntegra da delação de Nestor Cerveró, ex"-diretor
da Petrobras. Nas primeiras delações era visível o conteúdo partidário
imprimido ao interrogatório. Se vigorar o padrão Cerveró, as delações
terão conteúdo mais objetivo, sem a malha de insinuações das anteriores
que serviam apenas para jogar gasolina na fogueira.
