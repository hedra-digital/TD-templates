\chapterspecial{05/\allowbreak{}09/\allowbreak{}2016 Xadrez da volta das vivandeiras dos quarteis}{}{}
 

Há em andamento uma tentativa de jogar as Forças Armadas na tarefa menor
da repressão interna.

Entenda.

\section{Peça 1 -- a volta das vivandeiras}

Tem"-se um quadro completo, com os principais personagens para a montagem
desse jogo:

\subsection{Os black blocs}

Um grupo de ultraesquerda infiltrado nas manifestações, promovendo
quebra"-quebra e sendo tratado com um certo paternalismo por setores da
esquerda. No seu primarismo político, os black bloc livram a \versal{PM} do
trabalho sujo de simular quebradeiras.

\subsection{A \versal{PM}}

Uma \versal{PM} capaz de bater em adolescentes, mas incapaz de reprimir os black
blocs, porque são parte integrante da sua estratégia. Quem assistiu o
documentário sobre a batalha de Seattle -- as manifestações anti"-\versal{OMC} de
1999 -- aprendeu bem as manhas da repressão. Infiltravam agentes para
comandar quebra"-quebra, visando desmoralizar as manifestações e
encontrar álibi para a repressão. É~prática corriqueira. Dispondo do
apoio dos black blocs, o trabalho fica enormemente facilitado.

 

\subsection{A imprensa vivandeira}

Completa o quadro uma imprensa vivandeira que faz o jogo da \versal{PM} e clama
pelos quarteis, porque a \versal{PM} não coíbe a quebradeira. Foi o caso do
editorial da Folha (\url{https:/\allowbreak{}/\allowbreak{}is.gd/\allowbreak{}uhtpu4)}~entrando no jogo da \versal{PM}.

Os ataques de baixo nível a Ricardo Lewandowski e a atitude vil de expor
o apartamento da mãe de Dilma Rousseff -- e ainda taxá"-lo falsamente de
``apartamento de luxo'' -- comprovam que nem a vitória ajudará a impor
um mínimo de grandeza à velha mídia. Continuarão apostando na guerra de
extermínio sem receio do ridículo, como mostra reportagem da Folha
acusando a Bolsa Família pela crise das comunidades indígenas
(\url{https:/\allowbreak{}/\allowbreak{}is.gd/\allowbreak{}r6l\versal{REB}}).

O governador Geraldo Alckmin recuou de sua intenção de proibir as
manifestações de domingo. Mas é questão de tempo.

Do lado de dentro, espera"-se o caldeirão entornar para o grande objetivo
traçado pelo governo Temer: devolver às Forças Armadas o papel de
repressão interna.

Antes disso, uma pequena explanação sobre a política de defesa na última
década.

\section{Peça 2 -- a Política Nacional de Defesa}

No final dos anos 90, os Estados Unidos se apresentavam como o
guarda"-chuva do mundo nos campos cibernético, nuclear e espacial. Por
sua concepção, as Forças Armadas dos demais países serviam apenas para
cuidar de bandido, da repressão e do combate às drogas.

Na última década, no Brasil houve~ enorme avanço na Política Nacional de
Defesa graças a três gigantes: José Genoíno, Nelson Jobim e Celso
Amorim, três grandes brasileiros de formação política distinta, mas que
entenderam perfeitamente o papel das Forças Armadas em uma nação
moderna.

A concepção de defesa nacional baseou"-se no modelo de países
desenvolvidos e foi centrada em três tópicos principais:

\begin{enumerate}
\itemsep1pt\parskip0pt\parsep0pt
\item
  A energia nuclear, sob responsabilidade da Marinha, com as usinas, o
  enriquecimento do ciclo do urânio e os submarinos nucleares.
\item
  A aeroespacial, por conta da Aeronáutica, com os novos caças e os
  satélites brasileiros.
\item
  A cibernética, sob responsabilidade do Exército, com a
  responsabilidade de preparar o país para a guerra eletrônica.
\end{enumerate}

Em cima desse tripé foram traçadas as grandes estratégias de defesa:

\begin{itemize}
\itemsep1pt\parskip0pt\parsep0pt
\item
  Defesa do pré"-sal
\item
  Defesa da Amazônia
\item
  Defesa do cone"-sul
\end{itemize}

Essas definições foram relevantes para garantir a governabilidade nessas
regiões. Há um princípio internacional de montar forças inernacionais em
países que não tenham condições de cuidar de suas áreas. Por isso mesmo,
a Amazônia e a Bacia do Prata foram objeto de estudos e acordos. E~o
Exército sempre recusou aos Estados Unidos treinamento de tropa no
centro de treinamento da Amazônia.

Quando se decidiu ampliar a cooperação continental, o Brasil trocou a
relação bilateral com os Estados Unidos por uma relação de cooperação.

A definição fixada na Unasul e no Conselho de Defesa Sul"-Americano foi
de que apenas para fora haveria a dissuasão; para dentro, a arbitragem e
a cooperação. Todas as disputas haviam sido superadas pelo conceito de
cooperação, definindo operações conjuntas para patrulhar essas áreas
sensíveis.

\subsection{A cooperação regional}

Nesse período, o Brasil assumiu o papel de país central, exercendo a
função de aconselhamento das nações vizinhas. Foi assim que seguraram"-se
por algum tempo as pirações na Venezuela e fortaleceu"-se a ação mais
responsável de Evo Moralez na Bolívia.

O acordo de paz entre as \versal{FARC}s e o governo colombiano foi conquistado
assim. Jobim aconselhou diretamente o governo colombiano e alertava para
os riscos de se colocar bases norte"-americanas no continente:

-- Toda guerrilha quer alvo. Bases americanas são um baita alvo. Vocês
têm que impedir as bases, negociar, porque guerrilha na selva e
cordilheira é imbatível.

Juan~~Manuel Santos, o presidente da Colômbia, aceitou os conselhos,
negociou e o país está saindo da mais prolongada crise política da sua
história.

Agora, com o presidente argentino Maurício Macri abrindo o país para
bases norte"-americanas, e José Serra exibindo uma subordinação
desinformada e primária em relação aos Estados Unidos quebra"-se o
modelo.

\section{Peça 3 -- o desmonte da Defesa}

O governo Dilma Rousseff ``esqueceu'' a política de defesa. Avançou na
licitação \versal{FX}, manteve os programas em execução, mas sem o olhar do
presidente.

Com Michel Temer, vem o desastre.

Nas Relações Exteriores colocou um Ministro, José Serra, totalmente
jejuno na matéria, subserviente aos Estados Unidos, uma ignorância
rotunda. Para ficar no campo conservador, troca"-se o conservadorismo
culto e altivo de um Afonso Arinos, Celso Lafer e Rubens Ricúpero por um
chanceler com o cérebro de Maguila.

Na Defesa, entra um Ministro inexpressivo, Raul Jungman, em que pese ter
pensamento mais sofisticado que o de Serra. Por que se saliento a
inexpressividade? Porque, na outra ponta, o desmonte está sendo
comandado de fora para dentro, através do Ministro"-Chefe do Gabinete de
Segurança Institucional, general Sérgio Etchegoyen, e do Ministro da
Justiça Alexandre de Moraes.

Na gestão Nelson Jobim na Defesa, tentou"-se criar um Centro de Estudos
que aproximasse a academia das Forças Armadas. Jobim criou o Instituto
Pandiá Calógeras, visando:

\begin{enumerate}
\itemsep1pt\parskip0pt\parsep0pt
\item
  Produzir reflexões acerca dos aspectos políticos e estratégicos nos
  campos de segurança internacional e defesa.
\item
  Atrair recursos humanos no campo da defesa.
\item
  Estreitar relacionamento Defesa com meio acadêmico e internacional.
\end{enumerate}

A primeira decisão conjunta de Jungmann e Etchegoyen foi fechar o
Instituto.

\section{Peça 4 -- a segurança interna}

No modelo norte"-americano, há três órgãos distintos trabalhando a
segurança interna: o \versal{FBI}, a \versal{CIA} e a Guarda Nacional. No Brasil, os
correspondentes seriam a Polícia Federal, a \versal{ABIN} (Agência Brasileira de
Inteligência) e a Guarda Nacional -- que jamais foi criada.

A Estratégia Nacional de Defesa tratou o capítulo referente à garantia
da lei e da ordem com a proposta de institucionalização da Força
Nacional de Segurança. O~modelo proposto era o da criação da Guarda
Nacional para cuidar das fronteiras, portos e regiões de conflagração
urbana.

Em um primeiro momento, o Exército entraria, tomaria conta da área e
imediatamente a repassaria para a Guarda Nacional. As Forças Armadas
toparam, mas a Polícia Federal reagiu. E~como no governo Dilma não havia
Ministro da Justiça nem da Defesa, ficou"-se por isso mesmo.

Hoje em dia, a \versal{PF} usa armas camufladas e uniformes de combate
simbolizando um poder a mais.

Não se ficou nisso.

Movimento 1 -- a criação do \versal{GSI}

No presidencialismo, o Presidente é o comandante supremo das Forças
Armadas. Ele é eleito como chefe de governo e de Estado. A~segunda
autoridade é o Ministro da Defesa. Depois, o Chefe do Estado Maior das
Forças Armadas, os Comandantes Militares e os Comandantes de Área.
Assessorando o presidente tem também a Casa Militar.

Criado no segundo governo \versal{FHC}, o Gabinete de Segurança Institucional
(\versal{GSI}) foi confiado ao general Alberto Cardoso e tornou"-se um corpo
estranho. Entrando Lula, este foi alertado para os riscos da criação de
um poder paralelo. A~saída encontrada foi entregar ao general Jorge
Armando Félix, pouco efetivo.

No governo Dilma, o \versal{GSI} foi corretamente extinto, ficando apenas o
Gabinete Militar

Mal assumiu o interinato, Michel Temer recriou o \versal{GSI} e nomeou o general
Sérgio Etchegoyen, colocando debaixo dele a \versal{ABIN} e a Sisbin (Sistema
Brasileiro de Inteligência).

Consciente do novo poder que iria exercer, Etchgoyen imediatamente
convocou para um jantar em sua casa o presidente da República, o
Ministro da Defesa e os três Ministros militares, colocando"-se
simbolicamente acima deles. Em Brasília, na diplomacia e no poder,
jantares têm significado em si.

Movimento 2 -- a segurança nas Olimpíadas

Nas Olimpíadas, Temer nomeou o \versal{GSI} responsável pela segurança,
atropelando os responsáveis naturais, Ministro da Defesa ou da Justiça.
O~Chefe do Estado Maior conjunto sequer foi convidado para a abertura
das Olimpíadas.

A segurança foi organizada pela burocracia das Forças Armadas --
acantonada em Brasília -- não pelas tropas de combate.

Movimento 3 -- a desagregação da defesa

Na Estratégia Nacional de Defesa, a primeira preocupação foi criar o
conceito de comando conjunto. Por exemplo, não adianta submarino sem
satélite ou comando aéreo na Amazônica sem o Exército. Por isso foram
criados comandos em cada região crítica, permitindo a integração das
três forças e a definição de estratégias em cada região.

A criação de Unidades Militares de Combate, seja na Amazônia, Haiti ou
África, deixa claro o verdadeiro papel das Forças Armadas er os
malefícios advindos de sua transformação em polícia. Há levantamentos
internacionais mostrando que, nos países em que se tornaram polícia,
foram sucateadas, com os equipamentos tecnológicos de ponta -- para a
defesa nacional -- substituídos por investimentos em tanques, brucutus,
algemas, granadas e revólveres.

A diluição desse modelo começou com as \versal{UPP}s (Unidades de Policias
Pacificadoras). No início, pareceu dar certo no Rio, devido ao fato do
Secretário de Segurança José Mariano Beltrame ser da \versal{PF} e respeitado por
ela. Ainda no governo Dilma, houve financiamento do governo federal e a
parceria com o Exército.

O Exército burocrático gostou, porque dá visibilidade, nome e prestígio
à força. O~Exército de combatentes -- inteligência, ciência e tecnologia
-- sabia que seria o início do sucateamento, com a burocracia voltando a
tomar conta.

Com o abandono do modelo de integração, voltou"-se à visão
compartimentalizada, com cada tropa lutando por seu quinhão e perdendo a
visão de conjunto e os objetivos nacionais.

\section{Peça 5 -- os caminhos da repressão}

A repressão irá se tornar mais aguda devido a um conjunto de fatores
adicionais.

Na sua gestão no Ministério da Justiça, o Ministro Tarso Genro cometeu
um dos grandes erros estratégicos, ao descentralizar a inteligência na
Polícia Federal. Em nenhum país do mundo comete"-se essa imprudência,
devido ao risco concreto de criar ilhas de poder em cada canto.

O mesmo ocorreu no Ministério Público Federal. A~Constituição já
garantia a autonomia de decisão de cada procurador. A~\versal{AP} 470 e a Lava
Jato, no entanto, internalizou o conceito do direito penal do inimigo.

Aparentemente, o poder conferido pela parceria com a mídia e com a ralé
(no sentido sociológico do termo) conquistou corações e mentes, ainda
mais depois que erros sucessivos de governos do \versal{PT} consagraram o
corporativismo na eleição do Procurador Geral da República (\versal{PGR}). A~corporação passou a se comportar como classe média convencional,
estimulando as ações de repressão contra o inimigo comum e criticando
internamente as manifestações de crítica.

O \versal{MPF} e o \versal{PGR} são responsáveis diretos por esse estado de exceção, ao
permitir a entronização no poder de Michel Temer, Geddel Vieira, Eliseu
Padilha, Moreira Franco. Mas não se espere deles nenhuma ação visando
coibir essa escalada antidemocrática. É~mais fácil ver o \versal{MPF} como linha
auxiliar da repressão do que como baluarte da legalidade.

O \versal{STF} (Supremo Tribunal Federal) dispunha de um quarteto legalista:
Ricardo Lewandowski, Teori Zavascki, Luís Roberto Barroso, o corajoso
Marco Aurélio de Mello e, eventualmente, Luiz Edson Fachin e Rosa Weber.
Barroso e Fachin foram desencorajados por uma campanha infame produzida
por blog de Curitiba e repercutida pelos blogueiros da Veja. Zavascki
alvo de escrachos e de ameaças a ele e seus filho Bastou a Dias Toffoli
aderir a Gilmar Mendes para, de alvo, se transformar em querido da
mídia.

Bastam pequenas decisões contrárias à ralé para serem alvos de campanhas
desmoralizadoras sem que os órgãos de controle atuem. Se alguém
considerar esse estado de coisas incompatível com a normalidade
democrática, que se cale para não sofrer as mesmas represálias.

Se o \versal{STF} deixou na gaveta todas as denúncias contra políticos, não será
agora que se poderá esperar uma atuação mais ativa.

Por outro lado, a cada dia que passar se verá um governo cada vez mais
sem rumo e sem limite, explicitando progressivamente suas tendências
autoritárias; na outra ponta, uma juventude a mil por hora, pegando o
bastão da resistência democrática das mãos dos mais velhos. Eles não têm
a força, mas tem a convicção. A~rapaziada sabe que não está lutando por
Dilma, mas pelas liberdades democráticas. Sabe que na outra ponta -- dos
grupos de poder -- está o preconceito, a insensibilidade social, a
arrogância de quem conquistou o poder sem passar pelo teste dos votos.

E daqui a pouco, as manifestações contra o impeachment chacoalharão todo
o país.

Quem souber o resultado desse Xadrez, está blefando.

\section{Atualização}

Foi uma manifestação cívica, civilizada, disciplinada, alegre, com a
presença de jovens, crianças, moças grávidas.

No final da manifestação, segundo relato de todos os veículos que
acompanhavam --- \versal{G}1, \versal{UOL}, Band, Globo --- a Polícia Militar começou a
jogar bombas sobre manifestantes, sem que nada tivesse ocorrido antes.

Completa"-se o ciclo. A~Folha pediu a violência, Gerlado Alckmin
autorizou, a \versal{PM} cumpriu à risca e dificilmente o \versal{MI}nistério Público
atuará para investigar os fatos e punir os culpados.

Como se dizia nas equações, \versal{CQD} (Como Queríamos Demonstrar)

 

 
