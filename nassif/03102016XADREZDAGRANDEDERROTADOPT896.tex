\chapterspecial{03/\allowbreak{}10/\allowbreak{}2016 Xadrez da grande derrota do \versal{PT}}{}{}
 

As eleições municipais trazem consequências variadas para a oposição e
para a situação.

As principais conclusões a serem tiradas:

\section{Peça 1 -- a derrota de Fernando Haddad}

A derrota no primeiro turno em São Paulo foi a maior demonstração, até
agora, da eficácia da política de ``delenda \versal{PT}'', conduzida pela Lava
Jato junto com a mídia. Não se trata apenas de uma derrota a mais, mas a
derrota do mais relevante prefeito da cidade de São Paulo desde Prestes
Maia.

Enquanto Prestes Maia e Faria Lima ajudaram a consolidar a era dos
automóveis, com suas grandes obras viárias, Haddad trouxe para São Paulo
a visão do cidadão, colocando a política metropolitana em linha com as
mais modernas práticas das grandes capitais do mundo. Não houve setor em
que não inovasse, da gestão financeira responsável à Lei do Zoneamento,
das intervenções viárias às políticas de inclusão.

Teve defeitos, sim. Foi excessivamente tolerante com secretários
medíocres, excessivamente personalista, a ponto de impedir que os bons
secretários ganhassem projeção, descuidou"-se da comunicação e da
presença na periferia. E~não soube difundir de maneira eficiente as
políticas transformadoras que construiu, vítima que foi de um massacre
cotidiano da mídia.

Haddad também era o último grande nome potencialmente presidenciável do
\versal{PT}.

Sua derrota sepulta definitivamente as pretensões do \versal{PT} de ser líder
inconteste das esquerdas, acelerando a necessidade de uma frente de
esquerdas acordada. E~aumenta as responsabilidades sobre o governador de
Minas Gerais, Fernando Pimentel, o do Piauí, Wellington Dias, o do
Maranhão, Flávio Dino.

Esse rearranjo exigirá medidas rápidas do \versal{PT}, a mais urgente das quais
será a de eleger uma nova Executiva em linha com os novos tempos, aberta
à renovação, aos acordos horizontais, antenada com a nova cultura das
redes sociais e dos coletivos. Ou o \versal{PT} se refunda, ou se tornará
politicamente inexpressivo.

Agora, trata"-se de aguardar o segundo turno no Rio. Em caso de vitória
de Marcelo Freixo, do \versal{PSOL}, completam"-se as condições para a unificação
das esquerdas em uma frente negociada, sem aparelhismos, sem pretensões
hegemônicas de quem quer que seja.

\section{Peça 2 -- o novo desenho do golpe}

A vitória de João Dória Jr desequilibra as disputas dentro do \versal{PSDB}. José
Serra já era uma miragem comandado um exército de três ou quatro
pessoas. Como chanceler, tem acumulado uma sucessão inédita de gafes.
Não há nenhuma possibilidade de se recuperar politicamente.

Aécio Neves ainda se prevalece de sua condição de presidente do partido,
mas foi alvejado seriamente pelas delações da Lava Jato -- apesar da
blindagem garantida pelo seu conterrâneo Rodrigo Janot, Procurador Geral
da República (\versal{PGR}).

Com a vitória de Dória, quem sobe é Geraldo Alckmin e sua extraordinária
capacidade de influenciar o homem médio, isto é, o homem medíocre.

Se for adiante a tese do golpe dentro do golpe, com a impugnação total
da chapa Dilma"-Temer, é provável que se decida colocar na presidência da
República alguém com competitividade para se candidatar à reeleição em
2018. Nesse caso, o nome seria Alckmin.

O discurso da anti"-política consegue, assim, dois feitos. O~primeiro, o
de eleger prefeito da maior cidade da América Latina uma pessoa armada
dos conceitos mais anacrônicos possíveis sobre gestão metropolitana. E~coloca como favorito provavelmente o governador mais inexpressivo da
história moderna de São Paulo.

\section{Peça 3 -- o desenho das esquerdas}

No ``Xadrez de Fernando Haddad e a frente das esquerdas''
(\url{migre.me/\allowbreak{}v8ftE)}~levanto a necessidade da explicitação de
um padrão de governança das esquerdas, para reconstrução de uma
alternativa de poder.

Nos próximos meses, à destruição das políticas sociais do governo
federal, corresponderá à destruição das políticas implementadas em São
 Paulo pela era Haddad.

O caminho seria construir uma instância de articulação suprapartidária,
uma espécie de Conselho de Gestão juntando os principais estados e
municípios governados pelos ditos governos progressistas. Os acordos se
fariam acima das idiossincrasias das Executivas (especialmente do \versal{PT}), e
em cima de propostas concretas, de implementação de políticas públicas.

Seria a maneira de juntar os vastos contingentes que despertaram
novamente para a política, depois de excluídos dela pela burocratização
do \versal{PT} -- jovens, intelectuais, especialistas diversos. Nesse caso,
Haddad poderia ser o grande articulador, devido à experiência acumulada
em seus tempos de Ministro da Educação e prefeito de São Paulo, sua
aceitação por jovens e intelectuais.

\section{Peça 4 -- o acirramento da repressão}

Nem se pense que a vitória de Doria em São Paulo reduzirá a gana
repressora.

Nos últimos dias, houve as seguintes ofensivas:

\begin{itemize}
\itemsep1pt\parskip0pt\parsep0pt
\item
  ·~~~~~~ Indiciamento de Lula.
\item
  ·~~~~~~ Prisão de Guido Mantega.
\item
  ·~~~~~~ Prisão de Antônio Palocci.
\item
  ·~~~~~~ Novo inquérito contra Lula, para investigar a colocação de uma
  antena de celular pela Oi, próxima ao sítio em Atibaia. Esses quatro
  itens precedendo as eleições.
\item
  ·~~~~~~ Decisão do \versal{TRF}4 (Tribunal Regional Federal da
  4\textsuperscript{a}~Região) de consagrar o Estado de Exceção.
\item
  ·~~~~~~ Demissão de professor de direito no Mackenzie, por artigo
  contra o golpe.
\item
  ·~~~~~~ Ameaça de demissão a quem insistir no golpe, formulada por
  diretor de redação da revista Época.
\item
  ·~~~~~~ Demissão de José Trajano, comentarista símbolo da \versal{ESPN}, por
  manifestações políticas.
\item
  Afastamento do ex"-\versal{MI}nistro Carlos Gabas dos quadros do Ministério da
  Previdência por ter ajudado no pedido de aposentadoria de Dilma
  Rousseff
\end{itemize}

Não há sinais de que essa escalada enfrentará resistências de jornais e
jornalistas.

A fragilidade financeira dos jornais está submetendo"-os a episódios
outrora impensáveis, frente à camarilha dos 6 que assumiu o poder. Em
outros tempos, com toda sua dose de conservadorismo, o Estadão se
insurgiu contra práticas da ditadura.

Agora, dificilmente.

De grandes monstros tentaculares da estatização, por exemplo, Banco do
Brasil e Caixa Econômica Federal se transformaram nas empresas mais
admiradas pela mídia.

Na primeira semana após o golpe, a Folha convidou a diretoria da \versal{CEF} e
do \versal{BB} para almoços sucessivos. E~o Estadão inventou um novo prêmio para
as empresas mais admiradas. No segmento bancário, deu \versal{CEF} e \versal{BB} na
cabeça.

O balanço dos repasses publicitários à velha mídia, por parte do Miguel
do Rosário, contemplou apenas a publicidade institucional, aquela dos
ministérios. Quando se consolidar com os dados das estatais, se verá o
desenho da bolsa"-mídia.

A ideia do direito penal do inimigo está sendo aplicada em todos os
cantos.

Na matéria sobre publicidade nos blogs considerados de esquerda, a Folha
encampou a tese de Michel Temer, de que a publicidade somente seria para
quem veiculasse notícias de interesse. Ou seja, notícias que atendessem
a um público anti"-esquerda.

Com o Estado de Exceção explicitamente endossado pelo \versal{TRF}4, a colunista
Mirian Leitão, de O Globo, depois de colocar gasolina na fogueira que
fritou Mantega e Palocci, teimou em explicar que não se pode comparar os
tempos atuais com os da ditadura. Reviveu a tese da ditabranda. Depois
justificou que ela foi torturada etc. etc., o que lhe dá obviamente
enorme autoridade moral para explicar que Estado de Exceção não é bem
isso e para colocar mais gasolina na fogueira.

Em ambos os casos, fica nítida a pesada herança inquisitorial de uma
cultura -- a portuguesa -- que aceitava toda sorte de inclusão, desde
que quem chegasse professasse a fé católica.

Esse será o maior desafio do Brasil moderno: a luta pela volta do estado
de direito, independentemente de quem esteja no poder.
