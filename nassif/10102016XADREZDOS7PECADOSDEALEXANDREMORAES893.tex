\chapterspecial{10/\allowbreak{}10/\allowbreak{}2016 Xadrez dos 7 pecados de Alexandre Moraes}{}{}
 

Como a série do Xadrez visa mapear os principais pontos de poder do
golpe, cabe aqui a retificação sobre uma das peças do Estado de Exceção
que se aprofunda no país: o papel do Ministro da Justiça Alexandre de
Moraes.

Ao contrário do que inicialmente se supunha, Moraes não é um risco à
democracia, não por falta de instintos autoritários, mas pela pequena
dimensão política e administrativa. Ele é bem menor do que a sombra que
projeta.

De acordo com os manuais de gestão de recursos humanos, Alexandre de
Moraes incorre em vários pecados abominados pela boa gestão.

\section{Pecado 1 -- o funcionário do ``eu é que fiz''}

Bom chefe é o que sabe reconhecer e premiar os feitos dos subordinados.

Dia desses, assisti a uma palestra"-almoço dele no Instituto dos
Advogados de São Paulo.

No almoço, Moraes apresentou seu projeto para segurança pública.

Primeiro, falou do policiamento nas fronteiras -- papel que efetivamente
lhe cabe, como chefe da Polícia Federal.

Apresentou como grande novidade de seu plano a parceria com governos
vizinhos para atacar o tráfico nos próprios países produtores. Essa
``novidade'' já é implementada há tempos pela \versal{PF}. A~ponto de que, mal
chegou ao Ministério, pediu para participar de uma ida ao Paraguai para
aparecer em vídeos ridículos cortando arbustos de maconha.

Aliás, chegou no almoço acompanhado de duas equipes de cinegrafistas.

Esse exibicionismo já o havia indisposto com a Polícia Civil paulista.
Os delegados apelidaram"-no de Kojak, pela calvície precoce e atração
desmedida pelas câmeras. Ou, então, como o tio soturno da família Adams.
Exigia ser informado sobre todas as operações de impacto da Polícia
Civil para se apropriar do mérito.

\section{Pecado 2 -- o superdimensionamento dos pequenos feitos}

No almoço"-palestra apresentou outra prioridade do Ministério da Justiça:
o combate à violência doméstica. Educados, os comensais contiveram o
riso. A~troco de quê um Ministro, acantonado em Brasília, iria coordenar
uma ação federal de cunho eminentemente local?

O mote foi apenas para apresentar uma estratégia tecnológica
desenvolvida pela \versal{PM} paulista -- que ele assumiu como sua. Consiste em
pegar o Google Maps e anotar os pontos que registram mais casos de
violência doméstica. Depois, ampliar o policiamento nos locais, além de
delegacias especializadas. Aliás, é o primeiro caso mundial de
policiamento preventivo para repressão à violência doméstica que, como o
próprio nome indica, ocorre dentro de casa.

Hoje em dia, estados menores recorrem a sistemas de geo"-referenciamento
muito mais sofisticados, cruzando bancos de dados próprios com os do
Google, identificando regiões de maior criminalidade, analisando
aspectos socioeconômicos, prédios públicos, rede escolar. Mas o bravo
Moraes se jactava para um público de advogados do fato de marcar no
Google Maps as regiões de maior violência doméstica.

Outro episódio do gênero foi a transformação de um grupo de
simpatizantes do Estado Islâmico em perigoso grupo terrorista disposto a
explodir o país, mas contidos a tempo pela atuação do bravo Kojak.

\section{Pecado 3 -- a incapacidade administrativa}

Quando Secretário de Segurança de São Paulo, Moraes -- que se jactava de
usar o Google Maps --- não conseguiu colocar de pé o Detecta, um
supersistema capaz de identificar até gestos suspeitos de criminosos,
anunciado com pompa na campanha eleitoral de Alckmin.

Moraes foi incapaz de implantar o sistema, adquirido da Microsoft, nem
outro sistema caseiro, desenvolvido pela Polícia Militar. Nas
entrevistas da época, não demonstrava sequer conhecimento mais
aprofundado sobre os sistemas, nem sabia explicar as razões do atraso da
implementação (\url{migre.me/\allowbreak{}vbQ12)}.

Ganhou sobrevida com Alckmin devido ao fato do governador não dispor de
sistemas de avaliação de seu secretariado. Seu único termômetro é o que
sai na mídia, terreno preferencial para a atuação de Moraes.

Menor sorte teve como secretário de Gilberto Kassab, até 2010. No
início, Kassab impressionou"-se com os modos ``deixe"-que"-eu"-chuto'' de
Moraes. Este se tornou supersecretário das pastas de Transportes e de
Serviços, presidente do Serviço Funerário, da \versal{SPT}rans e da Companhia de
Engenharia de Tráfego (\versal{CET}).

Insurgiu"-se contra a criação da Autoridade Metropolitana de Transportes,
órgão vital para o planejamento do trânsito; entregou apenas um dos
cinco corredores de transporte prometidos, falhou na criação da
ciclofaixa do 23 de maio. Mais: anunciou a proibição de estacionamento
de carros em ruas do centro, sem comunicação prévia ao prefeito, criando
outra crise política.

\section{Pecado 4 -- os grandes lances sem planejamento}

Como Secretário de Justiça do primeiro governo Alckmin, ensaiou alguns
lances de defesa dos direitos humanos.

Seu ato mais expressivo foi despedir 1,6 mil funcionários da Febem,
recém contratados através de concurso. Cometeu uma arbitrariedade que
foi revogada pelo \versal{STF} (Supremo Tribunal Federal), trazendo enormes
prejuízos aos cofres do Estado.

Depois, na gestão Kassab, os próprios funcionários da prefeitura
perceberam o jogo, a dissintonia completa entre o discurso para consumo
do prefeito e os resultados efetivos.

Por exemplo, decidiu reduzir em 20\% o valor dos contratos de lixo.
Segui"-se uma greve de garis, com o lixo de acumulando por toda a cidade
e ajudando a ampliar a tragédia das enchentes -- agravadas por outro
seguidor do estilo Moraes, o governador José Serra, que suspendeu as
verbas para desassoreamento do Tietê. As enchentes acabaram resultando
em 59 mortos e mais de 900 pontos de alagamento em três meses de chuva.

\section{Pecado 5 -- a aliança com insubordinados}

Trata"-se de uma fraqueza indesculpável em chefias. Quando não conseguem
se impor sobre os subordinados, aderem a eles.

Assim que assumiu o Ministério, correu para visitar a Força Tarefa da
Operação Lava Jato e anunciou uma série de reuniões com
superintendências da Polícia Federal, hipotecando apoio total à
operação, como se tivesse alguma ascendência sobre a \versal{PF}.

Adotou esse mesmo comportamento em relação à Polícia Militar paulista.
Para não parecer que tinha perdido o controle, preferiu comprometer sua
biografia, apresentando"-se como chefe inconteste dos massacres.

Antes de chegar à Secretaria, Moraes foi titular de ações judiciais da
\versal{PM} contra o procurador da República Matheus Baraldi Magnani, que
denunciou as violências da corporação (\url{migre.me/\allowbreak{}vbSmW)}~após
18 assassinatos pela Rota. E~crucificou um \versal{PM} por denunciar as
violências da corporação. Está sendo processado pelo \versal{PM} por isso.

Assim como em relação à Polícia Civil paulista, irritou os delegados da
\versal{PF} com o exibicionismo, situação que se agravou quando, em Ribeirão
Preto, se exibiu aos correligionários do \versal{PSDB}, anunciando grandes
operações para a semana seguinte, pré"-eleições.

A retaliação veio na forma de um vazamento da Operação Acrônimo, da
própria \versal{PF}, mostrando que seu escritório recebeu pagamento de empresas
envolvidas com o caso (\url{migre.me/\allowbreak{}vbSap)}. Até agora não
explicou adequadamente que tipo de serviço prestou. Nem convenceu sobre
sua suposta ascendência sobre a \versal{PF}.

\section{Pecado 6 -- a traição com o antigo empregador}

Demitido por Kassab, tempos depois foi acusado de vazar notícias para a
imprensa, sobre supostas irregularidades na coleta de assinaturas para a
fundação do \versal{PSD}.

Em seguida, aliou"-se a Michel Temer, que se tornou seu padrinho na
indicação para Secretário da Segurança de Alckmin. Em maio de 2015, o
blog ``Flit Paralisante'' -- porta"-voz se parte da Polícia Civil --
informava sobre a possibilidade do Secretário de Segurança de Alexandre
Moraes ser prefeiturável em 2016 (\url{migre.me/\allowbreak{}vbvxr)}. Nomeado
Secretário, passou a privilegiar delegados ligados ao ex"-Secretário
Antônio Ferreira Pinto, também do \versal{PMDB}.

\section{Pecado 7 -- os episódios controvertidos}

Em seu tempo de Secretário da Segurança de Franco Montoro, Michel Temer
saiu com suspeita de ligação com bicheiros, que teriam ajudado no
financiamento de sua campanha para deputado.

Mal saiu da Secretaria dos Transportes de Kassab, Moraes tornou"-se
advogado da Transcooper, a cooperativa de vans controlada pelo \versal{PCC}. Sua
defesa foi a de que a Transcooper nào estava envolvida em nenhuma
irregularidade -- só seus associados.

\section{Conclusão}

Pelo acúmulo de abusos, pelo comportamento desajeitado como Ministro da
Justiça, fica claro a pequena dimensão política de Moraes. Uma das
principais características dos verdadeiramente poderosos é justamente o
de não alardear seu poder.

O poder da repressão, de fato, está em mãos mais profissionais, do
general Sérgio Etchegoyen comandando os serviços de inteligência das
Forças Armadas.

Moraes é apenas um personagem atrás de holofotes e de um roteiro melhor.
