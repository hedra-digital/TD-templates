\chapterspecial{06/\allowbreak{}12/\allowbreak{}2016 Xadrez do confronto entre Supremo e Senado}{}{}
 

\section{Peça 1 -- os sindicatos que ajudaram na Lei de Abuso de
Autoridade}

O \versal{PLS} (Projeto de Lei do Senado) Sobre Lei de Abuso de Autoridade teve o
apoio de dois sindicatos, o Sindicatos dos Agentes Penitenciários do
Estado de São Paulo e o Sindicato dos Metalúrgico das Agulhas Negras.

O primeiro, devido à situação insustentável dos presídios paulistas,
abarrotados por prisões preventivas abusivas praticadas por juízes.
Tratando diretamente com a massa carcerária, viam o equilíbrio delicado
dentro dos presídios e entendiam que só um conceito mais amplo de
Justiça impediria uma explosão.

Já o Sindicato dos Metalúrgicos reagia contra o que denunciavam uma
perseguição implacável contra o setor, de procuradores estaduais e
desembargadores, a serviço do então governador do Rio, Sérgio Cabral.

As denúncias ao Conselho Nacional do Ministério Público revelaram"-se
inúteis. O~Sindicato demonstrou a perseguição empreendida por quatro
procuradores do estado do Rio, que concorreu para a demissão do
presidente do Sindicato, que era também presidente da Câmara dos
Vereadores de Rezende.

No Senado, há uma montanha de documentos, em \versal{PDF}, demonstrando a atuação
política dos procuradores que --- segundo o sindicato --- serviriam aos
propósitos políticos do então governador Sergio Cabral.

Como o juiz Sérgio Moro estava longe dos acontecimentos, acabou
ordenando a prisão de Cabral, ocasião em que apareceu o trabalho de Luiz
Zveiter --- o polêmico desembargador carioca, principal responsável
pelos abusos do Judiciário fluminense, que ontem mesmo foi eleito
presidente do Tribunal de Justiça do Rio.

\begin{center}~\end{center}


O terceiro Sindicato foi o dos bombeiros, alvo de uma perseguição
implacável dos homens de Cabral, quando ousaram uma greve da categoria.

Os sindicatos acabaram se aproximando e entrando conjuntamente no
Supremo, para uma ação de repercussão geral visando o reconhecimento do
reajuste anual. Como policiais, bombeiros e metalúrgicos recebem salário
insalubridade, a ação proposta por São Paulo acabou juntando as três
categorias.

Marco Aurélio foi o relator, comoveu"-se com o movimento dos bombeiros e
deu"-lhes voto favorável. A~partir daí os sindicatos perceberam que
poderiam peitar os abusos do Poder Judiciário em algumas localidades.

Ontem mesmo, os três sindicatos se articularam para ingressar no Supremo
como~\emph{amicus curiae}, visando reverter a decisão de Marco Aurélio.

\section{Peça 2 -- sobre Marco Aurélio de Mello}

Não se sabe a motivação do Ministro Marco Aurélio, ao tomar uma decisão
que joga gasolina pura na fogueira do conflito entre Legislativo e
Judiciário.

Marco Aurélio não é de esquemas corporativos, nem de manobras políticas.
Trata"-se de um autêntico escoteiro no Supremo, cumprindo com
independência e coragem suas atribuições.

Para os sindicatos, desde que a filha se tornou desembargadora no
Tribunal de Justiça do Rio, Mello teria mudado sua posição em relação
aos esbirros da Justiça.

De qualquer modo, sua decisão tem um tom cinza, que ainda não deu para
decifrar.

Concede"-se liminar para evitar situações em que possam ocorrer danos
irreversíveis. O~senador Renan Calheiros conduz três projetos
absolutamente polêmicos.

Projeto 1 -- o Projeto de Lei das 10 medidas.

Projeto 2 -- O projeto da Lei de Abusos de Poder. Somado à decisão de
investigar os salários dos juízes e procuradores acima do teto.

Por aí, se poderia concluir que Mello tomou partido da Justiça e do
Ministério Público contra o Congresso.

Por outro lado, vota"-se o \versal{PEC} 55, sobre o teto de despesas, e a reforma
da Previdência. Renan é peça"-chave com que conta Michel Temer para essas
aprovações. Saindo Renan, assume a presidência do Senado o senador
petista Jorge Vianna.

Sob esse prisma, Mello estaria impedindo abusos de uma maioria
financiada a ouro e cargos no governo, visando atacar o funcionalismo
público e as políticas sociais.

ou poderia ser apenas uma reação às manobras protelatórias de Dias
Tofolli e Gilmar Mendes

O mistério fica no ar.~O mais provável é que tenha evitado
a~\emph{embromation~}que levou o \versal{MI}nistro Teori Zavascki a postergar por
um tempo infindável a queda de Eduardo Cunha,~~

\section{Peça 3 -- os procuradores e a Lei de Abuso de Autoridade}

O irresponsável e imaturo procurador Deltan Dallagnoll estimulou a
violência fascista contra os deputados que votaram contra as Dez
Medidas. Não cuidou de negociar, de procurar o Congresso. Tratou de se
valer da violência das ruas para estimular ataques ao Congresso. Foi o
que ocorreu no aeroporto de Fortaleza, Entre manifestantes e
parlamentares.

 

No Rio, camisas amarelas agrediram o desembargador Luiz Zveiter.

Ontem, o Procurador Geral Rodrigo Janot e seu estado"-maior procuraram
jogar água na fervura, com um comunicado
(\url{https:/\allowbreak{}/\allowbreak{}goo.gl/\allowbreak{}8UxYtQ)}~no qual se colocaram à disposição do
Congresso para um diálogo construtivo, a Declaração de Brasília:

``Ainda segundo a Declaração de Brasília, o Ministério Público
brasileiro expressa, finalmente, que é favorável ao aperfeiçoamento da
Lei de Abuso de Autoridade de 1965, colocando"-se à disposição para
colaborar com o Congresso Nacional, mediante diálogo construtivo. Todos
os procuradores"-gerais do \versal{MP} brasileiro concordaram com o teor da
Declaração''.

Estendem uma bandeira branca, mas nem um pio sobre a irresponsabilidade
e a indisciplina de um procurador imaturo, que joga com o nome do
Ministério Público Federal, e com um cargo para o qual foi alocado, para
se tornar dono da moralidade pública.

\section{Peça 4 -- ~Rodrigo Janot e a delação da \versal{OAS}}

Tem"-se então, os seguintes ingredientes para um rascunho de mapa do
inferno:

\begin{enumerate}
\itemsep1pt\parskip0pt\parsep0pt
\item
  A ampliação do conflito entre os poderes. Se o Senado recuar, será
  fatalmente engolido pelo Judiciário. Terá que reagir, o que ocorrerá
  inevitavelmente nos próximos dois dias, ampliando o grau de
  imprevisibilidade do momento.
\item
  Ontem, um juiz federal decretou o bloqueio de bens do casal Eliseu
  Padilha, devido à devastação produzida em área de preservação
  ambiental.
\item
  As delações da Odebrecht começam a vazar por todos os poros.
\item
  Depois que o Ministrio Teori Zavascki liquidou com uma das maiores
  conquistas democráticas -- o juiz natural -- tudo pode acontecer.
\item
  A economia continua derretendo.
\end{enumerate}

Enquanto isto, Rodrigo Janot impediu pela segunda vez a delação da \versal{OAS}
de avançar. Após a primeira tentativa de brecá"-la, a Lava Jato retomou
as negociações.

Há dois tipos de irregularidades no relacionamento
empreiteiras"-políticos.

O primeiro é o do chamado caixa dois: financiamento de campanha vindo do
caixa dois, mas sem contrapartida explícita da parte do financiado.

O segundo é a corrupção a seco, com o pagamento de percentuais sobre as
obras contratadas.

Como a Odebrecht avançou tudo o que podia sobre caixa dois, a única
forma da \versal{OAS} apresentar novidades seria escancarar a corrupção. Janot
barrou a primeira tentativa.

Agora, os delatores decidiram contar, por exemplo, que a propina era de
5\% na primeira gestão de Geraldo Alckmin. José Serra entrou e anunciou
uma redução de 4\% no valor dos contratos. As empreiteiras, então,
reduziram a propina para menos de 1\% do valor das obras. Imediatamente,
Serra enviou Paulo Preto para renegociar as propinas.

Tudo isto estará ao alcance da Lava Jato, assim que Janot parar com a
política de blindar políticos aliados.
