\chapterspecial{30/\allowbreak{}09/\allowbreak{}2016 Xadrez da delação da Folha contra jornalistas}{}{}
 

\section{Passo 1 --- a matéria sem gancho da Folha}

Na quinta"-feira, a Folha de S Paulo publicou uma não"-matéria.

A não"-notícia é que o governo Temer resolveu proibir toda publicidade de
empresas públicas nos blogs críticos a ele. Ora, esse fato ocorreu no
primeiro dia em que Temer assumiu o cargo. Qual a novidade para
justificar a matéria? Nenhuma.

O gancho da matéria é o não"-fato de que, desde que Temer anunciou a
proibição de publicidade de empresas públicas nos blogs, nenhum órgão
público anunciou mais nos blogs. Cadê a notícia?

A reportagem usou a não"-notícia como álibi para falar dos valores
aplicados nos blogs no ano passado, recusando"-se a comparar com os
valores investidos na velha mídia, tanto antes como depois dos vetos
políticos aos blogs.

Tem lógica? Do ponto de vista jornalístico, não. Do ponto de vista
político, sim.

Entende"-se a reportagem juntando outras peças do jogo:

1. A~\versal{PGR} (Procuradoria Geral da República) insiste na tese da
organização criminosa nacional, infiltrada em todos os poros da
República, submetida a um comando central. A~Lava Jato reitera essa
versão.

2. O~\versal{TRF}4 (Tribunal Regional Eleitoral da 4a Região) autoriza o Estado
de Exceção e consequentemente a aplicação do direito penal do inimigo.

Com base nesses dois pontos, qualquer crítica à Lava Jato pode ser
enquadrada como articulação da organização criminosa.

Todos os sinais são ~de que se arma ~uma ofensiva contra todos os pontos
críticos à Lava Jato e ao golpe. O~próximo passo será investir contra os
blogs, que a Folha trata como se fosse uma organização vinculada ao
comando central.

Até agora, esse estímulo à barbárie se dava de forma diluída, com
colunistas endossando as arbitrariedades, colocando lenha na fogueira,
mas sem nominar os ``inimigos'' a serem alvejados.

Com Otávio Frias Filho, a Folha abre mão dos pruridos e torna"-se o
primeiro veículo a partir para a delação explícita, sendo secundada por
blogueiros especializados em deduragem.

O que está por trás disso?

\section{Passo 2 --- a incompatibilidade do golpe com a liberdade de
expressão}

Como alertei em vários capítulos da série Xadrez, os passos iniciais do
golpe permitiam o exercício da hipocrisia. Para Fulano A apoiar o golpe,
sem afetar sua imagem, as pessoas têm que acreditar que ele acredita que
o golpe foi feito para combater a corrupção. Para Fulano B apoiar, as
pessoas precisam crer que ele acredita que a Lava Jato é politicamente
isenta. Para apoiar o governo Temer, o Fulano C tem que esquecer as
denúncias que já saíram contra Eliseu Padilha, Geddel Vieira Lima e o
próprio Michel Temer e fingir que acredita que eles sào verdadeiros
varões de Plutarco.

Por mais que seja difícil acreditar, a sociedade brasileira ainda está
submetida a algumas formas de restrição de consciência, mesmo com a
velha mídia tratando"-a como fraqueza moral em sua linha editorial.
Grandes atos de canalhice, falta de isonomia, delação, posições
arbitrárias, injustiças ainda têm o condão de despertar sentimentos de
indignação.

Ainda há algumas linhas morais que não podem ser ultrapassadas, sob
risco de manchar indelevelmente a reputação de quem o fizer.

Com a liberdade de expressão da Internet e das redes sociais, esse jogo
da hipocrisia, de fingir que não se sabe, torna"-se inviável.

Dia desses, um post sobre a delação de Eike Baptista -mostrando como os
procuradores da Lava Jato fugiam de qualquer menção ao \versal{PSDB} --- rendeu
mais de 420 mil visualizações apenas no \versal{GGN}, sem contar a reprodução em
vários outros sites independentes e blogs. A~série do Xadrez tem
conseguido de 60 a 100 mil visualizações únicas no \versal{GGN}, também sem
contar a republicação por outros sítios. O~mesmo acontece com artigos de
outros blogs.

Com o avanço do Estado de Exceção, esse jogo de delações praticado pela
Folha cria a possibilidade concreta de mandar blogueiros para a cadeia,
submetê"-los a processos, ou intimidá"-los ante o risco de serem alvo de
arbitrariedades. Vivem"-se tempos de exceção.

Mesmo sabendo disso, a Folha não vacilou. Aparentemente, Otávio decidiu
atravessar a tal linha moral.

Hoje em dia, os blogs independentes são a última cidadela contra a
escalada do arbítrio. É~nesses blogs que as consciências individuais,
nas diversas áreas profissionais, vêm buscar ânimo para romper com a
cultura do medo que se instalou no seu meio e no país

Se conseguirem nos calar --- como pretende Otavinho --- os próximos
alvos serão os próprios jornais, assim que ousarem se colocar
minimamente no caminho do arbítrio. Não existe arbitrariedade que se
esgote em si. Cada ato desses cria jurisprudência, abrindo espaço para
sua reiteração.

\section{Passo 3 --- a bolsa mídia e o pacto com Temer}

A segunda razão dessa armação contra os blogs tem a ver com a
bolsa"-mídia, em preparação no governo Temer. Seu preparo já foi
suficiente para segurar as denúncias contra o Ministro"-Chefe da Casa
Civil Eliseu Padilha. De uma das grandes capivaras do meio político,
Padilha tornou"-se um douto senhor, pontificando sobre temas complexos e
sendo tratado com respeito reverencial.

Como expliquei em outros lances do Xadrez, provavelmente a bolsa"-mídia
consistirá nas seguintes etapas:

\begin{enumerate}
\itemsep1pt\parskip0pt\parsep0pt
\item
  1. Financiamento do \versal{BNDES}.
\item
  2. Publicidade oficial maciça, com campanhas e cadernos especiais
  bancados pelos Ministérios.
\item
  3. Operação fisco. Na crise, a primeira conta a não ser paga é com o
  fisco. Em outros tempos, grupos de mídia conseguiram se safar de
  multas bilionárias através de diversos expedientes, como advogados da
  União perdendo prazo, redução retroativa de alíquotas de impostos,
  desaparecimento de processos na Receita.
\end{enumerate}

Com liberdade de expressão, esses caminhos tornam"-se complexos. Por isso
mesmo, a agressão contra os blogs não irá parar na reportagem de hoje.

Lembro que foi uma firme posição moral que, nos idos dos 80 e 90
transformou a Folha no maior jornal do país.

Nos tempos em que havia um comercial lembrando:

\begin{quote}
``É possível contar um monte de mentiras, dizendo só a verdade. Cuidado
com a informação do jornal que você recebe. Folha de São Paulo: o jornal
que mais se compra. E~que nunca se vende''.
\end{quote}

Agora, uma frouxidão moral inesculpável poderá ser o seu epitáfio.

\versal{PS} -- Meus respeitos ao repórter desconhecido, que se recusou a assinar
a matéria na Folha, sabendo que seria utilizada como instrumento de
delação de colegas.
