\chapterspecial{25/\allowbreak{}06/\allowbreak{}2016 Moro e a exploração da ingenuidade imperial}{}{}
 

É irreal o grau de ingenuidade de setores do \versal{PT} e do governo, aliviados
com as informações que constam nos documentos que embasaram a Operação
Acarajé, de que não há nenhuma evidência de que o marqueteiro João
Santana tenha recebido dinheiro ilegal para as campanhas de Lula e
Dilma.

Então, o juiz Sérgio Moro teria autorizado a prisão de Santana por
suspeita de financiamento oculto para as campanhas presidenciais na
República Dominicana?

É evidente que o objetivo de Sérgio Moro é derrubar o governo. É~evidente que Moro está alinhado à oposição e à estratégia de Gilmar
Mendes no \versal{TSE} (Tribunal Superior eleitoral).

Pessoas minimamente antenadas teriam percebido desde o início a
estratégia de Moro -- porque é óbvia, pouco sutil. É~impressionante, no
entanto, a facilidade com que iludiu seus principais alvos, a presidente
Dilma Rousseff e seu conselheiro"-mor, o Ministro da Justiça José Eduardo
Cardozo.

Há duas lógicas na Lava Jato: uma, a lógica político"-midiática, de criar
o clima para o golpe final; e a lógica jurídica.

Juízes só julgam sobre o que está nos autos. Pode"-se fazer a campanha
política mais radical, mais evidente, desde que não conste dos autos.
Constando dos autos, assim que os processos saírem do Paraná os
tribunais superiores poderão reconhecer a parcialidade do juiz e a
intenção política da Lava Jato.

A estratégia da Lava Jato foi simples. De posse das informações
levantadas, com o poder de editar como bem entendesse, já que aliada à
mídia, a Lava Jato direcionou todas as investigações para o lado de Lula
e alimentou a mais pertinaz campanha de desconstrução da imagem de uma
pessoa pública, desde a campanha contra Getulio Vargas em 1954.

A prisão da nata da malandragem teve como único objetivo recolher
informações de ordem política. Depois, todos foram liberados, assim como
Alberto Yousseff na Operação Banestado.

A campanha midiática teve por objetivo estimular o clamor popular e
demolir as resistências do Judiciário contra os abusos da operação, até
que o consenso popular faça o Judiciário assimilar qualquer Fiat
Elba.~Enquanto os delegados vazavam toda sorte de factoides, e os
procuradores todo tipo de discurso político, os autos permaneceram
impolutos: não há nada contra Lula, Lula não está sendo investigado,
Lula não é suspeito.

No reino colorido de Brasília, o conselheiro José Eduardo Cardozo
acalmava a Rainha Dilma e lhe dizia:

— Viu? Não há o que temer. A~operação é republicana.

Ontem, na véspera de se consumar o estupro, começou a cair a ficha das
virgens do Planalto de que havia algo de estranho no comportamento de
Sérgio Moro.

Provavelmente, é o mais ingênuo governo da história do país. Nem
Wenceslau Braz conseguiu superá"-lo.
