\chapterspecial{25/\allowbreak{}06/\allowbreak{}2016 O tempo político de Dilma está se esgotando}{}{}
 

Uma análise realista dos possíveis cenários futuros indicam que, em
breve, o impasse político"-econômico será rompido, ou com um governo de
coalizão, ou com o caos.

As peças que compõem esse jogo são as seguintes:

\textbf{Peça 1\emph{~-- O tempo político de Dilma Rousseff encurtou
consideravelmente.}}

Há uma crise fiscal acelerada, no meio de uma crise política que tem
paralisado todos os passos do governo. A~aprovação da \versal{CPMF} (Contribuição
Provisória Sobre Movimentação Financeira) é essencial para o equilíbrio
fiscal e para reverter a queda perigosa do \versal{PIB}. Com ameaça de nova queda
de 4\% do \versal{PIB}, as receitas fiscais caindo vertiginosamente, os estados
entrando em default, não há muito tempo pela frente para a hora da
verdade.

\textbf{Peça 2}~--~\textbf{\emph{Cresce a convicção de que Dilma não
conseguirá montar um plano política e economicamente viável.}}

O Ministro da Fazenda Nelson Barbosa precisaria ser suficientemente
ousado para apresentar um grande plano que não implicasse em riscos
fiscais, que não aprofundasse a recessão e, ao mesmo tempo, passasse a
ideia de previsibilidade -- para contornar as resistências ao seu nome
-- sem descontentar a base do governo, mais à esquerda.

Montou duas propostas que não impactando o curto prazo poderiam acenar
para o longo: reforma da Previdência e limites para as despesas
públicas. Não foi suficiente para demover a direita e provocou rupturas
no lado esquerdo da base de apoio.

Para conseguir apoio à \versal{CPMF}, o governo concordou com as pressões do
presidente do Senado Renan Calheiros, flexibilizando a lei do petróleo.

No momento, está por um fio a única base política efetiva com quem Dilma
pode contar.

\textbf{Peça 3}~--~\textbf{\emph{Dilma não conseguirá se equilibrar
entre mercado e base.}}

Exemplo claro foi o anúncio da reforma da Previdência. O~mercado ouviu
com pé atrás; à esquerda reagiu. No momento seguinte ela escala o
Ministro do Trabalho Miguel Rossetto para explicar que não era bem
assim.

Queimou"-se com o mercado e com a esquerda.

\textbf{Peça 4}~--~\textbf{\emph{Mesmos nos círculos próximos a Dilma,
aumenta a convicção de que a crise é grande demais para ela.}}

Mesmo os habilidosos Jacques Wagner e Ricardo Berzoini têm enorme
dificuldade em convencê"-la de medidas óbvias. O~termo mais usado no
Palácio é ``não adianta dar murro em ponta de faca''.

\textbf{Peça 5}~--~\textbf{\emph{Há um amplo espaço para aprofundamento
da radicalização política e policial.}}

Tanto no \versal{STF} como no \versal{STJ}, qualquer Ministro que ouse uma postura mais
garantista acaba vítima de ataques à reputação ou pelos jornais ou redes
sociais. E~poucos têm estrutura emocional para enfrentar a barbárie.

\textbf{Peça 6}~--~\textbf{\emph{Ruim com Dilma, o caos com o
impeachment.}}

Suponha que Gilmar Mendes atropele leis e regulamentos e emplaque o
impeachment via \versal{TSE} (Tribunal Superior Eleitoral). O~caso iria para o
\versal{STF}. Se Dilma e Temer fossem cassados, quem assumiria? O presidente do
Senado, Renan Calheiros, alvo da Lava Jato? Eduardo Cunha e sua imensa
capivara? O país entraria em ebulição.

\section{O governo de coalizão}

Juntando todas essas peças, chega"-se à conclusão de que a única saída
seria um governo de coalizão com Dilma, tipo o que foi montado por
Itamar Franco,~ quando pegou o pepino de suceder a Fernando Collor.

Ocorre que um governo de coalizão exige que o presidente efetivamente
abra mão de poder.

No momento, Dilma tenta montar a coalizão em cima de medidas goela
abaixo, ~mantendo o comando, recusando"-se a abrir mão de qualquer espaço
de poder. Não funciona.

Como diria Ricardo Berzoini -- justificando o acordo de flexibilização
do pré"-sal -- o governo tem que ser realista e entender quando perde as
condições políticas e negociar uma política de menor dano.

O aprofundamento da crise obrigará Dilma a cair na real em um ponto
qualquer do futuro.

Para acelerar a transição, quando a hora chegar, sugere"-se aos
articuladores políticos responsáveis que comecem a elaborar as ideias
para tornar a transição a menos traumática possível.

Como suas pretensões políticas acabam em 2018, Dilma terá facilidades em
arbitrar uma coalizão que garanta igualdade de condições a todas as
partes.
