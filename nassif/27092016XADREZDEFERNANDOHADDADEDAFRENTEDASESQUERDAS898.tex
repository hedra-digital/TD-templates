\chapterspecial{27/\allowbreak{}09/\allowbreak{}2016 Xadrez de Fernando Haddad e da frente das esquerdas}{}{}
 

\section{Peça 1 --- a esquerda e o culto à generosidade}

Recentemente, o filósofo de direita Luiz Felipe Pondé escreveu um artigo
dizendo que a direita tinha que aprender a conquistar as moças.~ Passou
uma lição importante na forma de uma deboche, provavelmente a maneira
que encontrou para chegar ao seu público que, além do economicismo
estéril, aprecia bobagens machistas.

Em síntese, diz ele que o fascínio da esquerda sobre a juventude está na
generosidade, na solidariedade, no fator humano, enquanto a direita se
prende a um economicismo vazio.

Matou a charada.

Tenho um bom laboratório familiar. As meninas se dizem, agora, de
esquerda. O~que as motiva é a solidariedade para com os mais fracos, o
combate à intolerância, o direito de cada um de ser o que quiser, desde
que não prejudique o próximo, e o fato de encontrarem, nesses grupos,
jovens solidários entre si e dispostos a tomar posição. E~intuem que
essa luta só será bem"-sucedida através da organização política.

Nem se pode dizer que haja influência paterna. Pelo contrário, estão
conhecendo melhor o pai através dos coleguinhas.

Esse é o sentimento identificado por Pondé, que é orientador em um
ambiente ``coxinha'' --- a \versal{FAAP}. Em cima disso, ele mantém a crítica
contra a esquerda, que instrumentaliza os jovens etc.

O grande desafio é como transformar esses valores em ações concretas,
devolvendo à rapaziada a crença na política.

\section{Peça 2 --- os fantasmas de 2013}

Essa explosão da nova vitalidade política da juventude nasce em junho de
2013, com a fantástica mobilização do \versal{MPL} (Movimento Passe Livre).
Estavam ali as sementes para uma reenergização da militância, um
rejuvenescimento dos partidos.

Infelizmente, essa sede de participação esbarrou em duas muralhas
intransponíveis: na presidência da República, Dilma Roussef; na
presidência do \versal{PT}, Ruy Falcão.

Dilma nunca escondeu uma profunda impaciência para tratar com movimentos
sociais. Desde os anos 80, no Sindicato dos Jornalistas de São Paulo,
Ruy se revelou o aparelhista clássico: aquele que, quando assume o poder
em uma organização, toda sua energia não é para projetar seu poder para
fora, mas para se consolidar para dentro, fechando as portas para
impedir competição interna.

Tem"-se agora uma movimentação extraordinária, a esquerda se recompondo
nas ruas, tendo como elemento aglutinador a defesa da democracia e das
políticas sociais. E~a volta dos intelectuais e dos jovens às batalhas
civilizatórias, das quais foram expulsos pelo envelhecimento dos
partidos.

O desafio está no plano institucional, a luta autofágica entre os
diversos partidos de esquerda.

\section{Peça 3 --- uma estratégia para as esquerdas}

Há duas lutas políticas no momento.

Uma delas, da esquerda contra a esquerda.~ O \versal{PT} envelheceu, fechou as
portas do partido para qualquer arejamento, e crises sucessivas
promoveram o crescimento de outros partidos à esquerda, dos quais o \versal{PSOL}
é o mais destacado. Agora, há o sonho de partidos menores de aproveitar
a perda de rumo do \versal{PT} para assumir a liderança das esquerdas.

Cometem dois erros. O~primeiro, de supor que a nova etapa da política,
com sua multiplicidade de coletivos, grupos de interesse, redes sociais,
comporta posições hegemônicas. Não comporta, nem do \versal{PT}, nem de seus
sucedâneos.

O segundo, de não perceber que a grande luta, hoje em dia, é contra uma
direita ferozmente antissocial que, associada ao fisiologismo mais
deletério, pretende desmontar todos os sinais de políticas sociais do
governo, apagar a memória recente do país, deletar um novo modo de
governar que se desenvolveu nas últimas décadas.

Daqui para frente, haverá dois caminhos para a reestruturação das
esquerdas.

O primeiro é prosseguir no jogo atual, de cada qual por si e o \versal{PT} para o
Ruy. Seria começar do zero uma dura caminhada para reinventar um modo de
fazer política de esquerda, com a maior parte da energia sendo consumida
em disputas estéreis. Depois, uma longa caminhada para conquistar
prefeituras e estados para, só aí, testar novos modelos de gestão.

A maneira mais rápida e objetiva seria através de um pacto em torno de
governos já existentes, e que atuariam como âncoras nessa reconstrução.
É~o caso do Piauí, Maranhão, Bahia, mas especialmente de Minas Gerais e
da prefeitura de São Paulo, em caso de reeleição de Fernando Haddad.

Esses governos"-âncora serão fundamentais para a consolidação de práticas
administrativas bem"-sucedidas, sua padronização e a formação de quadros
para disseminação por prefeituras e governos de Estado que elegerem
candidatos de esquerda. Além disso, servirão de retaguarda para
movimentos sociais e grupos de resistência à tentativa de ditadura que
se avizinha.

Obviamente, há um conjunto de pressupostos para uma estratégia
bem"-sucedida:

\begin{enumerate}
\itemsep1pt\parskip0pt\parsep0pt
\item
  1.~~~ Um pacto inicial entre prefeitos e governadores, talvez debaixo
  de um Instituto, um think tank suprapartidário para pensar o novo
  tempo, composto por integrantes de partidos de esquerda, movimentos
  sociais, coletivos, organizações da sociedade civil e lideranças
  jovens.
\item
  2.~~~ Um ambiente de confiança entre as partes e a definição de regras
  de partida que impeçam a prevalência de interesses particulares sobre
  o geral. Enfim, uma institucionalização da frente das esquerdas,
  obviamente abrindo espaço nas administrações para técnicos de outros
  partidos.
\item
  3.~~~ A renovação da executiva do \versal{PT} será peça central. O~partido
  possui alguns políticos com amplo trânsito e histórico de compromisso
  com participação social, como Patrus Ananias, Gilberto Carvalho, o
  próprio Jacques Wagner.
\item
  4.~~~ Acordos de colaboração entre os diversos governos, para seleção,
  padronização e multiplicação das experiências bem"-sucedidas.
\item
  5.~~~ Há que se criar uma nova marca de gestão, ousando práticas
  inovadoras para enfrentar velhos problemas, aprimorando as boas
  práticas desenvolvidas nos últimos anos. Foi mais ou menos o que
  ocorreu nos anos 80, com o tal modo \versal{PT} de gpvernar, com suas práticas
  de orçamento participativo e outros.
\end{enumerate}

Esse novo modelo poderá surgir de Minas Gerais, quando o governador
Fernando Pimentel superar os baques emocionais dos últimos anos, do
Piauí e do Maranhão. Mas, principalmente, da prefeitura de São Paulo.

É por aí que ganha especial relevância a reeleição Fernando Haddad.

\section{Peça 4 --- o fator Haddad}

Haddad não é apenas um dos últimos remanescentes do \versal{PT} em alto cargo
administrativo:~ é o administrador que melhor desenvolveu um método de
governo de esquerda moderna. Hoje em dia, São Paulo é o mais importante
laboratório para novas práticas de gestão social.

Tem princípios, valores e visão transformadora. Foca nos resultados
finais e não se deixa atrapalhar por idiossincrasias ideológicas.

No \versal{MEC} (Ministério da Educação), Haddad~ foi responsável pelos programas
de maior impacto e eficácia no governo Lula. Na prefeitura, montou
programas consistentes voltados para as minorias, colocou de pé
políticas públicas importantíssimas.

Sua administração é passível de críticas, óbvio. Poderia ter se aberto
mais para os conselhos de saúde, para os coletivos jovens, se aproximado
da periferia não apenas com obras, mas com sua presença física
sinalizando um protagonismo maior delas.

Mas conseguiu interlocução com setores modernos da sociedade, teve a
ousadia de enfrentar tabus e colocar São Paulo em linha com as modernas
políticas de humanização das metrópoles.

Mais que isso, em nenhum momento subordinou a prefeitura às
idiossincrasias da máquina partidária. Embora tenha demorado a perceber
a onda jovem que chegou com as redes sociais, provavelmente é a
personalidade pública mais admirada pela juventude paulistana. E~certamente, ao lado de Patrus Ananias, a liderança de esquerda com maior
empatia com a intelectualidade.

Seu desempenho nas eleições será de importância fundamental para a
construção dessa frente de esquerda. Se passar para o segundo turno,
poderia ser o momento para o pontapé inicial no grande pacto
progressista.
