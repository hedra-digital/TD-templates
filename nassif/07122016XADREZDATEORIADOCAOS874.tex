\chapterspecial{07/\allowbreak{}12/\allowbreak{}2016 Xadrez da teoria do caos}{}{}
 

\section{Peça 1 -- guerra entre poderes}

O quadro atual transcende as análises lógicas, as correlações racionais.
O~golpe em cima do pilar máximo de uma democracia -- o mandato de uma
presidente eleita -- desestruturou o equilíbrio institucional do país.

Os eventos se sucedem aleatoriamente e criam uma dinâmica própria, sem
que surjam forças moderadoras. Procuradores insuflam as ruas, presidente
de Supremo se comportam como líder sindical, no Executivo um presidente
tatibitate cercado de assessores do pior quilate. E, nessa salada, uma
confluências de episódios potencializando a crise.

Na segunda"-feira, o Ministro Marco Aurélio de Mello tomou a decisão
isolada de acatar a liminar da Rede e afastar o presidente do Senado.
Havia antecedentes no próprio caso Eduardo Cunha. Aceita a denúncia
contra Cunha, não poderia mais ser titular de um cargo -- no caso,
Presidência da Câmara -- na linha da sucessão presidencial. Sucedeu o
mesmo com Renan, com algumas especifidades.

Há procedimentos políticos a serem respeitados. No caso, decisão desse
calibre, que envolve um conflito entre poderes, não poderia ser tomado
individualmente por nenhum Ministro.

Como resposta, o Senado decidiu não acatar a liminar, afirmando que
aguardaria a manifestação do pleno do Supremo.

Nesta quarta"-feira, o pleno do Supremo analisará a liminar de Marco
Aurélio. Há três possibilidades:

\textbf{Hipótese 1}~- Acatar integralmente o voto de Marco Aurélio de
Mello.

\textbf{Hipótese 2}~- Acatar em parte: ele continuaria na presidência do
Senado, mas não poderia se habilitar à linha sucessória.

\textbf{Hipótese 3}~- Não endossar a liminar.

A~\textbf{Hipótese 1}~é defendida por Marco Aurélio. A~linha sucessória
tem que ser analisada como um bloco único, no qual não poderia haver
autoridades processadas.

A~\textbf{Hipótese 2}~é de Celso de Mello: o político processado pode
permanecer no cargo até o momento que se configure qualquer forma de
sucessão. Só seria afastado nessa hipótese.

Com a decisão do Senado de não acatar a liminar -- embora em uma nota
ponderada, informando aguardar a decisão do pleno -- decide"-se a
questão: o Supremo endossará integralmente a~\textbf{Hipótese 1}.

No final da tarde, os Ministros estavam imersos em dúvidas
institucionais: e se o Senado não acatar? O Supremo convocará força
policial para cumprir a determinação? O Senado manterá Renan na
presidência, mesmo correndo o risco de todas as decisões serem anuladas?

A maior probabilidade é do Senado acatar a decisão final e afastar
Renan. Trata"-se do mais grave conflito institucional desde a
redemocratização e a humilhação imposta ao Senado, com sua autoridade
sendo questionada não por Marco Aurélio, mas pelos moleques da Lava
Jato, deixará sequelas que perdurarão no tempo.

\subsection{\textbf{Peça 2 -- a autocontenção e a Presidente sem noção}}

A Suprema Corte norte"-americana adota o sistema da autocontenção. Ou
seja, as votações devem observar o momento, as circunstâncias, para não
atiçar os ânimos, em tempos de conflagração política ou social.

Aliás, foi a observância desse princípio que levou o Ministro Teori
Zavascki a adiar por tanto tempo a decisão sobre Eduardo Cunha, com
efeitos deletérios sobre a democracia, saliente"-se.

Agora, tem"-se na presidência do Supremo uma Ministra, Carmen Lúcia,
obcecada por manchetes, sem noção dos impactos de suas palavras e atos
sobre o ânimo nacional.

Em meio ao terremoto pós"-impeachment, colocou sucessivamente na pauta o
julgamento de Renan Calheiros, a \versal{ADPF} (Ação de Descumprimento de
Preceito Fundamental) sobre a manutenção no cargo de presidente do
Senado para alguém processado.

Fez mais, ao contrário do que apregoou em sua posse, não definiu um
prazo para os pedidos de vista. Com seu pedido de vista, Dias Toffoli
paralisou o julgamento da admissibilidade de abertura de processo contra
Renan, provocando a irritação de Marco Aurélio.

Carmen Lúcia só não colocou em votação com a ação de José Eduardo
Cardozo questionando o mérito do impeachment, porque ela tem lado.

Assim, cria"-se o caos rapidamente:

\begin{enumerate}
\itemsep1pt\parskip0pt\parsep0pt
\item
  Renan anuncia a intenção de apurar salários acima do máximo
  constitucional e a lei contra abuso de autoridade.
\item
  Há uma reação de alguns setores do Judiciário e , imediatamente,
  Carmen Lúcia se apresenta como a frasista sindical, sem a menor noção
  sobre o peso institucional do cargo, e coloca em pauta o julgamento de
  uma das ações contra Renan.
\item
  Nos momentos seguintes, os procuradores da Lava Jato colocam lenha na
  fogueira, mirando no presidente do Senado.
\item
  A turba assimila o discurso e investe contra Renan.
\item
  Marco Aurélio aceita a liminar e ordena o afastamento de Renan.
\item
  O Senado reage e não aceita a ordem, aguardando o julgamento do pleno.
\item
  Para não criar um grave precedente -- uma ordem do Supremo
  desobedecida --o Supremo deverá engrossar na decisão.
\item
  Ao mesmo tempo, o Procurador Geral da República pede o afastamento de
  Renan e Romero Jucá, as duas âncoras de Temer no Senado.
\item
  Gilmar Mendes atropela todos os limites, ataca Marco Aurélio, opina
  sobre matéria que irá julgar e consolida"-se definitivamente como o
  intocável -- para vergonha de um Supremo tíbio, que há tempos deveria
  ter imposto limites em sua atuação alucinada.
\end{enumerate}

E, nessa balbúrdia inédita, uma presidente de Supremo sem noção que
poderá irromper a qualquer momento com uma frase de efeito, açulando
ainda mais os ânimos, acrescida de um Ministério Público sem comando.

\section{Peça 3 -- os desdobramentos}

Como salientado no início, não há mais previsibilidade nos eventos
políticos e econômicos.

De mais certo, tem"-se o fracasso precoce do governo Temer e do Ministro
da Fazenda Henrique Meirelles -- uma espécie de Felipão da economia.
Temer se tornou disfuncional.

Ao mesmo tempo, há indefinição completa sobre os próximos passos. De um
lado, o \versal{PSDB} ajuda na desmoralização de Temer, mas não ousa pegar para
si o cálice da recessão econômica. A~imprensa se tornou uma biruta de
aeroporto. Não mais conduz: passou a ser conduzida pela irracionalidade
das ruas.

Nas ruas, há pólvora no ar, tanto pela ultradireita comandada pela Força
Tarefa da Lava Jato, quanto dos funcionários públicos sem salários no
Rio de Janeiro e, em breve, por outros estados da federação.

O jogo se encaminha para duas possibilidades:

\begin{enumerate}
\itemsep1pt\parskip0pt\parsep0pt
\item
  A localização de um nome capaz de conduzir a transição. Até agora, o
  mais forte é o de Nelson Jobim.
\item
  A possibilidade de eleições diretas, cada vez mais concreta.
\end{enumerate}
