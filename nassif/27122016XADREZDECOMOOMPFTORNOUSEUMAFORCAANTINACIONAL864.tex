\chapterspecial{27/\allowbreak{}12/\allowbreak{}2016 Xadrez de como o \versal{MPF} tornou"-se uma força antinacional}{}{}
 

\section{Peça 1 --- o cenário pré"-Lava Jato}

A Lava Jato vai revelando dois aspectos do estágio de desenvolvimento
brasileiro.

O primeiro, a corrupção endêmica e generalizada que foi apodrecendo o
sistema político sem ser enfrentada por nenhum partido. Era o tema à
vista de todos e há décadas percebido pela opinião pública, o único tema
capaz de provocar a comoção geral.

O segundo, as indicações de que o país estava a caminho de se
transformar em uma potência média, repetindo a trajetória de outras
potências, inclusive no atropelo das boas normas.

Como potência média, ainda não havia desenvolvido internamente
legislações e regulamentos que disciplinassem o financiamento político,
que blindassem as empresas que representassem o interesse nacional, os
procedimentos que impedissem ~que o combate à corrupção comprometesse
setores da economia. Enfim, todo esse aparato jurídico"-político com que
as nações desenvolvidas desenvolvem e blindam suas empresas e até tratam
com tolerância, criando uma zona de conforto para que possam pular os
limites, nos casos de ampliação do chamado poder nacional.

O Brasil trilhava o caminho de potência média, mas sem essas
salvaguardas e sem os cuidados necessários.

Os arquivos da Odebrecht revelam influência no México, Peru, Equador,
Argentina, Colômbia, Guatemala, República Dominicana e Panamá, nas
eleições de vários países da região, na esteira da ampliação da
influência diplomática brasileira, além da notável expansão das
empreiteiras na África e América Latina (\url{https:/\allowbreak{}/\allowbreak{}goo.gl/\allowbreak{}oyxNpa}).

Por outro lado, desenvolvia"-se uma indústria de defesa autônoma, com
absorção de tecnologias avançadas e inúmeras possibilidades abertas com
a quase consolidação dos \versal{BRICS} e das parcerias com a China e seus bancos
de desenvolvimento. Avançava"-se nos submarinos, nos satélites e na
informática.

Com a descoberta do pré"-sal, o país se projetava como um dos futuros
grandes produtores de energia, desenvolvendo paralelamente uma indústria
naval potente e uma grande cadeia de fornecedores para as mais diversas
necessidades, de máquinas, equipamentos, caldeiraria a sistemas
informatizados de ponta.

Nascia uma nova potência.

Mas havia uma pedra no meio do caminho: a falta de foco interno sobre o
chamado interesse nacional e uma corrupção generalizada na política. Em
cima dessa vulnerabilidade, desse calcanhar de Aquiles, o Reino foi
buscar seus campeões, os candidatos a Paris, os jovens mancebos do
Ministério Público Federal capazes de, a pretexto do combate à
corrupção, liquidar com as pretensões nacionais.

É assim que se inicia nossa história. Antes de prosseguirmos, um pouco
das disputas históricas entre potências estabelecidas e candidatas a
potência.

\subsection{Peça 2 -- o complexo de vira"-lata}

Qualquer obra de história da economia identificará o desenvolvimento
como um processo gradativo. A~estratégia de cada país deve se dar de
acordo com suas circunstâncias, com seu grau de desenvolvimento, com o
nível de competitividade da sua economia.

Desde a primeira metade do século 19 consagrou"-se o conceito do
``chutando a própria escada'' na economia política.

Coube ao economista alemão Friedrich List (1789--1846) decifrar o jogo
das potências. Com um diagnóstico correto dos fatores de
desenvolvimento, List ajudou a Alemanha a desenvolver o Sistema Nacional
de Inovação e a consagrar o conceito da união nacional como fator
essencial de consolidação econômica e política.

A nova ciência preconizava que da ambição de cada indivíduo se faria o
progresso. List rebatia que nem toda iniciativa era virtuosa e caberia
ao Estado definir um projeto de país no qual pudessem ser canalizadas as
iniciativas de seus cidadãos.

Para se tornar a primeira superpotência da era industrial, a Inglaterra
se valeu de todos os recursos que tinha à mão. Praticou pirataria, impôs
acordos comerciais lesivos aos parceiros, protegeu seu mercado da
invasão dos produtos têxteis indianos, criou reservas de mercado para
sua armada, e demanda para seus estaleiros.

Montou um mercado global para seus produtos. Consolidado o mercado,cada
fazendeiro que resolvesse mudar de ramo adquiria uma pequena máquina
têxtil. O~mercado era tão grandioso que em menos de um ano triplicava
sua produção, principalmente porque o setor era protegido da invasão dos
têxteis indianos, de muito melhor qualidade.

\subsection{Chutando a própria escada}

Depois de consolidado seu poder sobre o mercado global, a Inglaterra
passou a defender o livre mercado, a abolição de práticas
protecionistas, insurgiu"-se contra o tráfico negreiro, não por razões
humanitárias --- que não cabiam em quem impôs à Índia um imperialismo
sangrento \mbox{---,} mas puramente econômicas.

A maneira de chutar a própria escada foi com a cooptação de políticos e
intelectuais de outros países. Através de cursos e visitas à Inglaterra
voltavam deslumbrados com o avanço do país e passavam a vender a ideia
que a modernidade consistia em emular o estilo que a Inglaterra adotara
depois de ter se tornado potência.

Mais arguto observador do seu tempo, List teve papel relevante para
convencer seus conterrâneos que o processo de desenvolvimento se dava em
estágios. Daí, a impossibilidade de países pré"-industriais emularem
estratégias de países já plenamente industrializados, se desarmando de
todos os instrumentos de defesa da produção e do mercado internos antes
de atingirem o estágio dos países desenvolvidos.

Em 1792, o então secretário do Tesouro norte"-americano, Alexander
Hamilton, apresentou o ``Report of Manufactures'', o primeiro projeto de
defesa das manufaturas norte"-americanas, em reação ao protecionismo que
havia na Europa. As tarifas iniciais foram insuficientes. Mas em 1808,
com a guerra explodindo, o comércio com a Europa foi interrompido. Em um
ano, o número de indústrias têxteis saltou de 8.000 para 31 mil. Quando
o livre comércio foi retomado, veio de novo a crise.

Eram essas evidências que List ia buscar para desenvolver os princípios
de sua economia política

List não conhecia o termo ``vira lata'' para descrever os
internacionalistas deslumbrados de seu tempo. Mas descreveu de forma
definitiva a maneira como as sub"-elites intelectuais alemãs aderiram ao
discurso inglês, por modismo, ignorância ou para poder ascender social
ou profissionalmente junto aos setores ligados ao exterior. Em suma, o
avesso do avesso desse rapaz deslumbrado, o Deltan Dallagnoll.

 No entanto, foi através desse deslumbramento de procuradores,
procurando emular os yuppies do mercado financeiro, que a geopolítica
norte"-americana conquistou seu mais notável feito: o da judicialização
da política nos países democráticos, promovendo a maior quantidade de
desestabilizações políticas da história, sem envolver um míssil sequer
nos embates. E~o instrumento utilizado foi o instituto da cooperação
internacional contra a corrupção.

Afinal, ser contra o combate à corrupção, quem haveria de?

\section{Peça 3 --- a cooperação internacional contra a corrupção}

Nas últimas décadas, Síria, Egito, Líbia e Iraque se constituíram na
aliança mais expressiva contra o eixo Estados Unidos"-Israel no Oriente
Médio.

Contra a Líbia, se buscou o álibi da derrubada do ditador sanguinário; o
mesmo na Síria e no Egito; no Iraque, o combate às armas químicas de
alta letalidade, que jamais foram encontradas. Países inteiros foram
destruídos e submetidos a sistemas muito mais cruéis.

Paralelamente, contra a Índia, a socialdemocracia portuguesa, espanhola,
alemã e francesa, montaram"-se campanhas com denúncias a granel,
produzidas pela cooperação internacional.

Essa nova forma de atuação geopolítica surge no momento em
redesenhava"-se a geografia mundial.

Nos anos 80, a estratégia norte"-americana de abrir mão de setores
industriais permitiu a explosão de novos centros industriais pelo
planeta. Criou"-se um quadro acomodatício com os \versal{EUA} criando empregos na
China e na Ásia e os chineses financiando o consumo norte"-americano.

O sonho acabou em 2008 e, ali, a China já se projetava como potência
industrial tornando"-se o chão de fábrica do mundo enquanto a Índia se
convertia no chão de escritório, com seus serviços de informática. Os
\versal{BRICS} se projetam criando seu próprio banco de desenvolvimento e
anunciando o lançamento próximo de sua própria moeda e o Brasil, além de
potência agroexportadora, se projeta com suas siderúrgicas e
empreiteiras ocupando espaços na América Latina e África.

Por outro lado, desde os anos 80 a liberalização financeira provocara a
proliferação de paraísos fiscais, por onde circulavam recursos dos
petrodólares, dos magnatas japoneses, dos narcotraficantes colombianos,
dos plutocratas russos, dinheiro de corrupção política e pública. A~maneira de enfrentar essas práticas foi através da globalização da
repressão.

Dos anos 90 para cá foram construídas três grandes convenções
internacionais contra a corrupção, que serviram de alavanca principal
para o processo global de judicialização da política.

\begin{itemize}
\itemsep1pt\parskip0pt\parsep0pt
\item
  A.~~~~ Convenção Interamericana contra a Corrupção, concluída em
  Caracas, Venezuela, em 29 de março de 1996, patrocinada pela
  Organização dos Estados Americanos (\versal{OEA}).
\item
  B.~~~~ Convenção sobre o Combate da Corrupção de Funcionários Públicos
  Estrangeiros em Transações Comerciais Internacionais, concluída em
  Paris, França, em 17 de dezembro de 1997, patrocinada pela \versal{OCDE}.
\item
  C.~~~~ Convenção das Nações Unidas contra a Corrupção, aprovada pela
  Assembleia Geral das Nações Unidas em 31 de outubro de 2003, assinada
  pelo Brasil em 9 de dezembro de 2003 e promulgada pelo Decreto n\,  5.687, de 31 de janeiro de 2006. Também conhecida como \versal{UNCAC} (United
  Nations Convention Against Corruption) ou ainda como Convenção de
  Mérida, cidade do México onde foi assinada.
\end{itemize}

Essas convenções passam a estimular a cooperação recíproca entre países,
por meio de assistência técnica, treinamento, cooperação jurídica
internacional, parcerias formais e trocas de informações por vias
informais. E~passaram a promover o envolvimento da sociedade civil,
através das organizações não governamentais (\versal{ONG}s).

Dois pontos saltaram à vista na consolidação dessas políticas.

\begin{enumerate}
\itemsep1pt\parskip0pt\parsep0pt
\item
  Os interesses econômicos explícitos, na criação de regras
  internacionais para impedir que atos de corrupção pudessem atrapalhar
  a livre competição. A~preocupação inicial era com a concorrência
  desleal no comércio exterior. Tanto que foi a partir de estudos da
  ~\versal{SEC} (a \versal{CVM} dos \versal{EUA}) que surge a Convenção sobre Corrupção de
  Funcionários Públicos em Transações, bancada pela \versal{OCDE}.
\item
  O conceito de soberania nacional como principal adversário da
  cooperação. Inicialmente, devido à dificuldade em extraditar
  criminosos, por conta de conceitos tortos de soberania.
\end{enumerate}

\section{Peça 4 --- a demonização do conceito de Nação~}

Para a área de direitos humanos, o conceito de Nação sempre foi
negativo. Era através dele que se criavam distinções entre cidadãos da
terra e imigrantes, que se proibiam fluxos migratórios, que se impedia a
extradição de criminosos comuns, de guerra ou aqueles que cometeram
crimes contra a humanidade.

Nos anos 70, era comum o Brasil abrigar criminosos estrangeiros,
protegidos pela não existência de tratados de extradição.~ Em 2003, o
\versal{STF} negou a quebra de sigilo bancário no país, dizendo que o pedido
atentava contra a ordem pública brasileira. Este ano, mesmo, o Supremo
impediu a deportação de um criminoso de guerra argentino.

Com o tempo, passou"-se a demonizar o próprio conceito de interesse
nacional.

Vários artigos sobre o tema foram publicados no caderno ``Temas de
Cooperação Internacional'' da Unidade de Cooperação Internacional do
\versal{MPF}. Como mencionado em um dos textos: ``A cooperação jurídica
internacional constrói a ideia de um espaço comum de justiça, com
reconhecimento mútuo de jurisdições. Embora não se exija para ela a
harmonização de legislações, é evidente que a transformação do mundo em
uma aldeia global termina por promover essa ideia, inegavelmente ligada
à relativização do dogma da soberania''.

Os setores do \versal{MPF} ligados à cooperação internacional passaram a tratar
de forma negativa todo conceito de soberania como se, em todas as
circunstâncias, fosse um obstáculo à inevitabilidade da nova ordem
global. Como se soberania significasse o atraso e globalização a
civilização. E~interesse nacional fosse apenas um álibi para atrapalhar
o trabalho dos justiceiros globais.

De repente, procuradores caboclos e delegados tupiniquins esquecem as
origens, e são alçados à condição de polícias do mundo, ombreando"-se com
colegas norte"-americanos, suíços, ingleses. As novas tropas globais
passam a ser enaltecidas em séries de \versal{TV} e, pouco a pouco, vão criando
uma superestrutura acima dos poderes nacionais, dando partida à
judicialização da política em nível global.

A criação de uma ideologia internacionalista e antinacional no \versal{MPF} foi
um trabalho bem mais meticuloso, no qual as conferências tiveram papel
central.

\section{Peça 5 -- os controles legais nacionais}

No início da década de 2000, no Brasil, surgiram três órgãos voltados a
certos aspectos de contenciosos internacionais: em 2003, o Departamento
Internacional (\versal{DPI}) da Advocacia"-Geral da União; em 2004, o Departamento
de Recuperação de Ativos e Cooperação Jurídica Internacional (\versal{DRCI}) do
Ministério da Justiça; em 2005 a Secretaria de Cooperação Internacional
(\versal{SCI}) do Ministério Público Federal. No \versal{MPF} foram criadas unidades
especializadas.

A autoridade central para a cooperação passou a ser o \versal{DRCI} ~da
Secretaria Nacional de Justiça (\versal{SNJ}), do Ministério da Justiça. Apenas
abria"-se exceção para o acordo do Brasil com Portugal e com o Canadá,
casos em que a autoridade central é a Procuradoria Geral da República.

Era através do \versal{DRCI} que o Ministro da Justiça poderia exercer o controle
sobre os pedidos da cooperação. Caberia a ele o suporte e orientação e o
ponto de contato entre as autoridades brasileiras e internacionais para
inquéritos policiais e processos penais. Permitindo, também, o controle
de todas as cooperações pelo Ministro da Justiça.

No governo Dilma Rousseff, o Ministro da Justiça José Eduardo Cardozo
abriu mão completamente desse trabalho, por inércia acabou entregando o
controle total da cooperação à Procuradoria Geral da República.

Para se preparar para a cooperação, o \versal{MPF} havia criado o Centro de
Cooperação Jurídica Internacional (\versal{CCJI}), ainda na gestão de Cláudio
Lemos Fonteles. Em dezembro de 2010, na gestão de Roberto Gurgel, foi
substituído pela Assessoria de Cooperação Jurídica Internacional
(\versal{ASCJI}).

Em setembro de 2013, em um dos primeiros atos do novo \versal{PGR} Rodrigo Janot,
foi criada a Secretaria de Cooperação Jurídica Internacional (\versal{SCI}), pela
primeira vez sob o comando de um procurador em regime de dedicação
plena, contando com grupos de apoio para cada área de atuação.

Havia uma razão de ordem prática e outra de ordem política para a
criação desses grupos especializados.

\subsection{Peça 6 -- a criação da comunidade das polícias do mundo}

As Conferências constatavam que a posição dos países poderia variar, de
acordo com o presidente ou parlamentares eleitos, atrapalhando a
continuidade dos trabalhos.

Juntavam procuradores, delegados, fiscais de todas as partes do mundo,
tendo em comum a ameaça da subordinação ao poder do Executivo, a quem
caberia sempre a última palavra sobre a cooperação. Bastaria entrar um
presidente avesso à cooperação internacional, para a estrutura interna
desmoronar.

Para se impor sobre a vontade do Executivo, decidiu"-se recomendar a cada
país a criação de estruturas permanentes, comunicando"-se entre si e
articulando os trabalhos de juízes, procuradores, fiscais e delegados de
polícia, de maneira a dar um~\emph{by pass}~nas limitações jurídicas e
políticas convencionais, com suas estruturas burocráticas, processos
lentos de decisão e interesses particulares ou nacionais.

A troca direta de informações deveria ser pontual. No Brasil, tornou"-se
uma constante, principalmente devido à anomia do Ministério da Justiça.

A cooperação passou a estimular cada vez mais as comunicações diretas
entre seus membros. Cada vez mais foram assinados tratados (ou
iniciativas baseadas na reciprocidade) prevendo a comunicação direta
entre órgãos do Judiciário, com eliminação das autoridades diplomáticas.

O objetivo principal foi colocar os inquéritos fora do alcance das
autoridades do Executivo. Como diz um dos artigos: ``Com as comunicações
diretas, evita"-se ainda o inconveniente de fazer com que autoridades do
Executivo assumam atividades sem conexão com suas tarefas principais,
participando dos atos de cooperação de forma demasiadamente
desinteressada, formal ou burocrática. ''

Surge, então, uma organização supranacional, que gradativamente tenta"-se
colocar acima dos governos nacionais. Os encontros anuais, as redes de
relacionamentos, os sistemas de premiação oficiais ou de blogs
internacionais especializados, tornam"-se a bússola desse novo poder. A~Convenção de Palermo induz à formação de equipes conjuntas de
cooperação, ampliam"-se as formas de contato direta, através de
videoconferências e da criação de redes, como a Rede Judicial Europeia e
a Rede Ibero"-americana de Cooperação Jurídica Internacional.

A~\emph{accountability}~(prestação de contas) desses poderes envolvidos
na luta contra a corrupção, passa a ser para os acordos de cooperação,
não para os governos nacionais. Os vira"-latas passam a disputar as
premiações internacionais. E~o tamanho do prêmio dependia dos recordes
obtidos de prisões e de desmonte da economia dos seus países.

Em um quadro de ampla dissipação moral na política, bastava apenas
apontar os adversários da globalização que o \versal{MPF} se encarregava de
decapitar, poupando e aliando"-se aos aliados dos interesses centrais. É~o que explica a ampla blindagem do \versal{PSDB}.

\section{Peça 7 -- o conceito de Nação}

 Um presidencialismo de coalizão que se enlameou com a corrupção, um
Legislativo totalmente comprometido, um Supremo medroso, uma imprensa
venal, Forças Armadas burocratizadas, tudo isso convergiu para abrir um
espaço sem precedentes para o desmonte do país.

É em cima desse vácuo que cresceu a Operação Lava Jato. Em vez de
instrumento para o saneamento amplo da política brasileira, tornou"-se a
responsável pelo maior trabalho de destruição da história da economia
brasileira.

Nunca o sentimento de lesa pátria foi tão explícito em um dos poderes da
República, provavelmente nem no Banco Central, quando promoveu o maior
crescimento da dívida pública da história.

Jovens procuradores deslumbrados, com complexo explícito de vira"-lata,
juízes provincianos, uma corporação cega, sem um pingo de inteligência
corporativa, chefiada por um Procurador Geral medíocre, sem ~visão de
país e dos jogos globais do poder, comandaram o primeiro tempo do jogo:
o da destruição.

Haverá novos tempos. O~poder político se reconstituirá, com partidos de
extração política diversas.

Com um Congresso revigorado, ou um Executivo forte, haverá a prestação
de contas. Não escaparão de uma \versal{CPI} para analisar sua conduta
antinacional. E~essa conduta não está nos corruptos e corruptores que
foram presos, nem mesmo nos abusos cometidos, na parcialidade flagrante
das investigações. Mas em uma ação deliberadamente antinacional.

A \versal{CPI} terá condições de analisar todos os acordos de cooperação, abrir
as gavetas indevassáveis do Procurador Geral, levantar o que estava por
trás dessa fúria antinacional, conferir o que ele foi fazer no
Departamento de Justiça e em outros órgãos do governo dos \versal{EUA}, levando
informações contra a Petrobras e trazendo contra a Eletronuclear.

Mesmo antes disso, a imprudência com que o \versal{PGR} atuou nesse período já
está promovendo a volta do cipó de Aroeira: basta conferir a quantidade
cada vez maior de reportagens tratando procuradores e juízes como
marajás.

Antes da luta aberta, haverá o sufoco financeiro do \versal{MPF}, prejudicando
enormemente o trabalho sério e patriótico dos procuradores que
continuaram acreditando no \versal{MPF} como fator de defesa dos direitos dos
vulneráveis e da modernização do Brasil.
