\chapterspecial{25/\allowbreak{}06/\allowbreak{}2016 O xadrez da Lava Jato e a incógnita Janot}{}{}
 

Vamos por partes, para fechar o raciocínio. Começando por questões já de
conhecimento geral.

\textbf{Peça 1~- a campanha contra Lula tem caráter eminentemente
político.}

No início os vazamentos da Lava jato se valiam do álibi de que era
necessário criar a comoção popular para superar os obstáculos nas
instâncias superiores. Hoje em dia, com a operação sendo amplamente
avalizada nos tribunais superiores, a continuidade do vazamento há muito
deixou de ser uma estratégia jurídica para se tornar uma arma política.
Especialmente analisando"-se o nível dos vazamentos, buscando muito mais
expor a vida privada de Lula do que levantar aspectos jurídicos.

\textbf{Peça 2~- a política de vazamentos é avalizada por toda a força
tarefa da Lava Jato.}

Desde o início, a Lava Jato tem pautado sua atuação por total disciplina
e concordância de todas as partes em torno das estratégias traçadas.
Portanto, as decisões --- inclusive quanto aos vazamentos --- são
coletivas, tendo o endosso das partes.

\textbf{Peça 3~- os vazamentos estão claramente articulados com a
estratégia pro"-impeachment da oposição.}

O xadrez é nítido:

1.~~~ A campanha do impeachment esfria no final do ano e com o desgaste
dos opositores, devido ao fato, entre outros, da enxurrada de denúncia
do ano passado ter virado notícia velha. Sem carne fresca não haverá
como estimular a besta.

2.~~~ Dilma tenta retomar o protagonismo, com o reinício do Conselhão, a
mudança no Ministro da Fazenda, a articulação política com novo fôlego,
com Jacques Wagner e Ricardo Berzoini.

3.~~~ No dia 13 de março haverá o próximo desafio das manifestações de
rua pró"-impeachment. Se esvaziadas enterram de vez a tese do
impeachment.

Nesse intervalo, procuradores e delegados articulados com a mídia
garantem munição para um bombardeio incessante e diuturno.

E aqui se faz uma pequena pausa para relembrar alguns princípios de
estratégia militar que foram largamente assimilados no século 20 na
disputa política pelo mercado de opinião.

Inicia"-se a guerra com as chamadas batalhas de exaustão, aquelas em que
se recorre maciçamente a bombardeios aéreos ou em terra, visando exaurir
as energias e a vontade de batalhar dos adversários. No caso do mercado
de opinião, a artilharia de exaustão é a mídia com a chamada publicidade
opressiva.

Depois, entram em cena a cavalaria (os tanques), abrindo espaço para a
infantaria. No caso, a formalização dos inquéritos através de processos
na Justiça e \versal{CPI}s no Congresso.

A vitória final se dá apenas quando a infantaria consegue controlar o
espaço adversário. Isto é, quando os aliados do grupo conseguem levar a
cabo o impeachment.

Mídia, procuradores e delegados estão nitidamente na fase inicial, das
chamadas guerras de exaustão.

\textbf{Peça 4~-- o principal beneficiário de um eventual impeachment
seria o senador Aécio Neves.}

Impeachments não se fazem no vazio. A~não ser a besta -- a massa de
manobra -- ninguém entra em um processo de impeachment sem ter noção
clara sobre os vencedores. O~\versal{PSDB} tem três candidatos a candidatos em
eleições presidenciais. O único deles que ganharia com a antecipação das
eleições -- na hipótese de impeachment -- seria Aécio Neves.

Até aí, nenhuma novidade. São tão nítidos esses movimentos que não há
prazer intelectual nenhum em desvendá"-los. Os mistérios que rodeiam a
Lava Jato estão alguns degraus acima, no Executivo e nas cúpulas do
Ministério Público Federal, Polícia Federal e Poder Judiciário.

\section{Os personagens dessa trama}

Grosso modo, há cinco tipos de personagens nessa trama. Contra o
impeachment os militantes do \versal{PT} e os defensores da legalidade. A~favor,
os conspiradores ostensivos, os conspiradores que desempenham papel
ativo na conspiração, mas sem se revelarem, e os intimidados pelo rugido
da besta (a opinião pública nas ruas).

Não é tarefa difícil identificar em qual dos escaninhos da história
colocar personagens como Luís Roberto Barroso, Marco Aurélio de Mello,
Celso de Mello, Gilmar Mendes, Dias Toffoli, Carmen Lúcia, Fernando
Henrique Cardoso, José Serra, Ayres Britto, Globo, Folha, Estadão,
Abril.

Enfrentar a besta -- a voz das ruas -~ exige mais coragem do que
enfrentar as baionetas, especialmente para aqueles que prezam sua
reputação. ~Enfrentar as baionetas sujeita a pessoa até a torturas
físicas, mas engrandece a reputação. Enfrentar as ruas, e os ataques à
reputação, exige uma coragem e desprendimento apenas disponíveis nos
grande homens, como o Ministro Luís Roberto Barroso.

Torna"-se muito mais complicado analisar o papel de três personagens: a
presidente Dilma Rousseff, o Ministro da Justiça José Eduardo Cardozo e
o Procurador Geral da República Rodrigo Janot. Os três são responsáveis
diretos pelo nível do abuso em que incorre diariamente o Ministério
Público Federal e a Polícia Federal, como partícipes do linchamento
midiático de Lula.

Dilma e Cardozo são um pouco menos difíceis de entender.

Dilma é uma mulher de uma coragem à toda prova, mas desde que saiba o
que fazer. As sutilezas do jogo político decididamente não são a sua
praia. Ela está imobilizada não pela falta de coragem, mas por não saber
como agir. Mais: é capaz de enfrentar as piores torturas físicas sem
ceder. Mas não tem a menor estrutura psicológica para enfrentar ataques
à reputação. Foi por aí que a frente pro"-impeachment conseguiu
imobiliza"-la.

Quanto a Cardozo, não tem a menor vocação para tomar decisões. Perdeu o
controle da Polícia Federal muito antes da Lava Jato. Por ocasião do
episódio da Telexfree altos funcionários da \versal{PF} já reclamavam quase
abertamente do abandono a que foi relegado o órgão na sua gestão.

Aí ele escolheu a pior maneira de contemporizar. Em troca da \versal{PF} não
exigir nada dele, ele não exigiria nada da \versal{PF}, nem controle
administrativo. Não faz nada contra para não ter que fazer nada a favor.
A~liberdade dada não amainou as mágoas; e a autonomia conferida
potencializou os atos de represália. E, assim, a \versal{PF} tornou"-se uma
polícia política.

Jogou todo esse desastre na conta de um duvidoso republicanismo.
Aparentemente, Dilma foi na sua conversa.

\section{Janot, a grande interrogação}

E aí se entra na grande interrogação: o Procurador Geral da República
Rodrigo Janot.

Janot é fundamentalmente um procurador político. E~não se imagine essa
qualificação como depreciativa. Promoveu uma revolução no \versal{MPF} ao acabar
com a postura autárquica do \versal{PGR}, implementar a modernização nos
processos e procedimentos internos. Como se recorda, todos os inquéritos
que versavam sobre políticos com prerrogativas de foro eram analisados
exclusivamente por Roberto Gurgel, seu antecessor, e por sua esposa.

Instituiu uma série de decisões colegiadas e trouxe para sua assessoria
pessoal um grupo conceituado de procuradores de todas as partes do país.

Também endossou uma série de temas relevantes, como a revisão da Lei da
Anistia e outros temas que ajudaram na legitimação do \versal{MPF} como defensor
de bandeiras civilizatórias.

Peça 5~- Janot tem pleno domínio do \versal{MPF}, não apenas hierárquico como de
liderança.

Ao contrário de José Eduardo Cardozo e de Leandro Daiello,
delegado"-geral da \versal{PF}, Janot é uma liderança incontestável.

Peça 6~-- Janot é \versal{PGR} por voto da maioria dos procuradores.

O fato de Lula e Dilma terem tornado automática a indicação do \versal{PGR} mais
votado pela categoria acabou subordinando o \versal{MPF} ao chamado democratismo.
Em vez de responder ao Presidente da República, tornando"-se
corresponsável pelo equilíbrio político"-institucional do país -- como
ocorre nas democracias maduras --- o Procurador passa a responder
preponderantemente para sua própria categoria.

Peça 7~- Janot tem adotado medidas legais em defesa do mandato de Dilma

Nas arremetidas da oposição, assumiu posições fortes em defesa da
legalidade, seja mantendo no \versal{TSE} o procurador Eugênio Aragão -- capaz de
enfrentar as maiores baixarias de Gilmar Mendes sem mover um músculo da
face e sem ceder -- seja nos pareceres no \versal{STF}, não embarcando nas teses
golpistas.

Por outro lado, sua atuação em relação à Lava Jato é para lá de dúbia. E~aí mais três peças no nosso xadrez para completar o jogo:

Começando o jogo

Temos, agora, 7 peças para jogar nosso jogo de interpretar Janot.

A~Peça 1~indica que o vazamento reiterado de notícias obedece a uma
estratégia eminentemente política.

A~Peça 2~mostra que essa política de vazamentos é endossada pelos
procuradores que participam da Lava Jato.

A~Peça 5~sustenta que Janot tem pleno domínio sobre as práticas dos
procuradores da Lava Jato. Sendo assim, ele não pode interferir nas
investigações, mas poderia disciplinar os vazamentos, especialmente
quando ficou nítido seu caráter político.

Em outras palavras, se Janot quisesse, um mero gesto de sua parte
interromperia esses abusos. Como nada faz, é evidente que é cúmplice
dessa política.

Mas falta saber a razão.

A~Peça 6~-- que versa sobre o democratismo no \versal{MPF} -- poderia ser uma
explicação. ~Como a Lava Jato conferiu um prestígio inédito ao \versal{MPF},
Janot teria receio de se insurgir contra seus eleitores. A~corporação
dos procuradores é maciçamente anti"-\versal{PT} e anti"-Lula. É~só conferir as
manifestações nas redes sociais e as diversas representações de
procuradores em torno de factoides plantados pela mídia.

Entregando Lula às feras, Janot satisfaria a sede de sangue da oposição
-- e do seu eleitorado \mbox{---,} mas se preservaria para defender os direitos
constitucionais de Dilma, na presidência da República.

É uma hipótese, mas que fica prejudicada pelas lances seguintes.

De acordo com a~Peça 4, ~Aécio Neves é o principal beneficiário do jogo
do impeachment, agora ou em 2018.

Aí o quadro fica mais comlplicado para o lado de Janot.

Há pelo menos três medidas de Janot que blindaram Aécio:

1.~~~ Não ter transformado em denúncia ao \versal{STF} a delação de Alberto
Yousseff, de que Aécio era um dos beneficiários do esquema de Furnas.

Em lugar de Aécio, alvo de denúncias meticulosas, denunciou o
ex"-governador Antônio Anastasia, em cima de uma denúncia imprecisa. Quem
conhece a política mineira sabe que Anastasia é uma figura política
impoluta e insuspeita. Seu envolvimento pareceu muito mais uma maneira
de Janot dar satisfações à opinião pública por ter livrado Aécio, sem
submeter o \versal{PSDB} ao risco de se descobrir algo contra Anastasia.

Nao foi Sergio Moro e a Lava Jato que blindaram Aecio: foi Janot. A~aceitação da denúncia teria permitido à Lava Jato entrar mais cedo no
setor elétrico.

Depois disso, a \versal{PF} insistiu em continuar no pé de Anastasia e Janot
empenhou"-se -- como a nenhum outro suspeito -- em derrubar o processo.

2.~~~ Ter mantido na gaveta do \versal{PGR} denúncia do Ministério Público
Federal do Rio de Janeiro, sobre uma conta em Liechtenstein, de
titularidade de uma offshore das Bahamas, tendo como proprietários
familiares de Aécio,~{a famosa Operação
Norbert}, que resultou na condenação do ex"-corregedor do Tribunal de
Justiça do Rio de Janeiro, Carpena do Amorim.

Dois dos três procuradores autores da denúncia hoje em dia fazem parte
do estado maior de Janot. No jantar de posse da Dilma, troquei algumas
palavras com Janot e cobrei"-lhe o prosseguimento dessa ação. No início,
dizia não se lembrar. Depois, se lembrou e disse que daria um parecer
até abril -- de 2015, ou 9 meses atrás.

Em todo caso, o \versal{GGN} recorreu a Lei de Acesso à Informação para saber o
destino da denúncia.

3.~~~ Ter endossado a posição dos procuradores de restringir as delações
aos malfeitos do \versal{PT} e da base aliada.

Faltam peças no jogo para entender essa sua posição de blindar Aécio.

A maneira quase íntima com que se dirigiu a Aécio em sua ida ao Senado
-- ``como dizem lá no nosso estado, senador'' -- causou estranheza. Pode
ser um tique mineiro.

Outra possibilidade seria uma estratégia política, de não pretender
abrir duas frentes de desgaste, com o \versal{PT} e contra o \versal{PSDB}. Especialmente
para não atrapalhar as relações com o maior aliado do \versal{MPF}, a velha
mídia. E~sem a mídia, a Lava Jato morre na primeira instância.

Explica, mas não justifica, como se diz lá em Minas, conterrâneo.

Não se afasta a possibilidade do que se poderia denominar de ``a lei da
menor porrada''. Mostre o máximo de atrevimento possível contra quem não
impõe nenhum risco de retaliação, para se poupar de ousar contra quem
oferece risco.

Investir contra o governo de Dilma e Cardozo não exige nenhuma prova de
coragem. Caso mirasse sua espingarda em Aécio, levaria tiros do \versal{PSDB}, da
mídia e da própria presidente e de seu Ministro da Justiça, que não
perderiam a oportunidade de proclamar seu republicanismo.

E, por óbvio, não se pode afastar a hipótese de que esteja, de fato,
articulado com o grupo de Aécio.

Permanece a incógnita da~Peça 7,~que alimentaria a visão conspiratória
de que Janot poderia estar aliado a Dilma e Cardozo visando ajudar a
enterrar a herança Lula, para dar lugar à era Dilma, tendo como bandeira
a defesa intransigente da ética. Nessa versão, o crescimento da campanha
do impeachment teria sido fruto da perda momentânea de controle. Nâo
endosso a versão, mas tem a utilidade de trazer uma explicação para a
manutenção de Cardozo no Ministério.

Lá atrás, a maneira como o \versal{MPF} e a \versal{PF} invadiram o escritório da
presidência em São Paulo, teve como único objetivo escancarar as
relações pessoais de Lula com a secretária Rosemary.

A prova do pudim de Janot

A prova do pudim será a segunda delação envolvendo Aécio com Furnas --
agora, da parte de Fernando Baiano. E~não se trata de vendetta ou coisa
do gênero. Investigando Aécio se dará à Lava Jato sua verdadeira
dimensão republicana: a de investir contra os vícios do modelo político
como um todo, sem intocáveis, e não de se valer da luta contra a
corrupção escolhendo lado.

E aqui vai uma historinha mineira para Janot, o conterrâneo de Aécio.~

Em 2004, houve a inauguração do \versal{PCH} (Pequena Central Hidrelétrica) Padre
Carlos, em Poços de Caldas. Compareceram o presidente Lula, a Ministra
das Minas e Energia Dilma Rousseff e o governador de Minas Aécio Neves.

Aécio era uma alegria só. No palanque, até brincou de coçar a barriga do
Lula, segundo me contaram testemunhas. Achei um certo exagero, mas
pesquisando nos arquivos da Folha, conferi ~que o repórter mencionou os
``afagos'' de Aécio a Lula (\url{bit.ly/\allowbreak{}20U31Au}). Chamou Dilma
de ``conterrânea'' e saudou os inúmeros mineiros que participavam do
Ministério de Lula.

Por sua vez, Lula lembrou os passeios de charrete, quando foi a Poços
pela primeira vez em lua"-de"-mel. E~elogiou as \versal{PCH}s, lembrando que o país
tem mais de 1.500 pequenas hidrelétricas desativadas, que poderiam ser
reativadas.

Vendo o entusiasmo de Lula, o \versal{PT} da cidade tentou emplacar um diretor em
Furnas. Escolheu um conterrâneo, técnico, apolítico, dono de vasta
reputação no setor, e apresentou o nome a Lula, como sugestão para a
Diretoria de Operações.

A informação que receberam é que não daria. A~Diretoria de Operações já
estava prometida a Aécio Neves, e seria entregue a Dimas Toledo.

A delação dos executivos da Andrade Gutierrez é o caminho. Segundo a
Lava Jato, a delação visará identificar a corrupção no setor elétrico.~

Dependendo de como Dimas, Furnas, a troca de ações entre Cemig e Andrade
Gutierrez serão tratados, será possível colocar no nosso jogo a peça
final sobre o conterrâneo Rodrigo Janot. E~será possível, finalmente,
saber qual escaninho a história reservará para Janot: se a casa dos
conspiradores discretos, se dos que se assustaram com a besta ou se dos
que resistiram à barbárie.

\section{A herança da Lava Jato ao país}

A vida política nacional não termina este ano, nem com as eleições de
2018. Virão outras eleições e outras lideranças. E~as novas lideranças
já estão nascendo nos movimentos na rua, na ação dos secundaristas, nos
passes livres da rede. E~sob o signo do ódio que o Ministério Público
Federal, a Polícia Federal e os grupos de mídia estão plantando na
opinião pública, nessa busca desatinada de destruição de Lula.

A campanha não visa apenas apurar suspeitas contra Lula: trabalha
diuturna e sistematicamente para enterrar o mito Lula.

Não será surpresa se um dos indicadores de sucesso acompanhado por
procuradores e delegados não forem as pesquisas de opinião, analisando
em que nível se encontra a destruição da imagem de Lula.

E, no entanto, em que pese todos os pecados do \versal{PT} e de Lula, o lulismo
--- como ideologia --- foi abraçado por defensores de direitos humanos,
de políticas sociais universais, das políticas de cotas, os militantes
do \versal{SUS} e da educação e um amplo espectro de eleitores reunidos em torno
de princípios da socialdemocracia e dos direitos sociais, temas que
jamais frequentaram a pauta dos principais líderes da oposição. Em caso
de destruição de Lula, a herança de ódio se voltará contra o \versal{MPF} e
contra a biografia de Janot.

É só conferir quais os aliados que a Lava Jato procura para atiçar
novamente a bandeira do impeachment: é a besta, a multidão disposta a
voltar às ruas tangidas pelo ódio e o preconceito, os filhotes de
Bolsonaro, os playboys do Leblon, os grupos de mídia que se colocaram
contra as políticas sociais, a \versal{FIESP} de Paulo Skaf, a \versal{LIDE} de João
Dória. Esses são ~os aliados preferenciais da Lava Jato e de Aécio.
Janot tem a mais leve ilusão que manterá o espaço do \versal{MPF} em uma quadra
política dominado por essa coalizão ?

Hoje em dia, internacionalmente, o mito Lula é colocado no mesmo nível
de outros grandes pacificadores que ajudaram a construir a civilização
no século 20, como Ghandi, Mandela, Roosevelt.

Quando Obama chamou Lula de ``o cara'', foi por ter conseguido o que
ele, Obama, não conseguiu na política norte"-americana: incluir pessoas,
superando o profundo grau de intolerância criado nesses tempos de
globalização, redes sociais e grupos de mídia desvairados. Com Lula, os
pobres, os movimentos sociais, os sindicatos, entenderam que seria
possível crescer econômica e politicamente seguindo as regras do jogo
democrático e não apelando para a radicalização. Tornou"-se um símbolo
mundial da paz.

É essa noção de pax que está sendo varrida do mapa político brasileiro,
sob os olhares acomodatícios de pessoas como Janot. É~esse símbolo que
está sendo pisoteado diariamente por procuradores e delegados incapazes
de entender sequer a dimensão do personagem na história do século 20.

A história há de cobrar seu preço. E~cobrança não será do procurador
malicioso que fantasia"-se de roupa a caráter para receber seu prêmio das
Organizações Globo, e vocifera que existe um pacto das elites do outro
lado do balcão. É~um pequeno, cuja história se perderá nas dobras do
tempo.

A cobrança virá sobre aqueles personagens que, podendo deter a barbárie,
fugiram de seus compromissos.

No momento, Janot é a esperança do Brasil, mas não no sentido dado pelos
manifestantes que foram aplaudi"-lo em sua casa. Mas agindo de acordo com
os valores que norteiam o que se pensava ser o pensamento majoritário do
\versal{MPF}, contra a barbárie.
