\chapterspecial{19/\allowbreak{}10/\allowbreak{}2016 Celso Amorim chocou o ovo da serpente da intervenção militar}{}{}
 

Sobre a ampliação da participação das Forças Armadas nas operações
internas do país, recebo a seguinte mensagem de Erico Perrela, um dos
estudantes presos no Centro Cultural de São Paulo.

\section{De Erico Perrela}

\begin{quote}
Nassif, fui um dos estudantes presos no Centro Cultural de São Paulo em
operação envolvendo o capitão do exército Willian Lina Botelho (o
famigerado Balta).

Tenho acompanhado sua série de escritos a respeito da militarização do
Brasil e não só tenho concordado, mas eu e meus amigos vemos sendo
vítimas dessa militarização crescente.

Aparentemente, a julgar pelas frases ditas pelo comandante chefe do
estado maior do exercito general Villas Boas em entrevista ontem à radio
Jovem Pan (\url{migre.me/\allowbreak{}vhx2C}~) e também pela coletiva de
imprensa dada pelo comando da \versal{PM} de São Paulo no próprio dia de nossa
prisão, nós fomos presos em meio ao que é chamado no meio militar de
Operação de Garantia da Lei e da Ordem.

 Não sei se você teve oportunidade de observar essa ``norma'' militar,
publicada pelo ministério da defesa (\url{migre.me/\allowbreak{}vhx48}~) para
ações integradas entre as forças públicas de segurança, guiadas
pessoalmente pelo comandante do estado maior do exercito.

 Vou destacar alguns pontos do documento:

\emph{~2.1.1 As Operações de Garantia da Lei e da Ordem (Op \versal{GLO})
caracterizam"-se como operações de ``não guerra'', pois, embora
empregando o Poder Militar, no âmbito interno, não envolve o combate
propriamente dito, mas podem, em circunstâncias especiais, envolver o
uso de força de forma limitada, podendo ocorrer tanto em ambiente urbano
quanto rural.}

\emph{~2.1.4 Os planejamentos, para a execução de Op \versal{GLO}, deverão ser
elaborados no contexto da Segurança Integrada, podendo ser prevista a
participação de órgãos: a) do Poder Judiciário; b) do Ministério
Público; e c) de segurança pública.}

\emph{2.1.5 Outros órgãos e agências, dos níveis Federal, Estadual e
Municipal, poderão se fazer presentes em alguns casos. Desta forma, é
fundamental o conhecimento dos princípios das Operações Interagências
constantes de publicação específica.}

\textbf{Nota 1 -- Essa Portaria Normativa foi elaborada em 2013, na
gestão Celso Amorim no Ministério da Defesa, visando regulamentar pontos
previstos da Constituição~}

\textbf{Nota 2 -- A participação de demais órgãos, como Judiciário e
Ministério Público, não é obrigatória.}

\emph{4.3 Forças Oponentes}

\emph{4.3.1 Em Op \versal{GLO} não existe a caracterização de ``inimigo'' na
forma clássica das operações militares, porém torna"-se importante o
conhecimento e a correta \versal{MD}33-M-10 29/\allowbreak{}68 caracterização das forças que
deverão ser objeto de atenção e acompanhamento e, possivelmente,
enfrentamento durante a condução das operações.}

\emph{~4.3.2 Dentro desse espectro, pode"-se encontrar, dentre outros, os
seguintes agentes como F Opn:}

\emph{a) movimentos ou organizações;}

\emph{b) organizações criminosas, quadrilhas de traficantes de drogas,
contrabandistas de armas e munições, grupos armados etc;}

\emph{c) pessoas, grupos de pessoas ou organizações atuando na forma de
segmentos autônomos ou infiltrados em movimentos, entidades,
instituições, organizações ou em \versal{OSP}, provocando ou instigando ações
radicais e violentas; e}

\emph{d) indivíduos ou grupo que se utilizam de métodos violentos para a
imposição da vontade própria em função da ausência das forças de
segurança pública policial.}

 \textbf{Nota -- cabe tudo nessas classificações, desde passeatas de
\versal{MTST} até ações do \versal{PCC}.}

\emph{4.5.3. Principais ações Entre outras, podem"-se relacionar as
seguintes ações a serem executadas durante uma Op \versal{GLO}:}

\emph{a) assegurar o funcionamento dos serviços essenciais sob a
responsabilidade do órgão paralisado;}

\emph{b) combater a criminalidade;}

\emph{c) controlar vias de circulação urbanas e rurais;}

\emph{d) controlar distúrbios;}

\emph{e) controlar o movimento da população;}

\emph{f) desbloquear vias de circulação;}

\emph{g) desocupar ou proteger as instalações de infraestrutura crítica,
garantindo o seu funcionamento;}

\emph{h) evacuar áreas ou instalações;}

\emph{i) garantir a segurança de autoridades e de comboios;}

\emph{j) garantir o direito de ir e vir da população;}

\emph{k) impedir a ocupação de instalações de serviços essenciais;}

\emph{l) impedir o bloqueio de vias vitais para a circulação de pessoas
e cargas;}

\emph{m) interditar áreas ou instalações em risco de ocupação;}

\emph{n) manter ou restabelecer a ordem pública em situações de
vandalismo, desordem ou tumultos;}

\emph{o) permitir a realização do pleito eleitoral dentro da ordem
constitucional;}

\emph{p) prestar apoio logístico aos \versal{OSP} ou outras agências;}

\emph{q) proteger os locais de votação;}

\emph{r) prover a segurança das instalações, material e pessoal
envolvido ou participante de grandes eventos;}

\emph{s) realizar a busca e apreensão de materiais ilícitos;}

\emph{t) realizar policiamento ostensivo, estabelecendo patrulhamento a
pé e motorizado;}

\emph{u) restabelecer a lei e a ordem em áreas rurais; e}

\emph{v) vasculhar áreas.}

\textbf{~Nota -- Até operações de ocupação de escolas por secundaristas
se enquadram ai.}

\emph{5.3 Emprego 5.3.1 O emprego das \versal{FA} nas Op \versal{GLO} é de
responsabilidade do Presidente da República, que determinará ao Ministro
de Estado da Defesa a ativação de órgãos operacionais.}

\emph{5.3.2 Caberá aos Comandantes da Marinha, do Exército e da
Aeronáutica: a) fornecer os meios adjudicados pelo Ministro de Estado da
Defesa aos Comandos Operacionais Conjuntos, quando ativados; b)
assegurar o suporte logístico necessário aos Comandos Operacionais; e c)
emitir diretrizes, visando ao planejamento operacional para emprego,
quando da ativação de um Comando Operacional Singular a eles
subordinado.}

\textbf{Nota -- não se tem notícia de que a operação em São Paulo tenha
passado pelo presidente da República.~}

\emph{4.2.6 Emprego de Operações Psicológicas.}

\emph{4.2.6.1 O apoio das Operações Psicológicas (Op Psc) exige
planejamento prévio, minucioso e centralizado no mais alto escalão e
será básico para a conquista e manutenção do apoio da população, de
sorte a desenvolver uma atitude contrária às F Opn e outra favorável em
relação às forças envolvidas nas Op \versal{GLO}.}

\emph{4.2.6.2 Em Op \versal{GLO}, as Op Psc revestir"-se-ão de suma importância e,
sempre que possível, antecederão o emprego da tropa por meio de campanha
psicológica a ser desenvolvida sobre o público"-alvo considerado. Elas
permanecerão ativas durante a operação e após seu término, perdurarão
pelo tempo que for necessário podendo, inclusive, extrapolar a área de
operações.}

\emph{4.2.6.3 As Op Psc deverão ser desenvolvidas em consonância com as
atividades de Com Soc e de Inteligência, buscando"-se obter a desejada
sinergia na execução dessas atividades.}

\emph{4.2.6.4 Os principais objetivos das Op Psc são:}

\emph{a) obter a cooperação da população diretamente envolvida na área
de operações, desenvolvendo uma atitude contrária às F Opn e outra
favorável às forças empregadas;}

\emph{b) estimular as lideranças comunitárias favoráveis às operações;}

\emph{c) enfraquecer o ânimo e o moral das F Opn compelindo"-os à
desistência voluntária; e}

\emph{d) fortalecer o sentimento de necessidade do cumprimento do dever
na força empregada, aumentar o seu potencial de engajamento e torná"-la
imune às atividades de cunho psicológico das F Opn.}

\textbf{Nota -- infiltração nos Black Blocs para estimular atos
violentos enquadra"-se perfeitamente nesse capítulo.}

Qualquer um que ler com atenção esses itens, consegue imaginar que eles
podem ser manipulados para tentar conter \versal{QUALQUER} movimento contrário
aos que tem o poder de aplicação de uma operação como essas.
\end{quote}
