\chapterspecial{16/\allowbreak{}09/\allowbreak{}2016 Xadrez dos desdobramentos do Power Point}{}{}
 

\section{Peça 1 --- a luta política global}

Anos atrás, o ex"-presidente espanhol Felipe Gonzáles alertou Lula,
conforme testemunhou o governador do Piauí Wellington Dias:

-- Lula, prepare"-se que eles vão querer te processar, cassar ou prender.
Se não conseguirem, vão tentar te matar.

O alerta, com pitadas trágicas, não épara ser ignorado. O~próprio
Gonzáles fora alvo de uma caçada implacável, parceria da mídia com o
Ministério Público espanhol. Não apenas ele. Trata"-se de uma luta
política global que tem vitimado, uma a uma, as principais lideranças da
socialdemocracia mundial. No caso brasileiro, de forma mais explícita
devido ao baixíssimo nível dos principais protagonistas políticos,
jurídicos e midiáticos envolvidos.

Por uma questão de realismo, para tentar traçar qualquer cenário futuro
é importante que sejam consideradas as seguintes premissas:

\begin{enumerate}
\itemsep1pt\parskip0pt\parsep0pt
\item
  1. Não se está definitivamente em um estado democrático de direito.
  Portanto, manifestações de isenção serão a exceção, não a regra.
\item
  2.~ A pantomima montada pela Lava Jato de Curitiba é a comprovação
  cabal de que a Procuradoria Geral da República e a Lava Jato são
  personagens de um enredo maior, cujo objetivo final é liquidar com
  Lula e o \versal{PT}.
\item
  3. A~anulação de Lula exige um aumento dos abusos; e esse aumento dos
  abusos poderá despertar a consciência jurídica de setores até agora à
  margem dessa disputa.
\item
  4. Por tudo isso, o sub"-show de ontem, em Curitiba, não encerra a
  temporada de caça a Lula e significa uma nova etapa na disputa
  política. Do mesmo molde que o vazamento das conversas de Dilma e~
  Lula -- que expôs Rodrigo Janot \mbox{---,} e sua condução coercitiva -- que
  expôs a Lava Jato. Agora, obrigará o Judiciário a tomar uma atitude,
  coibindo o arbítrio, ou rasgar a fantasia e assumir"-se definitivamente
  como poder discricionário.
\end{enumerate}

\section{Peça 2 --- a acusação}

A peça de acusação leva aos limites da teoria do domínio do fato,
comprovando definitivamente a orquestração política, da qual o \versal{MPF} é
peça central.

O que faz ela?

\begin{enumerate}
\itemsep1pt\parskip0pt\parsep0pt
\item
  1. Descreve todas as injunções do presidencialismo de coalizão,
  mostrando que as barganhas são fundamentais para a governabilidade.
\item
  2. Depois, lista várias barganhas no governo Lula --- entre as quais
  as diretorias da Petrobras --- admitindo que eram essenciais para a
  governabilidade. Mas… no caso de Lula a barganha serviu para
  financiar os partidos, para enriquecimento pessoal e para perpetuar o
  \versal{PT} no poder. Pouco importa os diversos depoimentos sustentando que
  barganhas com a Petrobras existem há décadas. E~o maior beneficiário
  pessoal da corrupção foi Lula.
\end{enumerate}

A prova do pudim é provar que Lula enriqueceu com dinheiro ilícito. Bate
no tríplex:

\textbf{Diz a acusação}: Lula visitou o tríplex com o presidente da \versal{OAS}
e com dona Marisa. Logo é prova de que é dono do tríplex.

\textbf{Diz a defesa}: dona Marisa tinha cotas do edifício em questão. O~tríplex foi oferecido a ela pela \versal{OAS}. Lula visitou"-o com Léo Pinheiro e
dona Marisa. Não gostou e não ficou com o imóvel.

\textbf{Prova do pudim}: qualquer documento que comprove que Lula algum
dia teve a propriedade do imóvel. A~acusação não apresentou nenhum,
porque ``não temos provas, mas temos a convicção''. A~defesa apresentou
as provas de que o apartamento tem outro proprietário.

A segunda ``acusação'' foi a de que a \versal{OAS} bancou a guarda dos bens que
Lula acumulou, enquanto presidente.

\textbf{Prova do pudim}: provar que os bens têm valor monetário para
Lula.

\textbf{Realidade}: são bens da Presidência de República sob guarda do
ex"-Presidente. Portanto, inegociáveis.

A comprovação final seria a investigação das contas do escritório Mossak
Fonseca, especializado em lavagem de dinheiro. O~tríplex em questão era
de alguém com conta em paraiso fiscal montada pelo escritório. Mas,
assim que se depararam com uma conta offshore em nome da família
Marinho, a Lava Jato interrompeu as investigações sobre a Mossak
Fonseca. Nada mais se disse, nada então vazou.

Montaram um edifício retórico em cima de uma estrutura de bambu com
requintes de crueldade, ao indiciar dona Marisa. E~agora?

\section{Peça 3 --- os desdobramentos políticos}

O objetivo do carnaval foi influenciar as próximas eleições municipais e
preparar a cama para a denúncia de organização criminosa que o \versal{PGR} está
prestes a apresentar ao Supremo. ~Mas a ~insuficiência da acusação cria
um enorme problema para o juiz Sérgio Moro e o \versal{TRF}4.

O envolvimento do juiz com a acusação --- fato que afronta qualquer
norma de direito --- foi saudado como sinal de profissionalismo, do juiz
que sabe o que está acontecendo e impõe mudanças no rumo das
investigações, quando considera que não estão bem embasadas.

Moro derrotaria Moro, não aceitando a denúncia? Evidente que não.

A bomba, então, será transferida para o \versal{TRF}4. O~endosso à acusação
significará um passo largo em direção ao arbítrio e um tiro no coração
do argumento de que Moro jamais foi questionado pelos tribunais
superiores por ser dotado de uma técnica jurídica superior. ~Jamais foi
questionado ou por afinidade política ou por receio do rugir da besta
das ruas.

A prova dos 9 será a tramitação dessa denúncia que traz desdobramentos
complexos para o nosso Xadrez.

Ela foi atacada pelo \versal{PT} por razões óbvias; e por blogueiros
estreitamente ligados ao Gilmar Mendes e José Serra, por razões sutis. O~grupo de Gilmar se valeu da acusação para enfraquecer a Lava Jato,
prevenindo eventuais futuras ações contra Serra e Aécio Neves. De um
lado, festejam mais uma ofensiva midiática contra Lula. De outro,
celebram a fraqueza penal da acusação.

Conseguindo emplacar a tese da mediocridade da peça acusatória ---
tarefa facilmente demonstrável --- se fortalecerá a reação, quando, em
um ponto qualquer do futuro, a Lava Jato se dignar a olhar para o \versal{PSDB}.
Saliente"-se que a peça é vergonhosa, mesmo.

Mas, por outro lado, poderão prejudicar a estratégia macro, de
inabilitação de Lula para 2018. Preso por ter cão; preso por não ter
cão.

Como pano de fundo, tem"-se movimentos tectônicos na política, com o
quadro partidário começando a ser redesenhado após o terremoto.

\section{Peça 4 --- o fator \versal{PSDB} e a frente do golpe}

A lógica do xadrez é insuficiente para abarcar as múltiplas
possibilidades que se abrem, pelo fato de haver vários atores aliando"-se
taticamente em um momento, entrando em conflito no momento seguinte, sem
nenhuma coerência ideológica, nem histórico de lealdade pessoal. É~quebra"-pau de saloon de faroeste. O~mais bonzinho traiu o melhor amigo
no dia seguinte ao da sua nomeação.

A primeira grande confusão são as expectativas de cada ator que se uniu
para deflagrar o golpe:

\begin{itemize}
\itemsep1pt\parskip0pt\parsep0pt
\item
  ·~~~~~~ A camarilha dos 6, de Temer: apostando em ir além de 2018.
  Para tanto atuará em duas frentes: evitará medidas que possam ampliar
  a impopularidade; jogarão para adiar as eleições de 2018, inclusive
  apostando no endurecimento do regime. E, consequentemente, serão
  gradativamente deserdadas pelo mercado.
\item
  ·~~~~~~ Os \versal{PSDB}s:~de Aécio e Alckmin em conflito cada vez maior com a
  camarilha dos 6 e entre si. Seu principal agente, o Ministro Gilmar
  Mendes, tem poder de fogo no \versal{TSE} (Tribunal Superior Eleitoral),
  podendo se dar por ali o desfecho dos conflitos. José Serra está fora
  do jogo maior, louco para ser abrigado pelo \versal{PMDB} de Temer, e também
  tem Gilmar como aliado.
\item
  ·~~~~~~ Temer equilibra"-se entre os dois grupos. É~político menor que
  se move por sobrevivência política de curto prazo e precisa ser
  guiado. Antes, o cão"-guia era Eduardo Cunha. Agora é Eliseu Padilha e
  Romero Jucá. Se precisar se apoiar no \versal{PSDB}, aderirá.
\item
  ·~~~~~~ O presidente da Câmara, Rodrigo Maia, começando a se
  entusiasmar com a possibilidade de ser um substituto de Michel Temer,
  caso o ``Fora Temer'' e os conflitos com o \versal{PSDB} forcem o \versal{TSE} a
  impichá"-lo também.
\item
  ·~~~~~~ O \versal{PGR} Rodrigo Janot, jogando preferencialmente com o \versal{PSDB} de
  Aécio Neves, mas tendo como objetivo maior a destruição de Lula e do
  \versal{PT}. Adiará o máximo possível qualquer denúncia contra Aécio. É
  ambicioso e abraça apaixonadamente qualquer causa que lhe garanta
  poder, mesmo que seja totalmente contrária à paixão anterior.
\item
  ·~~~~~~ O \versal{STF} (Supremo Tribunal Federal), preso em suas contradições e
  temores.
\end{itemize}

Todas as alternativas abaixo são possíveis, dependendo das
circunstâncias do momento:

\begin{enumerate}
\itemsep1pt\parskip0pt\parsep0pt
\item
  1.~~~~ Uma aliança entre Gilmar"-Aécio"-Rodrigo Maia"-Janot visando
  impugnar Temer a abrir espaço para um governo do \versal{PSDB} tendo Maia como
  presidente, Gilmar atuando junto ao \versal{TSE} e Janot junto ao \versal{STF}, tirando
  da maleta mágica denúncias contra a camarilha.
\item
  2.~~~~ Uma aliança entre Gilmar"-Temer"-\versal{PSDB}"-Janot, com recrudescimento
  político, com o aumento da ofensiva do Ministro da Justiça e do
  Gabinete de Segurança Institucional com as \versal{PM}s estaduais em torno da
  figura do inimigo interno. O~\versal{PGR} ampliaria o exercício do direito
  penal do inimigo, tentando conferir algum formalismo legal ao jogo e
  ajudando a dizimar os políticos recalcitrantes.
\item
  3.~~~~ Uma aliança Temer"-Serra"-Renan, esvaziando a camarilha sem abrir
  espaço para o \versal{PSDB}.
\end{enumerate}

\section{Peça 5 --- o fator Lula e \versal{PT}~~}

É nesse quadro confuso, de um grupo de poder heterogêneo, sem lealdades
e sem projeto de poder -- a não ser o de leiloar o país -- que se
entende mais facilmente a ofensiva contra Lula. Depois do vexame de
ontem, a Lava Jato virá com outro inquérito secreto, usando o modelo
Gilmar Mendes no \versal{TSE}, pretendendo investigar \versal{UMA} empresa que contratou
\versal{UMA} palestra de Lula. Repito: \versal{UMA}. A~síndrome do Fiat Elba -- que serviu
para condenar Fernando Collor --- não os abandona.

Nos próximos meses haverá mudança drástica no panorama dos partidos
políticos, especialmente os de esquerda, com o \versal{PT} dizimado pelo
``mensalão'' e o ``petrolão'', e com uma direção incapaz de entender os
novos tempos e sem a iniciativa de abrir o partido para a renovação,
sequer para o belo think tank representado pelo Instituto Perseu Abramo.

Há dois caminhos possíveis:

\begin{enumerate}
\itemsep1pt\parskip0pt\parsep0pt
\item
  1. A~manutenção do \versal{PT} atual, com algum arejamento na direção,
  ambicionando manter o protagonismo de uma frente de esquerda.
\item
  2. A~criação de um novo partido, juntando o \versal{PT} e partidos menores e
  políticos progressistas ainda aninhados no \versal{PMDB}, \versal{PSB} e outros.
\end{enumerate}

O segundo caminho é defendido por lideranças expressivas do \versal{PT}, como o
governador do Piauí Wellington Dias. Seria a maneira de arejar o partido
e permitir a montagem de uma grande frente.

Se o \versal{PT} insistir em se colocar à margem, mantendo a gerontocracia que o
governa, e pretender liderar essa frente de esquerdas, será engolido
rapidamente por algum novo partido que surgir com esse propósito.

Por isso mesmo, prepare"-se para, dentro de algum tempo, conviver
possivelmente com uma nova sigla de esquerda.

Em qualquer quadro, a presença política de Lula é componente central: as
esquerdas se recompõem sem o \versal{PT}; mas demorarão muito mais a se recompor
sem Lula.

Quem ouviu o discurso de Lula, ontem, saiu com a certeza de que, se o
deixarem solto, em pouco tempo arregimentará seguidores para a frente
das esquerdas. Por isso mesmo, seria medida de prudência ficar atento
aos alertas de Felipe Gonzáles e reforçar a segurança de Lula.

A política ingressa definitivamente em um novo ciclo e Lula é a única
liderança nacional sobrevivente desses tempos de terremotos e redes
sociais.
