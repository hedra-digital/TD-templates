\chapterspecial{25/\allowbreak{}06/\allowbreak{}2016 A torcida por um 2016 melhor}{}{}
 

2016 começa sob o signo da esperança -- como todo ano, aliás. Há alguns
fatos novos no ar, depois dos problemas enormes que o país enfrentou em
2015.

O primeiro é a vontade geral de que os problemas políticos sejam
superados e a economia volte a se recuperar. Em cima dessa expectativa,
há uma reavaliação ampla da atuação de vários personagens públicos.

A estabilização da economia tornou"-se matéria de interesse nacional. Não
adianta se apelar para esse jogo malicioso de dividir a estabilização
entre governistas e oposicionistas. Em determinado momento, ganhou corpo
a ideia de que a saída de Dilma Rousseff atendia mais ao interesse
nacional.

Agora, a situação é outra. A~bandeira do impeachment murchou e qualquer
tentativa de prorrogar essa novela passa a ir contra o interesse
nacional e a vestir a carapuça do golpismo.

Gilmar Mendes, Dias Toffoli, Aécio Neves entram definitivamente para o
duvidoso panteão dos personagens políticos nefastos, ao lado de Eduardo
Cunha, daqueles que colocam interesses pessoais ou políticos,
idiossincrasias e oportunismos acima do interesse nacional.

Perderam a capacidade de derrubar governos, mas mantem o poder de
continuar atazanando o país.

\asterisc{}

O grande desafio será, agora, na esfera político"-econômica. E~está nas
mãos do Ministro da Fazenda Nelson Barbosa.

Nelson tem mais realismo do que as excentricidades desenvolvimentistas
de Guido Mantega ou a mentalidade de contador de Joaquim Levy. Sabe que
o principal desafio econômico será interromper a queda da atividade
econômica. Por outro lado, tem claro os limites fiscais.

Para reativar a economia, precisará de boa dose de imaginação para
articular instrumentos legais que não impliquem em mais custos fiscais.
Por outro lado, tem a necessidade de impor segurança ao mercado, sim. E~segurança não consiste em adotar medidas heroicas pró"-cíclicas. Medidas
heroicas são para enganar o freguês e permitir a economistas de jornal
jogar para a plateia. Segurança consiste em apresentar um plano
factível, lógico, que acene com o equilíbrio fiscal no médio prazo. E~equilíbrio fiscal significa recuperar as receitas fiscais através da
melhoria da atividade econômica.

\asterisc{}

O grande desafio de Nelson será equilibrar"-se ante as demandas dos
movimentos sociais e sindicatos e as do mercado. Em geral, Ministro que
entra tem a fase de carência, de pelo menos seis meses para mostrar a
que veio.

Por conta da crise a agenda ficou mais estreita. Mas seria importante
que as forças mais à esquerda entendessem as limitações da política
econômica e desse um sinal verde para o Ministro.

A recuperação da economia não pode depender dos esforços únicos de um
Ministro, mas de uma ação de governo, agitando todos os Ministérios em
torno de metas claras e factíveis de crescimento.

É papel que cabe à Presidente da República.
