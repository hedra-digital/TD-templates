\chapterspecial{29/\allowbreak{}08/\allowbreak{}2016 Xadrez do balé da dança dos lobos de Janot e Gilmar para
livrar Serra}{}{}
 

\section{Peça 1 --- sobre a mentira e os vazamentos horizontais}

Conforme apontamos em artigos anteriores, o grande problema dos
vazamentos horizontais --- como no caso da Lava Jato --- é que a única
maneira de uniformizar o discurso é através da verdade, que comporta
menos versões que a mentira.

Quando se foge do script, há um curto"-circuito total no discurso.

Foi o que ocorreu nos dois últimos factoides: o ``factoide
informacional'' (apud Rodrigo Janot) da capa da Veja, com a não"-denúncia
sobre Dias Toffoli; e o ``factoide processual'', de suspender a delação
por causa do vazamento.

Versão 1~-- o \versal{PGR} acusou advogados da \versal{OAS} de terem vazado parte do
pré"-documento de delação de Léo Pinheiro, com o intuito de pressionar
para que a delação fosse aceita. Atribuiu o suposto vazamento aos
advogados da \versal{OAS} devido ao fato de que a informação vazada supostamente
não constaria dos documentos encaminhados ao \versal{MPF}, logo o suposto
vazamento não poderia ter partido de lá. A~versão não se sustenta
porque, além de ser ilógica -- é evidente que o vazamento comprometeria
a delação --- fica"-se sabendo que o caso Toffoli foi mencionado na
conversa entre Pinheiro e os procuradores. Logo, o vazamento poderia ter
partido de qualquer um.

Versão 2~-- imediatamente Janot suspendeu as negociações para a
aceitação da delação do presidente da \versal{OAS}, Léo Pinheiro. Para justificar
a não tomada de decisão ante as 17 delações anteriores vazadas, alegou
que a de Toffoli era diferente, porque a informação não existia. Ou
seja, tratou drasticamente um vazamento irrelevante (porque, segundo
ele, de fatos que não existiam) e com condescendência vazamentos graves.

Na atual edição de~Veja, tenta"-se emplacar uma nova versão: a de que o
anexo (com o suposto vazamento) existia, mas não constava da pré"-delação
formalizada. Como fica, então, o argumento invocado para livrar os
procuradores da suspeita de vazamento?

 

Fica um conjunto de indagações no ar:

\begin{enumerate}
\itemsep1pt\parskip0pt\parsep0pt
\item
  O que explicaria a reação instantânea, de bate"-pronto de Janot e de
  Gilmar a um factoide jornalístico, uma não"-denúncia sem nenhuma
  relevância?
\item
  Por que se tentaria anular a delação de Léo Pinheiro.
\end{enumerate}

A hipótese mais aceita é que Janot se surpreendeu com o vazamento e saiu
atabalhoadamente em defesa dos seus e suspendendo a delação, visando
arrefecer qualquer arremetida mais dura do \versal{STF}.

Será? Há outras suposições bastante verossímeis.

\section{Peça 2 --- o jogo de cena entre Janot e Gilmar Mendes}

Advogado que assistiu a sessão do \versal{CNMP} (Conselho Nacional do Ministério
Público) que analisou a pinimba com Gilmar Mendes surpreendeu"-se com o
depoimento absolutamente tranquilo de Rodrigo Janot, que, embora
rebatesse as críticas contra a Lava Jato e o \versal{MPF}, em nenhum momento
aparentou indignação com o episódio. Pode ser sangue frio. Mas pode ser
outra coisa.

Janot sempre atuou como radar dos humores do \versal{STF} em relação à Lava Jato.
É~ele que orienta sobre quando avançar, quando recuar, permitindo aos
procuradores atuar sempre no limite da irresponsabilidade.

E ele endossa o viés totalmente partidário das investigações. Esse
acerto ficou nítido no vazamento dos grampos ilegais de Dilma e Lula,
decisivo para a votação da admissibilidade do impeachment. Janot estava
na Europa. O~\versal{STF} reagiu, Sérgio Moro e a Lava Jato se deram conta de que
haviam cruzado a faixa de segurança. Naquele breve momento de
estupefação, espremidos por repórteres investigadores de Curitiba
defenderam"-se alegando que, antes do vazamento, haviam consultado Janot
e recebido sua autorização. Alguns amigos ensaiaram uma defesa tênue,
sustentando que Janot não tinha conhecimento sobre as últimas gravações
--- gravadas após o prazo de encerramento.

Na Lava Jato, o chamado ``macho alfa'' (do dicionário: em biologia, o
líder que conduz a manada) é o procurador Carlos Fernando dos Santos
Lima. ~Seu ego o tornou imprudente. O~questionamento de uma decisão de
Toffoli, em artigo na Folha, mostrou uma autonomia de voo temerária e o
expôs a uma estratégia sutilíssima.

Certamente houve choques internos como resultado do seu atrevimento. E~justo no momento em que a Lava Jato ingressa na fase mais delicada:
quando deixa de lado o ``inimigo comum'' e se depara com delações que
atingem os atuais políticos alçados ao poder graças à atuação destemida
da \versal{PGR}.

Como enquadrar a tropa?

O ``factoide informacional'' caiu como uma luva. Cria"-se o barulho,
Gilmar tenta incendiar o Supremo contra a Lava Jato e, no meio da bruma,
surge a figura resoluta de Janot defendendo os seus e apontando os
culpados -- a \versal{OAS}. Como castigo, suspende a delação.

Depois, simulam"-se guerras verbais que permitem a Gilmar se consolidar
junto aos neo"-garantistas e a Janot junto à sua tropa. E, no final dos
embates, ambos celebram a paz, alcançando o mesmo objetivo: a suspensão
de uma delação que colocaria aliados no caldeirão da Lava Jato.

Faz sentido?

A hipótese alternativa é de alguém ligado ao \versal{PSDB} ter vazado a notícia e
Janot aproveitado o embalo para atingir seu objetivo.

Pelo resultado final do jogo, tendo a apostar na hipótese da combinação
do resultado. Graças a essa dupla manobra, Janot enquadra a tropa e
coloca a faca no peito de Léo Pinheiro: ou muda a estratégia (modo de
dizer: muda a narrativa) ou não tem delação premiada. E~os soldados na
ponta acatam as ordens do chefe que os está defendendo do único poder ao
qual se curvam: o Supremo.

\section{Peça 3 -- a relevância do depoimento da \versal{OAS}}

Como vocês se recordam, o \versal{GGN} antecipou a tentativa final de encerrar a
Lava Jato: a votação de uma lei pelo Congresso separando o financiamento
de campanha do enriquecimento pessoal. Haveria anistia a quem apenas
recorreu ao financiamento de campanha. No segundo caso, seria prática de
corrupção, expondo os autores à cadeira por formação de quadrilha.

No dia 11 de agosto passado, a sempre atilada Mônica Bérgamo deu pistas
importantes para entender os últimos episódios
(\url{migre.me/\allowbreak{}u\versal{MC}bH})''

\emph{``A revelação feita pela Odebrecht sobre dinheiro de caixa dois
para o \versal{PMDB}, a pedido de Michel Temer, e
para~{o
tucano José Serra}~(\versal{PSDB}"-\versal{SP}) tem impacto noticioso, mas foi recebida com
alívio por aliados de ambos. Como estão, os relatos poupam os
personagens de serem enquadrados em acusações mais graves, como
corrupção e formação de quadrilha.}

\emph{\versal{EM} \versal{CASA}}

\emph{Contribuição não contabilizada pode ser enquadrada como crime
eleitoral, de punição branda e chance mínima de resultar em prisão.}

\emph{\versal{CONTA} \versal{MAIS}}

\emph{Há, porém, uma pedra no caminho: a força"-tarefa da Operação Lava
Jato, que não aceita a versão de contribuições desinteressadas para
campanhas eleitorais via caixa dois. Os procuradores insistem na
revelação de contrapartidas, o que enquadraria a doação dos recursos em
propina pura e simples.}

\emph{\versal{PROCESSO} \versal{DINÂMICO}}

\emph{Por isso, a delação que envolve Temer e Serra pode ainda sofrer
alterações.}

\subsection{Caso Serra e Aécio}

É aqui o busílis da questão. Como o \versal{GGN} já informou, a delação da
Odebrecht se fixaria no financiamento de campanha e caixa 2. Se a Lava
Jato quiser correr atrás, que avance sobre Paulo Preto.

No caso da \versal{OAS}, sabia"-se desde o início que apresentaria dados mais
pesados. No material divulgado pela Veja fala"-se em pagamento a Serra em
dinheiro vivo.

Neste final de semana,~Veja~traz o conteúdo total da pré"-delação de Léo
Pinheiro.

Há informações seguras de pelo menos um depósito na conta de Verônica
Serra. Esse depósito não aparece na pré"-delação da \versal{OAS} divulgada
pela~Veja. Talvez apareça mais à frente, quando se avançar sobre os
sistemas de offshore.

\subsection{Caso Lula}

Que a \versal{OAS} pretendeu fazer mimo para Lula, não se discute. Ocorre que não
existem as duas pré"-condições para caracterizar crime:

\begin{itemize}
\itemsep1pt\parskip0pt\parsep0pt
\item
  A contrapartida de contratos escusos para os mimos recebidos. A~Lava
  Jato pretende que a cada graça corresponda uma operação de
  contrapartida. Ou seja, reformou o tríplex em troca de um contrato xis
  com a Petrobras.
\item
  Não existe o enriquecimento ilícito -- isto é, o crescimento do
  patrimônio da família Lula com esses mimos. O~usufruto do sítio, por
  si, não caracteriza enriquecimento pessoal. Insiste"-se nas bobagens de
  procurar corrupção em palestras remuneradas ou na influência notória
  de Lula sobre governantes estrangeiros. Trata"-se de prática usual para
  qualquer ex"-dirigente nacional, ao sul e ao norte da linha do Equador.
  E o preço estava de acordo com o status de Lula, de mais popular
  dirigente político do mundo naquele momento.
\end{itemize}

A única coisa que dá para acusar Lula é de estupidez política e
deslumbramento pessoal.

A semana terminou com Gilmar cobrindo a Lava Jato de elogios, sendo
acompanhado por Luís Roberto Barroso e esquecendo"-se o comentário do
Ministro Marco Aurélio, que pediu investigações sobre o supostos
vazamento da não"-notícia.

No fim de semana, o Valor Econômico traçou um perfil da atuação política
de Janot. Não chegou a avançar o tanto que poderia, tendo consultado
fontes bastante próximas e críticas a ele. Mas seu perfil encaixou"-se
como uma luva à definição do mineiro, segundo Guimarães Rosa:

\emph{''O mineiro espia, escuta, indaga, protela, se sopita, tolera,
remancheia, perrengueia, sorri, escapole, se retarda, faz véspera,
tempera, cala a boca, matura, engambela, pauteia, se prepara''.}
