\chapterspecial{28/\allowbreak{}10/\allowbreak{}2016 Xadrez dos impasses do governo Temer}{}{}
 

\section{Peça 1 -- a guerra dos vencedores}

O golpe teve dois movimentos.

No primeiro, bastou um governo inerte e um movimento de manipulação da
opinião pública para criar a figura do inimigo. Não havia riscos em
afrontar o poder e havia as vantagens de se alinhar ao bloco dos
vencedores.

Houve uma debandada geral do navio petista e dilmista, de parlamentares
aliados a magistrados, de autoridades que ascenderam nas ondas do
lulismo a petistas arrependidos.

Foi uma celebração que reuniu a malta da Câmara aos PhDs do Ministério
Público, dos torquemadas da Lava Jato aos bruxos do Senado, que permitiu
um exercício amplo da hipocrisia, com os presidenciáveis do \versal{PSDB}
propondo a fogueira aos colegas do \versal{PT}, pelos mesmos pecados que ambos
praticaram, enquanto Ministros do Supremo e o Procurador Geral se
confraternizavam no meio da turba, agraciados com placas moldadas em
chumbo quente e selfies apontando"-os como salvadores da honra nacional.

O grande porre da hipocrisia nacional chegou ao fim. Agora, entra"-se no
segundo tempo.

\section{Peça 2 -- os agentes (não) moderadores}

Perpetrado o maior ato de subversão da moderna história do país -- a
deposição por motivo fútil de uma presidente eleita -- romperam"-se os
cordões do equilíbrio institucional, tanto nas relações entre os
poderes, como na própria disciplina interna de cada poder.

Tem"-se agora, uma guerra intestina, entre e intra"-poderes, com poucos
agentes moderadores.

\subsection{O \versal{STF}}

Deflagrada esta semana, a Operação Métis -- que deteve policiais do
Senado que estavam fazendo varreduras em casa de senadores -- é um
fenômeno típico da desordem institucional do país. Os policiais cumpriam
ordens. Os mandantes eram senadores, mais que isso, o próprio presidente
do Senado. Qualquer intervenção teria que ter a autorização expressa do
Supremo.

Em vez disso, um juiz de 1à instância autorizou a prisão dos
funcionários do Senado, com a mesma sem"-cerimônia com que \versal{PM}s invadem
bares de São Paulo e Rio para espancar clientes.

O mais grave da história não foi o explícito abuso de autoridade, mas a
defesa do juiz pela presidente do Supremo, Carmen Lúcia. Ficou"-se à
beira de uma crise institucional, corrigida a tempo pela decisão do
Ministro Teori Zavascki de segurar a operação e, por vias indiretas,
enquadrar a presidente.

O episódio confirma a falta de noção de Carmen Lúcia. E~fornece sinais
claros sobre sua estratégia.

Não será um agente da paz e da concórdia, mas um instrumento para
acirrar a guerra política.

Ela tem lado -- o de Aécio Neves -- e uma vontade imensa de exercer o
mando e o protagonismo político, inversamente proporcional ao seu
conhecimento das estruturas de poder. Seu trabalho consistirá em
enfraquecer os aliados peemedebistas e investir contra o governador
mineiro Fernando Pimentel.

Portanto, tem"-se na presidente do Supremo um ponto de instabilidade.

\subsection{A Lava Jato e o \versal{PGR}}

Dia desses, um investigador da \versal{PF} ousou dar uma entrevista e
imediatamente foi enquadrado pelo comando. A~\versal{PF} tem lado, mas tem
comando.

Já o Procurador Geral da República (\versal{PGR}) Rodrigo Janot tornou"-se um
ponto morto. Em dias distantes, ousou confrontar a Lava Jato, recebeu de
volta a ameaça de demissão coletiva e definitivamente refugou.

Hoje em dia, é incapaz, sequer, de responder aos ataques que a
instituição recebe do Ministro Gilmar Mendes. Quem fala pelo Ministério
Público Federal é o procurador Carlos Fernando Santos Lima. É~ele que
investe contra os que tentam enfraquecer o \versal{MPF} e a Lava Jato.

A falta de comando de Janot será crucial quando se ingressar na guerra
mundial que se avizinha.

\subsection{A Polícia Federal}

A Operação Acrônimo, em Minas Gerais, é um abuso de poder e uma
politização da \versal{PF} mais grave que a própria Lava Jato. É~toda baseada em
um marqueteiro suspeito, o tal de Bené. Um dia, a \versal{PF} detém ~Bené e solta
trechos da sua delação. Depois, solta o Bené. Mais tarde, prende de novo
o Bené para novas averiguações e novas manchetes. É~um movimento
contínuo de marola, com o único intuito de impedir o governador de
governar.

O caso Bené é elucidativo. Montou"-se a operação em pleno segundo turno
das eleições. Pimentel já estava eleito, mas se pensava em influenciar a
campanha para presidente. ~Ele foi detido em um avião com uma pasta com
algum dinheiro. Bastou para se tentar tirar o mandato de um governador
eleito. Na mesma época, descobriu"-se um helicóptero com 500 quilos de
cocaína -- de propriedade de um senador mineiro -- e não resultou em
nada.

Tentou"-se o impeachment do governador através da operação. Uma decisão
do \versal{STJ} (Superior Tribunal de Justiça) matou a tentativa. Por 8 x 6 os
ministros decidiram que qualquer tentativa de processo teria que ter
autorização da Assembleia Legislativa.

Na semana passada, o Estadão já deu o mote: o Supremo, pautado por
Carmen Lúcia, poderá apreciar a decisão do \versal{STJ}.

\subsection{O Senado}

Em que pese as denúncias contra ele, o presidente do Senado, Renan
Calheiros, é um dos poucos agentes moderadores no quadro atual. Tem
noção da relevância do cargo, da responsabilidade institucional, da
importância de preservar o espaço de mediação Senado em meio ao
tiroteio. Mas está vulnerável.

\section{Peça 3 -- os fatores de instabilidade}

Esse quadro de poucos agentes moderadores, de mediocrização do alto
comando institucional, torna mais perigosos os movimentos
desestabilizadores.

\subsection{1.~~~~ Lava Jato}

Continuará tentando manter seu protagonismo político.

Até agora, os procuradores acreditavam que conseguiram romper a muralha
que protegia a elite brasileira graças ao seu extraordinário trabalho (e
a ajuda indispensável da cooperação internacional), com o apoio
desinteressado da mídia.

Daqui para frente irão se confrontar com as verdadeiras estruturas de
poder, o pacto tácito entre mídia, lideranças tucanas, a \versal{PGR} e os
Ministros mais combativos do \versal{STF}. E~perceberão a diferença entre bater
em um governo inerte e investir contra o poder real.

Será um embate curioso.

Embora haja ~uma clara tomada de posição de toda a Lava Jato contra
Lula/\allowbreak{}\versal{PT} e a favor do \versal{PSDB}, a mina Lula está prestes a se esgotar. Para
sobreviver, a Lava Jato terá que virar o disco. E~virar o disco
significará romper com o pacto de impunidade do \versal{PSDB}.

Não é por outro motivo que Gilmar Mendes e seus blogueiros começaram a
bater forte na operação e em Sérgio Moro.

Sempre haverá maneiras de procrastinar, valendo"-se da estratégia padrão.
Periodicamente, divulga"-se uma denúncia ou outra contra um parlamentar
aliado. Não permanece mais que um dia nos jornais. Procuradores e
delegados têm o poder de controlar os fluxos seguintes de vazamentos,
para uma mídia pouco propensa a dar espaço contra os seus. Depois, no
âmbito do \versal{STF} -- para onde irão os acusados com foro privilegiado -- a
\versal{PGR} poderá se valer do fator tempo de acordo com seus critérios. Bastará
colocar mais gás em alguns processos que nos outros, para controlar os
efeitos políticos.

Mesmo assim, a mega"-delação da Odebrecht significa artilharia pesada
contra todos os frequentadores do saloon, em um momento em que a
blindagem do \versal{PSDB} passou a incomodar até a opinião a pública não
alinhada e se tornar motivo de chacota nacional.

\subsection{2.~~~~ Economia andando de lado}

O único fator capaz de legitimar a gestão Michel Temer seria uma
eventual recuperação da economia. Mas não espere recuperação da economia
com o governo Temer. Não existe nenhum fator de demanda no horizonte:

\begin{itemize}
\itemsep1pt\parskip0pt\parsep0pt
\item
  ·~~~~~~ O aumento dos gastos fiscais não priorizou investimentos.
\item
  ·~~~~~~ A política monetária do \versal{BC} eleva de tal maneira a \versal{TIR} (Taxa
  Interna de Retorno) para novos investimentos, que praticamente os
  inviabiliza.
\item
  ·~~~~~~ A apreciação do câmbio inverteu novamente a balança comercial,
  desestimulando as exportações.
\item
  ·~~~~~~ Para 2017 espera"-se um aperto fiscal ainda maior, na véspera
  de um ano eleitoral.
\item
  ·~~~~~~ A questão da segurança jurídica foi para o espaço tanto pela
  perda de legitimidade do governo Temer como pela atuação do \versal{MPF} de
  criminalizar qualquer tipo de política pública e de contratos.
\end{itemize}

Virando o ano, ficará mais clara a incapacidade do governo Temer de
virar o jogo econômico.

\subsection{3.~~~~ A corrosão do governo Temer}

Talvez o governo Temer entregue a \versal{PEC} 241. Mas não vai entregar a
reforma da Previdência. Todos os membros da camarilha que tomou o poder
-- incluindo o próprio Temer -- estão nas delações da Odebrecht.

Além disso, Temer não possui a menor envergadura moral e política para
articular um arranjo entre os poderes, que permita baixar a fervura
política. É~politicamente pequeno, desinformado, domina as mesóclises,
mas não os rituais do cargo, e não tem envergadura para ser nem o líder
político, nem o mediador com os demais poderes de que o momento
necessita.

\section{Peça 4 -- o cenário imponderável}

Está"-se naqueles momentos de impasse, que abrem um leque enorme de
possibilidades.

Tem"-se, de um lado, centros difusos de poder, articulando"-se,
reorganizando"-se, mas de maneira caótica, sem um fio condutor, sem uma
liderança clara, com algumas peças sem nenhum controle, como o canhão
solto no convés do navio -- como é o caso da Lava Jato.

Movem"-se em um cenário caótico, sem a presença de agentes moderadores,
nem sequer de lideranças articuladoras.

Nesse terreno movediço, há duas variáveis chave: a delação da Odebrecht
e o julgamento do \versal{TSE} (Tribunal Superior Eleitoral) da chapa
Dilma"-Temer.

Ambas são armas relevantes nas mãos de quem conseguir se articular
melhor.

A eventual derrubada de Temer está condicionada aos seguintes fatores:

\begin{enumerate}
\itemsep1pt\parskip0pt\parsep0pt
\item
  1.~~~~ Nenhum partido quer pegar o pepino de um governo provisório, a
  não ser no bojo de uma conspiração mais ampla, destinada a interromper
  as eleições de 2018.
\item
  2.~~~~ Por outro lado, a deterioração do governo Temer é uma ameaça
  geral. A~desmoralização se acentuará com a não recuperação da
  economia, com os desmandos do seu ministério e com o aprofundamento do
  arrocho fiscal no próximo ano.
\item
  3.~~~~ Vive"-se um caos institucional, no qual a presença de um
  presidente fraco não interessa a ninguém.
\end{enumerate}

Há um conjunto de processos em andamento -- conforme relatado em outros
Xadrez \mbox{---,} como o maior protagonismo militar, mas ainda incipientes.
Ainda não há um quadro maduro permitindo apostar todas as fichas em uma
ou outra direção, embora ocorra um processo acelerado de deslegitimação
de Temer.

Portanto, aos cenários que traço abaixo deve ser dado o devido desconto.

\textbf{Cenário 1}~-- Temer se arrastando até 2018.

\textbf{Cenário 2}~-- Temer sendo derrubado no \versal{TSE} e sendo substituído
por um presidente eleito indiretamente.

A eventual queda de Temer significaria o \versal{PMDB} voltando ao seu papel de
adjunto do poder. Mas dificilmente entraria um presidente alinhado com o
\versal{PSDB}, tanto pela herança pesada, como pelas divisões entre os tucanos e
pelas resistências dos peemedebistas.

A hipótese de um tertius é a mais razoável.

Nesse caso, não deve ser menosprezada a alternativa Nelson Jobim,
aventada recentemente. Tem senioridade, autoridade, preparo, transita
bem por muitos setores.

A incógnita maior seria o conjunto de forças que o apoiaria e o perfil
de seu governo: se se acomodaria a um governo de transição ou não.
