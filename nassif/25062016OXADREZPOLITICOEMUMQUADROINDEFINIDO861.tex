\chapterspecial{25/\allowbreak{}06/\allowbreak{}2016 O xadrez político em um quadro indefinido}{}{}
 

O quadro político é o seguinte, após a inundação de grampos pela Lava
Jato. Há uma enorme guerra de informações superestimando atitudes contra
Dilma e escondendo as posições a favor.

Nesse momento, está"-se no auge da pressão pró"-impeachment.

Serão necessários alguns dias para ver se a poeira assenta
provisoriamente ou não, para poder avaliar melhor o cenário de curto
prazo.

No momento, o jogo de forças é o seguinte:

\textbf{Judiciário}

Por oportunismo, interesse corporativo, preferências políticas, o
sistema judiciário está rasgando todos os princípios de garantias
individuais.

Já se aceitam como normais 114 conduções coercitivas, grampos sobre
advogados de defesa, sobre a própria presidente da República. O~decano
do \versal{STF} prefere se indignar com frases extraídas de conversas informais
do que com a evidência de um estado policial em andamento.

Em Paris, em vez de deblaterar contra o grampo, o Procurador Geral da
República Rodrigo Janot também preferiu responder às conversas informais
gravadas

Em São Paulo, o Ministério Público Estadual ofertou uma delação premiada
em um episódio sem maiores evidências, de contratação de uma orquestra
espanhola pelo Teatro Municipal. E~não recorreu ao instituto da relação
no caso da merenda escolar ou dos trens, demonstrando um ativismo
político crescente.

O que se observa são Ministros exauridos pela tensão querendo que a
novela termine logo. Do outro lado, corporações -- como a dos
procuradores e delegados da Polícia Federal --- entrando ativamente na
política em favor da Lava Jato.

Todas as manifestações são de crítica ao conteúdo de conversas
informais, e não de reação à escalada policial.

Definitivamente, ocorreu a politização da Justiça é um jurisdicismo de
tratar a Constituição com um manual, de como atropelar os princípios sem
deixar marcas legais. Lembra o manual dos torturadores profissionais,
sobre como torturar sem deixar marcas.

Até que as figuras referenciais se apresentem, o sistema judiciário
continuará oferecendo respaldo à Lava Jato.

\textbf{Congresso}

Havia espaço para negociação com o \versal{PMDB}. A~divulgação dos grampos
colocou todo mundo em compasso de espera. Qualquer desculpa serve para
desembarcar da canoa do governo. A última foi a indicação de um deputado
do \versal{PMDB} para um Ministério.

Foi aprovada a comissão do impeachment. Mas ainda há água vai rolar até
o desfecho. O~governo precisa de 171 deputados para derrubar a tese do
impeachment.

\textbf{Empresários}

Há dois tipos de empresários: os que são contra o governo em qualquer
hipótese e os que são a favor da alternativa menos custosa. Neste
momento, se não surgir nenhum fato novo contra a escalada da Lava Jato,
a alternativa menos custosa será o impeachment.

Se a caminhada do impeachment endurecer, abrem"-se espaços para pactos de
governabilidade menos traumáticos.

\textbf{Movimentos sociais}

Hoje será o dia da demonstração de mobilização dos movimentos sociais.
Nas últimas semanas as diferenças com o governo Dilma foram deixadas
para trás e a defesa da legalidade serviu para unificar o movimento.
Mais do que em qualquer outro momento, a manutenção do governo dependerá
da capacidade de mobilização dos movimentos sociais.

A decisão do governador de São Paulo Geraldo Alckmin de permitir
manifestações contrárias hoje, é mais uma demonstração da escalada
irresponsável visando acirrar os conflitos.

\textbf{Lula}

Terá que aposentar a jararaca. O~momento exigirá, acima de tudo,
habilidade para fazer baixar a poeira e tempo para remontar as alianças
perdidas e perspectivas mínimas para a economia.

\textbf{Cenário do momento}

Há um quadro de radicalização crescente nas ruas e um desmoronamento de
todas as formas de mediação jurídica ou política. A~expressão estado
policial define bem o momento. A~segurança jurídica ganhou contornos
políticos: tem segurança jurídica quem escolheu o ``lado certo''.

Não se trata mais de uma disputa \versal{PT} x oposição, mas um questionamento de
todas as instituições e, especialmente, da democracia. A~frente
anti"-impeachment está sendo engrossada por liberais e pelas bandeiras de
defesa da democracia. Até que ponto a consciência democrática é forte no
país, mesmo entre luminares do Judiciário?

Esta é a resposta que vale 25 anos de estabilidade democrática.
