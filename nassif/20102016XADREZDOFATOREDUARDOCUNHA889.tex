\chapterspecial{20/\allowbreak{}10/\allowbreak{}2016 Xadrez do fator Eduardo Cunha}{}{}
 

A graça de um cenário é quando consegue identificar fatos pouco
conhecidos, montar ilações pouco percebidas, tirar conclusões
inesperadas.

Não é o caso da prisão do ex"-deputado Eduardo Cunha, respeitosamente
detido pela Polícia Federal, com autorização do juiz Sérgio Moro, e com
a recomendação de não fazerem espetáculo.

As conclusões unânimes são as seguintes:

\begin{enumerate}
\itemsep1pt\parskip0pt\parsep0pt
\item
  1.~~~~ Eduardo Cunha era pato manco desde o ano passado. Era um caso
  de prisão óbvia.
\item
  2.~~~~ Nunca pertenceu ao establishment político e midiático, como
  Aécio Neves e José Serra. Portanto, seria mínima a linha de
  resistência à prisão.
\item
  3.~~~~ O grupo da Lava Jato, juiz Sérgio Moro à frente, conta que, com
  a prisão, se consiga demonstrar um mínimo de imparcialidade, ampliando
  a força para uma futura prisão de Lula.
\end{enumerate}

Essas são as conclusões óbvias. Os desdobramentos, são mais
imprevisíveis.

Ninguém minimamente informado tem a menor dúvida sobre a parcialidade da
Lava Jato e sobre as estratégias políticas por trás de cada operação.
Nas vésperas das eleições municipais, foram mais três operações com
estardalhaço sobre alvos petistas.

Agora, uma operação discreta sobre um não"-petista.

Há os objetivos óbvios da Lava Jato e os desdobramentos ainda obscuros.

\section{Peça 1 --- Prisão e/\allowbreak{}ou inabilitação de Lula para 2018.}

Dias atrás, a Vox Populi soltou uma pesquisa sobre eleições
presidenciais. Em todas elas, dava vitória de Lula no primeiro turno.
Nenhum veículo de imprensa repercutiu.

Ontem, foi a vez da \versal{CNT}"-\versal{IBOPE} divulgar outra pesquisa com resultados
semelhantes.

Mais ainda. No segundo turno, o único em condições de enfrentar Lula
seria Aécio Neves (devido ao recall das últimas eleições) e mesmo assim
haveria empate técnico.

Com todos os demais candidatos, haveria vitória de Lula.

Um dado da pesquisa Vox Populi foi pouco notado. Na relação dos
brasileiros mais admirados, o primeiro é Sérgio Moro, com 50\%. O~segundo, Lula, com 33\%. O~terceiro, Dilma com 23\%. Os demais vêm mais
abaixo.

Hipoteticamente, a única pessoa capaz de peitar Lula seria Sérgio Moro.
E~em seu terreno, o Judiciário e no terreno comum da opinião pública.

\section{Peça 2 --- Os tucanos blindados}

Para analisar os desdobramentos da eventual delação de Eduardo Cunha, o
primeiro passo é identificar os que \versal{NÃO} serão atingidos.

Obviamente, serão as lideranças tucanas, devidamente blindadas pela Lava
Jato e pela Procuradoria Geral da República (\versal{PGR}).

\subsection{Aécio Neves}

Os jornais soltam fogos de artifício para demonstrar isenção. Foi o caso
da denúncia de que Aécio Neves viajou para os Estados Unidos com
recursos do fundo partidário, um pecadilho.

A dúvida que ninguém respondeu até agora: porque Dimas Toledo, o caixa
político de Furnas, jamais foi incomodado pela Lava Jato ou pela
Procuradoria Geral da República (\versal{PGR})?

Dimas é a chave de todo esquema de corrupção de Furnas.

Há o caso do helicóptero com 500 quilos de cocaína, que jamais mereceu
uma iniciativa sequer do Ministério Público Federal.

Em 2013, o \versal{MPF} aliou"-se à Globo para derrubar a \versal{PEC} 37, que pretendia
restringir seu poder de investigação. A~alegação é que o \versal{MPF} não poderia
ficar a reboque da Polícia Federal, quando percebesse pouco empenho nas
investigações.

A \versal{PF} abafou o caso do helicóptero. E~o \versal{MPF} esqueceu.

\subsection{José Serra}

A recente decisão da Justiça, de anular a condenação dos réus do chamado
``buraco do Metrô'', escondeu um escândalo ainda maior. Os réus eram
funcionários menores das três empreiteiras envolvidas -- Odebrecht,
Camargo Correia e \versal{OAS}.

Fontes que acompanharam as investigações, na época, contam que a
intenção inicial do Ministério Público Estadual era indiciar os
presidentes das companhias. Houve uma árdua negociação política,
conduzida por instâncias superiores do Estado, que acabou permitindo que
as empreiteiras indicassem funcionários de escalão inferior. O~custo da
operação teria sido de R\$ 15 milhões, divididos irmãmente entre as três
empreiteiras.

O governador da época era José Serra.

Na Operação Castelo de Areia (que envolveu a Camargo Correia, e que foi
anulada graças a um trabalho político do advogado Márcio Thomas Bastos)
havia indícios veementes do pagamento de R\$ 5 milhões pela empreiteira.
Agora, a delação da Odebrecht menciona quantia similar. Interromperam a
delação do presidente da \versal{OAS}, mas não seria difícil que revelasse os
detalhes.

São bolas quicando na área do \versal{PSDB} e que dificilmente serão aproveitadas
pela Lava Jato ou pelo \versal{PGR}.

\section{Peça 3 -- os desdobramentos da delação de Cunha}

\subsection{Desdobramento 1 --- Temer}

Eduardo Cunha é obcecado, mas não rasga dinheiro. Tem noção clara de
seus limites. Sabe que uma delação só aliviará suas penas se aceita pela
Lava Jato ou pelo \versal{PGR}.

Como existe o privilégio de foro para políticos com mandato ou cargos, o
árbitro para as delações envolvendo o andar de cima é o \versal{PGR} Rodrigo
Janot. O~conteúdo das delações dependerá muito mais das intenções de
Janot e da Lava Jato do que do próprio Cunha.

Portanto, todos os desdobramentos da prisão de Cunha dependerão nas
relações entre \versal{PSDB}"-mídia"-Judiciário e a camarilha dos 6 (Temer, Cunha,
Jucá, Geddel, Padilha, Moreira Franco) que assumiu o controle do país.

Poderá haver acertos de conta pessoais de Cunha com um Moreira Franco,
por exemplo, que poderá ser defenestrado sem danos maiores ao grupo de
Temer.

Mas qualquer ofensiva mais drástica sobre o grupo teria que ser
amarrada, antes, com a mídia (especialmente Globo), com o \versal{PSDB} e sentir
os ventos do \versal{STF} (Supremo Tribunal Federal). São esses os parâmetros que
condicionam os movimentos da Lava Jato e da \versal{PGR}.

Temer tem se revelado um presidente abaixo da crítica. Mas ainda é
funcional, especialmente se entregar a \versal{PEC} 241. A~cada dia, no entanto,
amplia seu nível de desgaste. Em um ponto qualquer do futuro se tornará
disfuncional. E~aí a arma Eduardo Cunha poderá ser sacada pelo \versal{PGR}.

\subsection{\textbf{Desdobramento 2 -- Lava Jato}}

A Lava Jato vive seus últimos momentos de glória. Seu reinado termina no
exato momento em que pegar Lula. Justamente por isso, é possível que
queira tirar alguns fogos de artifício da gaveta para o pós"-Lula.

À medida em que se esgote, os tribunais superiores passarão a rever suas
ilegalidades, a fim de poupar os políticos até agora não atingidos por
ela.

Mas ainda é uma caixa de Pandora.

\subsection{\textbf{Desdobramento 3 -- as novas lideranças}}

A prioridade total é a inabilitação e/\allowbreak{}ou prisão de Lula.

Só depois disso é que haverá o novo realinhamento político, e aí com
novo atores.

Do lado do \versal{PSDB}:

\begin{enumerate}
\itemsep1pt\parskip0pt\parsep0pt
\item
  1.~~~~ Geraldo Alckmin subindo, depois da vitória de João Dória Jr.
\item
  2.~~~~ Aécio em queda, pelos indícios de crime, mesmo não levando a
  consequências legais.
\item
  3.~~~~ Serra fora do jogo, tentando decorar siglas de organizações
  multilaterais, sem apoio no \versal{PSDB} e no \versal{DEM}.
\end{enumerate}

Do lado das oposições:

\begin{enumerate}
\itemsep1pt\parskip0pt\parsep0pt
\item
  1.~~~~ Já está em formação um núcleo de governadores progressistas,
  visando costurar estratégias e alianças acima das executivas dos
  partidos. Anote que daqui para a frente tenderão a ter um protagonismo
  cada vez maior na cena política, substituindo as estruturas
  partidárias, imobilizadas em lutas internas.
\item
  2.~~~~ Ciro Gomes é o opositor de maior visibilidade, até agora, mas
  mantendo o mesmo estilo carbonário da juventude. Suas verrinas contra
  Temer fazem bem ao fígado, mas preocupam as mentes mais responsáveis.
\item
  3.~~~~ Há uma tendência de crescimento de Fernando Haddad, prefeito
  derrotado nas últimas eleições. Na expressão do governador baiano Rui
  Costa, Haddad caiu para cima. Sua avaliação, no \versal{MEC} e na prefeitura de
  São Paulo, crescerá com o tempo.
\end{enumerate}
