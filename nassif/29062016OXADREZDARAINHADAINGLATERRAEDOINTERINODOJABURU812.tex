\chapterspecial{29/\allowbreak{}06/\allowbreak{}2016 O Xadrez da Rainha da Inglaterra e do interino do Jaburu}{}{}
 

A história é repleta de paradoxos. É~como uma espiral, sempre dá voltas
retornando ao mesmo lugar, mas alguns degraus acima, como dizia o músico
e filósofo Koellreutter. Há enormes semelhanças entre as crises das
primeiras décadas do século 20 e as atuais, culminando com o Brexit do
Reino Unido, a campanha pela saída do Reino Unido da Comunidade
Europeia, que foi vitoriosa no referendo.

Desde o século 19 há a disputa pelo controle das políticas econômicas
nacionais, entre a proposta globalizantes -- liderada pelo grande
capital internacionalizado -- e os projetos nacionais.

Esta disputa está na raiz da economia como ciência. De um lado, o
pensamento majoritário de crença no mercado, que nasce com Adam Smith,
com o mundo racionalmente integrado por economias nacionais, cada qual
fundando"-se em suas vantagens comparativas.

De outro, o desenvolvimento da economia política, a convicção sobre o
papel do Estado nacional para criar a competitividade sistêmica, a
partir das ideias do norte"-americano Alexander Hamilton, sistematizadas
depois pelo economista alemão Friedrick List. Nesse modelo, mercado
interno passa a ser tratado como ativo nacional, assim como a proteção
das indústrias nascentes, os investimentos estratégicos para conquistar
mercados etc.

Na base de tudo, sistemas eleitorais nos quais os dois lados irão vender
suas utopias, sobre qual modelo é mais eficiente para levar o bem"-estar
à maior parte da população eleitora.

 Primeiro passo --- a integração dos mercados

No século 19, a expansão da economia global, as novas rotas marítimas, a
integração continental com as ferrovias, permitiram alguma integração
internacional através do comércio.

O passo seguinte foi através dos fluxos de capitais, a primeira
articulação efetiva entre países, a partir da coordenação do Banco da
Inglaterra, tendo como parceiros os bancos centrais da Europa e dos
países periféricos -- no caso nosso, do Banco do Brasil cumprindo essas
funções.

A cooptação das elites nacionais se dava através de três personagens
centrais:

\begin{enumerate}
\itemsep1pt\parskip0pt\parsep0pt
\item
  Os capitalistas locais, que já mantinham relações com a banca inglesa.
\item
  Economistas portadores das últimas novas da nova ciência, incumbidos
  de criar a utopia de que a livre circulação de capitais traria a
  prosperidade geral.
\item
  Políticos eleitos, turbinados pelos recursos dos capitalistas e pelas
  utopias dos economistas.
\end{enumerate}

A globalização viceja fundamentalmente em países democráticos, em que o
jogo se decide pela cooptação dos vários agentes de opinião pública:
intelectuais, jornais, políticos, advogados.

No meu livro ``Os Cabeças de Planilha'' detalho melhor esse modelo e a
maneira como cooptaram Rui Barbosa, o primeiro Ministro da Fazenda da
República.

Com esse pacto instituiu"-se o predomínio do capital financeiro, abolindo
qualquer forma de controle e regulação de mercados em um longo período
que vai das três últimas décadas do século 19 até a Primeira Guerra
Mundial.

Permitiu"-se a criação de uma gama extraordinária de novas operações de
mercado, visando turbinar ainda mais a especulação.

No tempo de Rui Barbosa, já se batizara de ``tacadas'' as jogadas
possíveis com o controle da moeda, do crédito e a liberação do câmbio,
que incluíam jogadas em bolsa, concessões ferroviárias escandalosas,
operações de crédito com estados e União.

Esse modelo gera uma dinâmica que se espalha por várias economias até
implodir o próprio modelo: Força política --\textgreater{} Desregulação
de mercado --\textgreater{} Criação de novos instrumentos financeiros
--\textgreater{} Geração de bolhas especulativas --\textgreater{}
Implosão.

No caso brasileiro, o resultado foi a grande crise cambial do
encilhamento, no nascimento da República, que atrasou por trinta anos o
desenvolvimento do país.

\section{Segundo passo -- o choque de realidade}

Aí chega a conta. Sucessivas bolhas especulativas minam as economias
nacionais, mas o sistema político não consegue reagir porque, no período
de predomínio da financeirização, sufocam"-se as alternativas
democráticas de mudança de rota.

Os cidadãos são tomados de profundo ceticismo em relação ao modelo
político vigente, tanto interna quanto externamente, em relação às
instituições multilaterais, em geral criadas para impor o poder do
credor sobre os devedores.

As consequências fazem parte da história: Primeira Guerra, marcando o
início do fim do modelo; crise de 1929 assinalando seus estertores; as
disputas cambiais"-comerciais entre nações; o nascimento do comunismo na
Rússia (ainda uma economia feudal) e do nazi"-fascismo a partir das
disputas eleitorais na Alemanha, França e Espanha; a incapacidade da
Liga das Nações em arbitrar conflitos nacionais.Na sequência, a
consolidação de regimes ditatoriais até o desfecho final na Segunda
Grande Guerra.

Os tempos são outros, o desfecho certamente será distinto, mas os
sintomas são os mesmos.

Desde 1972, a financeirização passou a comandar as políticas nacionais.
A~expansão do capitalismo norte"-americano turbinou a China, da mesma
maneira que o inglês turbinou os Estados Unidos no século 19.
Montaram"-se os grandes blocos econômicos, abolindo as fronteiras
nacionais.

No plano socioeconômico, abriu uma enorme janela de oportunidades,
brilhantemente aproveitada pela China e pelos Tigres Asiáticos,
relativamente aproveitada pela América Latina.

Países com baixos salários começaram a se industrializar, como chão de
fábrica das grandes corporações. E~países que não lograram desenvolver
uma estratégia eficiente ficaram fora do baile.

Mais que isso, com o avanço das redes sociais e das diversas formas de
comunicação global, a expansão do mercado de consumo e dos valores
ocidentais, e sua contraposição, nos movimentos fundamentalistas em
países de pouca tradição democrática,abrem espaço para um redesenho da
geopolítica mundial. Nesse entrechoque de culturas, países inteiros
foram destroçados devido ao desmonte de suas instituições. Trocaram uma
ordem anacrônica, antidemocrática, pelo caos.

Em fins do século 19, as diversas guerras e crises europeias e do
Oriente Médio promoveram um formidável fluxo de migração para os
emergentes, beneficiando substancialmente \versal{EUA} e América do Sul com mão
de obra de qualidade superior.

No século 21, o fluxo migratório inverteu, com populações inteiras de
nações destroçadas ou que perderam o dinamismo, invadindo o mercado de
trabalho dos países centrais, já assolado pelas perdas de direitos,
consequência dos ajustes que tiveram que serem feitos para impedir a
quebra dos sistemas bancários nacionais.

Os efeitos são visíveis:

\begin{enumerate}
\itemsep1pt\parskip0pt\parsep0pt
\item
  Aumento do individualismo e da xenofobia.
\item
  Crise dos partidos tradicionais e das instituições internas.
\item
  Crescimento dos partidos de direita, estimulados pelas mídias
  nacionais, que pretenderam cavalgar a onda para ampliar seu poder
  político, ante as novas formas de comunicação.
\end{enumerate}

É o que explica o referendo britânico.

A integração europeia era defendida pelo establishment político,
financeiro, acadêmico. E~foi derrotada pelo voto de protesto difuso, no
qual se misturaram ~a ultradireita xenófoba e a esquerda
antiglobalização. Ou seja, a elite perdeu o controle das massas. O~regime democrático torna"-se disfuncional. E~a maneira encontrada para
controlar as pressões nacionais -- a camisa de força da União Europeia
-- começa a fazer água.

\section{Os desdobramentos no Brasil}

Todos esses episódios têm desdobramentos no Brasil.

De 2008 a 2012 o Brasil se beneficiou da estratégia anticíclica de Lula
e da sobrevida da especulação internacional com commodities, que
garantiu alguns anos a mais de fartura.

Quando a crise derrubou as cotações de commodities, depois de dois anos
de bom governo Dilma perdeu o rumo. Não conseguiu definir uma estratégia
econômica, política, ou social, como ocorreu na crise de 2008 com Lula.

A crise derrubou o ânimo nacional e incendiou as ruas, com multidões
insufladas pela mídia e compondo uma geleia geral ideológica: contra os
impostos e a favor da melhoria da educação e saúde públicas.

A insatisfação foi turbinada pela Lava Jato, pela piora nas expectativas
econômicas e pelos problemas com os serviços públicos.Mas não resultou
em um conjunto articulado de propostas, encampado por algum partido
político ou alguma liderança emergente. Houve apenas a insatisfação
generalizada que abriu espaço para a ação descoordenada de grupos
oportunistas de diversas espécies, como os grupos de Cunha"-Temer, a Lava
Jato, a mídia, os mercadistas. E~isso em uma quadra da história em que
escassearam as figuras referenciais, na política, na Justiça, no \versal{MPF},
nos partidos e na mídia.

Essa frente entregou o poder de bandeja para uma das organizações mais
suspeitas da moderna história política brasileira: o grupo de Michel
Temer, Eduardo Cunha, Eliseu Padilha, Geddel Vieira de Lima e Romero
Jucá.

A chance de dar certo é próxima de zero, conforme se verá a seguir.

\subsection{Um interino vulnerável moral e penalmente}

A notícia de Temer recebendo Eduardo Cunha reservadamente no Palácio
Jaburu, por si, seria motivo de impedimento de Temer. O~presidente
interino conversando reservadamente com um parlamentar cujo cargo foi
suspenso por suspeita de corrupção, apontado em vários desvios e
proibido de frequentar a Câmara, justamente para não conspirar contra a
Justiça. Certamente a conversa não girou sobre o Brexit nem sobre a
atual campanha do Vasco da Gama. E~só foi oficialmente divulgada após os
vazamentos sobre o encontro sigiloso.

Para o interino se expor dessa maneira, mostra uma relação nítida de
interesses.

A qualquer momento, Temer poderá ser fuzilado por uma das seguintes
alternativas:

\begin{enumerate}
\itemsep1pt\parskip0pt\parsep0pt
\item
  Uma delação de Cunha ou de outros membros da quadrilha.,
\item
  Uma denúncia da Procuradoria Geral da República.
\item
  Vazamentos de informações pelos jornais e redes sociais.
\end{enumerate}

Será possível ao país conviver com um interino com tais
vulnerabilidades, com uma biografia polêmica, uma companhia suspeita e
tendo nas mãos a mais poderosa caneta da República?

\subsection{Um interino sem dimensão política}

Dilma entendeu a dimensão da crise, mas não teve competência para
enfrentá"-la. Temer sequer logrou um diagnóstico consistente sobre o
cenário atual. É~surpreendente que, em algum momento de sua vida,
criasse fama de intelectual. Suas declarações públicas não conseguem ir
além dos ecos da imprensa,.

A maneira como se escora em Cristovam Buarque é deprimente. Alardeou aos
quatro ventos o grande elogio recebido de Cristovam, que disse que só
votaria pela volta de Dilma se ela mantivesse Henrique Meirelles e a
equipe econômica. Ou seja, o~\emph{aggiornamento}~de Cristovam não foi
apenas em relação ao \versal{PT}, mas à própria social democracia e à função do
Estado que um dia fizeram parte de sua biografia.

Cristovam é uma espécie de Eugenio Bucci do Senado, equilibrando"-se
permanentemente entre extremos através de declarações rasas de um
equilibrismo vazio.

A receita da lição de casa -- os sacrifícios impostos aos cidadãos ---
funcionou quando podia se invocar o fantasma da hiperinflação. Qualquer
sacrifício seria legítimo, pois todos eles visariam impedir a volta do
fantasma.

O momento é outro. Têm"-se uma população que experimentou períodos de
bonança, conquistou direitos, incluiu"-se no mercado e não aceita
retrocessos. Para ela, Temer acena com mudanças radicais na Previdência,
cortes nos gastos sociais com educação e saúde, aparelhamento da máquina
pública com o que de pior a fisiologia política criou, a corrupção
endêmica, profundamente enraizada na atuação política do grupo que
empalmou o poder.

\subsection{A democracia sem votos}

É nessa sinuca que se desenvolve a tese da democracia sem votos, um
sistema controlado pelas corporações públicas, pelo Ministério Público
Federal e Tribunais superiores, pelos Tribunais de Contas associados à
mídia.

É por aí que se entende a geopolítica norte"-americana, de aproximar"-se
das estruturas dos Ministérios Públicos e Judiciários nacionais. Aliás,
como bem lembrou Dilma na entrevista à Pública, a interferência externa
não é agente central do golpe, que é fundamentalmente coisa nossa.

Será impossível se aplicar as teses neoliberais a seco. Nem encontrar
políticos de discurso claro e vida limpa para conduzir o desmonte do
Estado social sem ter o que mostrar pela frente.

Olhando todas essas peças do jogo, há movimentos que tenderão a crescer
exponencialmente:

\begin{enumerate}
\itemsep1pt\parskip0pt\parsep0pt
\item
  Contra o golpe, ganhará fôlego a tese da constituinte exclusiva para a
  reforma política, suprapartidária, tendo como bandeira comum a crítica
  à crise de representatividade do Parlamento e dos partidos.
\item
  Como aprimoramento do golpe, inicialmente a tentativa de tucanização
  de Temer, esbarrando na dinâmica da Lava Jato, de criminalizar também
  as lideranças tucanas até agora poupadas. Todos fazem parte do mesmo
  balaio.
\item
  Como saída alternativa, o impedimento da chapa Dilma"-Temer seguido de
  eleições indiretas visando consagrar alguém fora da política
  tradicional para completar o trabalho.
\item
  Como lance final, maneiras de inviabilizar as eleições de 2018, pela
  óbvia impossibilidade de vencer eleições montado na velha lição
  neoliberal de desmonte das conquistas sociais.
\end{enumerate}
