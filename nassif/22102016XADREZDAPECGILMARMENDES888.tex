\chapterspecial{22/\allowbreak{}10/\allowbreak{}2016 Xadrez da \versal{PEC} Gilmar Mendes}{}{}
 

O Xadrez de hoje propõe a \versal{PEC} Gilmar Mendes, uma Proposta de Emenda à
Constituição que definisse um mandato máximo de 12 anos aos juízes de
tribunais superiores, de maneira a poupar o Judiciário das constantes
desmoralizações a que é submetido pelas intervenções do Ministro Gilmar
Mendes.

\section{Peça 1 -- o fator Gilmar Mendes}

A decisão da Frente Associativa de Magistratura e do Ministério Público
(Frentas) -- coordenada pela Associação dos Magistrados Brasileiros
(\versal{AMB}) -- de protocolar na Procuradoria Geral da República (\versal{PGR}) um
pedido de apuração e possível abertura de inquérito criminal contra o
Ministro Gilmar Mendes, do \versal{STF} (Supremo Tribunal Federal), é a primeira
reação efetiva, ainda que tardia, aos abusos cometidos reiteradamente
por ele contra os códigos de ética da magistratura
(\url{migre.me/\allowbreak{}vit7l}).

São inúmeras as ressalvas contra Gilmar.

\section{O empresário Gilmar}

Dono do \versal{IDP} (Instituto Brasiliense de Direito Público), Gilmar incorre
reiteradamente em uma série de ações que depõem contra a imagem de
neutralidade de um juiz.

Há inúmeras coincidências entre contratos firmados com o poder público e
privado e seus cargos no \versal{CNJ} (Conselho Nacional de Justiça), no \versal{TSE}
(Tribunal Superior Eleitoral) e no \versal{STF} (Supremo Tribunal Federal).

\subsection{O caso Paulinia}

Em abril de 2016 o \versal{IDP} firmou um contrato de R\$ 280 mil, sem licitação,
com a prefeitura de Paulínia (\versal{SP}), com direito a uma aula magna do
próprio Gilmar. Em maio assumiria a presidência do \versal{TSE} (Tribunal
Superior Eleitoral). O~prefeito Pavan Junior, de Paulínia, aspirava a um
terceiro mandato e tinha todo o interesse em conquistar as boas graças
do \versal{TSE}, para manter o afastamento do prefeito anterior e para garantir
um terceiro mandato. Segundo o jornal de Paulinia, ``sob a relatoria do
ministro Gilmar Mendes, atual presidente do Tribunal Superior Eleitoral
(\versal{TSE}), a instrução n\,535--95.2015 do \versal{TSE}, que trata dos candidatos às
Eleições 2016, em seu artigo 14, parágrafo único, é clara:~``O prefeito
reeleito não poderá candidatar"-se ao mesmo cargo, nem ao cargo de vice,
para mandato consecutivo no mesmo
município''~\emph{(\url{migre.me/\allowbreak{}viSiw)}.}

Antes da Aula Magna ambos -- Gilmar e Pavan --- ficaram uma hora e meia
em conversas reservadas. Na primeira reunião do secretariado, Pavan
informou que protagonizaria a primeira jurisprudência eleitoral no país,
como prefeito diplomado pela Justiça Eleitoral por duas vezes dentro de
um mesmo mandato, liberado para disputar em tese um terceiro mandato
(\url{migre.me/\allowbreak{}vis\versal{QI})}.

Pavan, de fato, se candidatou novamente e foi eleito.

Não há informações maiores sobre o papel de Gilmar nessa liberação do
candidato. Mas é evidente que o \versal{IDP} fechou contrato com um prefeito que
tinha enormes interesses no \versal{TSE}, dependendo do Tribunal para afastar o
prefeito anterior, de quem era vice, e para poder se candidatar
novamente.

\subsection{O caso \versal{TJ} Bahia}

Em 2014 o \versal{IDP} conquistou um megacontrato de R\$ 12 milhões com o
Tribunal de Justiça da Bahia. Na mesma época, o \versal{TJBA} tinha entrado na
alça de mira do \versal{CNJ} (Conselho Nacional de Justiça), entidade até pouco
antes presidida por Gilmar.

Durante a auditoria, constatou"-se que as maiores irregularidades estavam
em contratos sem licitação. E~o maior contrato era justamente com o \versal{IDP}.
A~dinheirama irrigou o \versal{IDP} justamente quando Gilmar terminava as
negociações para comprar a parte do sócio, avaliada em R\$ 8 milhões.

Não apenas isso. Os seminários do \versal{IDP} têm como patrocinadores grandes
grupos empresariais com interesses diretos no Supremo, alguns dos quais
com processos que têm Gilmar na relatoria. Além disso, o \versal{IDP} emprega,
hoje em dia, vários Ministros de Tribunais superiores com salários
relevantes.

\section{As incontinências verbais}

Nas sessões do \versal{STF} e do \versal{TSE} Gilmar dá um espetáculo diário de
grosserias.

Esta semana, em sessão do \versal{TSE}, incorreu em um indesculpável ato de
misoginia, ironizando um juiz que queria parecer tão sensível na
sentença, dizia ele, que teria ``alma feminina''. E~completou taxando o
juiz de ``lunático''. Nenhum de seus pares, nem o representante do
Ministério Público Federal, ousaram questioná"-lo.

Finalmente, acusou juízes e procuradores de chantagear políticos com a
Lei da Ficha Limpa, depois de ter taxado de ``bêbados'' os autores da
Lei.

Nas sessões do Supremo está sempre disperso, saindo no meio das falas de
colegas ou de advogados, concentrando"-se em mensagens de celulares e
computador, distribuindo grosserias a torto e a direito, a ponto de
induzir a desconfianças sobre seu equilíbrio mental.

Fica evidente que seu foco, hoje em dia, são extra"-Supremo, como os
projetos do \versal{IDP} e as articulações políticas, nas quais se excede em
declarações à imprensa.

Juízes de 1\textsuperscript{a}~Instância, desembargadores, Ministros de
tribunais superiores invariavelmente se contem, recusando"-se a comentar
fora dos autos. Gilmar se pronuncia sobre tudo, sobre processos em
andamento no \versal{STF}, no Tribunal Regional do Trabalho, no \versal{TSE}, e em ataques
a seus pares.

Partícipe direto do poder, não se vexa em visitar o presidente Michel
Temer, ``como amigo'', ou distribuir acusações midiáticas contra os
adversários.

\section{Peça 2 -- a \versal{PEC} da Bengala}

Por trás desses abusos está a falta de um prazo menor para mandatos de
magistrados -- especialmente de tribunais superiores. Eles são indicados
pelo presidente da República e referendados pelo Senado. Representam,
portanto, uma determinada visão de mundo do momento. Não é correto que
aquele momento seja perenizado por toda a vida útil do indicado, seja
aos 70 anos ou, com a \versal{PEC} da Bengala, aos 75. Hoje em dia é possível a
um juiz ser nomeado com 35 e ficar até os 75 -- 40 anos na atividade
máxima.

O segundo problema é a formação de facções, grupos com afinidades, a
exemplo dos politburos soviéticos.

A perpetuidade nos cargos induz a que se organizem em grupos. Há Câmaras
em São Paulo conhecidas por condenar todos os réus --- são chamadas de
Câmaras de Gás. Outras, com tendência mais garantista. Nos dois casos, o
tempo excessivo cristaliza procedimentos, impedindo um dos pontos vitais
do arejamento da Justiça, que são as diferentes visões de mundo, a
partir das quais vão se formando os consensos.

A permanência excessiva no poder traz problemas mais graves: as redes de
influência, como a tecida por Gilmar, empregando juízes de tribunais
superiores no \versal{IDP}. Ou parentes de magistrados atuando nos tribunais em
que o o magistrado atua.

Todos esses vícios são efeito direto do tempo de permanência no cargo.
Afinal, são redes de relacionamento, consolidação de opiniões,
aglutinação de pessoas com afinidades, que tendem a se estratificar com
o tempo.

O correto seria um mandato por tempo fixo, digamos 12 anos, ao final do
qual os juízes se aposentariam.

Uma \versal{PEC} dessa natureza poderia avançar, mesmo com o direito adquirido
proveniente da \versal{PEC} da Bengala. Bastaria colocar o magistrado em
disponibilidade com vencimentos integrais, permitindo a renovação.

Com isso haveria espaço para os mais jovens ascenderam na hierarquia e
tribunais claramente políticos terem renovação.

Uma proposta dessa poderia ser batizada como \versal{PEC} Gilmar Mendes. Afinal,
Gilmar se converteu no melhor exemplo dos prejuízos que mandatos muito
longos trazem ao poder judiciário.

\section{Peça 3 -- a lista tríplice do \versal{MPF}}

Outro ponto a ser revisto é a lista tríplice para a escolha do
Procuradoria Geral da República (\versal{PGR}).

O trabalho do \versal{PGR} Rodrigo Janot foi fundamental para a queda de Dilma
Rousseff e a ascensão de Michel Temer e seus aliados -- Eliseu Padilha,
Moreira Franco, Geddel Vieira Lima e, especialmente, Romero Jucá.

Aliás, Jucá é o homem forte do grupo, como era Eduardo Cunha. Os demais,
são pequenos varejistas sem noção maior das estratégias de poder.

Após a tentativa de Janot de pedir a prisão de senadores -- incluindo
Jucá, Renan e Sarney -- e do pedido ter sido negado pelo \versal{STF} (Supremo
Tribunal Federal), o \versal{PGR} entrou na linha de fogo, inclusive colocando em
risco a lista tríplice.

Agastado com parte da categoria, espremido entre a Lava Jato e o
Supremo, Janot luta, agora, para não ser o coveiro da lista tríplice.

Foi essa preocupação que o fez nomear para vice procurador Bonifácio de
Andrada, ligado ao \versal{PSDB} mineiro e, especialmente, a Aécio Neves. A~ideia
é montar sua campanha interna no \versal{MPF} de maneira a oferecer a Temer um
nome palatável que possa preservar pelo menos as aparências de
independência do \versal{MPF}.

A invasão do Senado pela Polícia Federal, para prender policiais será um
complicador a mais.

E fica"-se nesse dilema: como montar um modelo em que a \versal{PGR} nem seja uma
ameaça à estabilidade política, nem seja um joguete nas mãos do
Executivo.
