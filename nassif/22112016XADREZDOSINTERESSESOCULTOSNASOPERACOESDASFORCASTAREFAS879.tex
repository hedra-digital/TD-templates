\chapterspecial{22/\allowbreak{}11/\allowbreak{}2016 Xadrez dos interesses ocultos nas operações das Forças
Tarefas}{}{}
 

\section{A reportagem abaixo é fruto da pesquisa coletiva dos
comentaristas do \versal{GGN} ~}

\section{Peça 0 -- a título de prefácio}

Adapto o exemplo abaixo de um modelo de Júlio Sameiro, visando aplicar
os chamados princípios de Kant (\url{migre.me/\allowbreak{}vynAu)}

\emph{Um procurador descobre um malfeito de uma autoridade.}

\emph{Quais as três decisões possíveis?}

\emph{a)Não fez nada, porque a autoridade, além de muito influente, era
do seu time.}

\emph{b)Apurou rigorosamente o crime para ganhar reputação ou por outras
formas de interesse.}

\emph{c)Apurou o crime pelo fato de ser sua função apurar crimes.}

Das três opções, a única moralmente legítima é a C\,E é a menos
utilizada pelo Ministério Público Federal, desde que passou a disputar o
protagonismo político e a se envolver em pactos de sangue com a velha
mídia.

Na A, ele segue seus interesses pessoais, no caso ligados aos grupos de
interesses blindados. É~o caso da inação do Procurador Geral da
República (\versal{PGR}) e da Lava Jato com políticos do \versal{PSDB}.

Na B, ele apura rigorosamente, mas obedecendo a propósitos partidários
(eliminar o \versal{PT}), ou então para atender a interesses estratégicos da
corporação -- a ofensiva contra o presidente do Senado, Renan Calheiros,
dias depois de ele anunciar medidas contra os salários que ultrapassam
os tetos e votar o Projeto de Lei contra abusos de autoridade.

Apenas a alternativa C é moralmente legítima.

O fato é que o Estado de Exceção defendido por Luís Roberto Barroso, o
iluminista de Monsaraz, abre espaço para todo tipo de jogada. Cria"-se a
figura do inimigo e, por baixo da guerra santa, há guerras comerciais,
disputas partidárias, interesses corporativos e até confronto entre
organizações criminosas. Tudo escondido debaixo do manto da luta contra
a corrupção.

O caso Garotinho, até agora, é o exemplo mais flagrante do que acontece
quando se derrubam os limites legais à atuação dos agentes públicos.

\section{Peça 1 -- Garotinho é utilizado para diminuir o efeito"-Cabral}

A prisão com estardalhaço dos ex"-governadores do Rio de Janeiro Anthony
Garotinho e Sérgio Cabral são ingredientes típicos da Alternativa B\,
O Ministério Público Federal necessitava de um fato político de impacto
para pressionar senadores a acatar integralmente as tais 10 Medidas
contra a Corrupção, impedi"-los de votar o Projeto de Lei contra Abuso de
Autoridades e investigar os vencimentos acima do teto no serviço
público.

Soltam, então, a bomba Sérgio Cabral Filho, que vinha sendo maturada há
tempos.

Mas, aí, abririam um flanco contra a Globo.

O ex"-governador Garotinho ameaçava divulgar um dossiê contra autoridades
do Estado envolvidas no esquema de Cabral, assim que o adversário fosse
preso. Dentre elas, o ex"-futuro presidente do Tribunal de Justiça ~do
Rio de Janeiro, Luiz Zveiter, estreitamente ligado à Globo
(\url{migre.me/\allowbreak{}vyn\versal{PH}}).

Zveiter controla um amplo espectro de relações no \versal{TJRJ}. Ao lado da
esposa de Sérgio Cabral, Adriana Ancelmo, teve papel central para
convencer o governo Lula a indicar Luiz Fux para o \versal{STF} (Supremo Tribunal
Federal). Em retribuição, Fux tem procurado matar no peito as ações
contra Zveiter (\url{migre.me/\allowbreak{}vyn\versal{SB})}.

Dias atrás, uma não"-ação da presidente Carmen Lúcia, claramente alinhada
com a Globo, e do corregedor José Otávio de Noronha, do \versal{CNJ} (Conselho
Nacional de Justiça), livrou Zveiter de um Processo Administrativo
Disciplinar (\url{migre.me/\allowbreak{}vyn\versal{YA}})

Como impedir, então, que Garotinho levasse à frente seu intento e
denunciasse os esquemas em vigor na Justiça carioca, deixando em maus
lençóis seu parceiro, a Globo?

A maneira encontrada foi simultaneamente prender Garotinho para dividir
o foco das atenções e colocar mais gás na pressão sobre o Congresso.

\section{Peça 2 --- Os antecedentes}

Garotinho é acusado de crime eleitoral. Em nenhuma hipótese se
justificariam interceptações telefônicas e, menos ainda, prisão
preventiva. O~estardalhaço montado visou mais do que isso, o seu
assassinato político.

Não se pense que Garotinho seja figura ilibada.

No inquérito, é acusado de utilizar politicamente o programa Cheque
Cidadão nas últimas eleições. O~programa tinha 11 mil inscritos,
entrando após avaliação e estava fechado há tempos por limitações
financeiras. Segundo as denúncias, Garotinha teria inflado com mais 18
mil beneficiários, indicados pelos candidatos a vereador. E~teria
ameaçado testemunhas.

No despacho, o juiz Glaucenir Silva de Oliveira acusa Garotinho de se
valer dos meios de comunicação -- seu Blog e o jornal Diário, de Campos
-- ``para causar temor e insegurança jurídica perante os munícipes e
gerando também a descredibilidade da população nos ditames da lei e no
trabalho da Justiça Eleitoral''.

Recorre à estratégia da Lava Jato de desqualificar qualquer crítica como
obstáculo às investigações. Aliás, imita a Lava Jato até na menção ao
discurso de Roosevelt sobre a corrupção:~``Conforme discursou o
presidente Franklin Roosevelt, ao Congresso Americano no dia 7 de
dezembro de 1903, não existe crime mais sério que a corrupção''.

Naquele ano, Franklin Delano Roosevelt tinha 21 anos e nem pensava em
ser presidente. Quem disse isso foi Theodore Roosevelt.

\section{Peça 3 -- os abusos}

Como se pode conferir no despacho do juiz, Garotinho não é propriamente
flor que se cheire. Em episódios anteriores, inclusive, foi acusado de
se aliar ao delegado Paulo Cassiano -- seu atual verdugo -- para
manipular eleições da cidade de Santa João da Barra.

Mas o juiz, os procuradores e os delegados da Polícia Federal
atropelaram o Código Penal e as leis que definem os abusos de
autoridade. Montaram um espetáculo midiático indesculpável na prisão de
Garotinho e de Sérgio Cabral.

Armaram uma operação de guerra, avisaram a imprensa e promoveram um ato
de vingança, ordenando o envio de ambos para o Presídio de Bangu,
expondo"-os ao achincalhe da população. Cabral foi exibido com cabelo
raspado e roupa de presidiário.

Mas, como diz o Ministro Barroso, comendador da Ordem de Cosme Velho, o
momento exige mais exceção e menos direitos.

\subsection{A jovem advogada que é o que Barroso foi}

 

Aqui, uma pequena pausa para um momento de esperança.

Não partiu de Barroso, o iluminista do Projac,~ a condenação daquele
espetáculo circense. Mas de uma jovem advogada, Maria Eduarda Freire
Alves (\url{migre.me/\allowbreak{}vy9ni)}~com muito mais maturidade e respeito
pelos direitos, que comparou com as cenas do Coliseu romano e anunciou
que ela não compartilharia daquele espetáculo. Professor de centenas de
alunos, Barroso não conseguiu manter acesa a chama da democracia, que
jovens como Maria Eduarda carregam no peito.

\section{Peça 3 --- O problema cardíaco}

 

Toda a operação destinada a desumanizar Garotinho e transformá"-lo em
Inimigo Público Número Um esbarrou, contudo, em uma crise cardíaca não
prevista que acometeu Garotinho e o levou ao Hospital Souza Aguiar.

Mesmo assim, o juiz Glaucenir atropelou recomendações médicas e ordenou
que fosse literalmente arrastado para o hospital do presídio. Qual a
razão para tamanha radicalização?

Comportou"-se da mesma maneira impiedosa do juiz Ademar de Vasconcellos,
ao impedir que José Genoino, em crise cardíaca, após uma cirurgia de
alto risco, pudesse receber tratamento médico adequado
(\url{migre.me/\allowbreak{}vyf7j}). A~vida de Genoíno foi salva por uma moça
corajosa, supervisora dos presídios do Distrito Federal.

O episódio de Garotinho na maca, coberto por um lençol, lutando contra a
remoção, chocou até a opinião pública contrária a ele.

De repente, toda a construção jurídico"-midiática, de desumanização do
inimigo, de tirá"-lo da condição de indivíduo, com direitos, foi por
 água abaixo. Por falta de reflexo da cobertura, todos os holofotes
acabaram se concentrando na cena de Garotinho sendo arrastado para a
ambulância. Meramente porque não houve tempo para as chefias se darem
conta do estrago que a cena traria para a operação.

E aí entra em cena nem se diga a falta de humanidade, mas a falta de
esperteza desse espírito animalesco que rege os atos dos justiceiros: o
juiz Glaucenir tratou de ampliar a imagem de martírio ordenando que,
mesmo com crise cardíaca, Garotinho fosse transportado para Bangu.

De nada adiantaram os riscos à saúde e os riscos de vida -- como
ex"-Secretário de Segurança, Garotinho tem inúmeros adversários no
presídio, colocando sua vida em risco. Praticou"-se, ali, um ato de
vingança e, pensava"-se, um assassinato político, enquanto jornais como O
Globo e o Estadão celebravam a humilhação imposta a Garotinho.

No Estadão, fizeram um levantamento dos melhores memes criados em cima
da cena degradante de Garotinho sendo transportado para uma ambulância e
a família chorando
({migre.me/\allowbreak{}vygqk)}.
A~Rede Globo difundiu a imagem em seus telejornais. Deu"-se carne fresca
a uma opinião pública doente.

As imagens foram tão chocantes que, no lado saudável da opinião pública,
criou uma onda de indignação que superou a pós"-verdade da Globo. Menos
Luís Roberto Barroso, o renascentista da Globonews, para quem o direito
penal do inimigo é a maneira de transportar o país para o novo século.

\begin{center}~\end{center}


\section{Peça 4 --- a nova desconstrução}

Houve então a necessidade premente de nova desconstrução da imagem de
Garotinho, para retirar"-lhe o ar de vítima.

No plano familiar, o colunista Artur Xexeo -- há muitos anos
especializado em desmoralizar pessoas, especialmente mulheres, que ousem
criticar seu empregador -- cometeu um artigo covarde explorando o
desespero da filha de Garotinho
({migre.me/\allowbreak{}vyo8Y)}.~E
a força"-tarefa colocou em prática um conjunto de expedientes midiáticos,
que compõem uma espécie de manual tácito da contrainformação, para se
prevalecer do apoio da mídia.

Estratégia 1 -- amplie as suspeitas sobre o réu.

O juiz Glaucenir informa que recebeu uma proposta de suborno de
Garotinho e seu filho, entre R\$ 1,5 milhão e R\$ 5 milhões. No tribunal
utilizado para a denúncia -- a mídia -- não aparece nenhuma prova da
suposta tentativa. Nem se explica a razão do juiz ter mantido a
informação em sigilo por mais de um mês, nem o fato de não ter dado voz
de prisão ao subornador.

Estratégia 2 -- coloque sob suspeita qualquer autoridade que possa se
contrapor às arbitrariedades cometidas.

O juiz Glaucenir tratou de criminalizar a denúncia de Garotinho à
corregedoria da Polícia Federal e colocar sob suspeita o corregedor,
baseado em grampos, nos quais Garotinho meramente menciona que conversou
com o corregedor ((\url{migre.me/\allowbreak{}vyfiu})). Quem reforça as
suspeitas é o delegado Paulo Cassiano, sobre quem se falará mais abaixo.

Essa estratégia, aliás, foi aplicada pela Lava Jato, quando os delegados
-- chefiados por Igor Romário -- passaram a ser alvo de investigação
devido às denúncias de grampo na cela em que se encontrava o doleiro
Alberto Yousseff. Imediatamente procuradores saltaram em defesa dos
delegados, denunciando os colegas de terem pago o Estadão para a
publicação do dossiê dando conta das campanhas do grupo por Aécio Neves,
no Facebook.

Através do Fantástico, a Força"-Tarefa e a Globo fizeram o mesmo com a
Ministra Luciana Lossio, que ordenou a transferência de Garotinho para
prisão domiciliar. Matéria do Estadão tratou com suspeição o fato de
Garotinho manter contato com ela, para alertá"-la sobre os abusos
(\url{migre.me/\allowbreak{}vyfpw)}. Esse mesmo modelo de se valer de grampos
ou delações e criminalizar contatos entre inimigos e juízes foi
largamente utilizado pela Lava Jato para intimidar magistrados.

Nem isso sensibilizou o Ministro Barroso, a linha Maginot da democracia,
que foi colocado para correr com os primeiros ataques desqualificadores
que recebeu dos blogs de~Veja.

Estratégia 3 -- crie uma ameaça para fortalecer a imagem de heroísmo da
Força Tarefa.

O procurador que chefiava a operação, Sidney Madruga, solicitou medidas
de emergência contra ameaças que pairavam sobre o grupo
({migre.me/\allowbreak{}vyfsg)}.
A~manchete do Globo informava que ``prisão de Garotinho gera pedido de
segurança a promotores e juízes''. Vai"-se garimpar a informação e
fica"-se sabendo que um procurador recebeu um telefonema anônimo e não
sabia informar qual a motivação. Apenas isso.

No dia seguinte, confirma"-se a angioplastia em Garotinho, para a
implantação de stents.

\section{Peça 4 -- os personagens do caso Garotinho}

Mas a parte melhor da história é agora. Vamos conferir quem são os
personagens que mereceram cobertura total do Ministério Público Federal
e da Rede Globo.

O quebra"-cabeças será montado em torno de Tucuns, um mega"-escândalo de
15 anos atrás.

 

Personagem 1 -- o advogado Arakem Rosa.

Foi acusado de ter se apropriado e negociado uma área de reserva
ambiental na praia de Tucuns ((\url{migre.me/\allowbreak{}vygPt)}, no que foi
considerado o maior escândalo imobiliário de Búzios, um escândalo
graúdo, de disputa de terras, uma área de 5,6 milhões de m2, que acabou
promovendo remoção e punição de vários juízes e procuradores.

Personagem 2 -- Paulo César Barcelos Cassiano.

Pai do delegado Paulo Cassiano, principal algoz de Garotinho, Paulo
César foi nomeado interventor na Santa Casa de Misericórdia de Campos,
depois que o Tribunal de Justiça do Rio de Janeiro determinou o
afastamento do provedor Benedito Marques dos Santos Filho
(\url{migre.me/\allowbreak{}vyhKa}).

Personagem 3 -- Promotor Leandro Manhães

Na qualidade de promotor de Justiça de Tutela Coletiva de Campos, coube
a Leandro Manhães entrar com a ação cautelar que levou à intervenção na
Santa Casa (\url{migre.me/\allowbreak{}vyhTj)}, na gestão Rosinha Garotinho na
prefeitura. Leandro era um dos proprietários de terrenos no projeto
imobiliário de Araken Rosa.

Personagem 4 -- juiz Ralf Manhães

Intervém quando Rosinha, com base na opinião de outro procurador, ameaça
retomar a Santa Casa. ~Além de ameaçar os membros do \versal{MP}, Manhães ordena
à prefeitura que libere R\$ 3 milhões para a Santa Casa
(\url{migre.me/\allowbreak{}vyi5T}). Não se sabe o nível de parentesco com o
promotor Leandro Manhães.

Personagem 5 -- delegado Paulo Cassiano

Delegado da Polícia Federal, é um delegado polêmico. Evangélico,
exibicionista, em 2014 acusou a Universidade Federal de Santa Catarina
de ser ``um antro de maconheiros'' (\url{migre.me/\allowbreak{}vygxf)}.

Já mandou para a cadeia dois prefeitos do interior, em São Francisco de
Itabapoana e São João da Barra, ambas no Rio de Janeiro. Nos dois casos,
foi acusado de partidarismo político.

Em 2012, a poucos dias das eleições, chegou a propor a prisão preventiva
da prefeita Carla Machado, de São João da Barra, na Operação Machadada.
O~juiz Leonardo Antonelli indeferiu. No mesmo dia, Cassiano ordenou a
prisão em flagrante de Carla na véspera das eleições, claramente com a
intenção de influenciar o eleitorado.

Não apenas isso, como divulgou vídeos da prefeita
(\url{migre.me/\allowbreak{}vyiUu)}~a poucos dias das eleições.

Acusada de comprar votos, mais tarde a prefeita representou contra Paulo
Cassiano junto à corregedoria da Polícia Federal e conseguiu seu
afastamento da \versal{PF} de Campos.

Na ocasião, Carla Machado acusou"-o de trabalhar a serviço de Garotinho.
Na época da convenção do \versal{PMDB} local, Cassiano estacionou viaturas da
Polícia Federal em frente o almoxarifado da Secretaria da Saúde, sem
mandado judicial, criando um estardalhaço na cidade, segundo relatou
Carla em sua página no Facebook (\url{migre.me/\allowbreak{}vyiFt)}.

Tempos depois, um técnico de Campos confessou que tinha grampeado a
prefeita em conluio com o delegado Cassiano
(\url{migre.me/\allowbreak{}vyiRu}). Apesar da flagrante ilegalidade e de ter
atropelado diversos capítulos da lei que dispõe sobre crimes de abuso de
autoridades, nada de mais relevante aconteceu com Cassiano.

Nas últimas eleições, foi ~acusado por Garotinho de telefonar
pessoalmente para eleitores de Campos, pedindo votos para o candidato do
\versal{PPS} a prefeito, Rafael Diniz, eleito. Pelo WhatsApp comandou uma tal
``corrente do bem'' em favor de Diniz.

Personagem 6 -- Fabiana Rosa.

Filha de Arakem Rosa na época do escândalo em São João da Barra, era
Secretária da Saúde do município e foi acusada pelo delegado Cassiano
 de distribuir remédios com prazo de validade vencido.

Tempos depois, o pai de Cassiano, Paulo César Barcellos Cassiano, assume
a interventoria na Santa Casa de \versal{MI}serircórdia e leva Fabiana como
auditora. Em seguida, ela é nomeada Secretária da Saúde da gestão de
Rafael Diniz, o prefeito apoiado pelo delegado Cassiano.

Personagem 7 -- o juiz Glaucenir

Há um conjunto de fatos obscuros, descritos no post ``Os mistérios da
prisão de Garotinho'' (\url{migre.me/\allowbreak{}vyd\versal{PY})}.

Conforme \versal{GGN} já revelou, o juiz Glaucenir tem um histórico de
truculência. Há o caso da Guarda Municipal de trânsito que foi indiciada
por ele, após multa"-lo
({migre.me/\allowbreak{}vyg0U)}.

Antes disso, Glaucenir foi conduzido a uma delegacia em Vitória, acusado
de ter sacado a arma em uma boate, contra o namorado de uma moça que
teria sido incomodada por ele. Valeu"-se da posição de juiz para manter a
arma e o inquérito em sigilo (\url{migre.me/\allowbreak{}vydAw)}.

Assumiu há um mês o caso.

\section{Peça 5 -- as guerras de quadrilhas em um país sem lei}

Há muito ainda a investigar nessas operações.

Delegados, procuradores, juízes são ~agentes do Estado.~ A maneira de
controlar seus poderes é a obediência rigorosa às leis.

Quando o próprio~ Ministro Luís Barroso, plenipotenciário magistrado do
Jardim Botânico, defende o Estado de Exceção, significa o país abdicar
de qualquer forma de controle sobre os agentes públicos.

Dali para frente, tudo pode acontecer, especialmente quando se monta a
parceria com cartéis de mídia. Perde"-se o filtro, da mesma maneira que
as empresas quando passam a recorrer ao Caixa 2 e perdem o controle da
contabilidade oficial.

A tibieza do Procurador Geral Rodrigo Janot, a cumplicidade dos
Ministros do \versal{STF} com o arbítrio -- especialmente Carmen Lúcia e Luís
Roberto Barroso \mbox{---,} a irresponsabilidade da mídia instauraram um
faroeste em todo o país, do qual será muito difícil sair.
