\chapterspecial{05/\allowbreak{}10/\allowbreak{}2016 Xadrez do marco zero das esquerdas}{}{}
 

As eleições municipais simbolizam um marco zero para as esquerdas, de
fim da era de predominio ~absoluto do \versal{PT}.~

O partido nasceu moderno nos anos 80, como uma confluência de
coletivos.~

Nos anos 90 amoldou"-se para a luta política convencional, revendo
dogmas, aplainando radicalismos estéreis. Mas, para tanto, recorreu a um
centralismo que pouco a pouco foi~ inibindo o protagonismo dos
coletivos.~

Com a chegada ao poder, tentou manobrar as ferramentas de luta
institucional. Nesse período, perdeu quatro elementos centrais: José
Genoíno e Luiz Gushiken, tragados pela \versal{AP} 470; o ex"-Ministro Márcio
Thomaz Bastos; manteve ainda José Dirceu atuando como eminência parda,
mas sem dispor mais das ferramentas institucionais e afastado do centro
de poder pelo isolamento a que foi confinado pelo governo Dilma, crítica
de seus métodos.~

Finalmente, no governo Dilma Rousseff perdeu a identidade ideológica.

Assim, ocorreu uma quádrupla derrota nos campos político, institucional,
midiático e ideológico. A~esquerda terá que ser refundada.

As características dos novos tempos são as seguintes.

\section{Peça 1 --- fim do lulismo}

A legenda Lula compreendia um conjunto de símbolos pouco captados pelas
novas gerações: a criação da \versal{CUT}, os comícios da Vila Euclides, a
campanha de Collor, a campanha do impeachment.

Manteve"-se como o grande aglutinador dos grupos de esquerda e como o
maior símbolo da política brasileira do século. Mas tombou, vítima da
falta de lembrança das novas gerações, da campanha sistemática de
destruição pela aliança da Procuradoria Geral da República (\versal{PGR}), Lava
Jato e mídia. E~pelos erros tremendos de não ter entendido os aspectos
institucionais e midiáticos da guerra política.

Na verdade, o único líder do \versal{PT} com essa visão era José Dirceu.~
Explicitou demais o seu poder, atuou com excessiva desenvoltura em todas
as áreas, do Judiciário aos grandes grupos e terminou fuzilado por uma
armação: a tal~ ``teoria do fato'', magistralmente definida pela
Ministra Rosa Weber com seu célebre ``não tenho provas, mas a doutrina
me permite condená"-lo''.

Depois, foi alvo de todas as armações acusatórias possíveis e de uma
bala de prata real: suas relações com Milton Pascowitch.

O sonho de Lula 2018 está comprometido pelos resultados da campanha
política de desconstrução de sua imagem e pela continuidade do jogo
político escandaloso da Lava Jato, visando inabilitá"-lo juridicamente.

Continuará sendo figura referencial das esquerdas, a liderança capaz de
aglutinar os diversos setores. Mas os cenários possíveis para a esquerda
têm que começar a trabalhar com a hipótese concreta de não contar com
Lula em 2018.

\section{Peça 2 --- a mediocridade da direita}

Nos anos 80, ganhou popularidade uma velha piada sobre o inferno.~

O sujeito morre, vai para o inferno e precisa escolher entre três, o
inferno norte"-americano, o alemão ou o brasileiro. O~brasileiro, além de
incorporar todas as~ funcionalidades dos dois anteriores, ainda tem um
cardápio adicional de tortura, dentre as quais meia hora diária ouvindo
o José Serra falando sobre o perigo bolivariano no mundo.

O condenado se espantou:

\begin{itemize}
\itemsep1pt\parskip0pt\parsep0pt
\item
  Se o brasileiro é tão pior assim, porque está cheio e os dois outros
  vazios?
\item
  É porque no brasileiro nada funciona. O~Secretário de Governo desviou
  o carvão da fornalha, o demônio da Casa Civil montou uma concorrência
  fraudada e a cadeira do Dragão dá curto circuito, o Satanás só herdou
  as mesóclises de Jânio. E~o Serra nunca aparece porque dorme até tarde
  e passa o resto do dia tentando decorar siglas: Brics, \versal{NSA}, Mercosul,
  União Europeia… Brics, \versal{NSA}, Mercosul, União Europeia…
  Bracs, perdão, Brics, \versal{GSA}, perdão \versal{NSA}…
\end{itemize}

Piadas à parte, a direita brasileira não se mostra capaz de desenhar um
projeto minimamente articulado de país. Nos anos 90, embarcou na onda
Reagan"-Thatcher, que começava a dominar o mundo pós"-muro de Berlim.
Agora, nada tem, nem utopias globais às quais recorrer. Atualmente único
fator de aglutinação é atacar a velha esquerda e exalar toda forma de
preconceito. E~importar das ondas globais a intolerância mais
retrógrada.

Terá vida curta. Sua única saída será ampliar o Estado de Exceção. Mas
mesmo para isso teria que dispor de características~ morais e de
capacidade de desenhar o futuro. Só com mesóclises será insuficiente..

Portanto o novo tempo do jogo está próximo, de menos de uma década, com
novos atores que ainda estão em formação.

\section{Peça 3 --- a ampliação do estado de exceção}

Antes de ingressar no novo tempo político, há o enorme desafio de
enfrentar a maré do Estado de Exceção.

Quando comecei a apontar a participação do \versal{PGR} Rodrigo Janot no golpe,
procuradores bem intencionados preferiram se iludir com a presunção de
isenção. Nas suas entranhas, o processo jurídico é burocratizado e
lento. E~as regras de~\emph{accoutibility~}suficientemente vagas para
que os operadores do direito manobrem com prazos, com avaliações~
subjetivas sobre os inquéritos e, principalmente, com o uso seletivo dos
vazamentos.

O inquérito contra Aécio Neves dormiu por anos na gaveta do \versal{PGR}. O~julgamento do ``mensalão tucano'' foi atrasado por anos graças a um mero
``esquecimento'' do Ministro Ayres Brito.

Até hoje é impossível saber quais e quantos inquéritos repousam nas
gavetas da \versal{PGR} ou em pedidos de vista intermináveis do \versal{STF}.

Agora, o \versal{MPF} e a Polícia Federal se constituem na maior ameaça à
democracia. E~há razões de sobra para temer.

Qualquer avaliação sobre o avanço do Estado de Exceção tem que analisar
dinamicamente o que ocorre, levando em conta todos esses sinais.

O que Mirian Leitão fez foi uma mistificação histórica, ao comparar o
quadro politico atual com a ditadura pós"-\versal{AI}5, em plena maré de torturas,
para concluir que hoje em dia não existe regime de exceção.~

Para chegar a 1968, a ditadura passou por 1964, período no qual foram
plantadas as sementes da radicalização posterior --- principalmente
quando o regime entendeu que não tinha possibilidades eleitorais. E~surgiu porque avançou"-se dia a dia em medidas de exceção, criminalizando
os críticos, fossem comunistas ou liberais. E, na mídia, as Miriam
Leitão da época estimulavam a caça às bruxas recorrendo a um legalismo
de araque.

Se Mirian e outros colegas forem bem sucedidos em seu trabalho diário de
fomentar a caça às bruxas, é possível que dentro de pouco tempo possamos
chegar ao padrão \versal{AI}5.

O direito penal do inimigo está proliferando por todos os poros da
república, dos colunistas de jornais a procuradores da República,
diretores de escola expurgando ``comunistas'' e redações expurgando quem
ousar criticar o golpe.~

É um sentimento disseminado.

É possível que em um ponto qualquer do futuro erga"-se alguma onda de
resistência contra o arbítrio. No momento, não.

No \versal{CNMP} (Conselho Nacional do Ministério Público) as piores asneiras de
direita são perdoadas. Hoje em dia, no entanto, as ameaças do \versal{CNMP}
pairam sobre o pescoço do procurador que questionou a reforma
trabalhista em artigo, os bravos procuradores da República em São Paulo
que correram até a delegacia para defender jovens vítimas de
arbitrariedades policiais; a Procuradora Federal dos Direitos do
Cidadão, Deborah Duprat, por ter decidido filmar as passeatas para
coibir as truculências da Polícia Militar paulista.

Em quadro de normalidade democrática, de discernimento mínimo jurídico,
nem o mais obtuso procurador da República proibiria exposição de obras
de Paulo Freire na \versal{UFRJ} (Universidade Federal do Rio de Janeiro) ou
cartazes de ``fora Temer'' em colégios --- como fez ontem o procurador
da República no Rio de Janeiro, Fábio Moraes de Aragão.

Por tudo isso, o primeiro grande desafio será conter essa escalada da
violência, do dedurismo, que entrou por todos os poros da sociedade.

\section{Peça 4 --- a reconstrução de um modelo de esquerda}

Essa reconstrução passa não apenas pela recuperação dos valores centrais
--- inclusão social, políticas sociais, estado do bem estar social,
tolerância, defesa das minorias, defesa do meio ambiente etc. \mbox{---,} mas
por rediscussão sobre modelos de estado, instituições e mídia.

\subsection{\textbf{Sobre instituições}}

A maneira das corporações entrarem no jogo político foi atrás do
associativismo. Criado como uma instância de supervisão do Judiciário,
por exemplo, o \versal{CNJ} (Conselho Nacional de Justiça) só levanta dados para
justificar gastos do Judiciário. O~\versal{CNMP} (Conselho Nacional do Ministério
Público) é incapaz de ações mais severas contra abusos de procuradores.
Hoje em dia, o Judiciário já consome 1,8\% do \versal{PIB} e caminha para 2,1\%,
fato inédito em qualquer economia global.

Pior, há um desconhecimento amplo dessas corporações em relação a temas
políticos e econômicos, mais ainda em relação a projetos de país.

Nenhum governo será viável se não tiver o controle dessa agenda e se não
instituir uma~\emph{accountibility~}ampla nesses setores. Hoje em dia
não se tem acompanhamento nem sobre processos em tramitação no Supremo,
nem na Câmara, não se tem controle sobre a gaveta do \versal{PGR}.

Terá que se recuperar os princípios originais de independência do
Ministério Público e do Judiciário: para assegurar a independência de
julgamento do juiz e o trabalho independente do procurador. e não como
ferramenta de instrumentalização política de um poder de Estado.

\subsection{\textbf{Sobre gestão}}

Os governos Lula e Dilma revelaram grandes exemplos de boa gestão,
especialmente nos programas desenvolvidos por Fernando Haddad no \versal{MEC}
(Ministério da Educação), no Bolsa Família e no Brasil para Todos. E~alguns arremedos de gestão compartilhada no \versal{PAC} (Programa de Aceleração
do Crescimento).

Mesmo assim, não trataram de definir um padrão, a partir do acúmulo de
experiências bem sucedidas.

Mas, apesar dos avanços, ainda se deixou muito a desejar.~

\begin{enumerate}
\itemsep1pt\parskip0pt\parsep0pt
\item
  A institucionalização~ de políticas públicas passa por definir métodos
  claros de parceria com estados, municípios e terceiro setor. Só se
  consegue com a criação de protocolos de procedimentos para cada
  programa, facilmente assimiláveis na ponta, fiscalizáveis e
  reprodutíveis.~
\item
  Desenvolvimento de modelos de comunicação, para facilitar a
  compreensão da opinião pública sobre os benefícios dos projetos.~
\item
  Institucionalização de canais de participação da sociedade nas
  diversas instâncias de discussão das políticas públicas.
\end{enumerate}

\subsection{\textbf{Sobre mídia}}

Nenhuma democracia é compatível um mercado dominado por grandes grupos
de mídia atuando de forma cartelizada e com poder de fogo similar ao das
Organizações Globo.

Em todos os fóruns de direitos humanos, o direito à informação ganhou
status de direito fundamental, tão relevante quanto o direito à vida, à
alimentação e à saúde.

A Globo conseguiu seu maior feito politico ao ser a protagonista
principal de um golpe de Estado. Criou uma conta enorme a ser saldada em
um ponto qualquer do futuro. Desde então, se transformou no inimigo
número um de qualquer governo progressista que surja nas próximas
décadas.
