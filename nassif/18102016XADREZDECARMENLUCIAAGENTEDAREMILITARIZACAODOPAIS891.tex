\chapterspecial{18/\allowbreak{}10/\allowbreak{}2016 Xadrez de Carmen Lúcia, agente da remilitarização do país}{}{}
 

Há tempos venho juntando um conjunto de indícios que apontam para um
aumento da interferência militar nas políticas internas do país. O ápice
foi a atitude da nova presidente do \versal{STF} (Supremo Tribunal Federal),
Carmen Lúcia, de convocar as Forças Armadas (\versal{FFAA}s) para discutir
segurança interna.

A respeito do ``Xadrez das Vivandeiras dos Quarteis''
(\url{migre.me/\allowbreak{}vgKxM}), recebo o seguinte e"-mail do jornalista O,
que recebeu de L., brasileiro que mora na Argentina. Ambos pediram
sigilo de fonte.

\textbf{\emph{De L\,para O.}}

\begin{quote}
\emph{Prezado amigo,}

\emph{li o artigo de hoje do Nassif sobre a aproximação perigosa do
governo golpista com os militares. Por uma feliz coincidência, o Página
12 publicou hoje uma matéria intitulada ``Jugar con el fuego'' na qual
faz um resumo de um painel ocorrido recentemente aqui em Buenos Aires.}

\emph{O~evento reuniu acadêmicos e jornalistas para apresentar o
documento ``La riesgosa política del gobierno para las Fuerzas
Armadas''.}

\emph{O~texto foi publicado pelo Centro de Estudios Legales y Sociales
(\versal{CELS}), conhecida organização de direitos humanos aqui na Argentina, e
trata, adivinha do quê?, de uma paulada na clara intenção do governo
Macri de acabar com a saudável e histórica distinção entre Defesa e
Segurança Interna.}

\emph{Os argentinos conhecem bem a merda que é, para o conjunto da
sociedade, colocar as Forças Armadas para fazer papel de polícia. Com a
redemocratização, eles conseguiram separar bem os papéis institucionais
dos militares e da polícia, estabelecendo claramente, do ponto de vista
legal e administrativo, os limites das Forças Armadas.}

\emph{Agora, estão muito receosos das intenções do governo Macri de
acabar com essa conquista histórica. E, claro, com o que isso pode
representar: uma porta escancarada para a repressão política, para a
violação dos direitos individuais, sobretudo das populações menos
favorecidas.}

\emph{Veja você que, por razões semelhantes, o que se vive aqui na
Argentina é parecido com o que está ocorrendo no Brasil. A~diferença é
que aqui há mais vozes que gritam.}

\emph{No Brasil, há pouquíssimas. Nassif é um dos únicos a chamar a
atenção para isso. De modo que te envio abaixo os links para a matéria e
para o documento do \versal{CELS}. Acho que o Nassif vai adorar receber os dois.}
\end{quote}

 

Vamos a um breve apanhado do que ocorre na Argentina, para avaliar
melhor o papel da Ministra Carmen Lúcia.

A reportagem do Página 12 é sobre encontro realizado na Faculdade
Latino"-Americana de Ciências Sociais (\versal{FLACSO}) sobre o tema. No
encerramento do painel, Horácio Verbitsky, presidente do Centro de
Estudos Jurídicos e Sociais (\versal{CELS})~ denunciou documento do governo
Macri, procurando militarizar a polícia de segurança. Os demais
especialistas presentes alertaram para o ``enfraquecimento do princípio
da demarcação entre os conceitos de segurança e defesa''.

Os argumentos invocados por Macri para envolver as Forças Armadas nas
questões internas foram o da prevenção do terrorismo e do tráfico de
drogas e a contenção da agitação social e dos protestos.

Vamos ver como fica o nosso Xadrez com essas novas movimentações.

\section{Peça 1 -- a desmilitarização na Argentina}

A Argentina foi vítima de uma ditadura militar trágica, que deixou
milhares de vítimas e protagonizou a guerra das Malvinas e a derrota
ampla para a Inglaterra.

De lá para cá, houve um gradativo processo de reinstitucionalização das
Forças Armadas promovido desde 1983 pelo presidente Raul Alfonsin.

O principal ponto acertado foi o da clara demarcação entre os conceitos
de segurança e de defesa. Segurança é trabalho para a polícia; defesa,
para as Forças Armadas. Segurança trata de crimes; defesa trata de
inimigos externos.

A estratégia argentina consistiu, de um lado, na subordinação
constitucional dos militares ao poder civil. Foram punidos os crimes
contra a humanidade ocorridos na ditadura e desmilitarizados todos os
cargos do Ministério da Defesa.

\section{Peça 2 -- porque não envolver \versal{FFAA} com repressão interna}

O trabalho apresentado no seminário lista inúmeros argumentos para não
se envolver as \versal{FFAA}s com repressão interna.

A primeira razão é que as Forças Armadas atuam dentro do conceito de
guerra, na qual a lógica é do extermínio do inimigo. Os militares não
estão treinados para o uso gradual das forças, o que explica por que
suas intervenções sempre têm nível maior de letalidade.

A formação do militar é mais demorada que a do policial. Por isso mesmo,
não se resolve a adaptação com re"-treinamento ou com mudança de
equipamentos, diz o estudo.

Em 2006, o governo do México envolveu as Forcas Armadas na luta contra o
crime organizado.~ Segundo informe de janeiro de 2013 da Comissão
Nacional de Direitos Humanos (\versal{CNDH}) as denúncias por torturas,
assassinatos e desaparecimento de presos aumentaram em 1.000\% no
período de seis anos. E~não houve nenhuma redução no tráfico.

O mesmo ocorreu na Colômbia, com seguidas denúncias de violação de
direitos humanos chegando à prática de assassinatos como indicador de
eficácia.

Em vários países as próprias Forças Armadas se deram conta dos riscos de
entrarem na luta contra o narcotráfico.

Os riscos para as Forças Armadas estão no envolvimento de oficiais e
soldados com a corrupção. No México, as Forças Armadas foram infiltradas
por redes criminosas que acabaram controlando boa parte de sua estrutura
e adotando suas técnicas nas disputas com outras quadrilhas. O~mesmo
ocorreu na Colômbia, com integrantes das Forças Armadas envolvidas em
redes de narcotráficos e de armamentos.

\section{Peça 3 -- o Comando Sul e a luta contra o narcoterrorismo}

No seminário foram levantados um acerto e um erro da política argentina
pós"-ditadura. O~acerto foi a subordinação dos militares às lideranças
políticas, inclusive com a punição dos que foram responsáveis por crimes
contra a humanidade.

O erro foi não ter reconhecido a fundo o novo papel das forças armadas
como última linha de defesa nacional contra a agressão externa.

Ao permitir o financiamento externo através do Comando Sul --- uma tropa
multinacional coordenada pelo Departamento de Operações de Manutenção da
Paz das Nações Unidas \mbox{---,} mas na prática comandada pelo Departamento de
Defesa dos Estados Unidos
(\url{migre.me/\allowbreak{}vgGi3}~e~\url{migre.me/\allowbreak{}vgGiD}).

Originalmente, o Comando Sul foi criado para combater o narcotráfico.
Mas, segundo denunciou Horácio Verbitsky, já incluiu catástrofes
naturais e indigenismo como temas de preocupação.
(\url{migre.me/\allowbreak{}vgVn5)}.

Sediado em Miami, Flórida, o Comando Sul dos Estados Unidos é uma
organização militar regional unificada, ligada ao Departamento de Defesa
dos \versal{EUA}. Seu papel é o de organizar a cooperação com forças de segurança
da América do Sul, Central e Caribe, somando mais de 30 países da
região. É~comandado por um general 4 estrelas e organizado em
diretorias, comandos e forças tarefas militares.

Em 2011, em entrevista à Folha (\url{migre.me/\allowbreak{}vg\versal{GG}h}), o
ex"-embaixador do Brasil nos \versal{EUA} no governo \versal{FHC}, Rubens Barbosa, já
denunciava as interferências indevidas do Comando Sul nos assuntos
internos dos países. Acusava"-o de alimentar a imprensa com boatos sobre
terrorismo na Tríplice Fronteira (Brasil, Argentina e Paraguai), afim de
valorizar sua atuação. Sediado em Miami, o comando treinava militares
paraguaios, acenando com a ameaça dos ``brasiguaios''.

O chefe da Força, general James Hill, equiparava as drogas a armas de
destruição em massa e defendia o fim das restrições legais à
interferiria dos militares em assuntos internos.

Na Argentina, diz o trabalho, nenhum estudo sério comprovou que o
narcotráfico é o maior problema para a segurança interna. Mas é o álibi
para a ampliação das intervenções das Forças Armadas e da militarização
da estratégia de intervenção policial.

Nenhuma coincidência, portanto, nas atuações recentes do Ministro da
Justiça, Alexandre de Moraes, montando um carnaval em torno da suposta
rede terrorista brasileira, e da nova presidente do \versal{STF}, Carmen Lúcia,
convocando as \versal{FFAA}s para discutir a segurança interna.

Continuando nesse ritmo, em breve será uma ameaça à democracia
brasileira maior que o próprio Gilmar Mendes. Gilmar é partidário,
vale"-se de todos os instrumentos legais em defesa dos seus, mas tem
conhecimento suficiente sobre os riscos do excesso de poder de
corporações do Estado, do Ministério Público Federal às Forças Armadas.
Carmen Lúcia parece ser uma completa sem"-noção.

Aliás, o assessor de imprensa de Alexandre Moraes é um militar. E~o
chefe de gabinete do Ministro"-Chefe da Casa Civil, Eliseu Padilha, é um
general.

\section{Peça 4 -- os sinais da militarização na Argentina}

A lógica da Argentina de Macri é similar àquela desenhada nas primeiras
medidas do Brasil de Michel Temer. Trata"-se de envolver as Forças
Armadas nas disputas internas, a pretexto de combater o narcotráfico, o
terrorismo e as agitações populares. Dali para a repressão política
seria um pulo.

O seminário anotou os seguintes indícios:

\begin{itemize}
\itemsep1pt\parskip0pt\parsep0pt
\item
  ·~~~~~~ Em 19 de janeiro de 2016, através do Decreto 228/\allowbreak{}16, o governo
  declarou estado de emergência na segurança pública. Definiu um
  protocolo permitindo às Forças Armadas derrubarem aeronaves
  ``hostis'', algo não previsto em nenhuma norma de direito
  internacional. E~abriu a possibilidade de enfrentar ``novas ameaças'',
  caminho aberto para que seja envolvida no combate ao narcotráfico. O
  governo Temer vem ensaiando medidas nessa direção, agora com o apoio
  de Carmen Lúcia.
\item
  ·~~~~~~ Operação Fronteira, através do decreto 152/\allowbreak{}16, permitindo
  recursos militares tecnológicos e humanos para a Operação Escudo
  Norte, que atua nas fronteiras. No Sistema de Defesa Nacional não há
  nenhum tipo de atividade de fronteira que se enquadre em ameaça à
  integridade nacional. O~novo Plano Nacional de Segurança,
  semi"-divulgado por Alexandre de Moraes avança nessa direção.
\item
  ·~~~~~~ Instruções ao Ministério da Defesa para recolher informações
  sobre narcotráfico e terrorismo nos seus países de origem. No Brasil,
  o \versal{GSI} (Gabinete de Segurança Institucional), ligado à presidência da
  República, já articular os serviços de informações das Forças Armadas
  no monitoramento de movimentos sociais.
\item
  ·~~~~~~ Designação de militares como funcionários do Ministério de
  Defesa. Desde Raul Alfonsin vinha"-se desmilitarizando cada vez mais os
  cargos no Ministério da Defesa, entregando"-os a civis. Nos últimos
  meses os militares voltaram a ocupar o Ministério da Defesa. No
  Brasil, esse movimento foi iniciado por Aldo Rebelo, ainda como
  Ministro da Defesa do governo Dilma Rousseff. E~está sendo
  radicalmente ampliado pelo novo Ministro, Raul Jungman.
\end{itemize}

\section{Peça 5 -- a caixa de Pandora da remilitarização}

A ideia básica desse modelo é ajudar a fortalecer governos de direita,
contra movimentos populares e partidos de esquerda. Imagina"-se que
conferindo uma missão específica às Forças Armadas -- a luta contra o
narcotráfico e os ``subversivos'' -- ela vá se ater a esses campos,
sendo comandada por políticos para lá de suspeitos.

Com a desmoralização crescente do poder civil, o resultado óbvio será o
de, em algum ponto do futuro, as \versal{FFAA}s abolirem os intermediários.
