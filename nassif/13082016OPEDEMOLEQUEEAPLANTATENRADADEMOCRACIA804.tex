\chapterspecial{13/\allowbreak{}08/\allowbreak{}2016 O pé"-de"-moleque e a planta tenra da democracia}{}{}
 

O escritor O\,Henry tinha um conto sobre uma menina que precisa de toda
energia interna para superar uma grave doença no inverno rigoroso dos
Estados Unidos. Da sua janela, ela passa a acompanhar uma folha, de uma
trepadeira do muro em frente seu quarto.

 Todo dia ela abria a janela para conferir se a folha resistia. Outras
folhas caiam, mas a folha companheira resistia. E~resistiu até a menina
se curar.

Curada, foi até o muro e descobriu que alguém da família, percebendo a
simbiose emocional, simplesmente havia colado a folha no muro.

Vendo a crise econômica surgir, a democracia desmoronar, minha folha de
inverno é um pé"-de"-moleque na padaria perto de casa. Diariamente saio de
casa, passo na padaria, compro dois pés"-de"-moleque e dou minha caminhada
diária.

Em 2010 enfrentei uma audiência hostil, em uma palestra em Fortaleza
para um fabricante de produtos agrícolas. Coronéis vociferavam contra o
Bolsa Família. Calei"-os recorrendo ao pé"-de"-moleque: mensalmente, dois
pés"-de"-moleque por dia útil era mais do que o valor de uma pessoa
acolhida pelo Bolsa Família.

Através do pé"-de"-moleque, acompanhei dia a dia a vida do dono da
fábrica, mesmo sem saber quem era. Era uma empresa pequena, sim. Mas o
produto era gostoso, a embalagem caprichada. No início, ele tinha
dificuldades em definir o padrão. Às vezes acertava em cheio, às vezes
não. Com o tempo, conseguiu manter o mesmo padrão. E~eu acompanhando seu
aprendizado.

Quando a crise começou, o pé"-de"-moleque sumiu da prateleira perto do
caixa. O~caixa me indicou outro, de marca conhecida. Não era a mesma
coisa.

Fiquei imaginando a saga do empreendedor. Provavelmente uma pequena
empresa que decidiu dar um passo maior. Juntou dinheiro, conseguiu um
financiamento bancário, no curtíssimo período em que o crédito barateou,
adquiriu uma máquina, depois outra, montou um departamento comercial
perrengue e partiu para a luta, colocando o pé"-de"-moleque como quem
coloca um filho no mundo.

Lembrei"-me de meu pai em Poços de Caldas, com sua Farmácia Central Salva
Sempre, ou de meu tio Léo, com o Doces Mesquita, Coma e Repita, que
nasceu do talento de minhas tias Rosita e Marta de reeditar o
doce"-de"-leite argentino.

De repente, a crise surgiu como um ogro faminto, avançando sobre os
pequenos, arrebentando os sonhos de crescer, os credores avançando como
harpias ensandecidas.

Creio que foi essa loucura, fruto do plano de estabilização de Roberto
Campos e Bulhões, que me fez mais tarde enveredar pela economia. Queria
entender esse monstro que aparecia no horizonte durante as crises de
estabilização, que eliminava empresas, empregos, arrebentava com a
tranquilidade familiar.

O monstro que se abateu sobre a farmácia era filho direto de Campos.
Comprava"-se o remédio e se proibia de reajustar os preços, em um quadro
inflacionário. Quando o farmacêutico ia repor o capital de giro, a
indústria já havia aumentado de novo seus preços. A~cada renovação do
estoque, realizava"-se o prejuízo. ~Com giro maior, e adquirindo direto
nas fábricas, os grandes varejistas nadavam de braçada. Os grandes
tinham acesso a Campos; os boticários do interior, não.

Na época cheguei a compor uma peça com um título tirado de Carlos
Drummond de Andrade, o ``Congresso Internacional do Medo'', que começava
com um arremedo da Cavalgada das Valquírias, e as palavras rituais,
``juro, lucro, multa, mora, déficit /\allowbreak{} ágio, avo, quota, conta, prejuízo
/\allowbreak{} muralhas de Jericó, trombetas de Josué /\allowbreak{} o imprevisível''.

Agora, indo diariamente na padaria, fico conferindo o pé"-de"-moleque com
a mesma ansiedade da menina tísica apreciando a folha, com a mesma
ansiedade do adolescente testemunhando a agonia diária do velho.

Dias depois, voltou o pé"-de"-moleque original. Imaginei que o dono
tivesse se encalacrado com falta de capital de giro, fruto do desastroso
plano Levy. Ou então, tivesse voltado com outra razão social, para se
livrar do fisco. Era a mesma marca, a mesma razão social. Provavelmente
vendeu algum bem para recompor o giro.

De lá para cá, o pé"-de"-moleque aparece, some por algum tempo, mas volta,
persistente como a folha no muro, valente como a menina na cama.

Fico imaginando quanto tempo resistirá. E~fico imaginando a democracia
brasileira.

Desde a Proclamação, o país conviveu historicamente com golpes. Mas
havia explicações históricas, sociológicas. Floriano endureceu para
consolidar a República. Vargas veio no bojo da Aliança Nacional, uma
mobilização do país. Mesmo 1964 foi fruto de uma corporação militar
aliada com setores políticos. Mas, hoje? Hoje há o episódio inédito do
país ter sido entregue a um grupo que, se não estivesse com o poder,
estaria metido em inquéritos e processos, alguns deles mofando na
prisão.

É inédito, é humilhante.

Fico imaginando em qual folha me agarrar, enquanto persistir a escuridão
sem fim.
