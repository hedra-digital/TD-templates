\chapterspecial{22/\allowbreak{}08/\allowbreak{}2016 Xadrez de Toffoli e o fruto da árvore envenenada}{}{}
 

Entramos em um dos mais interessantes quebra"-cabeças da Lava Jato: a
operação fruto da árvore envenenada, possivelmente montada para livrar
Aécio Neves e José Serra das delações da \versal{OAS}. Trata"-se do vazamento
parcial da delação do presidente da \versal{OAS} Léo Pinheiro, implicando o
Ministro Dias Toffoli, do Supremo Tribunal Federal.

\section{Peça 1 -- o teor explosivo das delações}

Já circularam informações de que as delações da \versal{OAS} serão fulminantes
contra José Serra e Aécio Neves. Até um blog estreitamente ligado a
Serra -- e aos operadores da Lava Jato -- noticiou o fato.

Em muitas operações bombásticas, pré"-Lava Jato, os acusados valiam"-se do
chamado ``fruto da árvore envenenada'' para anular inquéritos e
processos. A~Justiça considera que se o inquérito contiver uma peça
qualquer, fruto de uma ação ilegal, todo o processo será anulado. Foi
assim com a Castelo de Areia. E~foi assim com a Satiagraha.

Na Castelo de Areia, foi uma suposta delação anônima. Na Satiagraha, o
fato dos investigadores terem pedido autorização para invadir um andar
do Opportunity e terem estendido as investigações a outro.

\section{Peça 2 -- os truques para suspender investigações.}

Vamos a um arsenal de factoides criados para gerar fatos políticos ou
interromper investigações:

Grampo sem áudio~-- em plena operação Satiagraha, aparece um grampo de
conversa entre Gilmar Mendes, Ministro do Supremo, e Demóstenes Torres,
senador do \versal{DEM}. Era um grampo às avessas, no qual o conteúdo gravado era
a favor dos grampeados. Jamais se comprovou a autoria do grampo. Mesmo
assim, com o alarido criado Gilmar conseguiu o afastamento de Paulo
Lacerda, diretor da \versal{ABIN} (Agência Brasileira de inteligência). Veículo
que abrigou o factoide: revista~Veja.

O falso grampo no \versal{STF}~-- Recuperada a ofensiva, matéria bombástica
denunciando um sistema de escutas no \versal{STF}. Era um relatório da segurança
do \versal{STF}, na época presidido por Gilmar Mendes. Com o alarido, cria"-se uma
\versal{CPI} do Grampo, destinada a acuar ainda mais a Polícia Federal e a \versal{ABIN}.
Revelado o conteúdo do relatório, percebeu"-se tratar de mais um
factoide. Veículo que divulgou o falso positivo: revista~Veja.

O falso pedido de Lula~-- em pleno carnaval da \versal{AP} 470, Gilmar cria uma
versão de um encontro com Lula, na qual o ex"-presidente teria
intercedido pelos réus do mensalão. O~alarido em torno da falsa denúncia
sensibiliza o Ministro Celso de Mello, o decano do \versal{STF}, e é fatal para
consolidar a posição dos Ministros pró"-condenação. Depois, a única
testemunha do encontro, ex"-Ministro Nelson Jobim, nega veementemente a
versão de Gilmar. Veículo que disseminou a versão: revistaVeja.

O caso Lunnus~-- o grampo colocado no escritório político de Roseane
Sarney, que inviabilizou sua candidatura à presidência. Caso mais
antigo, na época ainda não havia sinais da aproximação de Serra com
a~Veja.

O suborno de R\$ 3 mil~-- o caso dos Correios, um suborno de R\$ 3 mil
que ajudou a deflagrar o ``mensalão''. Veículo que divulgou:~Veja.
Fonte: Carlinhos Cachoeira, conforme apurado na \versal{CPI} dos Correios.

\section{Peça 3~-- a fábrica de dossiês}

Com base nesses episódios, procurei mapear os pontos em comum entre os
mais célebres dossiês divulgados pela mídia.

Confira:

Fato 1~-- na Saúde, através da \versal{FUNASA} o então Ministro José Serra
contrata a \versal{FENCE}, empresa especializada em grampo, o delegado da Polícia
Federal Marcelo Itagiba e o procurador da República José Roberto
Figueiredo Santoro.

Fato 2~-- em fato divulgado inclusive pelo Jornal Nacional, Santoro
tenta cooptar Carlinhos Cachoeira, logo após o episódio Valdomiro Diniz.

Fato 3~-- Cachoeira tem dois homens"-chave. Um deles, o araponga Jairo
Martins, seu principal assessor para casos de arapongagem. O~segundo, o
ex"-senador Demóstenes Torres, seu principal agente para o jogo político.
Ambos têm estreita ligação com o Ministro Gilmar Mendes: Demóstenes na
condição de amigo, Jairo na condição de assessor especialmente
contratado por Gilmar para assessorá"-lo.

Fato 4~-- todos os principais personagens do organograma -- Serra,
Gilmar, Cachoeira, Demóstenes e Jairo -- mantiveram estreita relação com
a~Veja, como fontes, como personagens de armações ou como fornecedores
de dossiês.

Não se tratava de meros dossiês para disputas comerciais, mas episódios
que mexeram diretamente com a República. O~organograma acima não é prova
cabal da existência de uma organização especializada em dossiês para a
imprensa. São apenas indícios.

\section{Peça 4 -- a denúncia contra Toffoli.}

Alguns fatos chamam a atenção na edição da~Veja.

Fato 1~-- já era conhecido o impacto das delações de Léo Pinheiro sobre
Serra e Aécio (\url{migre.me/\allowbreak{}u\versal{JK}sj)}. Tendo acesso à delação mais
aguardada do momento, a revista abre mão de denúncias explosivas contra
Serra e Aécio por uma anódina, contra Toffoli.

Fato 2~-- a matéria de Veja se autodestrói em 30 segundos. Além de não
revelar nenhum fato criminoso de Toffoli, a própria revista o absolve ao
admitir que os fatos narrados nada significam. Na mesma edição há uma
crítica inédita ao chanceler José Serra, pelo episódio da tentativa de
compra do voto do Uruguai. É~conhecida a aliança histórica de Veja
com~Serra. A~reportagem em questão poderia ser um sinal de independência
adquirida. Ou poderia ser despiste.

\section{Peça 5 -- a posição do \versal{STF} e do \versal{PGR}}

Um dos pontos defendidos de maneira mais acerba pelo Ministério Público
Federal, no tal decálogo contra a corrupção, é a relativização do
chamado fruto da árvore envenenada. Querem -- acertadamente -- que
episódios irregulares menores não comprometam as investigações como um
todo.

Se a intenção dos vazadores foi comprometer a delação, agiram com
maestria.

Sem comprometer Toffoli, o vazamento estimula o sentimento de corpo do
Supremo, pela injustiça cometida contra um dos seus. Ao mesmo tempo,
infunde temor nos Ministros, já que qualquer um poderia ser alvo de
baixaria similar.

Tome"-se o caso Gilmar Mendes. Do Supremo para fora até agora, não houve
nenhum pronunciamento público do Ministro, especializado em explosões de
indignação quando um dos seus é atingido. E~do Supremo para dentro?
Estaria exigindo providências drásticas contra o vazamento, anulação da
delação? Vamos aguardar os fatos acontecerem. Mas certamente, Gilmar
ganha um enorme poder de fogo para fazer valer suas teses que têm
impedido o avanço das investigações contra Aécio Neves.

A incógnita é o \versal{PGR} Rodrigo Janot. Até agora fez vistas largas para
todos os vazamentos da operação mais vazada da história. E~agora?

Se ele insiste na anulação da delação, a Hipótese 1 é que está aliado a
Gilmar na obstrução das investigações contra Aécio e Serra. A~Hipótese 2
é que está intimidado, depois do tiro de festim no pedido das prisões de
Renan, Sarney e Jucá. A~Hipótese 3 é que estaria seguindo a lei. Mas
esta hipótese é anulada pelo fato de até agora não ter sido tomada
nenhuma providência contra o oceano de vazamentos da Lava Jato.

De qualquer modo, trata"-se de um ponto de não retorno, que ou consagra a
\versal{PGR} e o Ministério Público Federal, ou o desmoraliza definitivamente.

Afinal, quem toca a Lava Jato é uma força tarefa que, nas eleições
presidenciais, fez campanha entusiasmada em favor do candidato Aécio
Neves. Bastaria um delegado ligado a Serra e Aécio vazar uma informação
anódina contra um Ministro do \versal{STF} para anular uma delação decisiva.
Desde que o \versal{PGR} aceitasse o jogo, obviamente.

Será curioso apreciar a pregação dos apóstolos das dez medidas, se se
consumar a anulação da delação.

\versal{PS}1 --- A alegação dos procuradores, de que o vazamento teria partido
dos advogados de Léo Pinheiro, visando forçar a aceitação da delação não
têm o menor sentido. Para a delação ser aceita, os advogados adotariam
uma medida que, na práitca, anula a delação? Contem outra.

\versal{PS} 2 --- N semana passada o procurador Carlos Fernando dos Santos lima
já mostrava deconforto com a delação da \versal{OAS}, ao afirmar que a Lava Jato
só aceitaria uma delação a mais de empreiteiras. Não fazia sentido. A~delação depende do conteúdo a ser oferecido. O~próprio juiz Sérgio Moro
ordenou a suspensão do processo, sabe"-se lá por que. E~nem havia ainda o
álibi do vazamento irrelevante.~
