\chapterspecial{16/\allowbreak{}08/\allowbreak{}2016 Xadrez da esperteza do mercado e o fator Temer}{}{}
 

\section{Peça 1 --- a arte de prever o pensamento médio do mercado.}

O mercado trabalha sob a ótica do investidor racional, o que toma suas
decisões baseado em análises e na lógica. Não é assim. No mercado, há um
efeito manada similar ao que acomete o público comum em relação a
eventos de repercussão.

Há vários tipos de análises no mercado.

A mais consistente é a chamada análise fundamentalista --- que se
debruça sobre os fundamentos da economia e da empresa analisada. A~boa
análise permite ao analista prever como a economia estará daqui a algum
tempo. Mas não é suficiente para prever o chamado pensamento médio do
mercado. É~o pensamento médio que, em última instância, define o valor
da ação.

Os ganhos são obtidos por quem consegue prever o pensamento médio do
mercado -- ou seja, a irracionalidade dos movimentos de manada -- em
relação aos fatos concretos que ocorrerão na economia.

\section{Peça 2 -- como operam os profissionais}

Hoje em dia, as Bolsas estão subindo devido a uma nova onda de liquidez
internacional, que inclusive ameaça formar novas bolhas globais. Os três
índices da Bolsa de Nova York -- Dow Jones, S\&P 500 e Nasdaq --
alcançaram máximas históricas, algo que não acontecia desde 1999.

Aí analistas profissionais, como Luiz Carlos Mendonça de Barros, assumem
posições compradas -- que trazem ganhos ao investidor sempre que a bolsa
sobe. Ele sabe que o governo Temer é instável, que está longe de se
consolidar. Sabe, inclusive, que há uma probabilidade cada dia menos
remota do próprio \versal{TSE} (Tribunal Superior Eleitoral) caçar a chapa
completa de Dilma, levando Temer junto.

Mas o que interessa é o movimento de curto prazo da Bolsa até a votação
do impeachment. Ele passa, então, a atribuir a alta da Bolsa à suposta
consolidação do governo Temer (\url{migre.me/\allowbreak{}uG6Ds}). Esse
discurso é repetido por outros profissionais de mercado, criando o
efeito manada.

A Bolsa continuará subindo até o dia da votação do impeachment. No dia
seguinte, o país continuará o mesmo, não haverá mais dúvidas sobre a
permanência de Temer no curto prazo, mas mesmo assim a Bolsa despencará
porque Mendonça e outros operadores profissionais, começarão a vender
suas posições, realizando lucros. E~quem ficará com o mico são os
investidores menos informados, que acreditaram que a alta se devia a
fatores internos e ao enorme brilho do governo Temer.

Portanto, todas essas análises sobre o mercado subordinam"-se a visões de
curtíssimo, curto, algumas vezes médio prazo. E~os movimentos de manada,
quase sempre, são montados em cima de visões parcialíssimas da economia.

Peça 3 --- a análise fundamentalista

A rigor, o mercado é afetado por fatores externos e internos. A~análise
fundamentalista é aquela que tenta analisar as empresas de forma
sistêmica, com todos os fatores que poderão impactar seus resultados. E~mesmo ela, com toda sua racionalidade, dificilmente autoriza conclusões
taxativas sobre as cotações -- porque estas dependem do pensamento
médio, que não se move racionalmente.

Veja no quadro abaixo, com um pequeno resumo dos fatores que impactam a
Petrobras.

 

\subsection{Lado externo}

O principal fator é o preço do petróleo. Sobe o petróleo, sobe a ação da
Petrobras e vice"-versa.

As cotações do petróleo são afetadas pela oferta, demanda e pela
liquidez internacional.

Do lado da oferta, há inúmeros fatores geopolíticos presentes. Uma crise
fiscal na Rússia, um conflito no Irã, uma crise na Venezuela, são
fatores que impactam diretamente a oferta do produto.

Do lado da demanda, o desempenho econômico da China e dos países da
União Europeia.

Do lado da liquidez, a situação das taxas de juros internacionais e das
injeções de liquidez dos Bancos Centrais que, injetando muito dinheiro
na economia global, promovem movimentos especulativos com commodities.

Todos esses fatores estão entrelaçados. Um acirramento de ânimos entre
\versal{EUA} e Rússia, por exemplo, afetará a oferta de petróleo, aumentará as
avaliações de risco e, por consequência, a liquidez internacional.

Tem"-se hoje em dia um quadro internacional bastante similar ao
brasileiro, sem a presença de figuras referenciais capazes de apontar
rumos ou promover pactos. Cria"-se um campo extremamente volátil para
previsões. O~caso Brexit é sintomático da imprevisibilidade dos modelos.

Lado interno

A cotação final da Petrobras é dada pelos seus resultados de balanço e
pelos ecos do mercado de ações em geral.

Os resultados próprios são afetados pelo aumento da produção, cotação
dos derivados, redução do passivo. Nem sempre o que é bom para o
acionista é bom para a empresa. O~acionista se move exclusivamente pelos
resultados de curto prazo enquanto a empresa tem que se planejar no
longo prazo.

Os movimentos do mercado de ações, por sua vez, se refletem
automaticamente nas cotações das empresas. O~mercado é afetado por
eventos internacionais (a maior ou menor liquidez internacional) e por
eventos internos.

Grande parte do mercado atua de forma binária. São os investidores que
recorrem a robôs que disparam ordens de compra ou venda quando a cotação
atinge determinados patamares.

A maioria do mercado, mesmo os que não recorrem a robôs, atua de maneira
binária.

Se o impeachment passar, a bolsa sobe até o dia da votação. Depois, cai.

Se não passar à reforma da Previdência, a bolsa cai.

Se a Lava Jato detonar ainda mais o Lula, a Bolsa sobe.

Se a Lava Jato pegar o Temer, a bolsa cai.

Se passar o limite de gastos a Bolsa sobe.

Tudo isso é avaliado evento a evento. Se um determinado evento leva a
uma alta da Bolsa, no dia seguinte à ele, a Bolsa cairá por realização
de lucros.

\section{Peça 4 --- O mercado e o país}

A partir daí, é paradoxal a ideia de que o desenvolvimento virá de
políticas econômicas que atendam o mercado no curto prazo.

Voltemos à análise do Mendonça de Barros. Para não ficar mal com os
especialistas de fato, ele inclui na sua análise, ainda que em segundo
plano, a liquidez internacional e os juros estratosféricos
proporcionados pela taxa Selic. Enfatiza que as próximas mudanças
fiscais propostas por Temer aumentarão ainda mais a credibilidade do
país.

No plano real, de visão de futuro, o que ocorrerá?

\begin{enumerate}
\itemsep1pt\parskip0pt\parsep0pt
\item
  Juros altos: desestimulam os investimentos produtivos.

  \begin{enumerate}
  \itemsep1pt\parskip0pt\parsep0pt
  \item
    Aumentam a taxa de retorno requerida para os investimentos.
  \item
    Pressionam a dívida pública, absorvendo mais recursos do orçamento.
  \item
    Como consequência, são reduzidos os os investimentos públicos e os
    gastos.
  \end{enumerate}
\item
  Limites orçamentários, cortando na carne despesas com saúde e
  educação, interessam apenas aos rentistas. Os limites para a alta de
  juros são dados pela relação dívida/\allowbreak{}\versal{PIB}. Com os cortes orçamentários,
  ganha"-se mais tempo para a manutenção de juros elevados. Mas
  compromete o investimento produtivo:

  \begin{enumerate}
  \itemsep1pt\parskip0pt\parsep0pt
  \item
    Menos oferta da mão"-de"-obra especializada.
  \item
    Menos dinamismo no mercado interno.
  \item
    Menos atividade econômica, com menos interesse pelas concessões de
    infraestrutura.
  \item
    Cortes de gastos sociais alimentam a instabilidade política. De um
    lado, pelo aumento das tensões sociais e da criminalidade em geral.
    De outro, por inviabilizar politicamente o governo que aplica essas
    políticas, induzindo"-o a ações visando interromper o processo
    político. Para o investimento produtivo -- que planeja para
    horizontes de décadas -- é um forte desestímulo. Para o investimento
    especulativo, um chamariz, porque saídas abruptas de capital também
    permitem grandes ganhos financeiros.
  \end{enumerate}
\end{enumerate}

É evidente que Mendonça de barros sabe de tudo isso. Mas, como operador
de mercado, seu papel é buscar aquilo que é melhor para o mercado.

Esse vício contaminou toda a estrutura produtiva brasileira. As grandes
corporações trocaram o crescimento sustentado pelos ganhos de
tesouraria. É~o típico caso do cachorro se alimentando do próprio rabo.

E essa extravagância, própria de nações subdesenvolvidas, é tratada como
tema sofisticado.
