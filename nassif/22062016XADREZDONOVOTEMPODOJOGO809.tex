\chapterspecial{22/\allowbreak{}06/\allowbreak{}2016 Xadrez do novo tempo do jogo}{}{}
 

Os últimos meses foram os mais decisivos da moderna história política
brasileira.

De um lado, pelo fim inglório de um período no qual partidos políticos,
poderes e instituições públicas se esfarelaram em torno do mais
vergonhoso episódio político pós"-redemocratização: a forma como está
sendo conduzido o processo de impedimento.

Não se salva um, da presidente afastada ao interino usurpador, de
ex"-presidentes da República a mandatários do Judiciário, dos velhos
coronéis nordestinos aos supostamente intelectualizados coronéis
paulistanos de má catadura.

Nunca o peso do subdesenvolvimento foi exposto de forma tão cruel quanto
agora. Praticamente não há mais nenhuma figura referencial em nenhum
setor. Executivo, partidos políticos, Supremo, Ministério Público,
Congresso, empresariado, mercado, mídia foram tomados pela mais medíocre
geração de dirigentes da história. Suas lideranças estão preocupadas em
preservar interesses miúdos, de curto prazo, eximir"-se de
responsabilidades em relação ao país.

A tentativa de dourar Michel Temer com a aura de estadista tem sido um
fiasco. O~próprio Delfim Neto apelou para que Temer esquecesse sua vida
até agora e começasse a interpretar daqui por diante o papel de
estadista. Lembra um clássico do cinema italiano, com Vitorio De Sicca:
``De crápula a herói''.

Não dá. Falta ao interino não apenas biografia como competência mínima
para se locomover no palco do poder.

Além disso, Temer pode esquecer seu passado, mas ele voltará
periodicamente a bater em sua porta.

Por outro lado, o processo do impeachment está sendo o catalizador de
uma movimentação política inédita, uma redefinição de valores e formas
de organização que irão dominar o cenário político"-eleitoral pelas
próximas décadas.

Nesse período consolidaram"-se as novas formas de organização, os
coletivos, ao lado dos movimentos sociais, fazendo"-se ao largo da
estrutura hierarquizada de sindicatos e partidos políticos. Essa mesma
horizontalidade se revela na notícia, com as redes sociais tornando"-se
cada vez mais influentes com seus múltiplos filtros substituindo o
filtro único da mídia.

As ideias"-chaves do que se imagina ser esquerda ou direita estão sendo
plasmadas nestes períodos turbulentos. Assim como as preocupações
centrais dos que zelam pelo aprimoramento da democracia.

Temas dos próximos anos

\textbf{Controle da mídia}

A fórmula trazida por Roberto Civita e protagonizada pela Rede Globo
definitivamente extrapolou. Deve"-se aos grupos de mídia não apenas a
deposição de uma presidente eleita, como o agravamento inédito da crise,
a apologia do ódio e a subversão das notícias. A~aposta no quanto pior
melhor tornou"-se marca muito forte da mídia.

 A apropriação da política pelos grupos de mídia, o uso das campanhas
extenuantes de fogo de exaustão contra os adversários, finalmente fez
com que o Brasil se equiparasse à Venezuela e à Argentina.

Hoje em dia, para pelo menos 30\% do país o controle da mídia tornou"-se
bandeira central. A~ideia do controle econômico da mídia, nos moldes de
qualquer país civilizado, será substituída por uma guerra permanente
entre partidos de esquerda e grupos de mídia.

Curiosamente, a esquerda sempre teve posições nacionalistas, em oposição
ao internacionalismo da direita. Mas, nesse caso, certamente abrirão os
braços para os grupos estrangeiros que começam a invadir o espaço com
jornalismo de alta qualidade -- como o El Pais, a \versal{BBC}, \versal{CNN}, \versal{ESPN}.

\textbf{O poder do \versal{MPF}}

Os abusos dos vazamentos da Lava Jato criaram um ambiente de completa
subversão política. Delegados, procuradores, vazam à vontade, vaza"-se em
Brasília e sempre de forma seletiva. Apenas quando o dedo de Gilmar
Mendes apontou em sua direção, o \versal{PGR} Rodrigo Janot manifestou"-se sobre o
tema.

No começo, os vazamentos eram encarados como abusos funcionais. A~partir
de janeiro de 2016, ficou nítido seu propósito político e a invasão da
política por pessoas investidas de poder de Estado recorrendo a práticas
ilícitas.

Passada a onda Lava Jato, não se tenha a menor dúvida sobre um conjunto
de medidas visando reduzir o poder de manipulação política da Polícia
Federal e do Ministério Público, em cima do vazamento de inquéritos. As
mudanças focalizarão especialmente a delação premiada e os vazamentos.

Haverá uma grande disputa para impedir que grupos de poder não se valham
da irresponsabilidade atual do \versal{MPF} para subtrair poderes relevantes para
a defesa da cidadania.

\textbf{As políticas sociais e a inclusão moral}

É o tema que delimita mais nitidamente o pensamento de esquerda e de
direita ou, mais que isso, o pensamento contemporâneo e o pensamento
ultraconservador de grupos religiosos e de ultradireita. Por aqui
abre"-se espaço para alianças mais amplas do que aquelas de cunho mais
ideológico.

\subsection{A luta de classes}

Nas últimas décadas, houve dois movimentos paralelos de ascensão. Na
classe média incluída, o aparecimento das primeiras gerações de PhDs,
poliglotas, muitos com cursos no exterior, que trouxeram um sentimento
de classe superior para todos os setores do país, das redações ao
serviço público, na forma dos concurseiros.

Mudou a natureza de muitas organizações, na medida em que passaram a ser
povoadas com essas gerações de vezo internacionalista, tendo a
perspectiva de mundo e com uma ambição pessoal acendrada.

Ao mesmo tempo houve uma ampliação das organizações sociais, entendendo
a luta política fundamentalmente como disputa de classes.

A maneira escancarada como a atual junta aboletou"-se no poder exibiu
didaticamente a forma como se dá a disputa pelo orçamento. A~aliança
entre sanguessugas políticos, do mercado, das corporações públicas e da
mídia, é curso profundamente didático intensivo sobre as disputas de
classe em torno do orçamento.

Na crise fiscal mais séria de décadas, os Ministérios estão com recursos
em caixa (provenientes do decreto que ampliou a previsão de déficit
público) para atender às suas demandas fisiológicas.

Esse componente será cada vez mais forte nas disputas políticas,
trazendo o chamado efeito Orloff: nós seremos a Argentina e a Venezuela
de hoje, cada vez mais radicalizados. Uma pena!

\textbf{A tentação do arbítrio}

Dois fatores levarão à tentação do arbítrio.

O primeiro, a enorme dificuldade da direita em apresentar um projeto de
país minimamente defensável. Com o discurso do ódio, antissocial, jamais
a direita será uma alternativa eleitoral competitiva -- incluindo aí seu
lado mais depauperado, o \versal{PSDB}. Quem não se sustenta pelos votos, precisa
encontrar outros caminhos.

O segundo fator é a dispersão de poder trazendo a demanda por governos
fortes.

O interino Michel Temer tem apostado nesse caminho. Nos últimos dias o
general chefe do Gabinete de Segurança Institucional Sérgio Etchegoyen,
passou a investigar até a movimentação de Lula. Enquanto o Ministro da
Justiça Alexandre de Moraes tenta fincar pontes com a Lava Jato, em
torno da bandeira do ``delenda Lula''.

Esse tipo de jogo não prosperará, devido à pequena dimensão política de
Temer e à composição extraordinariamente corrupta do seu bloco de poder

Mas certamente estimulará candidaturas bonapartistas nas próximas
eleições. O~caminho está aplainado para uma candidatura populista
autoritária, tipo Ciro Gomes.

\subsection{O aprofundamento da democracia}

Paradoxalmente, a ânsia por uma liderança forte será acompanhada também
por um aumento na sede de participação. O~país já experimentou formas
embrionárias de participação, com as Conferências Nacionais, conselhos
etc. Esse modelo foi deixado de lado no governo Dilma e~sob liquidação
no governo interino.

Mas, com o grau atual de diversidade e sofisticação da sociedade
brasileira, haverá uma demanda crescente por participação.

\versal{PS} --- Alguns comentaristas julgando ser depreciativa a designação
``populismo autoritário'' para Ciro Gomes. O~termo populismo não pode
ser visto como depreciativo. Significa quem se alia com o povo. E~autoritário define o estilo que busca a governabilidade pelo confronto,
em vez da conciliação, em um quadro em que a conciliação foi
inviabilizada pelo golpismo.
