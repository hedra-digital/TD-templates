\chapterspecial{13/\allowbreak{}01/\allowbreak{}2017 Xadrez do Brasil na era do narcotráfico}{}{}
 

\section{Peça 1 -- a cadeia produtiva do narcotráfico}

A característica central das novas organizações criminosas é a
descentralização e a flexibilização. O~avanço das novas tecnologias, a
integração das economias nacionais, abriu amplo espaço para
terceirizações, alianças, mudando o velho sistema das estruturas
hierarquizadas.

Houve quatro etapas na história recente das drogas.

Na~primeira etapa, investidores externos vinham até os países andinos --
Peru, Bolívia e especialmente a Colômbia -- adquiriam a coca e colocavam
nos países consumidores. Controlavam, assim, todo o processo de
comercialização.

Na~segunda etapa, os produtores decidem assumir o controle de toda a
cadeia produtiva. É~assim que se formam os cartéis de Medelin e de Cali
e, no caso brasileiro, a tentativa de Fernandinho Beira"-Mar de assumir
as duas pontas do comércio de drogas. Esse modelo torna os cartéis alvos
fixos da repressão norte"-americana.

 

\section{Peça 1 -- a cadeia produtiva do narcotráfico}

A característica central das novas organizações criminosas é a
descentralização e a flexibilização. O~avanço das novas tecnologias, a
integração das economias nacionais, abriu amplo espaço para
terceirizações, alianças, mudando o velho sistema das estruturas
hierarquizadas.

Houve quatro etapas na história recente das drogas.

Na~\textbf{primeira etapa}, investidores externos vinham até os países
andinos -- Peru, Bolívia e especialmente a Colômbia -- adquiriam a coca
e colocavam nos países consumidores. Controlavam, assim, todo o processo
de comercialização.

Na~\textbf{segunda etapa}, os produtores decidem assumir o controle de
toda a cadeia produtiva. É~assim que se formam os cartéis de Medelin e
de Cali e, no caso brasileiro, a tentativa de Fernandinho Beira"-Mar de
assumir as duas pontas do comércio de drogas. Esse modelo torna os
cartéis alvos fixos da repressão norte"-americana.

Aí se entra na~\textbf{terceira etapa}, que é a pulverização dos cartéis
e a montagem de alianças com organizações criminosas na ponta do consumo
e a terceirização ampla dos serviços intermediários.

A~\textbf{quarta etapa}~tem início quando os grandes distribuidores --
\versal{PCC}, \versal{CV} -- decidem escapar das limitações do fornecedor e passar a
controlar também a ponta da produção, mas adotando um modelo de gestão
flexível, baseado em códigos de conduta, uma máfia ou~\emph{cosa
nostra}~~na era do celular e da Internet.

Com algumas variações, o modelo da organização do narcotráfico é o
seguinte:

 

\begin{enumerate}
\itemsep1pt\parskip0pt\parsep0pt
\item
  Chamemos de Chefão do Tráfico o cabeça que organiza as diversas
  estruturas: os produtores, os consumidores, o fornecimento de produtos
  químicos, o financiamento. Embaixo dele há um Setor Operacional, que
  monta os contatos com as diversas pontas do sistema, preservando o
  Chefão e garantindo a compartimentalização -- uma peça não conhecendo
  outra.
\item
  Na ponta da produção: o Traficante Produtor organizando a produção
  interna, montando laboratórios ou adquirindo a produção de
  laboratórios independentes.
\item
  Na ponta da distribuição: parcerias com organizações criminosas
  locais, como o Comando Vermelho (\versal{CV}) ou o Primeiro Comando da Capital
  (\versal{PCC}). Estes, por sua vez, montam parcerias com organizações
  criminosas regionais ou municipais, como as estruturas do jogo de
  bicho. O~trabalho não é exclusivo. Além do tráfico, a parceria pode se
  estender para contrabando de armas, assassinatos pagos e outros
  produtos. Mas os chamados ``donos do morro'' dependem totalmente dos
  fornecedores para manter seu negócio.
\item
  No fornecimento de matéria prima: empresas especializadas ou suborno a
  funcionários de grandes empresas químicas para desviar parte da
  produção para o tráfico.
\item
  No transporte: empresas terceirizadas ou mulas.
\item
  No financiamento: o Chefão do Tráfico procura doleiros e apresenta
  oportunidades de negócios. Os doleiros, por sua vez, oferecem a
  investidores não diretamente ligados ao tráfico, mas perfeitos
  conhecedores das oportunidades e riscos.
\item
  Na lavagem: montagem de empresas da economia real, onde o dinheiro do
  Chefão do Tráfico possa ser lavado. Um caso emblemático é do bicheiro
  Carlinhos Cachoeira que investiu em laboratórios farmacêuticos em
  Anápolis -- setor com grande afinidade com o refino de coca.
\item
  Na blindagem: os contatos pessoais do Chefão com políticos e
  autoridades.
\end{enumerate}

\section{Peça 2 -- a guerra do tráfico e os episódios do norte}

O negócio do tráfico assumiu proporções bilionárias. No caso da
Colômbia, estima"-se em U\$ 6 bilhões as exportações para os Estados
Unidos, em \versal{US}\$ 5 bilhões para a Europa e em \versal{US}\$ 1,75 bilhão para
África e Ásia. Na ponta dos grandes atacadistas, a venda final faturaria
cerca de \versal{US}\$ 12,8 bilhões ano. Se incluir o preço final do varejo,
aumentaria substancialmente o faturamento.

Segundo reportagens de Mário Magalhães em 2000, quanto mais acima
entrava a cocaína (Pará, Roraima, Amazonas, Acre e Rondônia), maior a
possibilidade de se destinar ao exterior. ~Era a produção da Colômbia,
considerada de muito melhor qualidade. É~o que explica o crescimento de
uma quadrilha localizada, a Facção Família do Norte (\versal{FFN}).

Por Mato Grosso do Sul e Paraná vem a produção da Bolívia, mais
destinada ao mercado interno.

Uma das versões mais consistentes sobre a guerra do tráfico foi de
Carlos Wagner, repórter da região, para os Jornalistas Livre. O~desmonte
dos cartéis colombianos abriu os olhos da \versal{FFN} e do \versal{CV} para a
possibilidade de se livrarem dos Chefões do Tráfico e controlarem, eles
próprios, o ciclo total da produção e venda
(\url{https:/\allowbreak{}/\allowbreak{}goo.gl/\allowbreak{}u\versal{PQE}u9)}. Principalmente, o acordo de paz firmado
com o governo colombiano, retirou as Farc do jogo, deixando um vazio
para ser ocupado. O~pacto entre a \versal{FFN} e o \versal{CV} garantiria ao grupo o
controle da produção e da demanda.

Por sua vez, o \versal{PCC} teria o controle do presídio em Roraima, da produção
que vem da Bolívia e de outra que passa pelo Paraguai e entra pelo
Paraná.

\section{Peça 3 -- a repressão e os bagrinhos}

Na cadeia produtiva do narcotráfico, a repressão se concentra nas
pontas, dos plantadores e produtores e dos pequenos distribuidores e das
mulas (que transportam individualmente a droga) -- anotadas nas manchas
rosas.

Hoje em dia, até a parcela não"-colarinho branco do tráfico -- \versal{CV} e \versal{PCC}
-- consegue acordos com governos estaduais, conforme se observa em São
Paulo. A área verde, onde se concentra o colarinho branco, é preservada.

Todo o aparato repressivo entra em um jogo de enxugar gelo. Chega um
nigeriano no aeroporto de Guarulhos, em muitos casos, mero boi de
piranha -- para ser flagrado, enquanto o resto da turma passa incólume.
Preso, a Polícia Federal faz uma ocorrência. O~delegado o interroga e,
algumas vezes, até descobre ligações com outras pessoas que viajaram
juntas. Abre"-se o inquérito, com laudo toxicológico do que foi
apreendido. O~nigeriano vai preso e o inquérito vai para o procurador
que recebe, faz o pedido da conversão do flagrante em prisão preventiva
-- já que o réu não tem residência no Brasil. O~nigeriano fica preso,
somando"-se aos demais presos, o processo corre e quase nunca se chega ao
chefão.

Por outro lado, o próprio sistema prisional se incumbe de engordar as
organizações criminosas. De um lado, pela ampla liberdade concedida aos
chefes para comandarem suas organizações a partir dos presídios.

De outro, pelas próprias políticas penitenciárias. A~falta de
informações sobre o que ocorre na ponta ajudou a criar o monstro,
segundo o juiz Luiz Carlos Valois, de Manaus, explicou ao \versal{GGN}.

O \versal{PCC} e o \versal{CV} eram fenômenos confinados no Sudeste, São Paulo e Rio. Sua
expansão para o Norte e Nordeste se deveu a um erro de estratégia das
políticas de segurança do \versal{PT}, construindo presídios federais de
segurança máxima.

Quando algum marginal provocava problemas nos presídios do norte e
nordeste, é despachado para os presídios federais. Ele retornava
alardeando o contato com seus colegas do \versal{PCC} e do \versal{CV}, o que lhe conferia
um status especial no presídio de origem. Além disso, quando era
transferido para o presídio federal, criava um vácuo na liderança do
pavilhão, que não raras vezes resultava em disputas sangrentas até se
consolidar o novo líder. Voltando, disputava a liderança com seu
antecessor, gerando mais sangue.

Mais que isso, ajudava a disseminar a mística do \versal{PCC} e do \versal{CV}, com seus
códigos, discursos políticos próprios de irmandades criminosas.

Mesmo assim, a \versal{FFN} não tem expressão maior em Manaus, diz Valois, como a
de parar cidades, como o \versal{PCC} fez em São Paulo em 2006.

A situação é pior nos presídios paulistas e cariocas. O~sujeito é preso
por um assalto. No presídio, indagam de que facção ele é. Ele informa
que não é de facção nenhuma, que apenas perpetrou um assalto. Indagam,
então, de que bairro ele é. De acordo com o bairro, é enquadrado em uma
das facções existentes e confinado junto aos supostos colegas.

Referência no trabalho dos presídios, Valois sustenta que líder de
corredor ou de ala sempre existiu no sistema penitenciário. Quando as
facções nasceram, passaram a carimbar cada líder com a marca da facção.
Criaram lendas. Segundo ele, basta dois criminosos se comunicando com
walk talk para serem taxados de organização criminosa.

Todos esses equívocos, na base, ajudaram a construir a mística das
organizações criminosas, que acabaram infladas pela retórica e pela
explosão das populações carcerárias.

\section{Peça 4 -- as audiências de custodia para o baixo clero}

Hoje em dia, a população carcerária é de 600 mil pessoas. A~um custo
anual de R\$ 36 mil per capita, representam um custo total de R\$ 21,6
bilhões ano. Desses,

240 mil são presos sem condenação, que passam em média 6 meses em prisão
preventiva, sem julgamento e sem sentença.

A rigor, o único trabalho sério para reduzir a população carcerária
foram as audiências de custodia, na gestão Ricardo Lewandowski no
Conselho Nacional de Justiça (\versal{CNJ}).

Com base no Pacto de São José da Costa Rica, da Convenção Americana de
Direitos Humanos (\url{https:/\allowbreak{}/\allowbreak{}goo.gl/\allowbreak{}FbnCiW}), Lewandowski soltou uma
resolução obrigando que os presos fossem apresentados aos juízes no
prazo de 24 horas, mesmo em fins de semana. Nesse período, haveria como
o juiz conferir se o preso foi submetido a sevícias pela polícia e se
seu caso é de prisão.

As audiências revelaram que 45\% das prisões eram desnecessárias, além
de mostrar a viabilidade de manter fora dos presídios réus de baixa
periculosidade, recorrendo a medidas cautelares alternativas, como o uso
de tornozeleiras eletrônicas, a prisão domiciliar e restrições a
direitos.

Pelas projeções do \versal{CNJ}, se tentaria reduzir em 50\% o total de presos
provisórios. Isso permitiria uma economia anual de R\$ 4,3 bilhões na
sua manutenção e de R\$ 9,6 bilhões na redução de necessidade de
construção de 240 presídios (\url{https:/\allowbreak{}/\allowbreak{}goo.gl/\allowbreak{}BngKu4}).

Durante 2015, as audiências carcerárias lograram reduzir em 40.584
pessoas os presos provisórios.

 

A audiência com cada preso não leva mais que dez minutos.

O projeto mereceu o reconhecimento da Comissão Interamericana de
Direitos Humanos (\versal{CIDH}) (\url{https:/\allowbreak{}/\allowbreak{}goo.gl/\allowbreak{}\versal{Y}6axcs)}, ligada à
Organização dos Estados Americanos (\versal{OEA}). A~rigor, foi a única menção
positiva ao Brasil na última visita do observador do \versal{CIDH}.

Mesmo assim, a experiência foi amplamente ignorada pelos veículos de
mídia, por ser de iniciativa de personagem público, o Ministro
Lewandoiwski, tratado de acordo com o direito midiático do inimigo. E~mereceu resistências dos órgãos de classe de juízes, por significar
trabalho adicional, e do próprio Ministério Público, por seu viés
acusador.

\section{Peça 5 -- a blindagem do alto clero}

Se os presídios estão abarrotados com representantes do baixo clero, da
mais alta à mais baixa periculosidade, as celas especiais destinadas ao
colarinho branco estão vazias.

Hoje em dia, qualquer estratégia contra o crime organizado teria que
obedecer à máxima de ``seguir o dinheiro''. ~A figura"-chave, então,
passa a ser o operador financeiro, o doleiro, que une as pontas do Chefe
do Tráfico, os financiadores, a corrupção política.

 

Em 2003, o então Ministro da Justiça Márcio Thomaz Bastos desenvolveu a
Estratégia Nacional de Combate à Corrupção e à Lavagem de Dinheiro
(Enccla), juntando mais de 60 órgãos dos Três Poderes ligados a alguma
forma de fiscalização (\url{https:/\allowbreak{}/\allowbreak{}goo.gl/\allowbreak{}\versal{LT}jTlB)}.

O modelo foi amplamente utilizado na Operação Lava Jato. Nem resvalou no
combate ao tráfico. Um superdoleiro como Alberto Yousseff conseguiu sua
segunda liberdade provisória sem que o juiz Sérgio Moro -- veterano do
caso Banestado -- ousasse extrair nenhuma informação sobre o tráfico.
Ressalve"-se que o tráfico tem argumentos eloquentes para demover
delatores: quem delata, morre.

Mesmo assim, cada vez que se avança sobre doleiros, bate"-se de frente
com a economia formal, com grupos influentes, muitos deles parceiros
preferenciais do \versal{MPF} que se valem dos mesmos canais de lavagem de
dinheiro do narcotráfico. Quando chega aí, as investigações param.
Aliás, o fato do dinheiro transitar no mesmo duto teoricamente permite a
financistas montarem suas apostas também no jogo do tráfico.

Alguns exemplos:

\begin{enumerate}
\itemsep1pt\parskip0pt\parsep0pt
\item
  Em 2009, procuradores do \versal{MPF} estouraram o escritório de um casal de
  doleiros do Rio de Janeiro, na Operação Norbert. Toparam com uma conta
  de Aécio Neves em Liechtenstein. O~caso foi parar na gaveta do \versal{PGR}
  Roberto Gurgel em 2010 e só saiu em 2016, quando a parcialidade do \versal{PGR}
  Rodrigo Janot se tornou flagrante. Mesmo assim, o inquérito caminha a
  passos lentíssimos, com jogos de empurra entre o Ministro Gilmar
  Mendes, do \versal{STF}, o \versal{PGR} e a Polícia Federal
  (\url{https:/\allowbreak{}/\allowbreak{}goo.gl/\allowbreak{}\versal{KAS}ihP}).
\item
  Nas investigações sobre o suposto Triplex de Lula, a \versal{PF} implode os
  arquivos da Mossak Fonseca, escritório norte"-americano famoso pela
  lavagem de dinheiro de ditadores e políticos. Depara"-se, então, com
  contas da família Marinho. A~investigação desaparece e nao renasce nem
  com os Panama Papers (\url{https:/\allowbreak{}/\allowbreak{}goo.gl/\allowbreak{}Igv\versal{SC}b}).
\item
  Em um dos dos maiores escândalos dos últimos anos, o da \versal{FIFA}, doiis
  desembargadores polêmicos, da Tribunal Regional Federal do Rio de
  Janeiro (Paulo Espírito Santo e Abtônio Ivan Athié) determinaram a
  suspensão das remessas de dados do \versal{MPF} para o \versal{FBI}. O~\versal{STJ} (o Superior
  Tribunal de Justiça) manteve a suspensão. O~caso está parado no \versal{STF}.
\item
  Quando o caso \versal{HSBC} veio à tona, apareceram nomes de grupos de mídia
  aliados do \versal{MPF}. O~caso sumiu.
\item
  O caso Carlinhos Cachoeira foi exemplar. O~\versal{MPF} de Goiás montou a
  Operação Monte Carlo. Foram encaminhados ao Procurador Geral Roberto
  Gurgel os dados relativos aos parlamentares -- que tinham foro
  privilegiado. Gurgel paralisou as investigações. Chamado pela \versal{CPMI} de
  Cachoeira a dar explicações, informou que segurou o inquérito para não
  atrapalhar a Operação Las Vegas, destinada a investigar o bicho.
  Posteriormente, procuradores de Goiás informaram que as duas operações
  não tinham relação entre si (\url{https:/\allowbreak{}/\allowbreak{}goo.gl/\allowbreak{}oNyfk0)}. Cachoeira
  está solto e tendo, entre seus ativos formais, um laboratório
  farmacêutico (!).
\end{enumerate}

O episódio mais notório é o do helicóptero que caiu transportando 500
quilos de cocaína. Não há nenhuma prova concreta conhecida a ligar o
caso ao presidente do \versal{PSDB} Aécio Neves. Mas existe um conjunto de
circunstâncias que, no mínimo, exigiriam investigações e explicações:

\begin{enumerate}
\itemsep1pt\parskip0pt\parsep0pt
\item
  O helicóptero pertencia a Gustavo Perrela, filho do senador Zezé
  Perrela, estreitamente ligado a Aécio.
\item
  Conforme o \versal{GGN} publicou em 26 de julho de 2014,
  (\url{https:/\allowbreak{}/\allowbreak{}goo.gl/\allowbreak{}Om\versal{FYY}c}~), não haveria condições do helicóptero
  ter percorrido 1,17 mil km em linha reta sem uma parada para
  reabastecimento. E~o reabastecimento não ocorreu em nenhum aeroporto
  fiscalizado pela \versal{ANAC}. Por coincidência, o aeroporto de Cláudio,
  construído no governo Aécio perto de propriedades da família, ficava
  na rota percorrida pelo helicóptero e ainda não fora homologada pela
  \versal{ANAC}.
\item
  Morador em Cláudio, o primo de Aécio, Tancredo Tolentino, foi acusado
  de negociar habeas corpus para traficantes, em conluio com um
  desembargador do Tribunal de Justiça.
\item
  Perrela injetou muito dinheiro no Cruzeiro Futebol Clube. Depois de
  conquistado o bu"-campeonato nacional, praticamente todo o elenco foi
  negociado com o exterior (\url{https:/\allowbreak{}/\allowbreak{}goo.gl/\allowbreak{}vmcF19)}. O~comércio de
  jogadores sempre foi uma peça relevante na lavagem de dinheiro.
\end{enumerate}

Hoje em dia, o inquérito está parado, graças a um \versal{HC} do Tribunal de
Justiça de Minas, apesar dos esforços de procuradores da República de
Minas de tocarem as investigações.

Não se ouve protesto no Supremo Tribunal Federal (\versal{STF}), nem da parte de
Ministros que passaram a defender o estado de exceção contra a
corrupção, como é o caso de Luís Roberto Barroso. Parceria do \versal{MPF} no
combate intransigente à corrupção, nenhum grande veículo da imprensa se
interessa pelo tema -- cuja cobertura está restrita aos blogs e portais
alternativos.

De fato, o crime organizado domina o país. Mas ele não está restrito às
celas infectas dos presídios.

\textbf{Correção}~- foram corrigidos os dados referentes ao \versal{HC} que
impediu o envio de dados sobre a \versal{FIFA} pelo \versal{MPF} para o \versal{FBI}.
