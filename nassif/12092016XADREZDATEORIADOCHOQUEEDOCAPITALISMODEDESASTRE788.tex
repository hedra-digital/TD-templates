\chapterspecial{12/\allowbreak{}09/\allowbreak{}2016 Xadrez da teoria do choque e do capitalismo de desastre}{}{}
 

Há um conjunto de peças soltas no golpe que, quando devidamente
organizadas, permitem entender de modo muito mais claro um dos aspectos
mais relevantes: a influência externa.

São elas:

\begin{enumerate}
\itemsep1pt\parskip0pt\parsep0pt
\item
  A campanha sistemática da mídia de destruição da autoestima nacional.
\item
  Recém instalado o golpe, a corrida do ouro entre Eduardo Cunha e José
  Serra, para ver quem se antecipava na aprovação da nova legislação do
  petróleo.
\item
  A ida repentina do senador Aloysio Nunes aos Estados Unidos, para
  conversar com membros do Senado.
\item
  Antes dele, a ida do Procurador Geral da República aos Estados Unidos,
  para reuniões com o Departamento de Justiça e outros setores
  sensíveis.
\item
  A bandeira mágica que acompanha o golpe, de colocar a salvação do
  Brasil no trinômio reforma da Previdência"-livre fluxo de
  capital"-desregulação/\allowbreak{}privatização.
\end{enumerate}

Para juntar as peças acima, vale a pena um mergulho no livro ``Teoria do
Choque'' da norte"-americana Naomi Klein.

\section{Peça 1 -- o início da ``teoria do choque''}

A base do livro é a descrição de estudos psicológicos nos Estados
Unidos, que teriam contribuído igualmente para o aprimoramento dos
métodos de tortura da \versal{CIA} e das intervenções político"-econômicos em
países conflagrados. Tratam"-se dos estudos de 1963 de Ewen Cameron e
Donald Hebb, sistematizando os princípios do que veio a ser vulgarmente
conhecido como lavagem cerebral através do uso de eletrochoques.~~A
conclusão principal era a de que ``a privação de estímulos (através da
tortura) induz à regressão, despojando a mente do indivíduo do contato
com o mundo exterior e forçando à regressão''.

Quando o prisioneiro mergulha em um estado de ``choque psicológico'', ou
``vivacidade interrompida'', é sinal de que está mais aberto a
sugestões, mais disposto a ceder. Em situações mais brandas, mas nem por
isso menos drásticas, mantem"-se o réu detido, sem contato com o mundo
exterior, com família, sem acesso a notícias, até que entre no estado da
``vivacidade interrompida''. Se o trabalho for bem feito, delata até o
casamento da princesa Leopoldina em Diamantina, onde nasceu \versal{JK} (apud
``Samba do crioulo doido'').

Para haver cura, seria preciso eliminar tudo o que existia antes.
``Cameron estava seguro de que se varresse para bem longe os hábitos,
modelos e lembranças dos seus pacientes, chegaria àquele espaço vazio
primitivo'', diz Naomi. Em geral, os resultados alcançados foram os de
deixar os pacientes com suas memórias fraturadas e sua confiança traída,
constata ela. Mas abrindo o bico, que é o que interessava.

Cameron transpôs suas teses para o campo das ciências sociais, através
de um livro onde pontificava sobre como preparar a reconstrução da
Alemanha no pós"-guerra. Propôs"-se a desenvolver uma nova ciência social
e comportamental, na qual os cientistas do comportamento passariam a
agir como planejadores sociais. Nessa nova utopia não haveria lugar para
os doentes e os fracos, que deveriam ser removidos para não influenciar
as novas gerações.

\section{Peça 2 -- o capitalismo de desastre}

Segundo Naomi, essa mesma tese da destruição"-reconstrução foi
desenvolvida pelo chamado capitalismo de desastre, a partir dos estudos
e da pregação de Milton Friedman, da Escola de Chicago, inspirada nas
teses de Cameron.

Em um de seus ensaios fundamentais, Friedman desenvolveu a estratégia de
como se prevalecer de situações de crise -- ``crise real ou
pressentida'' enfatizou. Quando a crise acontece, dizia ele, as ações
que são tomadas dependem das ideias que estão à disposição. ``Esta, eu
acredito, é a nossa função primordial: desenvolver ideias alternativas
às políticas existentes, mantê"-las em evidência e acessíveis até que o
politicamente impossível se torne politicamente viável''.~

Tão logo uma crise se instale, defendia Friedman, é essencial agir
rapidamente, ``impondo mudanças súbitas e irreversíveis, antes que a
sociedade abalada pela crise possa voltar à tirania do status quo''. Nas
suas contas, uma administração teria de seis a nove meses para realizar
as principais mudanças. ``Caso não agarre a oportunidade de agir de modo
decisivo durante esse período, não terá outra chance igual''.

A fórmula salvadora consistia em medidas irreversíveis que atendam ao
trinômio liberdade total para o capital"-privatização/\allowbreak{}desregulação"-cortes
nos serviços e benefícios sociais. Alguma semelhança com o caso
brasileiro?

Há inúmeros episódios em que se aplicou a teoria do choque, desde o
golpe contra Allende, no Chile, ao enorme fracasso da ocupação do Iraque
e ao desmonte total do sistema de educação pública de Nova Orleans, após
o terremoto Katrina.

Foi o que Naomi testemunhou na guerra do Iraque. ``Os arquitetos da
invasão norte"-americana e britânica imaginaram que o seu uso da força
seria tão chocante, tão esmagador, que os iraquianos mergulhariam em um
estado de vivacidade interrompida, muito parecida com aquele descrita no
manual Kubark (da \versal{CIA})''.

O certo é que parte dos grandes empresários norte"-americanos,
evangelizados por Friedman, se imbuíram do chamado ``destino
manifesto'', de levar o capitalismo em estado puro para os povos
primitivos. Personagens contemporâneos, como os irmãos Kock repetem os
W.R.Grace, católicos de origem irlandesa que, nos anos 60, bancavam o
padre Peyton e sua cruzada pelo ``rearmamento moral''.

Desde Adlai Stevenson, a \versal{CIA} tornou"-se a parceria fundamental nessa
cruzada capitalista, em que se misturam interesses empresariais, a
pregação evangélica, a síndrome do ``destino manifesto'' e a geopolítica
do Departamento de Estado. A última tentativa (fracassada) foi quando
Otto Reich, do Departamento de Estado, articulou com os grupos de mídia
venezuelanos a deposição do presidente Hugo Chávez.

\section{Peça 3 -- a ``vivacidade interrompida'' no golpe brasileiro}

Em algum lugar do passado recente, o Brasil era uma nação prestes a
entrar para o primeiro time. Indústria naval, cadeia produtiva do
pré"-sal, grandes empreiteiras, montagem de uma forte indústria nacional
de medicamentos, multinacionais brasileiras começando a conquistar o
mundo, a diplomacia brasileira se impondo nos principais fóruns globais.

Em pouco tempo o país entrou na fase da ``vivacidade interrompida'', a
sensação da crise terminal, da falta de saídas, o pessimismo repetido 24
horas por dia, o fim do mundo ao alcance da próxima manchete.
Instaurou"-se a tal crise pressentida.

O que aconteceu?

A partir da \versal{AP} 470, a imprensa explorou duas estratégias paralelas. Uma,
a da luta contra a corrupção, personificada no \versal{PT} e em Lula. Outra, a
luta de classes, levantando diuturnamente as ameaças chavistas,
estigmatizando as políticas sociais, enfatizando a falta de cultura e de
verniz dos adversários. É~por aí que tem início a cooptação das classes
médias, da elite das corporações públicas e do próprio Ministério
Público Federal, o reino dos PhDs contra o primarismo dos chavistas.

O brilhante desempenho de Lula e Dilma de 2008 a 2012 anulou a
estratégia. Mas a imprensa não interrompeu sua campanha massacrante,
sistemática, de desconstrução do que estava sendo feito. Valeram"-se de
diversos subterfúgios. Se se levantavam grandes obras, com 90\%
concluídas, enfatizavam os 10\% que faltavam. Em um programa social com
15 milhões de famílias assistidas, a manchete era a pequena corrupção
identificada em um ponto qualquer do país. Na Copa do Mundo, enquanto os
estádios e aeroportos eram construídos, destacava"-se o fato de não
estarem prontos. Entregues, o destaque era para a falta de sabonete nos
banheiros.

Com os sinais de bonança revertendo, as manifestações de junho de 2013
foram o primeiro alerta de que que o pêndulo da opinião pública começava
a inverter.

Instalada a crise, a ``teoria do choque'' pode colocar a cabeça de fora.
Parlamentares como Aécio Neves, no Senado, e Eduardo Cunha, na Câmara,
trataram de bloquear toda a atividade parlamentar, negando ao governo
Dilma as ferramentas mínimas para consertar os erros. Aécio, José Serra
e Fernando Henrique Cardoso tornaram"-se os porta"-vozes do caos
estimulando o movimento golpista nas ruas e nos jornais, enquanto a
parceria \versal{PGR}"-Lava Jato"-mídia tratava de incendiar a classe média com as
denúncias de corrupção focadas exclusivamente no \versal{PT} e em Lula.

Nesse período, para preparar o bote final o Procurador Geral da
República (\versal{PGR}) Rodrigo Janot foi ao Departamento de Estado pedir a
bênção e voltou com malas digitais repletas de informações sobre as
contas das empreiteiras no exterior e sobre a corrupção na
Eletronuclear. Ali, na cooperação internacional, os Estados Unidos deram
a contribuição mais ostensiva para o golpe. Outras contribuições
demandarão algum tempo para virem à tona.

O discurso anticorrupção foi o mote que juntou todas as pontas, criando
o sentimento da classe e fornecendo o álibi para quem pretendesse pular
no barco da conspiração.

\section{Peça 4 --- o fator Dilma}

Sob esse céu coalhado de bombas, há o fator Dilma Rousseff, é verdade.

Poucas vezes na história teve"-se governo mais desastrado e indefeso.

Montou a mais ousada política industrial desde o
2\textsuperscript{o}~\versal{PND} (Plano Nacional de Desenvolvimento) em torno do
pré"-sal e da Petrobras. Estaleiros, renascimento da indústria de
máquinas e equipamentos, atração de laboratórios de grandes
multinacionais ao país, montagem pela Petrobras de programas de compras
públicas que dotariam o país de competência interna imbatível para
tecnologia de extração de petróleo em águas profundas,
internacionalização das empreiteiras. Toda essa responsabilidade nas
costas da Petrobras. E, de repente, a Petrobras passa a ser sufocada
pelos sub"-reajustes de combustíveis, como parte da tática de empurrar a
inflação com a barriga.

Dilma se fechou para todos os segmentos, dos movimentos sociais aos
empresários, e ainda assistiu inerte a Lava Jato completar a destruição
de parte relevante do \versal{PIB} sem esboçar um gesto de resistência.

O ciclo se fecha com sua teimosia em se candidatar à reeleição e a falta
de vontade de Lula de enfrentar a bucha que surgia no horizonte.

Dilma não entendeu o terceiro tempo das eleições, que se iniciou no dia
seguinte à abertura das urnas, trancou"-se no Palácio, fez dieta e
reapareceu em público no dia da posse, com um ministério tirado do
colete, sem nenhuma espécie de articulação política e com um pacote
econômico desastroso.

Consumou"-se o desastre com o plano Joaquim Levy, uma tragédia óbvia e
cantada, de que ajuste fiscal com recessão seria um desastre econômico e
uma sinuca política.~~Aliás, mote repetido várias vezes por Dilma em sua
apresentação no Senado -- mostrando que sempre descobre o caminho certo
com alguns anos de atraso.

Dilma foi apenas o desastre que facilitou o golpe, mas que jamais
poderia servir de álibi para a implantação do estado de exceção. A~economia teria condições de se recuperar, não fosse o cerco do Congresso
e da mídia. Estava"-se longe do estado de caos retratado na cobertura
jornalística, especialmente na pregação massacrante da Globo. Mas, o
cavalo de Troia do governo -- a \versal{PGR} -- já tinha deflagrado a ofensiva
final.

\section{Peça 5 -- a corrida contra o relógio}

Agora se entra naquele período crítico previsto por Friedman, de seis a
nove meses sob a égide do ``vazio primitivo'' para enfiar goela abaixo
do país as reformas previstas. Daí esse braço de guerra, com jornais
abrindo manchetes esbaforidas, tipo se a reforma da Previdência não
acontecer nos próximos dias, o futuro estará comprometido, e outras
baboseiras destinadas à ralé da opinião pública. Ou a pressa de Serra e
associados de correr com a lei do petróleo e a privatização acelerada na
Petrobras, mesmo com a economia na bacia das almas.

A blitzkrieg esbarra, no entanto, nos seguintes fatores:

\subsection{O Fora Temer}

É o fato novo, que vem em um crescendo, entrando por todos os poros do
mercado de opinião, inclusive nas brechas abertas inadvertidamente pela
mídia, é o Fora Temer. Há o risco concreto de que o tema ganhe os
leitores de jornais. Daí a montagem do sistema de repressão e da
tentativa de envolver as Forças Armadas, através dos factoides dos
supostos terroristas islâmicos, como têm demonstrado as extraordinárias
reportagens de Marcelo Auler (\url{https:/\allowbreak{}/\allowbreak{}is.gd/\allowbreak{}kX67p6}).

Há alguma probabilidade de que pegue o discurso das ``diretas"-já''.

\subsection{A ilegitimidade das reformas}

Nenhum investidor minimamente informado apostará em reformas que
dependem de um golpe para serem implementadas. Lula e Dilma avançaram em
algumas reformas relevantes, pelo fato de possuírem credibilidade junto
aos movimentos sociais e às esquerdas. Temer não tem nem credibilidade
institucional nem pessoal. O~que acontecerá a partir de 2018?

\subsection{\textbf{A construção de Temer}}

Daí, a uma tentativa bisonha de construir uma imagem pública minimamente
defensável para Temer. Repare na foto ao lado. É~a cara do governo,
Eliseu Padilha, cercado pelos holofotes da mídia. Uma breve pesquisa no
Google mostrará uma extensa capivara do Ministro"-Chefe da Casa Civil.
Como tornar o governo legítimo? É o Eliseu de Canoas, do \versal{DNIT},
dizendo"-se defensor da Lava Jato e das reformas.

A tentativa de isolar Temer, como se fosse uma jovem virginal envolvida
por malandros, não cola. Só o eminente jurista Celso Antônio Bandeira de
Mello tenta acreditar nisso. A~vida política de Temer está estreitamente
ligada às de Eliseu Padilha, Eduardo Cunha, Moreira Franco, José Serra,
Geddel Vieira Lima.

Mesmo abstraindo a biografia, Temer não conseguirá compor o figurino do
estadista, ou meramente do presidente que paire acima das quizílias do
dia"-a-dia. É~miúdo, vingativo, tem um linguajar antiquado, baixíssimo
nível de informação, nenhuma empatia com o público. O~jornal~\textbf{O
Globo}~abriu uma enorme oportunidade para mostrar o lado ``humano'' de
Temer e ele jogou fora dando um golpe no rei Arthur e colocando em seu
lugar Carlos Magno, que, por sua vez, abriu mão dos Doze Pares de França
para comandar os Cavaleiros da Távola Redonda, provavelmente em uma
escaramuça lá em Diamantina, onde nasceu \versal{JK}.

A alternativa encontrada foi focar em uma primeira dama jovem, bonita,
discreta e… muda. Foi ridícula a solução encontrada, de tirar
conclusões políticas do ``look'' branco que ela utilizou em uma
solenidade qualquer. Ridícula por expor a necessidade dos jornais de
arrostar o impossível e o ridículo para atender o governo Temer e fazer
jus à bolsa mídia prometida por Eliseu Padilha.

\subsection{\textbf{O fator Lava Jato}}

Na Lava Jato há dois personagens acusados de jogo político, de
perseguição ao \versal{PT} e de proteção ao \versal{PSDB}: o \versal{PGR} Rodrigo Janot e o juiz
Sérgio Moro. Janot não conseguirá desvencilhar"-se do estigma
simplesmente por não ter nem vontade nem condição política de indiciar
Aécio Neves.

No entanto, há alguns sinais no horizonte de que Moro pretenda passar no
teste de imparcialidade investindo na delação de Eduardo Cunha.

Consumada a cassação de Eduardo Cunha, a maior probabilidade é de que em
poucos dias ele seja conduzido preso à Curitiba e submetido a uma
delação conduzida por Moro. Isso ocorrendo, sairia das asas de Janot e
se abriria alguma possibilidade de rompimento da blindagem sobre Aécio
Neves e de ameaças concretas contra o governo Temer.

Há uma probabilidade --- pequena, por enquanto --- de crescimento do
``diretas já'' e de abreviação do governo Temer.
