\chapterspecial{29/\allowbreak{}12/\allowbreak{}2016 Xadrez do Hommer Simpson e do desmonte nacional}{}{}
 

Nos últimos dias tive dois contatos marcantes. Um deles, com um
autêntico representante da ultradireita delirante. Outro, com um
representante típico do Homer Simpson.

 

Vamos por parte.

 

Fomos apresentados à direita delirante por um amigo gozador, que juntou
os três casais em uma feijoada. O~sujeito era oftalmologista, estudara
nos Estados Unidos, em uma universidade da qual não me recordo o nome,
mas, segundo ele, ~muito mais afamada que Harvard, tinha sido convidado
a trabalhar em um órgão do governo norte"-americano, muito importante, e
do qual não me recordo o nome, e cometeu outros feitos expressivos, dos
quais não me recordo a relevância.

 

Ele se informa em sites de ultra"-direita, não confia em nada do que sai
na imprensa e acredita em tudo o que lhe dizem seus pares.

 

Quando elogiou minha origem libanesa, por ser uma raça pura, percebi que
a conversa ia ser marcante.~

 

Ele é contra todas as raças impuras, diz que Donald Trump vai colocar as
coisas nos eixos (sem jogo de palavras). Garantiu, sem pestanejar, que
Michele Obama é transexual; que Barack Obama não é Barack Obama, mas um
sujeito que se faz passar por Barack Obama. Trata os negros como
macacos. E~me passou a mais retumbante das revelações que, segundo ele,
tem sido sonegada por toda a imprensa ocidental. Aliás, apostou comigo
como não conseguiria publicar nem no meu blog a relevante informação de
que não há mais peixes no Oceano Pacífico.E não adiantou argumentar que
desastre desse tamanho não seria sonegado nem pelo Estadão, mesmo se
fosse de responsabilidade do \versal{PSDB}.

 

Pulemos para o simpático Homer Simpson, que me aborda no boteco de
Poços.

 

Diz que os problemas no Brasil surgiram com o porto de Mariel, em Cuba.
Levaram para lá todos nossos empregos e nossas divisas.

 

Tento explicar que a construção do porto envolve inúmeros materiais e
equipamentos fabricados no Brasil, contratos com indústria mecânica,
siderúrgica e muitas outras. Portanto, gerou muitos empregos no Brasil.

 

E ele: mas o dinheiro foi para fora.

 

Explico que não, que a obra será paga e os lucros reverterão para o
Brasil, através da empresa construtora.~

 

E ele: não sei não.~

 

Pacientemente explico que se trata de exportação de serviço praticada
por todas as nações, pela China, pelos Estados Unidos. Se não fosse bom,
porque os grandes países disputariam mercado?

 

E ele, com a segurança de um procurador da Lava Jato: ``Pode ser bom
para a China e Estados Unidos, mas não para o Brasil''.

 

Aí desisto e, como no começo da conversa ele se apresentou como
astrólogo amador, interrompo a conversa com minha saída favorita:

 

— Eu não ouso discutir astrologia com você.

 

Ele entendeu, se despediu e foi embora. Educadamente, saliento.

 

\section{O fenômeno da desinformação}

 

Nos dois casos, a conversa -- embora surreal -- foi em bases
relativamente educadas. No caso do direitoso, um conteúdo de uma
violência extrema, mas dito socialmente em uma ``conversa de brancos''.
No Hommer Simpson, um senhor simpático, boa gente mesmo.

 

Mas o novo normal é a grosseria, o sujeito tratar sua opinião como um
bem de raiz, dedicando a ela o mesmo cuidado obsessivo com que cuida das
suas posses, seja o carro velho ou a casa a beira"-mar. E~reagindo
agressivamente contra qualquer tentativa de tirá"-lo da comodidade das
suas verdades estabelecidas.

 

Na convivência social, um dos primeiros fatores de contenção é o
conjunto de regras sociais ~consolidadas que impõe um padrão de
sociabilidade do restaurante granfino, ao boteco de família, da missa ao
estatuto da gafieira.

 

Cada ambiente tem seu conjunto de regras e seus limites. O~machismo e a
homofobia estão restritos a ambientes machistas, onde é de mau tom
defender transexuais. Mas, se saíssem fora da jaula, seriam coibidos por
olhares de reprovação. Nos botecos, as mesas separavam os grupos por
afinidade de opinião. Mas não havia interferência nas conversas, mesmo
por parte de quem ouvisse e reprovasse.

 

Nos ambientes públicos, não era de bom tom o preconceito, a
intolerância. Uma pitada de esquerda social dava até status intelectual.
E~havia um respeito (muitas vezes excessivo) pelo conhecimento técnico.

 

Todas essas barreiras caíram. Hoje em dia, a norma é a grosseria, a
opinião fechada, intransponível como a muralha chinesa, em torno do
senso comum mais primário ou da piração mais louca, como comprovaram
meus dois interlocutores.

 

Quais os fatores que levaram o mundo a essa balbúrdia?

 

\section{Os fatores de confusão}

 

Há um conjunto de fatores muito similar ao que conduziu o Ocidente de
fins do século 19 até a 2a Guerra:

 

• Uma fase de grandes avanços científicos e tecnológicos que não
resultaram em melhoria da condição de vida das populações, levando à
descrença em relação ao pensamento científico, especialmente dos
economistas.

 

• Um financismo desvairado impedindo a consolidação das economias
periféricas.

 

• Dissolução de estados nacionais, guerras internas, promovendo
gigantescos movimentos migratórios.

 

• Os imigrantes promovendo terremotos nas estruturas sociais
estratificadas das nações hospedeiras, com novos valores, novas
informações, novas maneiras de encarar a vida.

 

• O aparecimento de novos meios de comunicação, implodindo a ordem que
repousava nos sistemas tradicionais de mídia.

 

• A falência dos sistemas tradicionais arcaicos de política.

 

A crise atual decorre de uma soma similar de fatores:

 

Fator 1 -- a falência do conhecimento científico

 

A crise de 2008 não apenas matou a ilusão do neoliberalismo como fator
de promoção de desenvolvimento e bem estar. Levou junto a
respeitabilidade do conhecimento científico junto ao público leigo, da
mesma maneira que o atual estado de exceção está desmoralizando o
conhecimento jurídico.

 

A expansão do neoliberalismo, da ampla desregulação financeira, foi
fundada na adesão acrítica e interessada de vastos setores da academia,
especialmente dos economistas -- conforme atestam documentários
produzidos depois da crise nos Estados Unidos. Literalmente, o mercado
comprou a opinião da Academia.

 

O padrão de atuação do mercado, de braços dados com a mídia, sempre foi
a de construir reputações de seus vendedores. Alçados à condição de
celebridades, ajudavam na venda de produtos ou de ideias de seus
empregadores.

 

Nas discussões sobre a desregulação da economia, por exemplo,
economistas medíocres, repetidores de slogans, eram alçados pela mídia à
condição de grandes gurus da economia. Para o universo dos Hommers
Simpsons, um Mailson valia mais que um Paul Krugman.

 

Do mesmo modo, no apogeu da Nasdaq (a bolsa das empresas de tecnologia)
os bancos de investimento fabricavam gurus a torto e a direito,
fornecendo palpites para a manada.

 

O auge foi quando a Goldman Sachs recomendou a compra de ações da
Microsoft logo após a União Europeia tê"-la condenado por práticas
monopolistas. O~ganho do investidor não está em investir no tamanho da
empresa, mas em sua expectativa de crescimento. Aquele episódio, mais a
estabilização do mercado de desktops, decretava o fim do crescimento
exponencial histórico da empresa, registrado em um período de amplo
domínio do Windows.~

Para manter o mesmo ritmo de crescimento, teria que competir com os
japoneses em games, com a Oracle em bancos de dados, com as novíssimas
redes sociais que surgiam.

 

Era apenas uma jogada do banco. Ao perceber que as ações da empresa não
tinham mais atração, preparou o mercado para poder desovar seus estoques
de ações a um bom preço. E~os gurus fabricados pela mídia
norte"-americana ajudaram no jogo.

 

Para tudo isso serviam os gurus. E~toda ~essa catedral de papelão veio
abaixo com a crise de 2008. Menos em países intelectualmente
subdesenvolvidos, onde um economista pode virar gênio sem publicar um
trabalho acadêmico que preste..

 

\section{Fator 2 -- a implosão das regras sociais}

 

No início das redes sociais, perdi uma aposta para o neurologista
Danielle de Riva. Eu acreditava que a Internet e as redes sociais
permitiriam a construção coletiva do conhecimento, com a informação
libertando. Cético, De Riva apostava que liberaria todas as taras, com a
formação de grupos de doenças sociais variadas, de pedófilos a
terroristas.

 

Ganhou.

 

As redes sociais aboliram as barreiras naturais dos ambientes sociais
presenciais. Agora, o sujeito pode entrar em qualquer ambiente virtual
sem ser apresentado, sem os constrangimentos naturais, as regras sociais
consolidadas ~nos contatos presenciais, dando vazão aos seus instintos
mais primários. Liberou geral.

 

Mais que isso, o espírito animalesco passou a encontrar assemelhados e a
se organizar em alcateias, compartilhando as piores intenções e os
piores sentimentos. Saíram do armário, nus e peludos como os homens da
caverna, despidos de todo o verniz social e todos os princípios
civilizatórios acumulados em séculos de civilização.

Do virtual para a contaminação do presencial foi um pulo.

 

\section{Fator 3 -- a opinião leiga}

 

Essas hordas partiram para a guerra armados de slogans primários, mas de
alta eficiência.

 

No trabalho seminal de 1962, em que previu todos os passos do golpe,
Wanderley Guilherme dos Santos analisou o discurso da direita, na época
praticado por Carlos Lacerda. Apesar do primarismo da análise, ironizada
pelos acadêmicos, Wanderley anotava sua enorme eficácia junto às massas
leigas. As massas -- à esquerda ou à direita -- são sensibilizadas por
frases simples, slogans falsos como são as verdades definitivas que
cabem em uma frase.

 

Lembro, com 13 anos de idade, influenciado pelo meu avô udenista,
enfrentando frei Josaphat, do jornal Brasil Urgente, em um debate em
Poços de Caldas:

 

— Que governo é esse que impede a greve dos bagrinhos em Santos, em
defesa da sua sindicalização?, bradei, com uma frase retirada
diretamente da revista Ação Democrática.

 

E o frei, com a mesma impaciência que eu tive com o Hommer Simpsons:

 

— Meu anjinho, você é muito novo para entender dessas questões.

 

O slogan disseminado pela revista armava de um menino de 13 anos a um
adulto para participar de um debate ideológico -- mesmo não tendo o
menor conhecimento sobre o contexto discutido.

 

Dia desses, um conhecido, cientista social, contava o que se passou nos
seus encontros familiares. De repente parentes que nunca se
pronunciavam, por seu escasso conhecimento de temas políticos, passaram
a entrar vigorosamente na discussão com argumentos similares ao do meu
amigo Hommer Simpson. Construiu"-se um verdadeiro manual da idiotia,
conferindo a cada Hommer um tacape para utilizar em qualquer discussão.

 

\section{A utilização da pós"-verdade}

 

Nesse ambiente intelectualmente rarefeito, o discurso político da
direita passou a visar o órgão mais sensível do Hommer Simpson: o
fígado.

 

É o ambiente ideal para o uso do preconceito, a disseminação da
vingança, as bandeiras moralistas, o atropelo de todo o avanço jurídico,
retomando os princípios da Lei de Talião e do estado de exceção -- sob a
aprovação dos humanistas de butique, como o Ministro Luís Roberto
Barroso e o jurista Oscar Vilhena, agora convertidos em arautos do
direito penal do inimigo.

 

Quando esse desastre recai sobre nações institucionalmente pobres, em
que os valores civilizatórios dependem de uma mídia venal, da erudição
vazia e descompromissada de juristas, de um parlamento vergonhoso, de
partidos políticos não"-programáticos, dá no que deu.

 

Não se imagine que o fundo do poço está à vista. A~fragilidade
institucional brasileira, a mediocridade de suas elites pensantes -- à
direita e à esquerda \mbox{---,} a ausência mínima de noção de soberania, de
interesse nacional, de solidariedade nacional, sugerem que o desmonte
nacional pode não ter fundo.
