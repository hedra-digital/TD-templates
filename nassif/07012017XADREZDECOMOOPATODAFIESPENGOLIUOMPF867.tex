\chapterspecial{07/\allowbreak{}01/\allowbreak{}2017 Xadrez de como o pato da \versal{FIESP} engoliu o \versal{MPF}}{}{}
 

\section{Peça 1 -- sobre as responsabilidades do \versal{MPF}}

O Ministério Público Federal teve papel central no golpe parlamentar que
visou, em última instância, reduzir drasticamente as responsabilidades
do Estado e congelar as despesas públicas através da \versal{PEC} 55.

O congelamento implica em meramente corrigir as receitas pela inflação
do período, mantendo o mesmo valor real inalterado por 20 anos.

A lógica anterior consistia em definir um patamar de despesas e
corrigi"-la anualmente pelo índice de preço do período mais o crescimento
do \versal{PIB}.

A ideia civilizatória era a de que o país iria aumentando seus gastos na
medida do aumento da riqueza, como consequência do crescimento do \versal{PIB}.
Descontados os últimos dois anos de Dilma Rousseff com o festival de
subsídios, a regra vinha sendo cumprida.

Voltei aos meus tempos de planilha e montei uma simulação para saber o
impacto da \versal{PEC} 55 sobre o orçamento do \versal{MPF}. Escolhi o \versal{MPF} por sua
responsabilidade no golpe e no desdobramento da \versal{PEC} 55 e por ser um
microcosmo a partir do qual podem"-se se tirar conclusões sobre as
despesas públicas em geral.

Usei um conjunto de hipóteses de orelhada, apenas para chegar aos
chamados grandes números, que permitem vislumbrar tendências. Mas a
planilha permite recalcular tudo em cima de números mais próximos da
realidade.

Primeiro, cumpre ressaltar que, no dia a dia, o \versal{MPF} é uma estrutura
financeiramente bem gerida. No ano passado houve uma queda em
praticamente todos os itens das despesas operacionais, mostrando
responsabilidade..

A estrutura batalha por melhorias salariais, pelos melhores equipamentos
e instalações para seus membros. Mas não desperdiça dinheiro. Por isso o
modelo serve como boa análise de caso extensiva a todo serviço público.

\section{Peça 2 -- os números do \versal{MPF}}

Montei a simulação em cima das seguintes variáveis:


Vamos entende"-las.

\begin{enumerate}
\itemsep1pt\parskip0pt\parsep0pt
\item
  Imagine um crescimento de 5\% ao ano nos quadros da ativa, para
  atender às demandas da sociedade.
\item
  Supus uma taxa de aposentadoria de 2,9\% ao ano. Cheguei a esse número
  pegando o total de 100 e dividindo pelo período de aposentadoria 35.
\item
  Imaginei que 0,5\% dos aposentados morrem a cada ano.
\item
  Do aumento dos aposentados menos o falecimento cheguei à taxa de
  2,36\% de crescimento ao ano da taxa de aposentados.
\item
  Para suprir as vagas de aposentados e dar conta dos novos postos de
  trabalho, a cada ano os concursos tem que acrescentar 7,86\% a mais de
  procuradores.
\item
  Imaginei um custo operacional de 40\% sobre cada procurador na ativa e
  um subsídio médio de R\$ 30.000,00 mensais brutos, sem os
  penduricalhos.
\item
  Usei as hipóteses do procurador ingressando no \versal{MPF} com 25 anos,
  aposentando"-se aos 60 anos e vivendo até os 80 anos.
\item
  Imaginei sua contribuição para a aposentadoria (de 12\%) sendo
  aplicada a uma taxa real de 4\% ao ano ou 0,33\% ao mês.
\item
  Ao final dos 35 anos, caso tivesse aplicado o valor da contribuição, o
  procurador teria um saldo de R\$ 4.981.570,00. Para garantir uma renda
  equivalente à aposentadoria integral até os 80 anos, deveria ter
  acumulado um saldo de R\$ 3.126.921,00. Logo, em termos atuariais o
  contribuinte banca cerca de R\$ 1.854.658,00 por cada procurador
  aposentado.
\end{enumerate}

10.~ Como foi de ouvido, pode"-se recalcular mudando as variáveis.

Mas não é esse o objetivo maior das nossas contas, mas avaliar como
ficaria o \versal{MPF} em geral com a \versal{PEC} 55.

\section{Peça 3 -- os efeitos da \versal{PEC} 55 sobre o orçamento do \versal{MPF}}

Em cima dessas variáveis, tentei estimar o crescimento do orçamento do
\versal{MPF} apenas nos itens subsídios (o nome do salário para servidor público)
+ despesas operacionais. Deixei de lado investimentos.

\imagemmedia{}{/\allowbreak{}home/\allowbreak{}deploy/\allowbreak{}apps/\allowbreak{}dt/\allowbreak{}releases/\allowbreak{}20160911120637/\allowbreak{}public/\allowbreak{}ckeditor\_\allowbreak{}assets/\allowbreak{}pictures/\allowbreak{}content\_\allowbreak{}nassif21484306050.jpg} 

A tabela mostra a situação do orçamento do \versal{MPF} agora, daqui a 10 anos e
daqui a 20 anos.

(Como expliquei, pode ser que os números de procuradores da ativa e de
aposentados não bata. Mas colocando os números reais não vai mudar muito
a conclusão).

\begin{enumerate}
\itemsep1pt\parskip0pt\parsep0pt
\item
  A corporação total (ativa~aposentados) saltará de 2.300 para 5.255.
\item
  No Custo Mês estão os valores totais pagos aos da ativa e aposentados.
\item
  No custo operacional, calculou"-se os 40\% apenas sobre os procuradores
  da ativa.
\item
  No Custo Total Mês~\imagemmedia{}{/\allowbreak{}home/\allowbreak{}deploy/\allowbreak{}apps/\allowbreak{}dt/\allowbreak{}releases/\allowbreak{}20160911120637/\allowbreak{}public/\allowbreak{}ckeditor\_\allowbreak{}assets/\allowbreak{}pictures/\allowbreak{}content\_\allowbreak{}nassif31484306144.jpg} entram os valores dos subsídios
  mais os custos operacionais.
\item
  No Custo Total Ano, o valor mensal x 12 (poderia incluir o
  13\textsuperscript{o}., mas deixei assim).
\item
  Apenas com esses dados, o crescimento anual das despesas do \versal{MPF}, sem
  contar investimentos, será de 4,3\% a 4,4\% ao ano.
\item
  Daqui a 10 anos, para manter o mesmo nível de serviço atual, a despesa
  teria que ser 53\% maior; em 20 anos, em 136\% maior. Com a \versal{PEC} 55,
  ficará congelada. Com isso, o orçamento será de -34\% em relação ao
  atual. Em 20 anos será 57\% menor.
\end{enumerate}

Essa enorme economia não seria possível sem a preciosa colaboração do
próprio \versal{MPF} e do seu Procurador Geral. Mas, hoje em dia, cada \versal{PGR} e cada
procurador querem viver intensamente o presente. O~futuro que exploda
com o \versal{MPF} e o país.

\versal{PS} --- Usei um serviço indiano bem interessante, o Zoho, para publicar
aqui a planilha. Quem quiser, poderá brincar com as variáveis. Só peço
que não baguncem muito os números.

 

 

 
