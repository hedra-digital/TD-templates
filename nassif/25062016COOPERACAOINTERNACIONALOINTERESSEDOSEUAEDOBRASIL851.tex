\chapterspecial{25/\allowbreak{}06/\allowbreak{}2016 Cooperação internacional: o interesse dos \versal{EUA} e do Brasil}{}{}
 

Provavelmente a intenção da~{Folha, ao
entrevistar a brasilianista Barbara Weisntein}~-- da Universidade de
Nova York -- foi mostrar o estado deplorável de corrupção que assola o
Brasil

Mas a boa entrevistadora, Sylvia Colombo, arrancou uma informação que
reforça o que se tem dito aqui sobre a geopolítica da cooperação
internacional (o acordo de parceria entre órgãos de investigação dos
diversos países).

Perguntou sobre as comparações entre Brasil e \versal{EUA}:

\emph{``Os \versal{EUA} têm uma particularidade. Muita coisa que acontece aqui
não conta como corrupção porque está dentro da lei, mas é moralmente
errado e isso tem impacto na situação de pessoas menos privilegiadas'',
respondeu a brasilianista.}

Há tempos vimos alertando para o fato de que a cooperação internacional
se transformou em território da geopolítica norte"-americana.

Os Estados Unidos são historicamente uma sociedade organizada em torno
de interesses corporativos explícitos. O~Departamento de Estado, de
Justiça e o próprio \versal{FMI} sempre estiverem atrelados à lógica corporativa
do país. Desde sempre as grandes multinacionais se constituíram no braço
externo do poder nacional.

Esse pragmatismo norte"-americano criou uma regulação permissiva para as
formas de relacionamento corporações"-poderes públicos. O~lobby é
regulamentado, assim como as contribuições de campanha. Aceita"-se como
natural até a retribuição dos presidentes eleitos aos grupos econômicos
aliados, que chega ao limite das guerras nacionais. É~só lembrar a
guerra do Iraque e as empreiteiras da era Bush.

Em 1952, o Secretário do Tesouro norte"-americano George Humphrey só
concordou em liberar um financiamento ao Brasil depois que a Hanna
Mining foi compensada da perda da exploração do manganês do Amapá.

O pragmatismo norte"-americano chega ao ponto de permitir ao presidente
da República conceder o indulto a empresários acusados de corrupção, em
nome do chamado interesse nacional.

\section{O conceito de corrupção nos \versal{EUA}}

Quando o narcotráfico e o terrorismo obrigaram a um cerco econômico
sobre o crime organizado, houve restrições à ação corruptora de empresas
norte"-americanas, mas limitadas ao pagamento de propinas. Todos os
demais instrumentos de guerra comercial continuaram sendo empregados à
larga, principalmente o exercício do poder de Estado da nação mais
poderosa do planeta.

Dia desses, uma fonte do \versal{MPF} admitiu a um colunista da Folha haver
interesses econômicos por trás da cooperação internacional (ufa!). Mas
explicou que os \versal{EUA} apenas querem igualdade de condições, depois que a
legislação antiterror obrigou a restringir a ação corruptora de suas
multinacionais.

Igualdade de condições? Como haver igualdade de condições se as mesmas
práticas são ~toleradas por lá e colocadas sob investigação por aqui?

Nesses tempos de Lava Jato, os vazamentos tentaram criminalizar de tudo,
de ações diplomáticas na África até financiamentos às exportações,
financiamentos do \versal{BNDES}, incentivos regionais.

Nem lhes chamou a atenção o fato do Departamento de Justiça, sem que
fosse solicitado, passasse dados sobre a corrupção na Eletronuclear.

A insensibilidade em relação ao destino das empresas, dos empregos e da
tecnologia investida é espantosa. A~ideia de criminalizar pessoas
jurídicas -- e não pessoas físicas -- não encontra respaldo em nenhuma
legislação moderna. Não é racional.

A visão conspiratória de que, salvando as empresas, salvam"-se os
acionistas, já que os negócios são ~todos imbricados, não resiste a um
mero exercício de lógica:

1.~~~~ Os acordos de leniência pressupõem o ressarcimento ao Estado de
valores identificados com a corrupção. Tem que haver pagamento.

2.~~~~ Se tem que haver pagamento, e se o governo recebe, o dinheiro ou
sai do controlador (vendendo a empresa) ou da empresa.

Na medida em que o combate à corrupção se torna um valor maior, o \versal{MPF}
tem que desenvolver uma jurisprudência interna, uma discussão mais
objetiva sobre disputas comerciais, para diferenciar a corrupção
propriamente dita das estratégias comerciais e do chamado interesse
nacional. A~Lei de Leniência é um primeiro passo.

Não faltam no \versal{MPF} grandes procuradores especializados em direito
econômico para enriquecer a discussão.
