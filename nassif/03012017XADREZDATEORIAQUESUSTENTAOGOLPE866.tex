\chapterspecial{03/\allowbreak{}01/\allowbreak{}2017 Xadrez da teoria que sustenta o golpe}{}{}
 

\section{Peça 1 -- as ideias e a conspiração}

Nessa geleia geral em que se transformou o golpe, uma boa análise
estratégica exige a tipificação mais detalhada do papel de cada
personagem.

O poder de fato está em uma entidade chamada mercado.

É o mercado quem forneceu o fio agregador do golpe, o objetivo final, o
componente ideológico capaz de criar uma agenda econômica alternativa,
em torno dos quais se agruparam a mídia, o \versal{PSDB} e se induziu à
politização de instituições, como o \versal{STF} (Supremo Tribunal Federal) e o
\versal{MPF} (Ministério Público Federal), montando o círculo inicial que passou
a dar as cartas no governo Temer e, possivelmente, no pós"-Temer.

É a parte mais eficiente do golpe, seguindo um roteiro fartamente
descrito em obras como ``A Teoria do Choque'' de Naomi Klein. Confira, a
propósito, o ``Xadrez da Teoria do Choque e do Capitalismo de Desastre''
(\url{https:/\allowbreak{}/\allowbreak{}goo.gl/\allowbreak{}v\versal{ZYV}zy)}.

Dado o golpe, reza a teoria (importada da Escola de Chicago), se tem
seis meses para emplacar as medidas mais drástica e consolidar o novo
modelo.

A nova equipe econômica avançou como um bólido sobre os instrumentos
econômicos do Estado, com um plano de ação completo, meticulosamente
preparado desde que o \versal{PMDB} apresentou a tal Ponte Para o Futuro.

Não se trata de um plano de estabilização, capaz de reverter a crise,
mas de um desmonte do Estado que aprofundará a crise. É~a estratégia da
terra arrasada, visando sepultar qualquer vestígio do antigo modelo,
independentemente dos custos para o país e seu povo.

\begin{itemize}
\itemsep1pt\parskip0pt\parsep0pt
\item
   Apresentou a \versal{PEC} 55 que, aprovada, acaba com qualquer
  possibilidade de política fiscal anticíclica e manieta todos os
  futuros governos.
\item
   Se vale da crise fiscal para garrotear os governos estaduais.
\item
   Esvaziou o \versal{BNDES}, fazendo"-o pagar antecipadamente R\$ 100
  bilhões ao Tesouro.
\item
   Ampliou a degola das empreiteiras nacionais, proibindo
  financiamento à exportação de serviços e às empresas mencionadas na
  Lava Jato.
\item
   Prepara"-se para vender a carteira de ações do \versal{BNDES} na bacia das
  almas.
\item
   Montou uma queima de ativos da Petrobras, em um momento em que
  todos os ativos nacionais estão depreciados pela crise e os ativos
  petrolíferos depreciados pelas cotações de petróleo. Vende para
  reduzir passivo. Deixa de lado todos os investimentos na prospecção,
  nas refinarias e nos estaleiros (que garantiriam a expansão imediata e
  a longo prazo) para quitar antecipadamente (!) financiamentos
  contratados junto ao \versal{BNDES}. Nenhuma empresa com crise de liquidez
  quita antecipadamente financiamentos. No máximo, reestrutura passivos.
\item
   Começou a esvaziar o \versal{FGTS}, facilitando o saque das contas.
\item
   Com a ajuda da Lava Jato, jogou a pá de cal na cadeia produtiva
  do petróleo e gás, no sonho dos estaleiros nacionais, na expansão do
  capitalismo brasileiro para África e América Latina. Busca a
  destruição da maior empresa privada brasileira, a Odebrecht, a
  empreiteira que mais incomodava os concorrentes norte"-americanos.
\item
   Na diplomacia, acabou de matar o protagonismo do Itamarati.
\item
  Para atingir seus objetivos, o sistema tem permitido a proliferação
  das maiores jogadas que o Congresso e o Executivo já ousaram em sua
  história recente:
\item
   A iniciativa de entregar às teles os ativos acumulados durante o
  período de concessão. Aliás, o senador Jorge Viana (\versal{PT}"-Acre) deve
  explicações a seus eleitores e admiradores.
\item
   A jogada de transformar multas das teles em obrigação de
  investimento, reeditando estratagema utilizado pelo inacreditável
  Paulo Bernardo, quando Ministro das Comunicações. Na prática, equivale
  a perdoar as dívidas, já que os investimentos teriam que ser feitos de
  qualquer maneira, por obrigação contratual ou exigência de mercado.
\item
   A compra gigantesca de produtos Microsoft, interrompendo o
  trabalho de disseminação do software livre.
\item
   As jogadas escandalosas do senador Romário, de depositar nas
  mãos das \versal{APAE}s e das Sociedades Pestalozzi o controle de toda a
  educação inclusiva.
\item
   A tentativa de emplacar os cassinos e casas de bingo.
\item
   A enxurrada de dinheiro público despejado nos veículos de mídia,
  cujo melhor exemplo é a campanha milionária de prevenção da Zika e
  falta de remédios para as grávidas.
\item
   ~A \versal{MP} 754 que faculta à \versal{CMED} (Câmara de Regulação do Mercado de
  Medicamentos) autorizar reajustes a qualquer momento. A~lei que criou
  a \versal{CMED}, em 2003, autorizava"-a a determinar apenas reajustes anuais de
  preços. Agora, haverá reajustes, a qualquer momento, dependendo de uma
  plêiade de Varões de Plutarco: Ricardo Barros, Ministro da Saúde,
  Alexandre Moraes, da Justiça, Henrique Meirelles, da Fazenda, o pastor
  Marcos Pereira, do Desenvolvimento, Indústria e Comércio, e Eliseu
  Padilha, da Casa Civil, todos homens piedosos.
\item
   ~A tentativa de jogar a Fiocruz sob o comando de Ricardo Barros
  e Temer.~~~~~~~~~
\end{itemize}

\section{Peça 2 -- a economia de um país retardatário}

Toda essa conspiração política repousa em um edifício teórico que está
sob forte processo de questionamento em países culturalmente mais
avançados. No Brasil, os temas se tornaram matéria de fé.

Os ideólogos desse manual -- tão velho quanto a Escola de Chicago -- são
os economistas Marcos Lisboa e Samuel Pessôa, ambos competentes em suas
funções.

Lisboa é um brilhante economista que, na gestão Antônio Palocci, foi
responsável por vários avanços microeconômicos relevantes. Foi alçado à
condição de guru pelo megainvestidor Jorge Paulo Lehman. Ao perceber que
as eleições de 2002 marcariam o fim do período tucano, Lehman enganchou
Lisboa na campanha de Ciro Gomes, por indicação de Alexandre Scheinkman,
o brasileiro que dirigia o prestigioso Departamento de Economia da
Universidade de Chicago. Depois, coube a mídia o trabalho de, em pouco
tempo, torna"-lo conhecido e com fama de gênio -- seguindo o roteiro
conhecido de criação de gurus, mesmo sem uma produção acadêmica robusta.

Eleito Lula, o primeiro aceno de seu Ministro da Fazenda Antônio Palocci
ao mercado foi a nomeação de Lisboa como Secretário Executivo da
Fazenda. Quando canalizou seu talento para as questões microeconômicas,
conseguiu feitos notáveis, como o de destravar o Sistema Financeiro da
Habitação.

Agora, seu papel é o desmontar o Estado nacional e implementar um modelo
de mercado, não um plano de estabilização, menos ainda um projeto de
desenvolvimento equilibrado, que junte as virtudes de mercado com a de
Estado. O~objetivo único é ideológico, impor terra arrasada em todos os
instrumentos de intervenção do Estado na economia -- mesmo aqueles
consagrados em todos os países civilizados, e peças centrais na
recuperação da economia, como bancos de desenvolvimento, ou de comércio
exterior, compras públicas, financiamentos à inovação etc. -- ainda que
à custa de um aprofundamento maior da crise.

Dilma não soube transformar o Estado em um articulador do mercado.
Lisboa simplesmente quer abolir o Estado, como se fosse possível a um
país da dimensão do Brasil depender do mercado como agente originário
das expectativas, algo que nem os Estados Unidos ousam. E~tudo isso
jogando com o destino de milhões de trabalhadores, de empresários,
jogando fora anos de investimento em novos processos, novas tecnologias.

É chocante como a chamada pós"-verdade se infiltra até nos círculos tidos
como bem informados, com afirmações sobre o ajuste fiscal na União
Europeia, quando o próprio \versal{FMI} está revendo os problemas dos ajustes
recessivos.

\section{Peça 3 -- a política econômica de manual}

Durante o longo período de neoliberalismo -- que se inicia em 1972, com
a desvinculação das cotações do ouro e do dólar -- criou"-se a fantasia
de que a economia global se articularia passando ao largo das políticas
nacionais. Aboliu"-se a história econômica como vetor de análises. E, com
o advento dos microcomputadores e das planilhas, entrou"-se na era do uso
abusivo de estatísticas e fórmulas ilusórias em cima de macro"-números
que encobrem as realidades nacionais e de blocos, e que só trabalham um
conceito de equilíbrio utópico, sem nenhum diagnóstico para os grandes
stress econômicos.

Especialmente nas ciências humanas --a medicina, as ciências sociais ou
a economia -- as teorias são instrumentos para se analisar a realidade
local e suas circunstâncias. Não existem regras universais. O~exame de
laboratório não substitui a análise do paciente pelo médico, assim como
a teoria econômica não é um manual de aplicação universal. Para cada
circunstância, há um conjunto de medidas específicas.

A crise de 2008 abriu os olhos do primeiro time de economistas dos
países centrais. Percebeu"-se que a economia é muito mais complexa do que
as realidades captadas em modelos matemáticos que compensavam a escassa
sofisticação analítica com excesso de estatística.

Vale a pena ler a entrevista de Eric Beinhocker na Carta Capital
(\url{https:/\allowbreak{}/\allowbreak{}goo.gl/\allowbreak{}DirQsb)}. Para cada circunstância, há que se apelar
para os instrumentos de política econômica adequados, sem part"-pris
ideológico. E~recorrer também ao conhecimento empírico, especialmente
nos casos de stress agudo da economia que criam situações não
identificadas na história econômica recente. De tal modo, que o
exercício da política econômica é um misto de técnica e arte, de teoria
e intuição.

Nos 8 anos de Fernando Henrique Cardoso, por exemplo, todas as crises
econômicas, quase todas nas contas externas, eram tratadas do mesmo
modo, com ajustes fiscais severíssimos, que apenas agravavam a recessão.
A~política de juros e de câmbio produziu um dos períodos de maior
estagnação econômica da história.

Em 2008, Lula decidiu enfrentar a mega"-crise que se avizinhava
recorrendo a todos os instrumentos possíveis para reanimar a economia.
Saiu consagrado. E~também deu sorte. Se a crise não catapultasse o dólar
para as alturas, provavelmente o país teria quebrado em 2008, tal o
rombo nas contas externas promovido por uma política cambial imprudente
que, além disso, prorrogaria estagnação do período \versal{FHC}.

A crise do governo Dilma foi decorrência da incapacidade de montar
cenários e estratégias alternativas para o fim do ciclo das commodities.
Deveu"-se também à elevação imprevista de juros em 2013, à sucessão
infindável de subsídios que fragilizaram a parte fiscal e, depois, um
ajuste fiscal severíssimo, pró"-cíclico, que aprofundou a crise: medidas
tomadas nos momentos errados.

Em fins de 2015, quando aparentemente conseguira chegar a um diagnóstico
mais razoável, com uma estratégia racional de saída da crise, e os
analistas previam a recuperação a partir do segundo semestre, foi
fuzilada pela ação conjunta da Lava Jato e do Procurador Geral da
República, associados ao boicote do \versal{PSDB} e de Eduardo Cunha na Câmara e
no Senado.

As lições que ficam é que as medidas econômicas não são virtuosas em si:
dependem das circunstâncias em que são implementadas. Há um conjunto de
princípios de responsabilidade fiscal a serem seguidos por qualquer
governo. Mas, em períodos de recessão, a política fiscal precisa ser
anticíclica -- através do aumento dos gastos públicos \mbox{---,} caso
contrário a cada corte de despesas se seguirá uma queda maior da
receita. Em tempos de economia aquecida, pratica"-se política fiscal mais
severa.~Nenhum economista com um mínimo de bom senso deixaria de
considerar essas questões.~

Esse quadro era nítido no início de 2015, quando Joaquim Levy deu inicio
a seu plano suicida. Uma dose de conhecimento empírico seria suficiente
para mostrar que os cortes fiscais aprofundariam ainda mais a recessão,
ampliando o déficit fiscal via queda de receita.

Levy preferiu acreditar em estudos dos anos 90, que supostamente
atestariam que cortes de despesas têm pouco impacto no \versal{PIB}. Nem se deu
conta que, em 2012, o próprio \versal{FMI} tinha revisto essas conclusões.

Para os cabeças de planilha, conhecimento empírico não é ciência e as
experiências históricas não tem validade. Valem apenas as estatísticas
baseadas em séries históricas contemporâneas.

A cada situação nova, criam desastres monumentais pela incapacidade de
só recorrer a manuais montados em cima de situações passadas. Os
desastres só serão inteiramente compreendidos quando estudados a
posteriori. E, como aqui é o país do Macunaíma, nem mesmo grandes erros
recentes -- como o pacote Levy -- servem de lição para o pacote Lisboa.

\section{Peça 4 -- próximas etapas}

A fantasia do pote de ouro no fim do arco"-íris acabou. A~história de que
bastaria tirar Dilma para a economia se recuperar já está sendo
percebida como blefe pelo cidadão comum.

Tem"-se um presidente tão desmoralizado que, a maneira que a revista Veja
encontrou para retribuir o megapacote publicitário, foi uma
capa"-fantasia com a senhora Temer, tal a falta de atratividade em
qualquer outro aspecto do primeiro marido.

A economia não irá se recuperar com esse viés ideológico predominando na
política econômica. Pelo contrário, há no horizonte próximo o pior dos
mundos: o default dos Estados.

Em março o \versal{STF} (Supremo Tribunal Federal) deverá liberar os inquéritos
contra políticos. A~quantidade de jogadas planejadas pela camarilha de
Temer e pelo Congresso aumentará ainda mais a fragilidade do governo.

A oposição vê nas eleições diretas a saída para a crise. Ocorre que
Sérgio Moro, os procuradores da Lava Jato e o \versal{TRF}4 têm lado político. Ao
menor sinal de renascimento de Lula, tratarão de impugnar sua
candidatura através da condenação relâmpago em 1\textsuperscript{a}~e
2\textsuperscript{a}~instância.

Por outro lado, o presidente do \versal{TSE} (Tribunal Superior Eleitoral) Gilmar
Mendes, deixa transparecer seu cansaço com o Supremo e a possibilidade
de aceitar algum cargo executivo futuramente.

No momento, a aposta com maior probabilidade é a degola de Michel Temer
seguido de eleições indiretas sob controle do mercado"-\versal{PSDB}, com o \versal{PGR}
cumprindo o papel de agente intimidador de políticos recalcitrantes.

Há muita confusão e poucos personagens, para permitir a montagem de
cenários mais precisos.
