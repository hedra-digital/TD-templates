\chapterspecial{01/\allowbreak{}12/\allowbreak{}2016 Xadrez da República dos Procuradores}{}{}
 

Poucas vezes, na história de uma República permanentemente sujeita a
golpes, viu"-se uma espetáculo tão deprimente de falta de compostura
institucional, uma ópera bufa da pior espécie.

O país institucional tornou"-se uma verdadeira casa da Mães Joana, com
personagens indignos de representa"-lo~~à frente do Executivo, do
Congresso, do Supremo Tribunal Federal, da Procuradoria Geral da
República, do Judiciário e dos partidos políticos.

Brinca"-se com o poder, derruba"-se um presidente eleito, arma"-se contra o
interino que aboletou"-se do cargo, fazem cálculos sobre o momento de
impichar a chapa, se agora, se no ano que vem, valem"-se de seu poder
institucional para toda sorte de abusos.

Procuradores atuam politicamente; deputados lutam para legalizar o
crime; Ministros do Supremo e o Procurador Geral da República manipulam
prazos de inquéritos para proteger aliados; juízes de 1a instância
autorizam grampos a torto e a direito.

Mas era previsto, tal o grau de desordem institucional plantada no país
pela abulimia do \versal{STF}, ao permitir o atropelo da Constituição. Deve"-se ao
Supremo esse vale"-tudo.

Cada grupo deu sua contribuição para o golpe, Sérgio Moro e Rodrigo
Janot vazando grampos ilegais, a imprensa no exercício amplo da
pós"-verdade, o Supremo acovardando"-se e abrindo mão de seu papel de
guardião da Constituição e a presidente incapaz de defender seu próprio
mandato.

Consumado o golpe, sem dispor mais do agente aglutinador, passou"-se a
disputar o butim do poder. E~agora chega"-se a esse vale"-tudo vergonhoso,
sem um estatuto da gafieira para discipliná"-lo minimamente.

Vamos entender um pouco mais esse Xadrez da Ópera Bufa.

\section{Peça 1 -- o tempo político na era das redes sociais.}

Nessa era das redes sociais, das notícias online, o tempo político
tornou"-se impressionantemente curto. Processos que, antes, levavam meses
para amadurecer, agora acontecem em questão de dias ou horas.

Qualquer fato relevante espalha"-se em minutos pela opinião pública de
todo o país. Não é necessário mais aguardar a edição impressa do jornal
no dia seguinte ou a edição do Jornal Nacional à noite.

Reverberando em tempo real, o noticiário acelera não apenas a tomada de
consciência como a tomada de decisões.

É um dado relevante para nossos cenários, inclusive na análise do tempo
político do golpe e da atual onda conservadora"-liberal.

Mal se saiu do golpe, o jogo começa a afunilar e a tornar mais nítidos
os personagens reais do novo poder: a aliança
\versal{PSDB}"-\versal{PGR}"-\versal{STF}"-mídia"-mercado. Eles darão as cartas daqui para frente.

\section{Peça 2 --o \versal{PSDB} passa a tutelar Temer}

O primeiro lance foi em cima do Executivo, da camarilha de Michel Temer.

No dia 24 de novembro saiu a notícia das gravações do ex"-Ministro da
Cultura Marcelo Calero com várias autoridades do governo -- incluindo
Temer.

No mesmo dia, matamos a charada, no ``Xadrez do golpe no golpe͟''
(\url{https:/\allowbreak{}/\allowbreak{}goo.gl/\allowbreak{}f6\versal{BF}fS}), ao mostrar os vínculos de Calero com o
\versal{PSDB}.

No dia seguinte, haveria um almoço entre Temer e Fernando Henrique
Cardoso. Analisando a reação da Globo em relação ao episódio, o xadrez
ficou nítido. De um lado, a Globo incensando a coragem de Celeró e dando
ampla visibilidade ao encontro do \versal{PSDB}, com a cobertura centrada na
figura de Fernando Henrique Cardoso; do outro, no Jornal Nacional uma
catilinária que não poupou Temer.

Juntando peças no ``Xadrez do homem que delatou Temer''͟
(\url{https:/\allowbreak{}/\allowbreak{}goo.gl/\allowbreak{}wsMq0D)}, a conclusão era quase óbvia: usaram o
tombamento pelo \versal{IPHAN} do Secretário da Presidência Geddel Vieira Lima,
para a Globo montar sua dramaturgia com Celeró visando colocar Michel
Temer sob a curadoria do ex"-presidente Fernando Henrique Cardoso.

Ontem mesmo -- primeiro dia útil após o almoço com \versal{FHC} -- o pequeno
Temer colocou em campo o Secretário Moreira Franco para negociar com o
presidente do \versal{PSDB}, Aécio Neves, uma maior participação do \versal{PSDB} no
governo.~~E aceitou a condição apresentada por \versal{FHC}, do \versal{PSDSB} passar a
participar da formulação das políticas de governo
(\url{https:/\allowbreak{}/\allowbreak{}goo.gl/\allowbreak{}Kedkne}). Tudo devidamente combinado com \versal{FHC} no
almoço que aconteceu um dia depois de Caleró botar a boca no trombone.

   

Nas próximas semanas haverá reforma ministerial contemplando a nova
composição de forças. Provavelmente dançará também Eliseu Padilha, que
supôs ter comprado o silêncio da mídia com sua bolsa"-mídia. Os pactos de
tinta da mídia são mais flexíveis que os pactos de sangue da máfia. Esse
tipo de barganha garante a blindagem somente até a véspera do último
dia.

Em um governo em que há corrupção até na liberação de edifícios
residenciais, como manter no cargo um Ministro com a ficha de Padilha,
envolvido em falcatruas até em prefeituras do interior, sabendo que
passam por ele todas as indicações a cargos públicos, incluindo as
diretorias do Banco do Brasil e da Caixa Econômica Federal? As grandes
corrupções institucionais, complexas, sofisticadas, ao largo da
compreensão da opinião pública, não podem conviver com ladrões de
galinha. A~diferença entre a camarilha e os profissionais é a mesma
entre, digamos, um Carlinhos Cachoeira e um Madoff.

Por sua vez, um Padilha custa cerca de dez Geddel. E~o governo Temer não
tem consistência para suportar nem meio Geddel a mais.

Os valentes conquistadores que, nos primeiros dias no poder, ordenaram
uma Noite de São Bartolomeu no serviço público, sob pressão expõem sua
verdadeira dimensão: Geddel aos prantos pelos corredores do Palácio,
Padilha internado com pressão alta.

Mas o novo tempo de jogo está apenas começando. E~nem se incluíram as
perguntas de Eduardo Cunha e Michel Temer em seu julgamento pelo juiz
Sérgio Moro.

\textbf{Peça 3 -- o Judiciário tenta submeter o Congresso}

Não basta o enquadramento de Temer. A~ofensiva seguinte é em cima do
Congresso, no embate em torno das tais 10 Medidas contra a corrupção e
da Lei Contra Abuso de Poder.

Com ou sem Geddel, e com Padilha na linha de fogo, a bancada do \versal{PMDB},
somada ao Centrão, não entregaria facilmente a rapadura.

É por aí que entra o fator Judiciário.

As 10 Medidas não visam apenas conferir maior efetividade no combate ao
crime: visam a conquista do poder institucional pelo Ministério Público
e pelo Judiciário.

Estão em jogo as delações da Odebrecht e das demais empreiteiras. A~lista de nomes conferirá um poder inédito ao Procurador Geral da
República. Com o apoio da Globo, o \versal{PGR} tem o poder de engavetar
denúncias, inquéritos, definir o ritmo dos inquéritos em andamento,
escolher quem será ou não processado.

Além de livrar os seus do crime do caixa dois, o Congresso não quer
deixar os procuradores com esse poder ilimitado nas mãos, ainda mais
agora que ficou nítido que disputam efetivamente o poder.

Cada procurador, aliado a um juiz de 1\textsuperscript{a}~instância, tem
poder de mandar para a cadeia qualquer pessoa sem prerrogativa de foro,
expô"-la à humilhação pública, grampeá"-la e divulgar os grampos -- como
ocorreu no episódio Garotinho -- com a garantia de que será blindado
pelos escalões superiores. Não é por falta de lei. É~por solidariedade
de classe.

No Congresso, as únicas lideranças capazes de fazer frente a esse poder
avassalador do \versal{MPF}"-Globo seriam o presidente Renan Calheiros e o líder
da maioria Romero Jucá, ambos donos de uma biografia política polêmica.

Na quarta"-feira, Renan jogou sua grande cartada -- votar o regime de
urgência para a Lei de Abuso de Poder \mbox{---,} mas o Senado refugou. O~temor
dos Senadores -- tanto do \versal{MPF} quanto da opinião pública -- foi maior e
deixaram seu líder ao relento.

Na quinta"-feira, o Supremo analisará uma das ações contra Renan. O~caso
está nas mãos do Ministro Luiz Edson Fachin. Com a tibieza demonstrada
pela casa, nos últimos tempos, e com a frente ampla de defesa dos
juízes, é possível que Renan seja degolado. E~Jucá virá atrás. Ontem
mesmo, a Lava Jato tratou de retaliar e vazar denúncias contra Jucá.

Não que não mereçam. Mas, de imediato, se os dois comandantes
efetivamente forem deixados fora de cena, não haverá poder capaz de se
contrapor ao poder quase absoluto do Ministério Público. Abre"-se espaço
para a pessedebização final do governo Temer e para o início efetivo da
República dos Procuradores.

Mas como o homem põe e o destino dispõe, há o agravamento da crise
política colocando um complicador a mais no nosso xadrez.

\textbf{Peça 4 -- a ideologia dos economistas do \versal{PSDB}}

Cada escola de pensamento costuma de embaralhar em seus dogmas
ideológicos. Os desenvolvimentistas acreditaram que o excesso de
subsídios resultaria em um aumento da atividade que compensaria a queda
de receita. Os ortodoxos acreditam piamente no papel das políticas
monetária e fiscal, como únicos instrumentos de gestão da economia.

Mas há uma diferença fundamental entre os monetaristas históricos e a
geração de economistas de mercado que se torna hegemônica a partir do
plano Real usando o \versal{PSDB} como ``cavalo''. Os velhos economistas
ortodoxos, monetaristas, utilizam suas ferramentas teóricas tendo como
objeto de análise a realidade, assim como sua contraparte, os
desenvolvimentistas. Ou seja, desenham as estratégias que consideram
mais adequadas para o país.

Já os economistas de mercado, tendo como base a \versal{PUC}"-Rio e como
inspiradores os economistas do Real, sempre atuaram ideologicamente
pensando apenas nas estratégias de fortalecimento do mercado, que criem
novos negócios, que permitam o predomínio do financista em relação ao
restante da economia, independentemente dos efeitos sobre o país.

Banco Central, Fazenda e \versal{BNDES} já trabalham assim, procurando

\begin{enumerate}
\itemsep1pt\parskip0pt\parsep0pt
\item
  Instituir o teto para despesas, liberando o orçamento para os próximos
  vinte anos, para pagamento exclusivo dos juros e amarrando as mãos dos
  futuros governantes para qualquer outra tentativa de políticas
  pró"-ativas.
\item
  Mudar~o comando das grandes obras públicas, das empreiteiras para o
  mercado. Para tanto, além de esterilizar R\$ 100 bilhões do \versal{BNDES},
  estão proibindo"-o de financiar exportações de serviços ou mesmo de
  financiar qualquer empresa que esteja fichada na Lava Jato.
\item
  Ao mesmo tempo, articulam a criação de fundos para trabalhar dívidas
  públicas renegociadas, visando servir de lastro para os fundos de
  infraestrutura que passarão a comandar os investimentos.
\item
  O ajuste fiscal produzirá uma razia na atividade econômica, já afetada
  pela maior recessão da história. Pensa"-se, com isso, em reduzir a
  resistências às reformas e, ao mesmo tempo, derrubar preços de ativos,
  permitindo um redesenho da economia através dos processos de fusões e
  aquisições capitaneados pelos grandes fundos de investimento.
\item
  Manutenção dos juros altos e do câmbio baixo, para potencializar os
  ganhos do mercado.
\end{enumerate}

Algumas ideias são razoáveis, outras meramente ideológicas. Mas o
conjunto final é catastrófico: visando a disputa ideológica, de eliminar
definitivamente as empreiteiras como fontes de influência e de poder,
criam um vácuo na atividade econômica

Mesmo assim, esses economistas são as únicas fontes de idéias do \versal{PSDB}.
Os comandantes do partido -- Aécio, Serra, Alckmin, Aloysio, José Anibal
-- são incapazes de formular uma ideia estratégica sequer. Seu único
papel é o de meter"-se em disputas braçais contra os adversários, tarefa
que exige muita corda vocal e pouco cérebro.

\section{Peça 5 -- o recrudescimento da crise}

Há alguns pontos centrais na política econômica, comuns a qualquer
escola de pensamento.

Quando a economia está despencando, a retomada depende de alguns fatores
de demanda:

\begin{enumerate}
\itemsep1pt\parskip0pt\parsep0pt
\item
  Consumo das famílias
\item
  Consumo do Estado.
\item
  Exportações.
\item
  Investimentos
\end{enumerate}

Nenhuma dessas pré"-condições está presente.

Nenhum economista de peso aposta na recuperação da economia. O~fato do
governo ter jogado para o segundo semestre de 2017 significa apenas que
não há nenhuma perspectiva de recuperação no curto prazo.

O que diz o economista Afonso Celso Pastore (https:/\allowbreak{}/\allowbreak{}goo.gl/\allowbreak{}kFyv\versal{DQ})

\emph{Nada disso seria uma surpresa (a maior recessão dos útimos 25
anos) para quem evitasse fazer previsões sobre o \versal{PIB} dando um peso
excessivo aos índices de confiança, em vez de ponderar as perspectivas
das exportações, do consumo das famílias e da formação bruta de capital
fixo.}

\emph{Com as exportações mundiais e os preços de commodities em queda,
não podemos esperar que as exportações brasileiras impulsionem a
retomada do crescimento. Nem o consumo das famílias poderá exercer essa
função nos próximos trimestres, quer porque, após o encerramento da
recessão o nível de emprego e os salários ainda sofrerão quedas, quer
porque os bancos deverão continuar retraídos na concessão de crédito.}

\emph{Resta esperar que a retomada do crescimento venha dos
investimentos em capital fixo, mas, na grande maioria dos setores, há
uma enorme capacidade ociosa, e assistimos a um número recorde de
empresas em recuperação judicial.~}

Delfim Neto é mais sintético, mas com o mesmo pessismismo
(https:/\allowbreak{}/\allowbreak{}goo.gl/\allowbreak{}lh56Yh)

\begin{quote}
\emph{Há sérias dúvidas, por exemplo, sobre a eficácia da política
anunciada ``urbi et orbi'' que teríamos em 2016 uma política fiscal
fortemente contracionista.}

\emph{Primeiro, porque assustou o setor privado que sofreu o
contracionismo efetivo de 2015 e viu a demanda global desabar e,
segundo, porque há sérias dúvidas se ela será, de fato, contracionista.}

\emph{Quanto à política monetária, esta, sim, tem sido restritiva: houve
aumento da taxa de juro real e recusa a enfrentar a necessidade de
sustentar uma taxa de câmbio real competitiva e relativamente estável, o
que inibe o investimento e as exportações industriais, dois vetores do
crescimento.}

\emph{Sem uma acomodação do crédito para mitigar a alavancagem do setor
privado e sem a garantia de uma taxa de câmbio real adequada, é muito
pouco provável que se restabeleça uma ``expectativa'' de crescimento e
que volte à vida a indústria nacional.}

\emph{E, sem elas, o equilíbrio fiscal, apesar de ser absolutamente
necessário, continuará apenas uma ilusão…}
\end{quote}

A rigor, qual a única instituição que está efetivamente preocupada com o
quadro econômico? Justamente o Senado de Renan Calheiros:

\emph{Em face das crises recorrentes, o presidente do Senado Federal
reitera a imperiosidade de uma agenda a fim de superar o agravamento da
situação econômica que penaliza toda a sociedade brasileira.
\redondo{[…]}}

\begin{quote}
\emph{Segundo o presidente, o ajuste que está sendo implementado é uma
obrigação para fazer frente ao momento econômico, mas precisa ser
complementado com medidas de retomada da atividade econômica, geração de
empregos, recuperação dos investimentos e, o principal, a redução dos
juros. Não é somente o limite de gastos e a reforma da previdência.
\redondo{[…]}}
\end{quote}

\section{Peça 6 -- desfechos possíveis}

Seja qual for o desfecho desses embates \versal{PSDB} x Temer, Congresso x
Justiça, a única maneira de superar a crise será um pacto nacional para
consolidar a única política econômica capaz de tirar o país do atoleiro:

\begin{enumerate}
\itemsep1pt\parskip0pt\parsep0pt
\item
  Aumento dos gastos públicos nas grandes obras públicas e nos programas
  sociais.
\item
  Uso dos bancos públicos -- \versal{BNDES}, \versal{BB} e \versal{CEF} -- para permitir a
  renegociação dos passivos das empresas.
\item
  Redução acelerada da Selic e manutenção do câmbio em patamar
  competitivo.
\item
  Agilização dos programas de concessões, assim que a recuperação da
  demanda definir um cenário mais favorável.
\end{enumerate}

A questão é que só se chegará a esses pontos quando a crise for vista
como suficientemente ameaçadora. A~informação que vale um bilhão é:
quando se chegará no fundo do poço que dispare o gatilho do bom senso
sobre o país?
