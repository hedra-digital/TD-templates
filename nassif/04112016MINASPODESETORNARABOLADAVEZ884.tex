\chapterspecial{04/\allowbreak{}11/\allowbreak{}2016 Minas pode se tornar a bola da vez}{}{}
 

\section{Peça 1 -- a falta de propostas da direita}

Tem"-se a seguinte situação em relação aos chamados governos
progressistas.

De um lado, a falta de perspectivas imediatas de volta ao poder. Ao
mesmo tempo, um governo sem a menor noção de políticas públicas, somado
a um pensamento neoliberal raso, incapaz de ir além do fiscalismo tosco
da \versal{PEC} 241.

Muitos e muitos anos atrás, o \versal{PSDB} era capaz de formular políticas
alternativas, dentro da ideia de Estado enxuto, porém forte. Hoje em
dia, só sabe exercitar a negação pura e simples do Estado.

Preso a superstições ideológicas, nos próximos dois anos o governo
\versal{PSDB}"-\versal{PMDB} irá praticar o ajuste fiscal e uma política monetária de juros
altos e de câmbio baixo em um ambiente recessivo Matará, com uma só
tacada, três motores de demanda: a demanda pública, as exportações e os
novos investimentos privados, que não conseguirão concorrer com uma taxa
de retorno real de 8 a 10\% ao ano.

Com isso, a aliança \versal{PSDB}"-\versal{PMDB} chegará em 2018 sem recuperar a economia e
sem um projeto de país.

De seu lado, a oposição terá que mostrar suas vitrines de políticas
alternativas. O~caminho é através dos estados, o que joga uma enorme
desafio"-oportunidade"-responsabilidade para os governadores.

\section{Peça 2 -- o fator Minas Gerais}

Há tempos Minas Gerais tem se destacado como um laboratório de
experiências administrativas.

Do antecessor, Antônio Anastasia, Pimentel herdou algumas metodologias
de planos gerenciais e um bom contingente de gerentes públicos formados
pela Escola de Administração Pública da Fundação João Pinheiro

Depois, se cercou de um grupo de acadêmicos da Cedeplar (Centro de
Desenvolvimento de Planejamento Regionais) da \versal{UFMG} (Universidade Federal
de Minas Gerais) (\url{migre.me/\allowbreak{}vp\versal{XNE}}), entre os quais
especialistas em desenvolvimento regional e em modelos administrativos
que rapidamente identificaram as falhas e os pontos de aprimoramento a
serem trabalhados.

Depois de um primeiro ano excepcional, no entanto, o governo de Fernando
Pimentel foi alvejado por dois tiros: a crise econômica e a ofensiva
política da Polícia Federal.

Se o governo Pimentel recuperar o pique, Minas poderá ser estados chave
como vitrine de um modelo de governança alternativa à centralização
autocrática de Brasília.

\section{Peça 3 -- as políticas de gabinete}

Os técnicos do Cedeplar identificaram dois vícios no modelo de gestão
anterior.

O primeiro, o modelo gerencialista, que optou por dirigir a gestão pelo
monitoramento de indicadores físicos, abstraindo o resultado final, a
entrega de políticas públicas.

O segundo, a falta de interlocução com as regiões, o que resultou em
políticas de gabinete, muitas vezes sem atender às especificardes
locais.

Estados imensos, como Minas, com dimensão de país, carecem de estruturas
intermediários mezo"-regionais.

Os consórcios vêm tentando preencher o vazio. Mas ainda não representam
uma leitura estratégica dos gestores, de como tratar as especificidades
territoriais dos estados.

Algumas regiões, como o sul de Minas e o Triângulo, têm sociedade
georreferenciadas. Mas não dispõem de estruturas institucionais às quais
recorrer.

Se o Norte quiser algum benefício, ou elege um deputado ou faz um
secretário emu ma pasta que o atenda. Daí a necessidade de uma estrutura
que lide diuturnamente com problemas.

Há as estruturas regionalizadas do Estado. A~Secretaria da Saúde tem 27,
a Educação 42 e a Segurança Pública 48. Mesmo assim, são estruturas
burocráticas, que não desenvolvem diagnósticos nem propostas de ação
levando em conta o território. Com isso, as políticas são empurradas
pelas necessidades imediatas dos municípios, sendo reativos, não
propositivos.

O primeiro passo foi resgatar o sentido da Política na formulação de
políticas públicas.

Para dentro da máquina, é necessário reconhecer o protagonismo dos
gestores, mantendo com eles interlocução permanente para planos,
orçamentos e \versal{PPA} (Plano Plurianual). Para fora, montar o processo de
participação regional conectado com processos internos.

O segredo está em ter um pilar hegemônico, em torno dos quais
gravitariam os demais pilares demandados pela sociedade. O~pilar em
questão foi o princípio de desenvolvimento com combate às desigualdades.

\section{Peça 4 Os instrumentos de gestão}

Em Minas, o sistema de gestão é montada em cima de três peças:

\subsection{Plano Mineiro de Desenvolvimento Integrado}

Trata"-se de um modelo que vem desde os anos 50, anterior à criação dos
\versal{PPA}s (Planos Plurianuais) pela Constituição de 1988.

Os \versal{PPA}s deveriam ser a ferramenta que asseguraria a continuidade das
políticas públicas, ao definir as prioridades do gestor, entrando pela
administração seguintes. Mas não cumprem esse papel porque não
conseguiram definir modelos eficientes de participação.

\subsection{\textbf{Fóruns Descentralizados de Gestão}}

Na gestão Pimentel, a maneira encontrada foi através dos Fóruns
regionais.

Todos eles trabalham tendo como foco o desenvolvimento econômico com
redução das desigualdades.

A montagem dos fóruns começou com um decreto regulamentador.

Depois, o governador visitou os 17 territórios, recepcionou prefeitos e
lideranças. Na sequência, foi explicado para eles o processo.

Um mês depois houve a segunda reunião, para montar o grande diagnóstico
por cinco eixos:

\begin{itemize}
\itemsep1pt\parskip0pt\parsep0pt
\item
  ·~~~~~~ Saúde e proteção social
\item
  ·~~~~~~ Educação e cultura
\item
  ·~~~~~~ Infraestrutura e logística
\item
  ·~~~~~~ Desenvolvimento econômico sustentável
\item
  ·~~~~~~ Segurança pública
\end{itemize}

Foram feitas reuniões por grupo, sem nenhuma pré"-orientação.

As lideranças entregaram questionários apontando problemas dos
territórios. Foram mapeados 12.689 problemas e necessidades.

Depois, o trabalho das lideranças foi priorizar ações dentro de cada
eixo. Em cada território foi escolhido um colegiado executivo com 25
lideranças da sociedade civil, um prefeito e um vereador por
microterritório. Ao todo, são ~80 microterritorios. Cada colegiado terá
um Secretário Executivo representante do governo, que será a correia de
transmissão entre as demandas regionais e a administração do Estado.

\subsection{Planos Mineiros de Desenvolvimento}

Da junção do \versal{PMDI} com os Fóruns nasceram 17 planos mineiros de
desenvolvimento.

Os Planos de Desenvolvimento Territoriais Integrados não são apenas
documentos com diagnósticos, mas carteiras pactuadas com sociedade civil
e poder público municipal e estadual, tendo como foco a diretriz de
vocação econômica regional.

Um dos grandes desafios desses modelos é desenvolver instrumentos de
coordenação. Trata"-se de um desafio que atormenta governos desde a
redemocratização. Existem várias arenas superpostas e com autonomia.

O próximo passo da gestão mineira terá que ser uma reforma
administrativa que defina melhor os papéis dos órgãos de articulação.

\section{Peça 5 -- a entrega dos produtos}

Dos Fóruns nasceu o \versal{PPA}, que foi submetido aos critérios de
sustentabilidade, não apenas ambiental, mas também econômica, de acordo
com a metodologia de pesquisa que o \versal{IBGE} está desenvolvendo para a \versal{ONU},
a \versal{ODS} (Objetivos de Desenvolvimento Sustentável), um conjunto de 17
indicadores, com metas a serem atingidas até 2030.

Esse modelo será colocado à prova nos próximos meses, devido à crise
econômica e fiscal. O~desafio exigirá do próprio governador uma atuação
proativa para negociar os projetos em ambiente de escassez.

Mesmo com a crise, o governo Pimentel tem conseguindo envelopar o que
existe e está sendo tocado com diretriz diferenciada: as preocupações
sociais e de atendimento a segmentos mais marginalizados. Inclusive
porque a burocracia mineira é bem instrumentalizada.

De dezembro a abril haverá as chamadas reuniões devolutivas, onde serão
prestadas contas dos acordos firmados. Nesses encontros haverá a
interação entre as estruturas regionais do Estado -- especialmente da
Saúde, Educação e a \versal{RISP} (Região Integrada de Segurança Pública), as que
estão mais direcionadas para pensar políticas de acordo com as
especificidades locais.

Se Pimentel conseguir superar os obstáculos atuais e retomar o fio da
meada do primeiro ano, Minas Gerais poderá fazer a diferença. Se se
entregar ao desânimo, poderá ser o coveiro do \versal{PT} em um de seus últimos
redutos.
