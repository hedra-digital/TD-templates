\chapterspecial{28/\allowbreak{}06/\allowbreak{}2016 O Xadrez dos fantasmas de Temer e as eleições indiretas}{}{}
 

\section{Definição 1 -- os novos inquilinos do poder}

Há dois grupos nítidos dentre os novos inquilinos do poder.

Um, o \versal{PMDB} de Michel Temer, Eliseu Padilha, Moreira Franco, Geddel
Vieira de Lima e Romero Jucá, grupo notório. ~O outro, um agrupamento em
que se somam grupos de mídia, Judiciário, Ministério Público Federal e
mercadistas do \versal{PSDB}. Vamos chama"-los de \versal{PSDB} cover, pois inclui as alas
paulistas e os mercadistas cariocas do \versal{PSDB}. A~banda de Aécio Neves é
carta fora do baralho.

Por vezes, o \versal{PSDB} cover provoca indignação. Já o grupo de Temer provoca
vergonha, um sentimento amplo de humilhação de assistir o país governado
por grupo tão suspeito, primário e truculento. E~menciono esse
sentimento não como uma expressão individual de repulsa, mas como um
ingrediente político que será decisivo nos desdobramentos políticos
pós"-impeachment, que rabisco no final.

O grupo de Temer quer se apropriar do orçamento com vistas às próximas
eleições. O~\versal{PSDB} cover quer se valer da oportunidade para reeditar as
grandes tacadas do Real.

Temer e seu grupo são mantidos na rédea curta, com denúncias periódicas
para mostrar quem tem o controle do processo. Deles se exige espaço
amplo para as articulações financeiras do \versal{PSDB} cover e o trabalho sujo
para desmontar qualquer possibilidade da oposição nas próximas eleições.

\section{Definição 2 --- a estratégia econômica}

Ao longo de 2013 e 2014 Dilma perdeu o foco da política econômica e deu
início à sequência de isenções fiscais, arrebentando com as contas
públicas. No final de 2014 havia um grande passivo das chamadas
``pedaladas''.

Um pouco antes de vencer as eleições, Dilma anunciou publicamente a
substituição do Ministro da Fazenda Guido Mantega por Joaquim Levy,
provocando ressentimentos em Mantega.

Passadas as eleições, foi aconselhada a zerar os passivos ainda em 2014.

Demitido em público, mas ainda Ministro, Mantega recusou"-se a tomar as
medidas necessárias. Indicado Ministro, mas ainda não empossado, Joaquim
Levy também preferiu postergar.

Assumindo Levy, Dilma anuncia a estratégia da chamada contração fiscal
expansionista. Ou seja, um enorme choque fiscal que devolveria a
confiança aos agentes econômicos que voltariam a investir.

Os empresários ficariam tão encantados com o choque fiscal que nem
ligariam para a queda da demanda, aumento da capacidade ociosa, taxas de
juros estratosféricas. Como diria Gil, ``andar com fé eu vou''. E~fomos.

O primeiro desastre foi o anúncio do plano a seco, como primeira
manifestação de Dilma. Foi um suicídio político.

No meio do ano estava claro o fracasso da estratégia que, ao derrubar
ainda mais a economia, ampliou a recessão, a queda de receitas e,
consequentemente, os desajustes fiscais.

Passou"-se todo o segundo semestre discutindo a revisão da política, sem
que nada fosse feito. Levy acabou saindo antes deixando armada a bomba
fiscal e a política.

Mudou"-se a estratégia para a flexibilização fiscal reformista.

Consistiria no governo assumir um resultado fiscal menor no curto prazo,
para absorver a perda de receita. E, para reconquistar a confiança do
mercado, em vez do ajuste fiscal, uma reforma fiscal.

Flexibilizaria no curto prazo, para devolver um pouco de fôlego à
economia. E~acenaria com reformas de médio prazo, visando devolver a
confiança no equilíbrio fiscal.

Em dezembro de 2015 a fogueira política parecia ter refluído. A~proposta
foi apresentada em janeiro de 2016, com os seguintes ingredientes:

\begin{enumerate}
\itemsep1pt\parskip0pt\parsep0pt
\item
  Pedido de autorização do Congresso para um déficit maior.
\item
  Limites de gastos orçamentários.
\item
  Reforma da Previdência.
\end{enumerate}

Na proposta Nelson Barbosa, os limites de gastos orçamentários seriam
definidos a cada quatro anos pelo Congresso. Substituir"-se"-iam os gastos
obrigatórios por metas obrigatórias a serem alcançadas. Seja qual fosse
o resultado, haveria a possibilidade de correção de rumos a cada quatro
anos.

Em relação à Previdência, haveria um aumento na idade mínima, mas com
uma longa regra de transição, de maneira a poupar quem já tivesse
ingressado no mercado de trabalho.

Mas, àquela altura, a governabilidade já tinha ido para o espaço, graças
à combinação da Lava Jato com Eduardo Cunha. A~cada semana, a Lava Jato
soltava uma bomba política e, após o recesso, Cunha soltava uma bomba
fiscal.

A equipe de Meirelles pegou as propostas e turbinou com Red Bull.

Hiperflexibilizou no curto prazo obtendo autorização para um déficit de
R\$ 170 bilhões para pagar a conta do impeachment. Produzindo um buraco
maior, pressionaria por reformas muito mais radicais do que as previstas
pelo governo Dilma.

Em relação ao limite de gastos pretende amarrar o orçamento por 20 anos,
em cima dos gastos de 2016, espremidos por dois anos de quedas de
receitas. Se passar a \versal{PEC} (Proposta de Emenda Constitucional), um grupo
que não recebeu nenhum voto nas últimas eleições, membros interinos da
junta de poder, definirá o orçamento para os próximos três presidentes
da República.

Não é apenas isso.

A deterioração das contas públicas abrirá espaço para as famosas
``tacadas'' -- termo que Rui Barbosa utilizava para as jogadas do
encilhamento; e que os economistas do Real praticaram na política
cambial e nas privatizações.

Os negócios estão caminhando a mil por hora.

\begin{enumerate}
\itemsep1pt\parskip0pt\parsep0pt
\item
  De cara, haverá a rentabilíssima operação de vendas de ativos públicos
  depreciados. O~Projeto de Lei apresentado pelo senador Tasso
  Jereissatti vai nessa direção, ao inviabilizar qualquer recuperação de
  empresa pública e colocá"-la à venda sem nenhuma estratégia setorial ou
  de valorização dos ativos.
\item
  Nessa panela entrarão as vendas de participação do \versal{BNDES}, com o
  mercado no chão.
\item
  Se acelerarão as concessões com margens altas de rentabilidade,
  abandonando de vez as veleidades de modicidade tarifária.
\item
  No caso da participação externa em companhias aéreas, por exemplo,
  havia estudos para autorizar até 49\% podendo chegar a 100\%, mas
  apenas dentro de acordos de reciprocidade com outros países. Já se
  mudou para autorização para 100\%, sem qualquer contrapartida. Altas
  tacadas e altas comissões.
\end{enumerate}

Estão no forno duas outras medidas complicadas. Uma, visando retirar do
\versal{BNDES} R\$ 150 bilhões de recursos não aplicados; outra vendendo R\$ 100
bi em ativos do Fundo Soberano.

No caso do \versal{BNDES}, o governo Dilma tinha pronto medida colocando à
disposição dos bancos comerciais os recursos não aplicados pelo \versal{BNDES},
nas mesmas condições. Seria uma maneira de impedir o travamento dos
investimentos.

Mesmo assim, a flexibilização do orçamento e a perspectiva do fundo do
poço ter sido alcançado no primeiro trimestre, promoverá algum desafogo
na economia nos próximos meses.

\subsection{Definição 3 -- os desdobramentos políticos}

E aí se chega no busílis da questão, no xeque pastor -- o mais rápido do
xadrez. Vamos compor esses quebra cabeças com as peças que se têm à mão.

Lembre"-se: não são apostas cravadas nas hipóteses abaixo, são
 possibilidades. As peças estando em determinadas posições, abrem espaço
para estratégias prováveis.

\subsection{Peça 1 -- o reino da democracia sem voto}

Hoje em dia, se está no mundo que o \versal{PSDB} cover pediu aos céus: uma
democracia sem votos. O~exército das profundezas, organizado por Eduardo
Cunha, está prestes a ser desbaratado. O~poder de fato é exercido hoje
pela combinação da mídia com o Ministério Público, Judiciário e Tribunal
de Contas, substituindo o sufrágio popular.

Essa combinação está permitindo mudanças constitucionais, derrubada de
presidentes sem obedecer às determinações constitucionais, destruição de
setores e empresas em torno da bandeira genérica da luta contra a
corrupção.

\subsection{Peça 2 --- Michel Temer é um interino inviável.}

A última edição da revista Época revela mais uma ponta da parceria de
Temer com o~ coronel da reserva da \versal{PM} paulista João Baptista Lima Filho,
sócio da Argeplan, incluída em obras da Eletronuclear sem possuir
experiência para tal. Lima foi citado pelo presidente da Engevix como
receptador de R\$ 1 milhão cujo destinatário final seria Temer.

Não é a primeira menção à parceria Lima"-Temer.

Anos atrás, em um processo de divórcio de um ex"-gestor do porto de
Santos, ao detalhar as formas como o ex"-marido amealhou patrimônio, foi
mencionado especificamente o que ele recebia de propinas e o que era
encaminhado para Lima e Temer.

Na época, o \versal{MPF} e o Judiciário pediram arquivamento do caso. Agora, Lima
reaparece na delação da Engevix. À~esta altura, jornalistas e
procuradores estão juntando mais elementos das parcerias.

Mais que isso: se a parceria com a mídia não impediu a denúncia das
relações tenebrosas de Temer, o que impedirá a colheita no manancial de
escândalos protagonizados por Eliseu Padilha e Geddel Vieira Lima? E
ainda não se chegou ao tema central, da delação de Marcelo Odebrecht.

Não haverá blindagem capaz de garantir Temer. É~uma relação ampla de
delações com seu nome obrigatoriamente envolvido. Mesmo em nome da
governabilidade, não será possível passar ao largo das evidências.

Em dezembro de 2014, por exemplo, a Secretaria de Aviação Civil (\versal{SAC}),
não mais sob controle de Moreira Franco, anulou licitação para
contratação de empresa consultiva de engenharia, para monitorar todas as
atividades do Fundo Nacional de Aviação Civil. O~consórcio vencedor era
formado pela Engevix e pela Argeplan Arquitetura e Engenharia.

\subsection{Peça 3 -- as eleições indiretas}

Chega"-se, finalmente, à perspectiva mais imediata de xeque, que não
inclui a volta de Dilma.

Primeiro, tem"-se o desafio da votação do impeachment. Passando ou não,
tem"-se a segunda barreira, no \versal{TSE} (Tribunal Superior Eleitoral).

Nos últimos dias, ventilou"-se a tese Gilmar Mendes, de montar uma
operação para supostamente legitimar Temer. Consistiria no \versal{TSE} barrar
Dilma e Temer. Pela Constituição, um mês depois haveria eleição indireta
pelo Congresso, sem obrigatoriedade de candidaturas de parlamentares,
mas com a promessa de Temer poder se candidatar e ser eleito.

Isto é o que se diz.

Se o custo Temer estiver muito alto, nada impedirá o \versal{PSDB} cover de
lançar Henrique Meirelles, abolindo os intermediários, ou alguma
articulação mais ampla passar pelo presidente do Senado Renan Calheiros.
