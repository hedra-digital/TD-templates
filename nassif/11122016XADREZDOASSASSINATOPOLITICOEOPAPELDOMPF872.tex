\chapterspecial{11/\allowbreak{}12/\allowbreak{}2016 Xadrez do assassinato político e o papel do \versal{MPF}}{}{}
 

\section{Peça 1 -- as peças iniciais do jogo}

É curioso a rapidez do tempo histórico nesses tempos de Internet e redes
sociais. Há o lado da desestruturação das informações, pela quantidade e
rapidez com que se sucedem os eventos. Mas há o lado da enorme rapidez
dos diagnósticos em cima de eventos históricos ainda em andamento.

É o caso da nova estratégia da geopolítica norte"-americana, montada a
partir do advento da Internet e das redes sociais.

Ao longo dos últimos anos, foi possível acompanhar passo a passo esse
jogo. No início, dada a aparente volatilidade dos fatos, íamos
registrando o passo"-a-passo, mas ainda mantendo dúvidas sobre as formas
de organização: havia uma lógica, algum conhecimento sistematizado, ou
apenas um ou dois eventos planejados e o resto se sucedendo de forma
aleatória?

Afinal, a cooperação internacional -- a troca de informações entre os
órgãos de segurança de vários países --- ~é praticada há anos em várias
instâncias, desde a cobrança de pensão alimentícia até extradição de
criminosos. Era de conhecimento público a cooperação entre \versal{FBI} e a
Polícia Federal. E, desde a constituição da Sistema Brasileiro de
Inteligência (Sisbin) a prática da integração dos diversos órgãos de
fiscalização em forças tarefas.

A maneira como a Lava Jato investiu contra a Petrobras e as
empreiteiras, como destruiu sistematicamente a cadeia do petróleo e gás
e a indústria naval, parecia, no início, apenas esbirros de um país
atrasado, de instituições frágeis, de uma mídia subdesenvolvida, que não
conseguiram avaliar a relevância das empresas para a geração de
impostos, emprego, tecnologia.

Jogava"-se já no golpe do impeachment e todos os prejuízos ao país eram
lançados na conta do golpe.

Com o tempo, percebeu"-se que havia método no trabalho.

\section{Peça 2 -- os primeiros indícios do jogo
antinacional~~~~~~~~~~~~~}

A ida do Procurador Geral da República Rodrigo Janot aos Estados Unidos,
no início de fevereiro de 2015, chefiando uma equipe de procuradores,
levando informações contra a Petrobras, despertou o primeiro alerta: a
cooperação internacional se dava de forma estranha, não seguindo as
formalidades.

No dia~\textbf{2 de fevereiro de 2015}, nosso colunista André Araújo, do
alto de sua experiência, antecipava os pontos centrais de questionamento
(\url{https:/\allowbreak{}/\allowbreak{}goo.gl/\allowbreak{}\versal{V}2Wrhv}):

\begin{enumerate}
\itemsep1pt\parskip0pt\parsep0pt
\item
  Como um agente do Estado brasileiro vai aos \versal{EUA} levando informações
  contra uma empresa controlada pelo Estado brasileiro? Quem deveria ter
  ido era a \versal{AGU} (Advocacia Geral da União).
\item
  Nenhum país minimamente consciente de sua soberania permite que suas
  empresas e cidadãos sejam processados no exterior. No caso brasileiro,
  não apenas se permitia como se alimentava a Justiça norte"-americana.
\item
  Cooperação internacional só pode se dar através do Ministério da
  Justiça. A~tropa de procuradores, comandada por Janot, não apenas
  atropelava o Ministério da Justiça como o próprio Ministério das
  Relações Exteriores, assumindo o controle completo da cooperação.
\end{enumerate}

André estranhava, principalmente, a visita de Janot ao Departamento de
Justiça: ``A única coisa sobre Petrobras que existe no Departamento de
Justiça é uma investigação criminal contra a empresa Petrobras, os
procuradores vão lá reforçar a acusação? É a única coisa que podem
fazer, defesa não é com eles, é com a \versal{AGU}''.

No dia 9 de fevereiro, a Procuradoria respondeu às indagações formuladas
(\url{https:/\allowbreak{}/\allowbreak{}goo.gl/\allowbreak{}Vs6lqz)}. Foi a única vez que se dignou a dar
informações para uma cobertura que não fosse chapa branca.

Na nota, duas informações significativas.

A primeira, a relação de instituições públicas que acompanharam o \versal{PGR}:
\versal{CVM} (Comissão de Valores Mobiliários) e \versal{CGU} (Controladoria Geral da
União), apenas instituições públicas fiscalizadoras, e não a \versal{AGU}
(Advocacia Geral da União) a quem caberia defender a Petrobrás. Não foi
um pecado solitário da \versal{PGR}, mas a prova mais evidente da forma
totalmente despreparada com que o governo Dilma Rousseff encarou a Lava
Jato.

Não faltaram alertas para que ela entrasse em contato direto com Barack
Obama, visando impedir ações contra a Petrobras -- vítima da corrupção,
e não autora.

A segunda, a informação de que o Ministério da Justiça não era a
autoridade central exclusiva nos acordos de cooperação. Dizia a nota:

\emph{``A obtenção de provas por meio de auxílio direto ou rogatórias e
a transmissão de documentos entre os Estados é feita pela autoridade
central, papel que, no Brasil, é desempenhado pelo Ministério da
Justiça~\textbf{\versal{OU}}~pela \versal{PGR}''.}

De nada adiantaram os alertas de que seria suicídio o Ministério da
Justiça deixar o controle total da cooperação nas mãos da \versal{PGR}, que era
peça da conspiração. O~Ministro José Eduardo Cardozo jamais quis correr
o menor risco em defesa da legalidade e do seu governo.

Em~\textbf{2 de abril de 2015}, dois meses após a visita de Janot aos
\versal{EUA}, saiu a denúncia contra o almirante Othon Luiz Pereira da Silva,
figura chave no desenvolvimento nuclear brasileiro
(\url{https:/\allowbreak{}/\allowbreak{}goo.gl/\allowbreak{}\versal{AVP}iw8}).

A maneira como chegaram em Othon foi apertar o presidente da Camargo
Correa Dalton Avancini, que já havia feito uma delação. Providenciaram
uma segunda delação onde o induziram a denunciar a Eletronuclear, com
base nas informações conseguidas junto às autoridades norte"-americanas.

A partir da reformulação de sua delação, deflagrou"-se a Operação
Radioatividade, para investigar suspeitas na área nuclear.

Indagamos da \versal{PGR} se trouxera da visita as informações contra a
Eletronuclear. A~resposta, dúbia, foi de que ``nós não saímos do Brasil
com essa intenção'', uma maneira de dizer que voltaram com a informação.
O~indiciamento do Almirante se deu em tempo recorde.

No dia~\textbf{2 de agosto de 2015}, quando já estavam mais nítidos os
sinais da articulação entre a \versal{PGR} e as autoridades norte"-americanas, o
\versal{GGN} resolveu investigar a trajetória do \versal{PGR} Janot nos Estados Unidos. E~descobriu que ele se encontrou com Leslie Caldwell, procuradora"-adjunta
encarregada da Divisão Criminal do Departamento de Justiça dos Estados
Unidos (\url{migre.me/\allowbreak{}q\versal{ZS}vO)}~e, até um ano antes, advogada de um
grande escritório de advocacia que atendia à indústria eletronuclear
norte"-americana.

A partir desse episódio, ficou nítido que havia uma estreita cooperação
entre autoridades de ambos os países e o que parecia uma aparente
ignorância do \versal{PGR} e do Ministério Público em relação aos interesses
nacionais em jogo, era uma articulação pensada e antinacional.

\section{Peça 3 -- o confronto com o que ocorreu em outros países}

Gradativamente, começaram a aparecer detalhes de casos envolvendo
líderes socialdemocratas em outros países do mundo, sempre tendo o
Ministério Público e a Justiça como elementos centrais de
desestabilização.

Em Portugal e Argentina ocorreu o mesmo processo
(\url{https:/\allowbreak{}/\allowbreak{}goo.gl/\allowbreak{}d\versal{JZHHZ})}. Em Portugal, uma campanha sistemática
contra o ex"-primeiro ministro socialista José Sócrates, um ~ano de
campanha, 9 meses de prisão preventiva. No final, nenhum elemento capaz
de condená"-lo, mas Sócrates estava politicamente destruído.

Na Argentina, o mesmo procedimento do \versal{MPF} brasileiro. Pega"-se uma
decisão de política econômica, identificam"-se ganhadores genéricos e
amarra"-se com algum financiamento de campanha para criminalizar Cristina
Kirchner que foi indiciada e precisou depor perante um juiz
(\url{https:/\allowbreak{}/\allowbreak{}goo.gl/\allowbreak{}no1iaC)}.

No dia 20 de fevereiro de 2016, uma entrevista extremamente elucidativa
de Jamil Chade~~ (\url{https:/\allowbreak{}/\allowbreak{}goo.gl/\allowbreak{}Bk2qJq)}, correspondente do
Estadão em Genebra. Autor de um livro sobre o escândalo da \versal{FIFA}, com
fontes no \versal{FBI}, Chade contava que foram as manifestações de junho de 2013
que convenceram o \versal{FBI} que o Brasil estaria preparado para enfrentar dois
mega"-escândalos. Um, foi a Lava Jato, com foco na Petrobras. O~segundo,
a \versal{FIFA}, visando romper os acordos esportivos que asseguram às empresas
nacionais blindagens de audiência contra a entrada de competidores
estrangeiros.

Ora, \versal{FIFA} é um escândalo brasileiro, que tem na Globo seu principal
formulador. Os agentes do \versal{FBI} diziam que o \versal{MPF} brasileiro era o menos
colaborativo no caso \versal{FIFA}, ao contrário da Lava Jato, onde as
informações fluíam torrencialmente.

Justamente nas manifestações de junho de 2013 houve o pacto entre a
Globo e o \versal{MPF} no combate à \versal{PEC} 37, que restringiria a capacidade de
investigação do \versal{MPF}.

No dia~10 de março de 2016, \versal{GGN} entrevistou o cientista político Moniz
Bandeira, que explicou de forma detalhada a nova estratégia
norte"-americana, abdicando das parcerias militares em benefício dos
pactos com o Judiciário e o Ministério Público. Sob o título ``Da
Primavera Árabe ao Brasil, como os \versal{EUA} atuam na geopolítica''
(\url{https:/\allowbreak{}/\allowbreak{}goo.gl/\allowbreak{}u1\versal{ISQ}8)}~Moniz disseca o novo modo operacional da
geopolítica norte"-americana.

No dia~20 de maio de 2016~participei de um debate na Fundação Escola de
Sociologia e Política com o acadêmico alemão Thomas Meyer, autor do
livro ``Democracia midiática: como a mídia coloniza a política''. ~Meyer
é intelectual de peso, membro do Grupo Consultivo da União Europeia para
a área de Ciências Sociais e Humanas e vice"-presidente do Comitê de
Princípios Fundamentais do Partido Socialdemocrata da Alemanha

No debate, contou em detalhes como se deu a campanha que levou à
renúncia do presidente socialdemocrata Christian Wullf. Durante quatro
anos, houve uma campanha de mídia na Alemanha que utilizava informações
inventadas, absurdas, segundo ele. Todos os veículos montaram um fluxo
único de informações, massacrando o presidente até renunciar.

\section{Peça 4 -- a explicitação da metodologia do ``lawfare''}

Nos embates contra a Lava Jato, os advogados de Lula decidiram levar a
perseguição ao Acnudh~(Alto Comissariado da \versal{ONU} para os Direitos
Humanos). No levantamento das práticas de abusos, houve uma discussão
com especialistas na Universidade de Harvard, que detalharam a tática
conhecida como ``lawfare'', ou guerra jurídica
(\url{https:/\allowbreak{}/\allowbreak{}goo.gl/\allowbreak{}28knxn)}.

Ali se percebeu que o fenômeno, global, já havia sido detectado pela
academia dos países centrais, que conseguiram sistematizar seu modo de
operação.

Consiste em uma parceria entre Ministério Público e mídia visando gerar
uma enorme quantidade de notícias e denúncias, mesmo sem maiores
fundamentos. O~objetivo é sufocar a defesa, destruir a imagem do réu
perante à opinião pública, atingindo seus objetivos de anulá"-lo para a
política -- seja pela destruição da imagem ou pelo comprometimento de
grande parte do tempo com a defesa.

Trata"-se, portanto, de um recurso utilizado em várias partes com
propósitos eminentemente políticos. A~mesma coisa que ocorreu em
Portugal, Alemanha, na Espanha, com o primeiro"-ministro Felipe Gonzáles.

E, aí, se junta a última peça para a explicitação da metodologia de
atuação: quem comanda o circo

 

No começo de tudo estão os interesses geopolíticos norte"-americanos,
fundados em alguns objetivos:

\begin{enumerate}
\itemsep1pt\parskip0pt\parsep0pt
\item
  Impedir o desenvolvimento autônomo de potências regionais e de modelos
  de socialdemocracia. Não é coincidência, a crise atual da Coréia do
  Sul, os ataques aos líderes socialdemocratas em vários países.
\item
  Atuar firmemente contra os \versal{BRIC}s. Brasil já é fato consumado.
  Tenta"-se, agora, a Índia.
\item
  Consolidar o livre fluxo de capitais já que, hoje em dia, a hegemonia
  norte"-americana se dá fundamentalmente no campo financeiro.
\end{enumerate}

O governo dispõe basicamente de três estruturas.

Em azul escuro, no topo, o Departamento de Estado (na época dirigido por
Hillary Clinton, estreitamente ligada ao establishment norte"-americano),
em cooperação com o Departamento de Justiça. Como braços operacionais, o
\versal{FBI} -- e suas parcerias com as polícias federais -- e a \versal{NSA} -- a
organização que se especializou em espionagem eletrônica, responsável
pelos grampos nos telefones de Dilma Rousseff e Ângela Merkel.

O Departamento de Estado dispõe de três ambientes de disseminação da
estratégia: as redes sociais, a cooperação internacional e o mercado.

Há anos, o Departamento de Estado atua nas redes sociais de vários
países. Recentemente, a Wikileaks revelou a atuação do homem de Hillary
nas redes sociais atuando junto a comunicadores brasileiros.

A cooperação internacional é uma estrutura antiga, de troca de
informações entre Ministérios Públicos e Policias Federais de vários
países. Após o atentado às Torres Gêmeas, tornou"-se peça central de
colaboração contra o crime organizado. Nela, o \versal{FBI} desempenha papel
central, por ser o órgão mais bem aparelhado para o rastreamento de
dinheiro em paraísos fiscais -- onde se misturam dinheiro do
narcotráfico, caixa dois, dinheiro de corrupção política. Com o controle
das informações, disponibiliza aquelas que são de interesse direto da
geopolítica norte"-americana.

Finalmente, o mercado, com sua extensa rede de entrelaçamento com
instituições financeiras, empresas e mídia nacionais, é o terceiro canal
de influência.

Nos círculos vermelhos, os três fenômenos que chacoalham as democracias
modernas.

O primeiro, a informação caótica, fato que aumenta com as redes sociais
e, especialmente, com os grupos de mídia praticando a chamada
pós"-verdade -- a invenção de notícias com propósitos políticos.

O desalento com a economia -- após a crise de 2008 -- gerou dois novos
sentimentos de massa: o desânimo com a democracia e a busca de saídas
autoritárias; e a exploração do mito do inimigo externo, que pode ser um
membro do Islã, um imigrante indefeso ou um perigoso agente da
socialdemocracia.

A falência do estado de bem"-estar social, a falta de alternativas,
promoveu um quarto sentimento, que é o do desmonte do Estado através do
enfraquecimento da política em favor do mercado.

Em verde, finalmente, os agentes nacionais desse golpe: a Lava Jato e a
\versal{PGR}, firmemente empenhados na destruição da estrutura atual de grandes
empresas brasileiras; a mídia e o mercado.

Com essas ferramentas à mão, monta"-se o ``lawfare'', visando
exclusivamente os adversários do sistema. E, no bojo das operações, o
conjunto de ideias econômicas que, no caso brasileiro, foi batizado de
``Ponte para o Futuro'': desmonte do Estado social, livre fluxo de
capitais, privatização selvagem.

No futuro, assim que se sair do estado de exceção atual, não haverá como
não denunciar o Procurador Geral Janot, o juiz Moro e os procuradores da
Lava Jato por crime contra o país. E, aí, haverá ampla documentação
devidamente registrada e que possivelmente será requisitada pelo
primeiro governo democrático brasileiro, pós"-golpe, junto à cooperação
internacional.
