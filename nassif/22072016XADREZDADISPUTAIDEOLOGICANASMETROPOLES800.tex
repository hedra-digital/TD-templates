\chapterspecial{22/\allowbreak{}07/\allowbreak{}2016 Xadrez da disputa ideológica nas metrópoles}{}{}
 

\section{Peça 1 -- A crise da democracia representativa}

A crise de 2008 derrubou o mito da democracia representativa como forma
de governo. Até a Primeira Guerra foi o sistema que permitiu a
internacionalização do capital e a derrubada de muros nacionais. De 1914
à Segunda Guerra houve o interregno autoritário, com o nascimento do
comunismo e do fascismo. Da Segunda Guerra até 1972 o grande pacto que
permitiu definir controles sobre os fluxos de capitais, equilibrando um
pouco o jogo entre interesses nacionais e sociais e os interesses do
capital, garantindo um desenvolvimento seguro de muitos países
emergentes e o florescimento do estado do bem"-estar na Europa.

De 1972 em diante houve a segunda grande investida do capital,
culminando com três grandes vitórias:

Vitória 1~-- o fim da paridade dólar"-ouro, devolvendo a liberdade aos
fluxos de capital.

Vitória 2~-- o advento da era Reagan e Thatcher influenciando as
democracias ocidentais~ a diminuir o papel do Estado.

Vitória 3~-- a queda do muro de Berlim, precedendo a derrocada do
Império Soviético.

Ao mesmo tempo, houve a explosão das redes sociais e um novo movimento
de concentração e de aparecimento de novas superempresas globais.

Aí reside o centro da crise da democracia representativa. Acabou o sonho
do pote de ouro no final do arco"-íris das reformas liberalizantes. Mas a
maior crise do capitalismo provocou, paradoxalmente, a maior crise das
esquerdas, com o crescimento de grupos de direita.

Tem"-se, então, o seguinte quadro:

\begin{enumerate}
\itemsep1pt\parskip0pt\parsep0pt
\item
  A democracia representativa não mais atende os interesses do grande
  capital. O~desmonte dos estados de bem"-estar social, no pós-2008,
  matou qualquer veleidade de reconstrução nacional através das velhas
  fórmulas financistas, pela óbvia impossibilidade de sustentar
  eleitoralmente o sonho neoliberal. A~maneira de contornar o problema
  foi apelar para um conjunto de fórmulas que afastassem de vez o
  controle do voto sobre as políticas econômicas. No caso da União
  Europeia, pela imposição das regras do Banco Central Europeu. Em
  países menos civilizados -- como o Brasil e a América Latina em geral
  -- pelo estratagema dos golpes parlamentares.
\item
  Encerrou"-se a era da conciliação, de pactos sociais costurados através
  de eleições diretas e a aposta dos movimentos populares na democracia.
  Agora, o que há, a ferro frio, é a disputa selvagem pelo controle do
  orçamento nacional. No caso brasileiro, de forma ostensivamente tosca,
  através da política monetária do Banco Central e da tentativa de
  privatizações a toque de caixa.
\end{enumerate}

\subsection{Peça 2 -- a reconstrução da política e a crise dos estados
nacionais}

Os movimentos pós-2008 mostram um enfraquecimento cada vez maior dos
estados nacionais. A~Grécia foi o primeiro estado nacional a ser varrido
do mapa, tornando"-se um passivo dos bancos internacionais. Foi uma
vitória ampla dos bancos sobre a política. E~a Grécia era apenas um
retrato na parede, uma pequena economia.

Com o golpe brasileiro, há uma vitória maiúscula do capital financeiro
sobre a política, juntando no mesmo balaio de alianças o Ministério
Público, a camarilha dos 6, grupos de mídia e o mercado.

Com as dúvidas lançadas sobre a política econômica de Dilma, a
interrupção da democracia, a falência dos partidos, a reconstrução da
narrativa social não se dará através das políticas nacionais. O~próximo
campo para o embate de propostas serão as regiões metropolitanas.

É nelas que se poderão recuperar alguns dos valores básicos da
democracia, como a participação dos moradores, a apropriação do espaço
público

E aí despontam duas propostas similares, a do prefeito de São Paulo
Fernando Haddad, e do candidato do \versal{PSOL} no Rio, Marcelo Freixo.

\section{Peça 3 -- a luta pelo território das metrópoles}

Nas discussões de políticas nacionais, as estratégias de apropriação do
orçamento e de captura da política econômica são sustentadas por
falácias teóricas que, para os leigos, passam a ideia de verdade
científica. A~participação dos cidadãos na política quase sempre é
restrita aos períodos eleitorais.

No ambiente das cidades, é muito mais fácil definir o que é e o que não
é interesse público, assim como as formas de participação direta.

Pelo menos no plano retórico, Freixo é o político que melhor tem
definido esse novo locus das batalhas conceituais.

As cidades são definidas conforme as relações de poder, diz ele. As
cidades globais vão ficando cada vez mais iguais, com os megaeventos, os
encontros em shoppings centers, as ruas segregados, as cidades
desiguais.

O fenômeno neoliberal não se caracterizou apenas pelo processo de
redução do estado, diz ele. Nas cidades, ele se consolida para
deslocamento de poder: mais que controle econômico, o capital passa a
dispor do controle político. Há um esvaziamento da ideia de poder
público. Quem decide pela cidade é o capital

O exemplo mais nítido é o Rio de Janeiro. Com toda sua estrutura de
ex"-capital, com o arcabouço público de que dispõe, não existe uma
Secretaria do Planejamento, porque o planejamento é dado pelo mercado.
Viram cidades"-mercados, cidades, caras, desiguais, tendo a remoção de
pobres como prática e a gestão do dinheiro público pelo setor privado
através das Organizações Sociais.

A mobilidade urbana é o grande debate de democracia social, porque
define o direito às cidades. E, no Rio de Janeiro, a mobilidade urbana
está nas mãos das empreiteiras. Elas administram os trens do Metrô, a
barca Rio"-Niteroi, a Zona Portuária, onde 75\% dos terrenos são
públicos.

\versal{OS} exemplos da cidade"-cara estão nos aluguéis, na falta de \versal{IPTU}
progressivo, na falta de habitação popular em áreas centrais.

A pouca importância dada às cidades é comprovada pelo fato do Rio de
Janeiro ter sido utilizado pelo \versal{PT} como moeda de troca com o \versal{PMDB}, E não
era qualquer \versal{PMDB}, mas o \versal{PMDB} carioca.

\section{Peça 4 -- as políticas democratizantes nas metrópoles.}

O caminho a ser trilhado será o da radicalização da democracia
participativa, através dos conselhos de moradores, de usuários, de
formas de participação nas prefeituras regionais.

No caso de Freixo, sua campanha criou o site ``Se a Cidade Fosse
Nossa'', montando um programa participativo. Há um ``Se a Cidade Fosse
Nossa'' para cada bairro, favela e serviço público.

Enquanto no Rio ainda é uma plataforma de um candidato -- que
provavelmente será endossado por outros candidatos populares, como
Jandira Feghali \mbox{---,} no caso paulista já há resultados concretos.

É o caso do Plano Diretor que definiu novas formas de construção e a
indução à construção de moradias populares em zonas residenciais, entre
outros avanços.

Outro ponto central foi o enfrentamento da ditadura dos automóveis,
através de iniciativas como o de apropriação da cidade (Paulista,
Minhocão) por pedestres e pelas bicicletas, além das faixas únicas de
ônibus e os novos limites de velocidade.

A experiência paulistana serviu para quebrar inúmeros tabus. Sabe"-se ser
possível criar opiniões majoritárias em favor de novas práticas.

Após o golpe, a nova política nascerá das metrópoles.
