\chapterspecial{25/\allowbreak{}06/\allowbreak{}2016 Quem é quem no xadrez do impeachment}{}{}
 

Os jogos em torno do impeachment não são de fácil diagnóstico. Que
existe um movimento articulado, não se discute. Mas existem também
tendências internacionais, estados de espírito internos que induzem as
pessoas a certas atitudes, de tal maneira que se torna difícil separar o
que é conspiração ou tendência induzida pelas circunstâncias históricas.

É evidente que a conspiração atua sobre as características políticas do
momento. Mas nem todos que endossam esse movimento agem com intenção
conspiratória. Meramente seguem tendências tornando"-se massa de manobra.

Para facilitar o raciocínio, vamos separar as principais peças do jogo
para tentar remonta"-las mais adiante.

\section{Primeiro conjunto: as tendências internacionais}

A institucionalização da bestificação do discurso político não é
meramente uma obra da mídia. Grupos de mídia são empresas comerciais,
com interesses econômicos claros, que atuam quase sempre
pró"-ciclicamente -- isto é, acentuando os movimentos de opinião pública.

Mas não são meros agentes passivos. Em tempos de alta intolerância, o
poder dos grupos de mídia se potencializa. Com os ânimos exaltados, os
nervos desencapados, a opinião pública fica muito mais suscetível à
manipulação. Quebram"-se os filtros da verossimilhança, qualquer denúncia
cola, avultam as teorias conspiratórias e consegue"-se manipular o
estouro da boiada através da recriação de alguns mitos históricos, como
o do inimigo externo, das ameaças insondáveis à família, do castigo
eterno aos ímpios e outros mitos que, tendo como pano de fundo a
superstição, alimentaram os piores episódios de intolerância do século
20.

É quando a besta -- esse sentimento de intolerância massificado -- sai
da jaula a passa a ser tangida por palavras de ordem emanadas da mídia
ou de lideranças populares. Aí, a ~mídia adquire poder de vida e de
morte sobre personalidades públicas. Vide o macarthismo, o uso da
informação de massa pelo fascismo ou, mesmo sem o modelo de mídia
ocidental, mas navegando nas mesmas águas da intolerância, a revolução
cultural chinesa.

\subsection{Os fatores de intolerância}

Se não é um fenômeno estritamente brasileiro, o que caracterizaria,
então, a universalização atual dessas ondas de intolerância?

Está"-se em um quadro claro de falência do modelo de economia liberal,
que começa em 1972, e de democracia representativa que vigorou em todo
século 20.

Nos modelos democráticos, o equilíbrio geral -- econômico, social e
político --- é uma percepção criada pelo trabalho articulado entre
quatro setores --- Executivo, Legislativo, Judiciário e Mídia \mbox{---,} por
um quadro econômico estável, e com válvulas de escape permitindo
administrar os conflitos internos, com relativa abertura para processos
lentos de inclusão.

A crise de 2008 matou a utopia e trouxe à tona diversos elementos
desestabilizadores, como a insegurança econômica e o medo de perda de
status social.

A globalização e os grandes movimentos de inclusão trouxeram uma nova
população invadindo os mercados de consumo, de lazer, de educação e de
opinião. Enquanto o mito econômico se sustentou, foi mais fácil
administrar as intolerâncias e preconceitos em relação aos
``invasores''. Com a crise e o fim das ilusões, a busca de bodes
expiatórios foi bater nas costas dos imigrantes e dos novos incluídos,
muito mais concretos para atiçar o primarismo da besta do que movimentos
financeiros sofisticados ou as grandes jogadas empresariais.

Somou"-se o desmantelamento dos sistemas tradicionais de mídia. . O~sistema que vigorou no século, a não ser nas fases iniciais da era do
rádio, embora alimentasse a intolerância, funcionava também como
descarrego das manifestações individuais de seus leitores.

Houve então um estilhaçamento de todas as formas de coordenação e
controle da opinião pública em um momento de conflitos étnicos e de ódio
interno nos países. A~besta arrebentou as grades e invadiu as ruas, as
cidades, até as conversas de família.

Principalmente, comprometeu radicalmente um dos elementos centrais dos
pactos democráticos: a hipocrisia da democracia representativa.

O primado da separação de poderes criou um conjunto de freios ao poder
absoluto. E~a ideia genérica de que ``todos são iguais perante a lei''
legitimou o modelo. Além disso, abriu espaço para a assimilação lenta e
gradual das políticas de inclusão, que deveriam acompanhar sempre o
pensamento médio nacional.

Cada grupo social precisava, antes, expandir suas ideias, viabilizar"-se
politicamente para, mais à frente, inserir seus princípios nas leis e na
política.

Esse modelo gradual, garantiu a disciplina das chamadas massas, mantendo
sob controle as disputas de classe e permitindo a prevalência do poder
econômico em todas as instâncias, em alguns casos amenizado por um
conjunto de regramentos.

Na política, o poder econômico avançou através dos financiamentos de
campanha. No dia"-a-dia da economia tornaram"-se os parceiros mais
influentes de todos os presidentes. Nos Estados Unidos, levaram à guerra
contra a Espanha, em fins do século 19, à guerra contra o Iraque, no
século 21.No Brasil, \versal{FHC} buscou seus aliados junto ao setor financeiro;
Lula, junto aos grandes grupos da economia real.

A própria prestação da Justiça desdobrou"-se em várias formas de proteção
aos poderosos, das apelações infindáveis às diversas maneiras de
interpretar o ``garantismo'' -- a defesa das garantias individuais --
dependendo de grandes escritórios de advocacia. Em alguns casos, como
nos \versal{EUA}, em nome do interesse nacional foi conferido até direito do
Presidente da República conceder indulto a crimes econômicos praticados.

Essa mesma parceria manifestou"-se em relação à mídia, com os diversos
modelos de financiamento dos grupos de mídia subordinando"-os a
interesses de grupos.

Apenas nas eleições o eleitor tinha condições de se manifestar. Mesmo
assim, submetido a formas variadas de controle e manipulação da
informação.

Todo esse aparato institucional visava criar uma mediação e controle das
demandas públicas. E~nem se julgue essa constatação um fator totalmente
negativo: não há nada pior para um país ou uma comunidade que uma
opinião pública descontrolada, reagindo aos estímulos de líderes de
torcida.

Esse mundo desabou.

Em cima da decepção com os modelos econômico e democrático, vieram as
novas formas de comunicação das redes sociais, passando a ilusão da
democracia direta em todas as instâncias.

Nas ruas, o grito sem a mediação dos partidos e da mídia. No mercado de
opinião, a atoarda das redes sociais, nas quais a mídia é apenas um
perfil a mais, com seus seguidores. Na Justiça, a busca do justiçamento,
a justiça com as próprias mãos e a interpretação de que toda forma de
garantismo como maneira de livrar os poderosos dos rigores da lei.

Em cada escaninho de poder, cada detentor de poder, pequeno, médio ou
grande, se julga com liberdade para exercitar seu voluntarismo. O~descarrilhamento das estruturas de poder se dá para fora e para dentro.

Nesse quadro, dois personagens emergiram exercitando uma violência
descontrolada: os grupos de mídia, atropelados pelas novas formas de
comunicação; e a oposição aos governos que conseguiram montar políticas
vitoriosas de inclusão.

Essas políticas geram novos consumidores, mas também novos cidadãos. O~partido que patrocina a inclusão ganha uma massa de votos que
desequilibra o jogo de alternâncias no poder, levando a oposição a uma
luta extra"-eleitoral encarniçada para se manter no jogo. E~as armas
principais às quais têm recorrido, seja na Fox News, seja na Veja, é a
exploração radical da intolerância existente na sociedade.

\section{Segundo conjunto: o caso brasileiro}

O caso brasileiro foi montado em cima dessas características globais
atuais acrescidas das particularidades internas. Alguns dos episódios
condicionantes do momento:

1.~~~ Roberto Civita importa dos \versal{EUA} o estilo escatológico de Rupert
Murdock. Em 2005 há o pacto dos grupos de mídia para enfrentar a
globalização do setor. A~campanha pró"-armamento descobre um mercado
promissor na exploração do discurso do ódio e em uma nova direita que
nascia.

2.~~~ A enorme inclusão social ocorrida na última década, cujos
conflitos foram amenizados pela fase de bonança econômica e explodem com
o fim da bonança mundial dos commodities e com os erros políticos e
econômicos cometidos por Dilma em 2014 e 2015.

3.~~~ O desmonte da base de apoio do período Lula, somado à corrosão na
popularidade da presidente, abrem uma vulnerabilidade inédita no
Executivo.

4.~~~ Antes disso, a cobertura intensiva do julgamento do ``mensalão'',
visando obscurecer a \versal{CPMI} de Cachoeira, testando pela primeira vez a
massificação das denúncias de corrupção de forma continuada. A~campanha
do ``mensalão'' ajudou a fixar na classe média a ideia de que a
corrupção estava no \versal{PT} e a solução, no seu extermínio.

Criou"-se o clima adequado para os grandes movimentos de manada.

A ira difusa em relação ao desconforto atual, ao sistema político, à
lentidão do Judiciário, tudo isto é canalizado contra o governo. E~a
Lava Jato bateu na imensa mina de corrupção montada em torno da
Petrobras e amplificou os ecos não esquecidos do mensalão.

A disfuncionalidade política, de governo e oposição, a desconfiança em
relação ao Judiciário (especialmente após a frustração das Operações
Satiagraha e Castelo de Areia) ampliaram os movimentos de ação direta,
nas ruas.

\section{Terceiro conjunto: a orquestração política}

Desse conjunto de fatores germinaram as ações radicalizantes que
passamos a analisar a seguir.

Na análise sobre os personagens envolvidos, haverá certa dificuldade em
identificar as movimentações.

Para facilitar o raciocínio, vamos dividi"-los em três grupos principais:

1.~~~ ~Aqueles cujo fator mobilizante é a indignação pura e simples.
Entram aí movimentos de rua.

2.~~~ O grupo motivado pela disputa corporativa por espaço político.
Inserem"-se aí procuradores, delegados, juízes de primeira instância,
técnicos do \versal{TCU} (Tribunal de Contas da União)

3.~~~ E há o terceiro grupo, o dos conspiradores efetivos, manobrando as
circunstâncias do momento.

Para nossa análise, interessa identificar esse terceiro grupo.

Os pontos que chamam a atenção, por induzir a uma ação concertada são os
seguintes, tendo como instrumento de guerra a parceria mídia"-Lava Jato:

1.~~~ A estratégia jurídica

A perfeita coordenação entre as estratégias de Gilmar Mendes e Dias
Toffoli no \versal{TSE} e Sérgio Moro na Lava Jato -- de encontrar indícios para
criminalizar o caixa 1 da campanha de Dilma.

A concatenação entre a Lava Jato e a Zelotes é outro indício de atuação
coordenada.

Além disso, a maneira como um juiz de Primeira Instância, no Paraná,
conseguiu deflagrar a mais abrangente operação criminal brasileira cujo
único elo com sua jurisdição era um doleiro que já tivera os benefícios
da delação premiada e voltara a prevaricar.

2.~~~ A estratégia política.

A concatenação entre o fluxo de vazamentos da Lava Jato e as estratégias
pró"-impeachment da oposição.

A blindagem aos nomes de oposição que surgem nas delações premiadas.

Em momentos mais críticos, a Lava Jato providencia um fluxo maior de
factoides destinados a estimular a opinião pública.

3.~~~ A estratégia econômica.

Um viés totalmente internacionalizante, no âmbito do Congresso --- toda
vez que o governo entra em sinuca, a saída apresentada consiste na
flexibilização da Lei do Petróleo e das políticas sociais -- e no âmbito
da própria Lava Jato e do Ministério Público Federal através dos acordos
de cooperação internacional. Parece haver um trabalho articulado para
atingir setores de interesse direto dos Estados Unidos: Petrobras com a
lei do petróleo, empreiteiras brasileiras (que se tornaram competitivas
internacionalmente) e setor eletronuclear.

Na visita do \versal{PGR} a Washington, por exemplo, levou informações contra a
Petrobras e trouxe informações de escândalos na Eletronuclear. Há um
ataque sem quartel a todas as políticas visando fortalecer a economia
interna, da mesma maneira que na Operação Mãos Limpas.

A ideologia do jogo -- expresso não apenas na oposição, na Lava Jato e
na própria Procuradoria Geral da República, através da chamada
cooperação internacional -- é a do internacionalismo. A~corrupção é
decorrência de uma economia fechada. O~mercado liberta, o Estado
corrompe.

A não ser os grupos ligados a direitos humanos, o grosso dos
procuradores provavelmente esposa essa visão reducionista de mercado.

\section{Os personagens do jogo}

Os personagens do jogo serão analisados com base nas informações que
tenho sobre eles e nas impressões deixadas pela forma como estão
jogando.

Há alguns pontos centrais de articulação -- como o Instituto Milenium,
que continua cumprindo à altura seu papel de sucessor do velho \versal{IBAD}
(Instituto Brasileiro de Ação Democrática). Fora ele, não há sinais de
locais mais expressivos de articulação.

Entendidos esses aspectos do jogo, vamos aos jogadores.

\subsection{Congresso}

Há um conjunto de personagens secundários que ganharam visibilidade por
ecoar a intolerância. Políticos como Carlos Sampaio, Mendonça Neto,
Agripino Maia, Aloisio Nunes, Ronaldo Caiado, Roberto Freire,
vociferantes, mas personagens menores que apenas atendem à demanda da
mídia por catarse. De certo modo, comportam"-se como as claques de
programas de auditório. Os profissionais se preservam.

São quatro os personagens a serem analisados.

O primeiro deles é Aécio Neves, o candidato do \versal{PSDB} nas últimas eleições
presidenciais. Tem importância apenas pelo recall das últimas eleições.

É politicamente inexpressivo, incapaz sequer de articular de forma
consistente interesses mais complexos.

É candidato a um indiciamento próximo por duas razões: em algum momento
o \versal{MPF} terá que mostrar isenção e a cada dia se avolumam mais evidências
contra ele. A~segunda razão é que ele se tornou uma liderança
disfuncional, incapaz de articular um corpo mínimo de ideias e
estratégias.

O grupo profissional tem três elementos: Michel Temer, Renan Calheiros e
José Serra.

No curto espaço de tempo em que se tornou protagonista político, Temer
não demonstrou maior envergadura política. Encampou a tal agenda
liberal, surgiu no horizonte político e desapareceu como um cometa
fugaz.

Renan é político com uma concepção muito mais sólida de poder. Fareja
como ninguém os centros de poder e sabe agir com rigor e objetividade.
Provavelmente sua aproximação com a agenda liberal e com as mudanças na
lei do petróleo se prendam a essa percepção mais apurada sobre poder.
Sob ameaça da Lava Jato, como estratégia de sobrevivência tratou de se
aproximar do foco mais influente do poder.

José Serra é o grande articulador. É~o político que transita pelos
grandes grupos internacionais -- lembrem"-se do Wikileaks com ele
prometendo à lobista da indústria petrolífera flexibilizar a lei do
petróleo assim que eleito. Transita também pela mídia e pelo submundo do
Judiciário -- Polícia Federal e procuradores, com os quais montou uma
verdadeira indústria de dossiês.~

Foi curioso o açodamento de Serra e de Eduardo ~Cunha assim que se
comprovou o desmanche da base de apoio de Dilma: ambos saíram correndo
para ver quem teria a primazia de primeiro apresentar o projeto para
flexibilizar a lei do petróleo, comprovando que são dois dos maiores
operadores políticos do Congresso

\subsection{Justiça}

Os dois personagens centrais dessa articulação são Gilmar Mendes no \versal{STF}
(Supremo Tribunal Federal) e no \versal{TSE} (Tribunal Superior Eleitoral) e
Sérgio Moro na Lava Jato. São os grandes estrategistas que provavelmente
estreitaram relações entre si durante o julgamento do ``mensalão''.

Suas estratégias se completam, assim como o recurso recorrente à mídia,
Gilmar em episódios ostensivamente manipulados, como o grampo sem áudio,
o grampo no Supremo ou o encontro com Lula, factoides que explodiam e
desapareciam como fogo fátuo; Moro de forma profissional, abastecendo a
mídia com jorros ininterruptos de notícias e factoides.

As ligações históricas de Gilmar com José Serra, o trabalho de cooptação
de Dias Toffoli, seu trabalho pertinaz no \versal{STF} e \versal{TSE}, o colocam como
personagem central da conspiração. O~que, convenhamos, não chega a ser
nenhuma novidade.

A Força Tarefa da Lava Jato, Moro, os procuradores e delegados, são o
epicentro operacional dessas articulações. Mas não conseguiram disfarçar
a posição ostensivamente partidária. Já viraram o fio há algum tempo.

Já o problema do \versal{MPF} é muito mais o de perda de controle sobre os jovens
procuradores, devido ao fato do Procurador Geral Rodrigo Janot responder
à sua base, e não à presidência da República, como determina a
Constituição.

Há três fatores que afetam a imagem do \versal{MPF} como um todo.

Um deles, as entrevistas políticas do procurador falastrão, Carlos
Fernando~dos Santos Lima. O~segundo, a excessiva politização do \versal{MPF} do
Distrito Federal. O~terceiro, o exibicionismo de jovens procuradores,
tentando de todas as formas se habilitar aos holofotes da mídia através
de representações estapafúrdias.

Mesmo a maneira como se insere na cooperação internacional -- na qual é
patente o alinhamento com interesses dos Estados Unidos -- parece muito
mais falta de reflexão interna sobre os aspectos geopolíticos da
cooperação, do que qualquer postura conspiratória.

Quanto a Janot, em que pese a blindagem de Aécio Neves, é uma figura
pública respeitável, preso a esses dilemas entre garantir a legalidade
e, ao mesmo tempo, não remar contra o sentimento de onipotência que
acometeu a categoria, após a Lava Jato. Vai acabar se queimando pela
incapacidade de disciplinar o exibicionismo de procuradores e de blindar
Aécio Neves.

\subsection{Mídia}

Aí se concentra o poder maior, que está na Globo. Veja, Folha e Estadão
são ~apenas agentes auxiliares, que fornecem as pautas para o Jornal
Nacional.
