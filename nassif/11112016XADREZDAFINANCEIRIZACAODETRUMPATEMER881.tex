\chapterspecial{11/\allowbreak{}11/\allowbreak{}2016 Xadrez da financeirização de Trump a Temer}{}{}
 

Não aposte no fracasso de Donald Trump. Esse americano truculento,
primário, grosseiro, tem trunfos na manga para algumas mudanças
fundamentais, que poderão relançar a economia norte"-americana.

Poderá ser um Franklin Delano Roosevelt às avessas. Enquanto Roosevelt
salvou a economia norte"-americana brandindo o discurso da solidariedade,
valendo"-se do triunfo econômico para consagrar a democracia, Trump
poderá trazer de volta o dinamismo aos \versal{EUA} e consagrar a intolerância e
o voluntarismo como fator determinante.

Como é uma questão intrincada, vamos por partes para compor o nosso
xadrez.

\section{Peça 1 -- a financeirização no século 19}

No século 19 há a internacionalização do sistema financeiro mundial,
comandando a globalização do período. Instituiu"-se o livre fluxo de
capitais e montou"-se uma articulação entre o Banco da Inglaterra e
demais bancos centrais, deles para com os financistas dos países
periféricos, visando conseguir o máximo de vantagens para o capital.

No início, o padrão ouro impedia jogadas especulativas mais profundas.
Mas ficava a liquidez global dependendo dos humores do Banco da
Inglaterra. Bastava o banco subir suas taxas para enxugar o ouro dos
demais países e, consequentemente, obrigá"-los a restringir as moedas em
circulação --- que tinham o ouro como lastro.

Essas restrições levaram ao abandono gradativo do padrão ouro,
substituído pelo papel moeda e pela desregulação total dos mercados. E,
aí, bolhas especulativas passaram a explodir em várias partes do
planeta.

\section{Peça 2 -- os principais personagens da financeirização}

O jogo político dos grupos financeiros dava"-se através dos seguintes
personagens, que descrevo em meu livro ``Os Cabeças de Planilha'':

1.~O financista interno, que faz o meio campo entre os capitalistas
nativos e o meio político e a ponte com o sistema bancário inglês.

2.~O economista, subordinado ao financista, que se apresenta como o
portador da nova ciência mundial, capaz de fazer os países atrasados se
desenvolverem.

De posse desses dois ativos --- recursos financeiros e a plataforma
política embasada no discurso do economista \mbox{---,} o financista monta os
pactos internos, com políticos e partidos..

3. O~terceiro personagem é o~político, presidente ou Ministro da
Fazenda.

4. A~mídia, influente nos centros de poder, mesmo para um país com
apenas 5\% de alfabetizados. Foi através dos jornais que Rui Barbosa
ganhou sua primeira vitrine, combatendo os planos de remonetização da
economia do gabinete de Ouro Preto, o último Ministro da Fazenda do
Império.

Deposto o Imperador, o primeiro presidente (golpista) Marechal Deodoro
da Fonseca convida Rui para Ministro da Fazenda. Sua primeira atitude
foi retomar o plano de remonetização da economia, mas aí escolhendo o
seu banqueiro, o Conselheiro Mayrink, espécie de Daniel Dantas da época,
banqueiro sem capital que passa a controlar a emissão de moeda no país.

O resultado final são bolhas especulativas, tacadas financeiras de Rui e
seus sócios e movimentos voláteis do câmbio que impedem a consolidação
de qualquer tentativa de estabilidade e de industrialização. O~país vai
se arrastando até os anos 30.

Foi só após mais uma moratória, no início dos anos 30, o governo Vargas
encontrou força para proibir definitivamente o livre fluxo dos capitais.
Sem ter como se movimentar, esses capitais desceram ao reino da terra,
financiando indústrias e novos empreendimentos. O~Brasil começava a
nascer ali.

\section{Peça 3 --- a 2a Guerra e os regimes que venceram o financismo}

Essa hegemonia massacrante do capital produziu concentração de renda por
todas as partes, dificultou a ascensão econômica de países, produziu
conflitos comerciais e se encerrou com a Primeira Guerra Mundial. Ganhou
uma sobrevida até a crise de 1929. E~foi sepultado no pós"-Segunda
Guerra.

Os abusos do financismo lançaram um manto de descrédito sobre o sistema
democrático em todo mundo. Percebeu"-se que as democracias deixavam
países desarmados, ante a influência do dinheiro nas eleições, através
da parceria de capitalistas locais com os internacionais e com políticos
locais, deixando o país indefeso ante o capital predador.

Em três frentes houve o rompimento com a financeirização. Na Europa, com
a ascensão do nazifascismo. Na Eurásia, com a ascensão do comunismo. Nos
Estados Unidos, com a new deal de Roosevelt e seu plano de grandes obras
públicas.

Nos três casos, os países se transformaram em potências econômicas.
Mesmo a Alemanha, com as enormes dívidas de guerra, encontrou
alternativas criativas que lhe permitiram adquirir matérias primas e
retomar o crescimento.

A decisão sobre o melhor regime se deu nos campos de batalha. ~Com a
vitória dos aliados, se consolida o modelo democrático de mercado
norte"-americano e o comunismo soviético.

O mundo se reorganiza em torno de novas instituições, disciplinam"-se os
fluxos de capitais com um novo modelo cambial e criam"-se instituições
para promover o desenvolvimento nas regiões mais pobres e amparar países
com problemas de solvência.

\section{Peça 4 --- o neoliberalismo e os desequilíbrios sociais}

As lições da Guerra encerram"-se em 1972. O~presidente norte"-americano
Richard Nixon promove a desvinculação do dólar com o ouro, reinstaurando
o primado da financeirização global.

Gradativamente a desregulação financeira vai estendendo o manto da
financeirização sobre os diversos países, impulsionada pelas mudanças
tecnológicas e pela falência dos modelos centralizados de planejamento.
Atinge o auge com a queda do Muro de Berlim e o fim da União Soviética.
E~não para.

As bolhas especulativas se sucedem. À~bolha da prata no início dos anos
80, se sucedem as bolhas do mercado imobiliário nova"-iorquino, no final
dos anos 80, a bolha das montadoras, as bolhas cambiais que balançam da
Inglaterra ao México, da Coreia ao Brasil, as bolhas de empresas de
tecnologia. E, à frente do \versal{FED} (o Banco Central norte"-americano), um
acadêmico, Alan Greenspan, sendo enaltecido como gênio das finanças,
capaz de controlar todas as bolhas e garantir a prosperidade eterna.

O sistema financeiro passa a reciclar recursos de potentados árabes,
empresários japoneses, do narcotráfico, da corrupção política e
empresarial. E~consolida"-se valendo"-se da adesão ampla com o meio
acadêmico, especialmente os economistas, desenvolvendo teorias cada vez
mais despregadas da realidade, visando iludir eleitores -- com o bordão
da ``lição de casa'' -- e preservar os ganhos do capital.

E também se beneficia com a adesão incondicional da mídia, tanto
norte"-americana quanto de outros países.

\section{Peça 5 -- os efeitos da crise de 2008}

Nas eleições de 2008, a mídia foi unanimemente anti"-Obama. Agora,
anti"-Donald Trump. O~que ambos têm em comum? Certamente, não as posições
morais, um Obama libertário e um Trump troglodita, mas a crítica ao
sistema financeiro e a ameaça ao reinado do financismo.

Mesmo assim, o financismo logrou preservar seu poder político. Quando
veio a crise de 2008, um pesado manto ideológico fez com que as ações de
salvamento focassem o sistema bancário, em detrimento dos demais gastos
públicos, inclusive dos devedores. Mais que isso, impôs cortes fiscais
draconianos em economias que afundavam, ampliando ainda mais a recessão.
Desmontou estados nacionais.

Tudo isso levou a uma enorme reação global contra o sistema, basicamente
o modelo financista que gerou concentração de renda, entregou a maior
crise desde 1929, e ainda arrebentou com os orçamentos nacionais.

Mais que isso, a globalização das corporações promoveu a precarização do
trabalho, a redução do emprego. No próprio país sede da financeirização,
a lógica do capital atropelou a lógica nacional. E~tudo isso era
revestido por um pretenso cientificismo.

O ponto que une todas as ações antiglobalização, portanto, não é a
xenofobia, o racismo explícito -- presente em grupos de ultradireita e
no discurso de Trump \mbox{---,} mas a crítica a essa financeirização.

O mundo racional, cartesiano, intelectual, obedecia a um pêndulo que ora
gerava políticas mais nacionais, ora políticas mais
internacionalizantes. Com os exageros de mercado, nos Estados Unidos a
bola da vez caiu no colo da malta, tangida por discursos racialistas.

\section{Peça 6 --- o desafio de Trump}

A subordinação total à financeirização, criou uma falsa ciência, com
tantos absurdos, que bastará romper com seus dogmas para revitalizar as
economias.

Há uma grande possibilidade da política econômica de Trump ser
bem"-sucedida, se romper de vez com a financeirização. Caso comprometa o
orçamento público com investimentos pesados em infraestrutura, práticas
industriais protecionistas e outros recursos antiglobalização,
conquistará uma vitória maiúscula para seu país, e problemas para o
resto do mundo.

Se vingarem suas propostas de política externa, reduzirá drasticamente
as intervenções norte"-americanas no mundo, principais responsáveis pelas
guerras e pelos êxodos globais.

A eventualidade de seu sucesso ampliará as práticas protecionistas por
todo o planeta. E~Trump poderá ser o grande inspirador dos racistas,
xenófobos, intolerantes por todos os pontos do planeta.

O modelo do cowboy solitário que se impõe, pela vontade, sobre partidos
políticos e sobre o sistema será um referencial perigosíssimo para os
avanços dos direitos sociais.

\section{Peça 7 --- o neoliberalismo à brasileira}

No Brasil, no momento em que se resolvesse a questão da dívida pública e
dos juros pagos pelo Tesouro, a recuperação da economia seria questão de
meses. A~política monetária do Banco Central, 14\% de taxa básica de
juros para uma economia que amarga a maior depressão da história, não é
apenas um erro de análise: é crime. Se um dia houver um tribunal para
julgar os grandes crimes financeiros da história, não isentará um
dirigente de \versal{BC} brasileiro desde Francisco Gros a Ilan Goldjan -- o
criador da planilha das metas inflacionárias.

Mais que isso, o eventual sucesso de Trump poderá demonstrar que o
sistema de freios e contrapesos das democracias acabou criando freios
demais, tornando as democracias disfuncionais e exigindo o estadista
capaz de atropelar leis, constituição, separação de poderes.

E o maior exemplo da crise da democracia está aqui mesmo, com um
Ministro do \versal{STF} (Supremo Tribunal Federal), Luis Roberto Barroso,
defendendo o estado de exceção, um mero procurador do Tribunal de Contas
definindo políticas fiscais, um juiz de primeira instância e
procuradores regionais atropelando direitos básicos, em cima do vácuo do
Supremo. E, principalmente, um golpe que resultou na ampliação da falta
de rumo, das benesses financeiras em um momento em que a potência
hegemônica do capitalismo financeiro diz não.

A eleição de Trump é o ponto de não retorno. O~que virá daqui em diante
é uma incógnita ampla, que passa pelo reposicionamento estratégico da
China, pela maneira como a União Europeia irá se comportar, pelos
impactos sobre o comércio global e, principalmente, sobre as maneiras
como a democracia irá se modificar, para ser preservada.~

E, em todos os campos institucionais, o Brasil irá para essa guerra
armado com personalidades públicas de pequena dimensão, Michel Temer,
Eliseu Padilha, Geddel, Henrique Meirelles, José Serra, \versal{FHC}, Luis
Roberto Barroso, Rodrigo Maia, Carmen Lucia, Rodrigo Janot, nenhum à
altura dos grandes desafios que se apresentam em seus campos.

Aumentam as possibilidades de uma solução bonapartista em 2018.
