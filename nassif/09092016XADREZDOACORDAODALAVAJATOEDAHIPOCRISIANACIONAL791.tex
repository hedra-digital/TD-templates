\chapterspecial{09/\allowbreak{}09/\allowbreak{}2016 Xadrez do acordão da Lava Jato e da hipocrisia nacional}{}{}
 

Conforme previsto, caminha"-se para um acordão em torno da Lava Jato que
lança a crise política em uma nova etapa com desdobramentos
imprevisíveis.

\section{Movimento 1 -- os ajustes na Lava Jato}

Trata"-se de um movimento radical do Procurador Geral da República (\versal{PGR})
Rodrigo Janot, que praticamente fecha a linha de raciocínio que vimos
desenvolvendo sobre sua estratégia política.

Peça 1~-- monta"-se o jogo de cena entre Gilmar Mendes e Janot. Janot
chuta para o Supremo a denúncia do senador Aécio Neves. Gilmar mata no
peito e devolve para Janot que se enfurece e chuta de novo de volta ao
Supremo. Terminado o jogo para a plateia, Janot guarda a bola e não se
ouve mais falar nas denúncias contra Aécio. Nem contra Serra. Nem contra
Temer. Nem contra Geddel. Nem contra Padilha.

Peça 2~-- duas megadelações entram na linha de montagem da Lava Jato: a
de Marcelo Odebrecht e de Léo Pinheiro, da \versal{OAS}. Pelas informações que
circulam, Marcelo só entregaria o caixa 2; Léo entregaria as propinas de
corrupção, em dinheiro vivo ou em pagamento em off"-shores no exterior.

Peça 3~-- advogados de José Serra declaram à colunista Mônica Bergamo
estarem aliviados, porque a delação de Marcelo Odebrecht só versaria
sobre caixa 2. Ficavam duas questões pairando no ar. Se Caixa 2 é crime,
qual a razão do alívio? E se ainda haveria a delação de Léo Pinheiro,
qual o motivo da celebração?

Peça 4~-- em um dos Xadrez matamos a primeira charada e antecipamos que
a Câmara estava estudando uma saída, com a assessoria luxuosa de Gilmar
Mendes, visando anistiar o caixa 2 e criminalizar apenas o que fosse
considerado dinheiro de corrupção, para enriquecimento pessoal.

Peça 5~-- a segunda questão -- de Léo Pinheiro delatando corrupção ---
foi trabalhada em seguida, quando monta"-se o jogo de cena,
de~Veja~publicando uma não"-denúncia contra Dias Toffoli e,
imediatamente, Janot acusando os advogados da \versal{OAS} pelo vazamento e
interrompendo o acordo de delação, ao mesmo tempo em que Gilmar investia
contra a Lava Jato, anunciando que o Supremo definiria as regras das
delações futuras.~Há movimentos do lado da Lava Jato, do lado da \versal{PGR},
Gilmar se acalma, diz apoiar a Lava Jato. A~chacoalhada, sutil como um
caminhão de abóboras que passa em um buraco, permitiu ajustar todas as
peças, como se verá a seguir~.

Peça 6~-- Avança"-se na tal Lei da Anistia do caixa 2 e o caso passa a
ser analisado também pelo \versal{TSE} (Tribunal Superior Eleitoral), presidido
por Gilmar.

Peça 7~-- Ontem Janot abriu mão das sutilezas, dos rapapés, das manobras
florentinas, dos disfarces para sustentar a presunção de isenção e
rasgou a fantasia, nomeando o subprocurador Bonifácio de Andrada para o
lugar de Ela Wieko, na vice"-Procuradoria Geral. Não se trata apenas de
um procurador conservador, mas de alguém unha e carne com Aécio Neves e
com Gilmar Mendes. Janot sempre foi próximo a Aécio, inclusive através
do ex"-\versal{PGR} Aristides Junqueira, com quem trabalhou e que é primo de
Aécio. Com Bonifácio, estreita ainda mais os laços.

Fecho~-- tudo indica que se resolve o imbróglio da Lava Jato da seguinte
maneira:

1.~~~~~Avança"-se na nova Lei da Anistia, com controle de Gilmar através
do \versal{TSE}.

2.~~~~~Bonifácio de Andrada faz o meio"-campo de Janot com Gilmar e
Aécio, ajudando na blindagem e evitando qualquer surpresa, que poderia
acontecer com Ela na vice"-\versal{PGR}.

3.~~~~~Monta"-se um acordo com a Lava Jato prorrogando por um ano seus
trabalhos e definindo um pacto tácito de, tanto ela quanto a \versal{PGR},
continuar focando exclusivamente em Lula e no \versal{PT}, exigência que em nada
irá descontentar os membros da força"-tarefa.

4.~~~~~Agora há um cabo de guerra entre a Lava Jato e Léo Pinheiro. Léo
já informou que não aceitaria delatar apenas o \versal{PT}. Janot interrompeu as
negociações sobre a delação afim de que Léo e seus advogados ``mudem a
estratégia'', como admitiram procuradores à velha mídia. Sérgio Moro
prendeu"-o novamente, invadiram de novo sua casa, no movimento recorrente
de tortura psicológica até que aceite os termos propostos por Janot e a
Lava Jato.

\section{Movimento 2 -- o teatro burlesco no Palácio}

Aí se entra em um terreno delicado.

A política move"-se no terreno cediço da hipocrisia. Faz parte das normas
tácitas da democracia representativa os acordos espúrios, os interesses
de grupo disfarçados em interesses gerais, a presunção de isenção da
Justiça.

Mas o jogo político exige a dramaturgia, a hipocrisia dourada. E~como
criar um enredo minimamente legitimador com um suspeitíssimo Geddel
Vieira Lima, e sua postura de açougueiro suado cuspindo imprecações? Ou
de Eliseu Padilha, e seu ar melífluo de o"-que"-vier"-eu"-traço? Ou de José
Serra e as demonstrações diárias da mais rotunda ignorância em
diplomacia e um deslumbramento tão juvenil com John Kerry que só faltou
beijo na boca? Ou de Temer e suas mesquinharias, pequenas vinganças,
incapaz de entender a dimensão do cargo e do poder que lhe conferiram?

O ``Fora Temer'' não se deve apenas à situação econômica precária, mas
ao profundo sentimento de que o país foi entregue a usurpadores. Mesmo
com toda a velha mídia encenando, não se conseguiu conferir nenhum
verniz a esses personagens burlescos.

Ressuscitou"-se até esse anacronismo da ``primeira dama'', tentando
recriar o mito do casal 20, de Kennedy"-Jacqueline, Jango"-Tereza e
filhos, peças do repertório dos anos 50.~~Ontem, o Estadão lançou a
emocionante questão: vote no melhor ``look'' da primeira dama que,
aparentemente, vestiu quatro ``looks'' durante o dia. Um gaiato votou no
``tomara que caia Temer''.

\section{Movimento 3 -- a \versal{PGR} rasga a fantasia}

Como fica, agora, com o próprio Janot abrindo mão da cautela e expondo
seu jogo?

Janot foi um dos artífices do golpe. Teve papel central para entregar o
país aos projetos e negócios de Michel Temer, Geddel Vieira Lima, Eliseu
Padilha, Romero Jucá, Moreira Franco e José Serra, entre outros.
Sacrificou"-se apenas Eduardo Cunha no altar da hipocrisia.

Sua intenção evidente é liquidar com Lula e com o \versal{PT}. Mas, para fora e
especialmente para dentro -- para sua tropa -- tem que apresentar
argumentos legitimadores da sua posição, como se o golpe fosse mera
decorrência de procedimentos jurídicos adotados de forma impessoal.~~Era
visível o alívio dos procuradores nas redes sociais, quando Janot
decidiu encaminhar uma denúncia contra Aécio. Porque não tem corporação
que consiga manter a disciplina e o espírito de corpo se não houver
elementos legitimadores da sua atuação, o orgulho na sua atuação, na sua
missão.

Ontem, em suas escaramuças retóricas, Gilmar escancarou a estratégia.
Mesmo contando com toda a estrutura da \versal{PGR}, Janot avançou minimamente
nas denúncias contra políticos.

Para todos os efeitos, há delações contra Temer, Padilha, Geddel, Jucá,
Moreira, Aécio e Serra. E, para todos os efeitos, o grupo está cada vez
mais à vontade exercitando as armas do poder. Dá para entender porque o
temível Janot não infunde um pingo de medo neles?

Com a nomeação de Bonifácio de Andrada abrem"-se definitivamente as
cortinas e o \versal{MPF} entra no palco onde se encena o espetáculo da
hipocrisia nacional.

\section{Movimento 4 -- as vozes da rua}

O jogo torna"-se sumamente interessante.

Os últimos episódios, a violência policial, os sinais cada vez mais
evidentes de se tentar fechar o regime, despertaram um lado influente da
opinião pública, que jamais se moveria em defesa de Dilma, mas começa a
acordar em defesa da democracia.

A ``teoria do choque'' exigia, na ponta, um governante com carisma, um
varão de Plutarco, um moralista compulsivo, que trouxesse o ingrediente
final na consolidação de um projeto fascista. O~enredo não previa o
espetáculo dantesco da votação na Câmara, a pequena dimensão de Michel
Temer, a resistência épica de Dilma Rousseff -- que, de mais fraca
governante da história, na queda ~tornou"-se um símbolo de dignidade da
mulher \mbox{---,} a massacrante diferença de nível entre José Eduardo Cardoso
e Miguel Reali Jr. e Janaina Pascoal.

O grito de ``Fora Temer'' torna"-se cada vez mais nacional.

Por outro lado, a violência das \versal{PM}s de São Paulo e outros estados
mereceu ~a reação do \versal{MPF}, através da \versal{PFDC} (Procuradoria Federal dos
Direitos do Cidadão). Agora, a intenção implícita de Janot é o
esvaziamento dos bolsões de direitos humanos da \versal{PGR}, liquidando com os
últimos pontos legitimadores da instituição.

O baixo nível de corrupção no \versal{MPF} deve"-se a um sentimento de missão que
está sendo jogado pelo ralo pelo jogo político. Sem as bandeiras
legitimadoras, será cada vez mais cada qual por si, com os mais
oportunistas procurando exercitar a dose de poder que o \versal{MPF} conquistou
com o golpe. A~médio prazo, dr. Janot vai promover o desmonte da tropa,
não se tenha dúvida.

Qual o desfecho? Aumento da violência em uma ponta, aumento da
indignação na outra. O~país institucional encontrará uma saída para essa
escalada de violência, ou nos conformaremos em ser uma Argentina de
Macri e uma Venezuela de Maduro?

Na mídia e em alguns altos postos do Estado, não se fica a dever quase
nada à Venezuela. E, em uma época que se tem os olhos do mundo sobre o
Brasil.
