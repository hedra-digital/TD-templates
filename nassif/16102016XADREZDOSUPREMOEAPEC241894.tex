\chapterspecial{16/\allowbreak{}10/\allowbreak{}2016 Xadrez do Supremo e a \versal{PEC} 241}{}{}
 

\section{Peça 1 -- Carmen Lucia e o oportunismo institucional
brasileiro}

Duas medidas temerárias da Ministra Carmen Lúcia, nova presidente do \versal{STF}
(Supremo Tribunal Federal), mostram que a corte suprema volta se alinhar
à mediocridade institucional brasileira, que já domina o Executivo e o
Congresso.

O país está nas mãos ou de figuras suspeitas, ou, como parece ser o caso
da Ministra, dos sem"-noção.

A primeira medida foi colocar o Supremo como mediador de problemas de
segurança nacional e convocar o Ministério da Defesa e as Forças Armadas
para discutir a questão. Esse assunto abordarei em um próximo Xadrez,
mas demonstra uma irresponsabilidade institucional ampla, que só pode
ser explicada pelo açodamento e despreparo da Ministra.

A segunda, o apoio, em nome pessoal -- mas falando na condição de
presidente do Supremo -- à \versal{PEC} (Proposta de Emenda Constitucional) 241.

\section{Peça 2 -- o Supremo como guardião dos direitos fundamentais}

Recentemente, o Ministro Ricardo Lewandowski definiu o século 21 como o
século do Judiciário. Dizia ele que, após os direitos fundamentais terem
se consolidado no século 20, no 21 haveria o avanço dos direitos das
minorias, da busca da igualdade de gêneros, raças e condição social.

Ora, a arena central para discutir direitos difusos é o orçamento
público. É~dele que advém os recursos para educação, saúde, defesa das
minorias e até para o próprio funcionamento do Sistema de Justiça.

Segundo muito autores, a \versal{PEC} 241 -- do limite nominal de gastos -- é uma
ameaça frontal aos direitos fundamentais. Se faltarem recursos para
saúde, educação, para o combate à fome, poderia jogar o país em um
cataclismo social. E~não dá para agir após o fato consumado. Os efeitos
dos cortes de recursos à saúde só se manifestam após o ocorrido, na
forma das epidemias, da ampliação das doenças, das estatísticas de
doenças e óbitos. E~estatísticas revelam o que já ocorreu, o fato
consumado. Portanto, está em jogo o destino de milhões de brasileiros.

Se o Supremo se pretende, de fato, o protagonismo civilizatório, não
haverá como não se pronunciar sobre essa ameaça, antes que se
concretize, não da forma superficial e irresponsável de Carmen Lúcia,
mas através de audiências públicas, convocando especialistas para
esmiuçar o tema.

Há plenas condições de estimar os efeitos da \versal{AP} 241 sobre os gastos com
saúde e educação, assim como as consequências para o setor de uma
redução expressiva dos recursos previstos.

\section{Peça 3 -- a economia como uma ciência ética}

O segundo passo seria questionar o poder dos economistas em se apropriar
da definição do mais relevante tema político de um país: a divisão do
bolo orçamentário.

Ao contrário do que supõe o Ministro Luís Roberto Barroso -- outro que
se apressou em hipotecar apoio cego à 241 --- a economia não é uma
ciência neutra, nem aética, nem exata. Nem uma religião só aberta a
iniciados.

Ela trata da distribuição da riqueza, da divisão do orçamento, dos
preços internos vs os preços internacionais, trata do emprego, das
políticas sociais e dos direitos fundamentais. E~se pretende ciência,
estando, portanto, sujeita a testes de racionalidade, de interpretação
de correlações visando checar os objetivos prometidos.

Mesmo as medidas ilegítimas precisam ser legitimadas com a promessa do
retorno social em um ponto qualquer do futuro, como foi o mantra do pote
de ouro no fim do arco"-íris para quem faz a ``lição de casa'' -- que
voltou a ser invocado com a \versal{PEC} 241.

Não existe uma resposta única para determinados problemas -- inflação,
contas fiscais, contas externas. Justamente por isso, cada decisão
precisa explicitar a relação custo"-benefício, inclusive na comparação
com políticas alternativas.

\section{Peça 4 -- as causas do aumento da dívida pública}

\subsection{Passo 1 -- o peso dos juros}

O primeiro ponto a se considerar é sobre os fatores que mais pesam na
dívida pública, já que o objetivo final da \versal{PEC} 241 é o da estabilização
da relação dívida/\allowbreak{}\versal{PIB}. Os juros respondem por 8\% do \versal{PIB}; despesas com
educação e saúde não passam de 1,4\% e 1,7\% do \versal{PIB}. Os juros beneficiam
70 mil pessoas; as despesas com saúde e educação atendem a milhões de
brasileiros, além de serem peça central para o desenvolvimento do país.

Na economia, há escolas que tratam a moeda como uma constante, em torno
da qual se subordinam todos os outros fatores econômicos. E~há os que
tratam moeda e câmbio como um elemento a mais na busca do equilíbrio
econômico.

Para ajudar no raciocínio dos sábios Ministros da Corte Suprema: nenhum
economista estrangeiro sério, de nenhuma escola, do mais intimorato
monetarista ao neoliberal mais empedernido, consideraria normal um
modelo em que os juros queimam anualmente o equivalente a 8\% do \versal{PIB}. Ou
que se necessite de uma taxa real de juros de mais de 7\% ao ano a
pregtexto de debelar a inflação.

Se os juros têm esse peso na composição da dívida e do déficit, é
questão de perguntar: a manutenção dos juros no nível atual é
imprescindível para o funcionamento da economia?

Aí se entraria na segunda parte da questão.

\subsection{Passo 2 -- o papel dos juros}

Hoje em dia,a maior justificativa teórica para juros altos vem da
chamada teoria das metas inflacionárias, que define uma relação entre
expectativa futura de inflação e taxa de juros. Se aumenta a expectativa
futura de inflação, há um aumento mais que proporcional da taxa Selic,
de maneira a ampliar o ganho dos rentistas.

Hoje em dia, a expectativa é de 4,5\% em 2017 para uma Selic de 14,25\%.
A~diferença corresponde aos juros reais, de longe a mais alta taxa do
planeta.

A lógica por trás dos juros altos é a seguinte:

\begin{enumerate}
\itemsep1pt\parskip0pt\parsep0pt
\item
  1.~~~~ Com mais juros, há menor demanda por crédito, por compras, por
  capital de giro.
\item
  2.~~~~ Com menos demanda, os preços tendem a cair.
\end{enumerate}

E há inúmeros argumentos contrários:

\begin{enumerate}
\itemsep1pt\parskip0pt\parsep0pt
\item
  1.~~~~ Quando o \versal{PIB} cai 5\%, não se pode atribuir a inflação ao
  excesso de demanda. Portanto, não há nenhuma justificativa para manter
  juros elevados com \versal{PIB} em queda.
\item
  2.~~~~ Choque de juros combate inflação de demanda. Nos últimos anos,
  a inflação brasileira tem sido de oferta (seca, quebra de safras), de
  origem externa (elevação de preços de commodities), inercial (inflação
  do passado corrigindo contratos futuros), fatores que não são afetados
  por políticas de juros.
\item
  3.~~~~ Portanto, a política monetária serve apenas para sugar o
  orçamento, promover a concentração de riqueza e desviar recursos que
  deveriam ir para infraestrutura ou para o atendimento das necessidades
  básicas da população.
\end{enumerate}

Se o Ministro Barroso substituísse a fé cega pela razão, convocaria uma
audiência pública com economistas que sustentam que a \versal{PEC} 241
desmantelará a saúde e os que garantem que promoverá o desenvolvimento.

Aí poderia esclarecer algumas dúvidas econômicas:

Por que razão, praticando os mais altos juros do planeta, o sistema de
metas inflacionárias não logrou dominar a inflação?

Os economistas dirão que é porque a inflação brasileira é influenciada
por choques de oferta (secas, quebras de safra), pelas cotações
internacionais (para as commodities), por contratos indexados à inflação
passada.

Os críticos da política monetária dirão que o modelo de dívida pública
(com títulos indexados à expectativa de inflação futura) induz o mercado
a sempre puxar para cima das expectativas inflacionárias, para ampliar
seus ganhos.

Com a ajuda de assessores, ou de outros economistas convidados para a
consulta pública, o Supremo poderia conhecer políticas alternativas e
saídas para os problemas estruturais apontados.

Poderia comparar o impacto da Selic sobre o custo do crédito com medidas
em que o Tesouro, em vez e pagar, ganha: como por exemplo, impor um \versal{IOF}
salgado sobre as operações de crédito; ou reduzir os prazos dos
financiamentos; ou exigir um pagamento à vista maior.

Ou poderia montar um estudo hipotético da infraestrutura brasileira, se
nela fossem aplicados os recursos oriundos da sobretaxa utilizada pelo
Banco Central na definição da taxa Selic.

\subsection{Passo 3 -- as amarras ideológicas}

Nem se pense que a 241 nasce exclusivamente das maldades da era Temer. A~proposta de desvinculação das despesas obrigatórias vem do governo Dilma
Rousseff, apesar das constatações de que o crescimento das despesas
aumentou por conta das isenções fiscais irresponsáveis concedidas no seu
governo~{(clique
aqui)}.

Os governos Lula e Dilma reduziram a taxa de juros básica da economia em
relação aos crimes monetários do governo \versal{FHC} -- que chegou a bancar
taxas de 45\% ao ano \mbox{---,} mas jamais tiveram coragem política de romper
com o nó górdio das taxas.

Uma posição firme do Supremo poderia obrigar o governo Temer a fazer,
por obrigação legal, o que os demais não fizeram, por receio político.
