\chapterspecial{19/\allowbreak{}12/\allowbreak{}2016 Xadrez da tacada final do golpe da Constituinte}{}{}
 

\section{Capítulo 1 -- o sentimento de salve o Brasil}

No Brasil real, meu amigo Vinicius, de 16 anos, criou um grupo no
WhatsApp com o título ``Salvar nosso Brasil''. O~histórico trabalhista
Almino Afonso, no meio da conversa deixa escapar um ``precisamos salvar
nosso país''. Minha tia Clélia reza todo dia para Deus salvar o Brasil,
embora não saiba exatamente como e de quem. O~jurista Celso Antônio
Bandeira de Mello diz, no meio do almoço, que precisamos salvar o
Brasil. Pelo Twitter, o governador maranhense Flávio Dino repete a mesma
expressão.

Testemunhei todos esses apelos em menos de meio dia. Em todos os pontos
do país há um sentimento difuso de que o Brasil derrete, se desmancha,
em um suicídio lento e ritual. E~a angústia de querer e nada poder
fazer, porque o destino do Brasil não está nas mãos do Brasil real, mas
de uma realidade virtual montada em Brasília.

\section{Capítulo 2 -- o Brasilguai da estufa Brasília}

Em um reino distante chamado Brasilguai, nos limites entre um Brasil que
já era e simulando um Paraguai que já foi, a realidade é envolta em
manchetes feéricas e em redes sociais onde borbulham babas de ira e
sangue, compondo o cenário em que as mais altas autoridades disputam os
clicks da turba e as dobras do poder.

É o pessoal da corte, do país improdutivo que vive em uma estufa onde
são cultivadas as intrigas, os jogos de lisonja e de poder, um mundo
invertebrado, em que as carreiras são construídas por artimanhas,
puxadas de tapete e de saco, palco de cenas desse nível:

\begin{itemize}
\itemsep1pt\parskip0pt\parsep0pt
\item
  A presidente do Supremo Tribunal Federal (\versal{STF}) brada: ``Onde um juiz é
  destratado, eu também sou'' (\url{https:/\allowbreak{}/\allowbreak{}goo.gl/\allowbreak{}rC9\versal{FJ}d}). Recebe
  1.450 cliques no Facebook.
\item
  O Procurador Geral da República adapta o brado: ``Atacar um procurador
  é atacar todo o Ministério Público'', e corre a se informar sobre a
  quantidade de cliques que a frase irá merecer.
\item
  O decano, que não é de ferro, aproveita e se exibe em um self em um
  shopping da moda, sabendo que o discurso anticorrupção é o creme de
  leite que, com sua densidade, disfarça qualquer massa de má
  consistência.
\item
  Não se sabe se o presidente do Senado aproveitou o mote e bradou:
  ``Quem ataca um senador, ataca o Senado''. Se não gritou, pensou. Se
  pensou e não realizou, perdeu o direito aos clicks.
\item
  E, como a realidade virtual não está sujeita às limitações enfadonhas
  da realidade, o Ministro da Justiça declara seu modesto plano de ação:
  erradicar toda a maconha do continente. A~julgar pelos selfies que
  tirou no Paraguai, desbastando uma a uma as mudas de cannabis.
  Repetindo o que disse a Comissão de Ética da Presidência: ``cala"-te,
  Magda!''.
\end{itemize}

É esse simulacro de país virtual que comanda o Brasil real, um país que
poucos anos atrás ambicionava o status de nação moderna. É~lá, que se
desenrola essa ópera que não é bufa, por trágica, cujos próximos
capítulos descreveremos a seguir.

\section{Capítulo 3 -- um apanhado do momento atual}

No Brasilguai, ora são denúncias contra Michel Temer, ora loas a um
simulacro de pacote econômico, ora o Procurador Geral investe contra o
presidente do Senado, ora o Ministro do Supremo contra a Lava Jato; o
\versal{PGR} e o Ministro se acusam mutuamente do objetivo comum de blindar o
presidenciável tucano, enquanto associações de juízes e de procuradores
pressionam o Congresso e procuradores e juízes cometem ``estudantadas'',
mostrando menos maturidade que os secundaristas do Brasil real.

Enfim, um retrato claro da balbúrdia institucional que se instalou no
país improdutivo, aquele que não gera riquezas, mas consome o futuro e
deixou de cumprir com sua responsabilidade constitucional.

Para não perder o fio da narrativa, o leitor deve se fixar apenas no
roteiro principal, que consiste no desmonte do Estado. Idas e vindas são
apenas variações em torno do tema principal. Aí se dará conta de que o
que era intuído há alguns meses, hoje em dia é ação consumada e
ostensiva, mostrando a coerência da narrativa central:

\begin{itemize}
\itemsep1pt\parskip0pt\parsep0pt
\item
  O jogo político da Lava Jato em benefício do \versal{PSDB}. Mais de dois anos
  de operação e nenhuma exceção sequer para confirmar a regra.
\item
  O desmonte do Estado.
\item
  O desmonte de garantias individuais, sociais e ambientais com o avanço
  do estado de exceção.
\item
  A porteira aberta para os negócios do Congresso. Deputados preparam a
  volta do jogo, ruralistas liquidam com regras ambientais, o senador
  Romário avança uma trapaça com \versal{APAE}s e Sociedades Pestalozzi, o
  governo ameaça rever as reservas indígenas.
\end{itemize}

Essa imoralidade ampla e irrestrita é aceita pelo polo condutor do golpe
como uma espécie de dano colateral inevitável para se alcançar os
objetivos finais: o desmonte total do Estado nacional, das políticas
sociais e dos modelos de atuação proativa do Estado, substituindo a ação
política pelo modelo tecnocrático"-jurídico globalizante, já identificado
em várias obras acadêmicas.

Antes de avançar neste Xadrez, aliás, é útil a leitura sobre as
estratégias em torno do chamado ``capitalismo de desastre''.~ Trata"-se
da fórmula já incorporada às estratégias globais, conforme você poderá
conferir aqui (\url{https:/\allowbreak{}/\allowbreak{}goo.gl/\allowbreak{}5NmavZ)}~e no ''Xadrez da Teoria do
Choque e do Capitalismo de Desastre'' (\url{https:/\allowbreak{}/\allowbreak{}goo.gl/\allowbreak{}\versal{V}4Xd\versal{VZ}))}.

A partir daí, fica mais fácil entender a tacada final do golpe: a
tentativa de Constituinte exclusiva com a atual composição do Congresso.

\section{Capítulo 4 -- O golpe final da Constituinte}

Em um Congresso integrado por trânsfugas, fisiológicos, conservadores e
preconceituosos de toda espécie, não há maior ameaça aos setores
populares e modernizantes que o deputado carioca Miro Teixeira e sua
proposta de Constituinte (\url{https:/\allowbreak{}/\allowbreak{}goo.gl/\allowbreak{}xsvtfp}).

Miro é um Cristóvão Buarque profissional. Hoje, é o principal homem da
Globo no Congresso.

Discreto, só interfere no jogo em momentos decisivos, como foi na \versal{CPMI}
de Carlinhos Cachoeira, quando a mídia, especialmente a revista~Veja,
entrou na linha de fogo.

Sua proposta de uma Constituinte com os atuais integrantes do Congresso
é, de longe, a maior ameaça aos direitos sociais desde que o golpe foi
desfechado.

A pretexto de discutir a reforma política, o fator Miro Teixeira se
impõe com essa tese esdrúxula de uma Constituinte com o Congresso atual.

O parágrafo 2 do artigo 101 define o que não poderá ser deliberado, de
tal maneira que abre espaço para transformar o pior Congresso da
história em poder constituinte (\url{https:/\allowbreak{}/\allowbreak{}goo.gl/\allowbreak{}y\versal{MNM}g0)}. ~O tal
parágrafo diz que a Assembleia nacional Constituinte deliberará
``preferencialmente'' sobre reforma política. Não é exclusivamente: é
preferencialmente. E~veda apenas as propostas abolindo ~I- o Estado
democrático de Direito \versal{II}- a separação dos Poderes; \versal{III}- o voto direto,
secreto, universal e periódico; \versal{IV}- a forma federativa de Estado V- os
direitos e garantias individuais e \versal{VI}- o pluralismo político.

Ou seja, não avançando até esse limite, tudo pode.

Tendo a demolição do Estado de bem"-estar como peça central, entram"-se
nas variáveis, a principal das quais é o destino do governo Michel
Temer.

\section{Capítulo 5 ~-- as peças do jogo principal}

Como o processo atual é essencialmente dinâmico, vamos, primeiro, a uma
avaliação do estágio atual dos principais personagens, para entender o
papel de cada um no jogo:

\textbf{Núcleo duro}

O núcleo duro continua integrado pela Globo e o mercado. O~Ministério
Público Federal, através do \versal{PGR}, e o \versal{PSDB} tornaram"-se meros caudatários.
O~endosso da opinião pública à Lava Jato é pontual e tem prazo de
validade, como em todo processo catártico.

A Globo implementou um discurso único para seus comentaristas, de ataque
continuado ao sistema político e de exaltação permanente à Lava Jato.
Joga com Michel Temer, mas aguarda sua queda. Manobra com a Lava Jato
para queimar Renan Calheiros e manter Lula sob fogo cerrado, enquanto
prepara o próximo tempo do jogo.

\textbf{A \versal{PGR} e o Judiciário}

A cada movimento vai ficando claro que o \versal{PGR} não possui dimensão
política para atuar como organizador do golpe. Teve papel decisivo no
golpe, sim, conforme antecipado pelo \versal{GGN}. Mas é um mero seguidor de
script desenvolvido por instâncias profissionais.

Com a narrativa midiática"-ideológica vitoriosa, construindo a figura do
inimigo interno (\versal{PT}"-Lula) e apresentando a fórmula salvadora (o desmonte
do Estado), cria"-se o movimento de manada e os fatos acabam se sucedendo
por inércia: Ministros do \versal{STF} intimidados ou deslumbrados, procuradores
e juízes de inquéritos paralelos procurando mostrar serviço e a escalada
rápida do estado de exceção, com juízes arremetendo contra qualquer
coisa que cheire esquerda. Vide tentativa de cassar os direitos
políticos do senador Lindbergh, imediatamente após ele ter enfrentado o
juiz Sérgio Moro, porque havia o logotipo da prefeitura em um saco de
leite distribuído.

\textbf{O \versal{PSDB}}

Não tem mais protagonismo algum. Assumiu a condição de terceirizado do
mercado, e é o que resta do quadro partidário. Por esse papel supletivo,
é poupado: do lado da mídia, registrando as menções dos delatores, mas
não insistindo na repercussão; do lado do Supremo e do \versal{PGR}, postergando
os resultados dos inquéritos.

Por exemplo, o Ministro Gilmar Mendes e o Procurador Janot, não se
bicam, mas estão na mesma missão de poupar o \versal{PSDB}. Quando um dos dois
atua em benefício do \versal{PSDB}, o outro se sente aliviado e aproveita para
espicaçar o adversário"-parceiro.

Ao mesmo tempo em que preserva o \versal{PSDB}, Janot investe pesadamente contra
o presidente do Senado Renan Calheiros e estimula perseguição implacável
a Lula, concentrando inquéritos e denúncias divulgados em momentos
estratégicos, como agora, diluindo um pouco as delações da Odebrecht.

\textbf{As forças políticas}

Não se ataca Renan por seu histórico, mas por ser o general maior da
resistência do Senado aos esbirros do \versal{MPF} e do Judiciário. Na verdade, o
bastão político está entregue, hoje em dia, ao comando solitário de
Renan no Senado, e à liderança de Lula no Brasil real. O~apoio de Renan
à \versal{PEC}~ 55 foi estratégico, para manter o Senado no jogo. Mas não infunde
confiança no comando do golpe, que sabe que Renan tem voo próprio.

É por aí que se entende a sequência de ataques aos dois polos.

\textbf{Michel Temer, o breve}

Um a um o golpe vai decepando cada braço político de Temer. Eliseu
Padilha e Moreira Franco estão com os dias contados. Romero Jucá será o
próximo alvo do \versal{MPF}. Aguarda"-se apenas o autor da bala de prata: se as
delações da Odebrecht ou de Eduardo Cunha.

Consumada, restarão dois destinos inglórios para Temer:

\begin{enumerate}
\itemsep1pt\parskip0pt\parsep0pt
\item
  O impeachment através do Tribunal Superior Eleitoral (\versal{TSE}).
\item
  A interdição, com o \versal{PSDB} assumindo o controle do governo e mantendo um
  presidente decorativo.
\end{enumerate}

Tivesse dimensão política mínima, se poderia esperar uma saída honrosa,
tipo comandar uma tentativa de salvação nacional através da convocação
de lideranças de todos os poderes e partidos. O~momento demanda saídas
assim. Mas Temer jamais deixará de ser um mero mercador das pequenas
demandas fisiológicas.

Assim, se entrará na próxima tentativa do golpe sem ninguém que possa
exercer o papel de algodão entre cristais.

\section{Capítulo 6 ~-- a desestabilização econômica}

Para que a proposta da Constituinte seja aceita, e os princípios
defendidos pela Globo"-Miro empurrados goela abaixo do Congresso, há a
necessidade de sucessivos choques que criem um estado de desorientação
tornando os parlamentares receptivos a propostas salvadoras de seus
mandatos.

Mas no meio do caminho tem uma baita crise econômica, trazendo ampla
desorganização econômica, social e política, podendo se desdobrar em
várias alternativas.

Como Temer está enfrentando a borrasca?

A ofensiva neoliberal esbarrou em um problema lógico.

Precisa contar com a desordem geral, a falta de perspectiva para
apresentar as propostas salvadoras, que consistem na redução radical do
papel do Estado. Aliás, precisa aproveitar o estado de desorientação
geral para ser bem sucedida.

No entanto, só se conseguirá sair da crise com uma intervenção decidida
do Estado, através dos investimentos públicos e de um amplo envolvimento
dos bancos públicos comandando o processo geral de renegociação dos
passivos de empresas e pessoas físicas.

O jogo consiste em ir até às últimas instâncias com o desmonte, através
da implementação de medidas irreversíveis. Mas como administrar a crise,
sem criar um caos de tal ordem que abra espaço para aventureiros?

O pacote contra a crise anunciado por Michel Temer é de um ridículo, que
nem a pós"-verdade da mídia conseguiu disfarçar. Fizeram a xepa, abrindo
as gavetas da Fazenda e do Planejamento e retirando de lá propostas
irrelevantes.

Qual o sentido de apresentar como grande saída contra a recessão a
diferenciação de preços para compras à vista e por cartão, ou a redução
em um ponto da multa do \versal{FGTS} (\url{https:/\allowbreak{}/\allowbreak{}goo.gl/\allowbreak{}\versal{PC}koL3)}?

Só a paixão repentina do Estadão por Temer para encontrar alguma
eficácia nesse pacote e tratar essa bobagem como ``uma boa surpresa''
(\url{https:/\allowbreak{}/\allowbreak{}goo.gl/\allowbreak{}9eIklw)}, algo que nem a Globo ousou endossar.

\section{Considerações parciais}

Esses são os dados do momento.

Por outro lado, o agravamento da crise produzirá um sentimento cada vez
maior de urgência para a busca de saídas. Dependendo das circunstâncias,
poderá levar ao entendimento ou ao endurecimento final do regime.

O grande problema desses personagens fakes do Brasilguai é a
incapacidade total de entender o dinamismo dos processos sociais,
políticos e econômicos. Assim como a parte mais superficial do mercado
financeiro, esses personagens projetam o futuro a partir dos dados do
presente, sem nenhuma sofisticação analítica para perceber a dinâmica
dos processos.

Vamos a alguns exercícios simples de futurologia

1.~Ministério Público e Judiciário

Daqui a alguns anos o mundo político estará pacificado. Seja qual for o
partido no poder, ou mesmo uma ditadura, a reação será o embate com o
\versal{MPF} e o Judiciário, pela simples razão de que nenhum sistema político
conseguirá funcionar minimamente com a margem de arbítrio conquistada
pelo \versal{MPF}, Tribunal de Contas e outras instituições do funcionalismo
público. Em caso de ambiente democrático, não haverá o embate direto,
mas o estrangulamento orçamentário gradativo e as restrições aos
salários e vencimentos da categoria. Será o fim da era dos concurseiros.
Em caso de ditadura, basta bater a bota para colocar os bravos para
correr.

2.~Ascensão social

O que ocorrerá com essa multidão que ascendeu socialmente, que
conquistou comida na mesa e filhos nas universidades? Aceitará
passivamente o recuo para as profundezas da miséria? É evidente que não,
o que acarretará distúrbios populares cada vez mais intensos.

3.~~~~ A Lava Jato

Na Itália, quando a opinião pública se deu conta do estrago produzido na
economia pelas Mãos Limpas, a operação morreu. A~destruição da economia
formal abriu espaço para um crescimento sem paralelo da máfia, que hoje
domina perto de 30\% da economia italiana. O~que ocorrerá com o
aprofundamento da crise, quando cair a ficha que não foi a corrupção,
mas o álibi da corrupção fornecendo os instrumentos para uma destruição
consciente e antinacional da economia pelo \versal{MPF}?

4.~~~~ Os movimentos sociais

Na era Lula"-Dilma os movimentos sociais foram institucionalizados.
Passaram a acreditar no jogo democrático, nas ações visando legalizar
terras devolutas e a pressionar por políticas públicas. O~que ocorrerá
com eles se passarem a ser reprimidos pela Polícia Militar e pelas
Forças Armadas?

5.~~~~ As Forças Armadas

No momento, comportam"-se profissionalmente. O~modelo de confronto, no
entanto, demandará cada vez mais sua intervenção. Como reagirão os
militares se novamente colocados como guardiões do regime, em um jogo em
que os principais atores civis e públicos estão envolvidos em suspeitas
de corrupção ou pensando exclusivamente em seus interesses
corporativistas? Continuarão aceitando as ordens passivamente?

Enfim, são perguntas óbvias e grandes desafios analíticos para
personagens públicos absolutamente medíocres, a pior geração de
burocratas em um país em que a burocracia pública, em outros tempos, se
constituía no principal instrumento de modernização. Apenas do papel
didático da crise se poderá esperar alguma surpresa. E~temo não ser
surpresa boa.
