\chapterspecial{06/\allowbreak{}09/\allowbreak{}2016 O Xadrez do golpe em marcha e a teoria do avestruz}{}{}
 

O processo de fechamento político obedece a uma lógica conhecida:

Etapa 1~-- o golpe inicial nas instituições, com a destituição do
presidente eleito.

Etapa 2~-- a perseguição implacável aos derrotados.

Etapa 3~--reação dos atingidos, na forma de protestos.

Etapa 4~-- superdimensionamento e criminalização dos protestos, para
induzir a mais repressão.

Etapa 5~-- o golpe final, com a suspensão formal das garantias
individuais.

\section{Peça 1 -- é golpe em preparação, sim}

O editorial da Folha, conclamando ao endurecimento contra os
manifestantes marcou a entrada na~Etapa 4. Some"-se a ela a coluna do
Secretário de Redação (\url{https:/\allowbreak{}/\allowbreak{}is.gd/\allowbreak{}x5gmKd)}~retomando todos os
bordões das guerras ideológicas dos anos 50: uma ``elite vermelha'' com
um ``comitê central'' mirando os alvos -- empresários, imprensa,
parlamentares, procuradores e juízes -- para planejar seus atentados. Os
atentados, em questão, consistem em chama"-los de ``golpistas'', ``dia e
noite''. Depois, nas ruas, as ``tropas de assalto'' entendendo o recado
e partindo para a ação.~``Nas derivações subletradas do marxismo de
hoje, o culto da revolução ---o banho de sangue que abriria caminho para
o mundo pacificado--- deu lugar ao prazer estético da depredação e do
confronto provocado com a polícia''.

Não há mau jornalismo, paranoia ou estratégia editorial que explique
esses artigos. Trata"-se de uma ação deliberada visando utilizar as
manifestações contra o impeachment como álibi para a suspensão dos
direitos civis.

Em São Paulo, o indiciamento de 16 adolescentes por formação de
quadrilha, colocando como indício celulares, gazes, algodão, vinagre e
um chaveiro do Pateta; no Rio, a \versal{PM} invadindo a sede do \versal{PC}doB, a
pretexto de estar investigando suspeitas de atentados nas Olimpíadas,
tudo isso configura um quadro claro de endurecimento político e de
interrupção das garantias individuais.

Essa estratégia está ligada ao rápido esvaziamento do governo Michel
Temer e à perda de perspectiva em relação a 2018. Especialmente à enorme
dificuldade encontrada pela Lava Jato para liquidar com Lula.

A campanha persecutória contra Lula entra na fase delicada, colocando em
risco a imagem do Brasil no mundo. É~este o dilema.~~~~~~~~~~~

\section{Peça 2 --- a encruzilhada da Lava Jato}

A ideia da Lava Jato era a de que Lula chefiava uma organização
criminosa e se locupletava disso. Julgava que bastaria uma acusação, a
quebra dos sigilos fiscais e bancário, dele e da família, uma prensa em
alguns delatores para entregar Lula de bandeja à opinião pública.

Ao longo do ano, o procurador Carlos Fernando dos Santos Lima, porta"-voz
da Lava Jato, deu várias entrevistas, prometendo entregar o serviço da
condenação de Lula.

Feita a devassa, não foi identificado nenhum elemento que comprovasse
corrupção. Começa aí uma sucessão de operações contra Lula, uma
perseguição implacável, meio sem nexo, que em breve submeterá o Brasil
ao julgamento das cortes internacionais de direitos humanos.

\subsection{Peça 3 -- a última bala contra Lula}

A última tentativa foi na semana passada, em um relatório da Polícia
federal com erros grosseiros e uma base factual fictícia que apresentou
diversas evidências da perseguição imposta a Lula.

Evidência 1~-- a caracterização do crime.

O relatório imputa um crime -- corrupção passiva -- que só se aplica a
funcionário público. Colocaram os supostos delitos na linha do tempo em
2014. Desde 1\textsuperscript{o}~de janeiro de 2011 Lula não é
funcionário público. Dona Marisa nunca foi. E~não se incluiu nenhum
funcionário público na lista dos indiciados.

Evidência 2~-- os inquéritos ocultos

O que mais surpreendeu foi o fato da denúncia ter ocorrida no âmbito de
um inquérito que tramitou de forma oculta na \versal{PF}.~

Há um inquérito público que apurava os verdadeiros proprietários dos
apartamentos no edifício Solaris. Foi relatado sem imputar crime algum a
Lula. O~inquérito oculto que só foi descoberto porque o Ministério
Público Federal (\versal{MPF}), talvez por engano, peticionou no inquérito
público indicando o número do inquérito oculto. Os advogados de Lula
fizeram pedido de acesso ao juiz Sérgio Moro. Que respondeu que só
poderia dar acesso com concordância do \versal{MPF}. Nesse ínterim, soube"-se da
existência de um terceiro procedimento, também oculto.

No dia 19 de agosto, os advogados ajuizaram a reclamação no \versal{STF} (Supremo
Tribunal Federal) No dia 24 de agosto Moro deu acesso ao inquérito. Dois
dias antes, sem permitir nenhuma possibilidade de esclarecimento, a \versal{PF}
anunciou o indiciamento de Lula e Marisa. Nada foi instaurado para
apurar os fatos relatados. A~rigor, ninguém apurou nada. Indiciamento em
si não tem o menor valor legal. Serve apenas para estigmatizar pessoas e
garantir palanque para delegados.

Lula e Marisa se tornaram alvo da cobiça de todas as partes, inclusive
da Associação dos Peritos da \versal{PF} que acusou o delegado de divulgar o
inquérito sem dar o devido crédito aos peritos.

O grande feito do delegado -- surrupiando o mérito da Associação dos
Peritos -- foi a descoberta de uma rasura em um documento privado. Quem
fez, por que fez, não se sabe e nem se foi atrás para saber. Mas graças
à rasura o delegado pode atribuir a Lula o crime de ``falsidade
ideológica''.

Enfim, uma cena de vaudeville em uma das dez maiores economias do
planeta.

\section{Peça 4 -- os abusos identificados pelo Supremo}

O Supremo reconheceu pelo menos duas ilegalidades graves na Lava Jato:

\begin{enumerate}
\itemsep1pt\parskip0pt\parsep0pt
\item
  A ilegalidade do grampo entre Dilma e Lula.
\item
  Ilegalidade na conduta de Serio Moro, de dar publicidade às
  interceptações telefônicas.
\end{enumerate}

Se o Supremo reconheceu que Moro agiu de forma ilegal, e afirmou que tal
conduta poderia configurar crime, de acordo com jurisprudência pacífica
caberia ao \versal{PGR} tomar providências. Afinal, confirmou"-se que o
monitoramento atingiu 35 advogados de defesa, atingiu a privacidade de
um ex"-presidente da República e teve papel relevante na votação do
impeachment.

Advogados estrangeiros consultados não conseguiram identificar episódio
semelhante em qualquer outro país civilizado. O~que de mais remoto se
levantou foi o juiz espanhol Baltazar Garzon que monitorou a conversa de
um réu preso com seu advogado. Sequer teve a ousadia de divulgar o
áudio. Mas foi punido.

No Brasil, o monitoramento de 35 advogados não resultou em nada, nenhuma
consequência, nem administrativa nem penal. Havia claro desvio funcional
com a lei definindo a conduta como criminosa. Diversas representações ao
\versal{CNJ} (Conselho Nacional de Justiça) foram arquivadas. Em junho foram
feitas representações ao \versal{MPF} para apurar os crimes de abuso de
autoridade e crime previsto no artigo 10 da Lei das Interceptações. Até
agora não houve nenhum desdobramento relevante. Foi feita uma
representação por abuso de autoridade dirigida ao \versal{PGR} Rodrigo Janot. A~medida que tomou foi reencaminhar para o \versal{MPF} do Paraná.

Todas as medidas nem foram no sentido de punir os abusos, mas de
paralisar os abusos contra direitos fundamentais de Lula. Em vão.

\section{Peça 5 -- a denúncia à \versal{ONU}}

Com o Estado se recusando a fazer a apuração, sem nada mais a fazer no
Brasil, a defesa de Lula decidiu levar o caso ao Comitê de Direitos
Humanos da \versal{ONU}. Esse recurso está previsto naqueles casos com ausência
de medidas eficazes para paralisar violações.

Agosto foi mês de férias. Em setembro as demandas passaram a ser
analisadas. A~primeira etapa é o juízo de admissibilidade da comunidade.
Aceito, faz"-se a instrução do caso e leva"-se a julgamento.

Se condenado, a \versal{ONU} monitora o país para verificar o cumprimento dessa
obrigação. Em 2005, ditou novas regras estreitando a análise do
monitoramento, com relatórios a serem encaminhados para a Assembleia
Geral afim de dar ciência sobre o cumprimento ou não do que for
acordado.

Nos tempos em que se apresentava como defensor dos direitos humanos, o
\versal{PGR} Janot deu parecer no sentido de que o Brasil tem obrigação de
cumprir todas as decisões proferidas por órgãos internacionais em
relação aos quais o país aceitou a jurisdição.

Se a \versal{ONU} identifica violação, a condenação envolve tanto a parte de
reparação aos danos acusados --- tanto moral como específica --- como a
punição individual aos culpados, e não apenas ao juiz que cometeu
violações. ~No Tratado da \versal{ONU}, aliás, há um capítulo específico sobre o
\versal{MPF}, indicando como procuradores e promotores devem atuar na persecução
penal, dando parâmetros de conduta.

A base da denúncia é a parceria procuradores"-Judiciário e a pressão da
mídia sobre o Judiciário.

Foi denunciado que a Lava Jato atropelou um princípio sagrado de
direito, que é a separação entre quem denuncia, quem investiga e quem
julga. Há entrevistas do procurador Deltan Dallagnol dizendo que eles e
Moro formam ~um time só.

Outra tese que poderá ser levantada pelos advogados de Lula será a da
``teoria do avestruz''.

Tenta"-se imputar a Lula a chamada ``teoria do domínio do fato'' --
segundo a qual seria impossível ao presidente da República não saber as
falcatruas cometidas na Petrobras. Nos Estados Unidos, um juiz isentou a
Price Watherhouse de responsabilidade nas falcatruas da Petrobras,
entendendo que ela não teria como saber.

Levado ao pé da letra, é possível que sobre para o Ministério Público.

A fiscalização da Petrobras passava pela auditoria interna, pelo
conselho fiscal, pela auditoria externa, pela Presidência da Petrobras,
pelo Ministério das Minas e Energia, Controlador Geral da União e
Tribunal de Contas da União.

Todos os órgãos têm em comum a presença de um procurador do Ministério
Público. Como alegar, então, que o \versal{MPF} não sabia das falcatruas? O
inquérito inicial é de 2006 e diz que desde então o doleiro Alberto
Yousseff era monitorado. Como nada se descobriu durante anos?

Esse conjunto de circunstâncias configuraria a chamada ``teoria da
avestruz'', da cegueira deliberada.
