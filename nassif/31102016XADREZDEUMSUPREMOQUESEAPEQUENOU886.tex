\chapterspecial{31/\allowbreak{}10/\allowbreak{}2016 Xadrez de um Supremo que se apequenou}{}{}
 

\section{Cena 1 --- de como Gilmar tornou"-se o condestável da República}

Gilmar Mendes tornou"-se o mais influente dos brasileiros, e faz questão
de exercer o poder em sua plenitude. Literalmente, Gilmar manda na
República.

Preside o \versal{TSE} (Tribunal Superior Eleitoral) e tem maioria de cinco
votos. Com ele votam Henrique Neves, Napoleão Nunes Maia Filho, Luiz Fux
e Antônio Herman de Vasconcellos Benjamin, um antipetista radical.
Napoleão é professor no \versal{IDP} (Instituto Brasiliense de Direito Público),
com salário estimado em R\$ 40 mil mensais.

Nessa condição, tem nas mãos o destino de Michel Temer, presidente.

No \versal{TSE}, também cabe a ela pautar as questões.

No domingo, andou dando entrevistas sugerindo que existiriam precedentes
para separar o julgamento da chapa Dilma"-Temer no \versal{TSE}.

Não existe precedente algum. A~Constituição diz claramente que os votos
da chapa embargada serão anulados. Poderá haver divergência na aplicação
das penas políticas, como inabilitação para cargos públicos. Mas jamais
para manter um dos dois no cargo.

No entanto, como presidente do \versal{TSE}, bastará postergar a votação --- como
vem fazendo, aliás --- para dar sobrevida a Temer.

Além disso, é presidente da 2a Turma do \versal{STF} (Supremo Tribunal Federal),
por onde transita a Lava Jato. A~turma é composta por cinco juízes. Dias
Tofolli é fechado com Gilmar. Celso de Mello é o chamado ``vagalume'',
de comparecimento instável --- provavelmente por problemas de saúde -- e
bastante influenciável por Gilmar em questões políticas, não
necessariamente doutrinárias. Como presidente da Turma, cabe a Gilmar
pautar os julgamentos em cada sessão.~ Para absolver qualquer réu, basta
programar para um dia de ausência de Celso de Mello. Com o voto de
Toffoli garantido, em qualquer caso haveria empate. E, em caso de
empate, pro reu.

Com esse poder, tem em suas mãos o presidente do Senado Renan Calheiros.
Há no Senado um processo de impeachment contra Gilmar. Se o presidente
do Senado ousar encaminhá"-lo, no dia seguinte dança na Lava Jato. Não
apenas Renan, mas qualquer parlamentar com processos nas costas.

Gilmar tem manifestado seu poder de forma ostensivo.

O Ministro da Justiça Alexandre Moraes só não caiu por sua interferência
direta.

O \versal{IDP} se tornou o think tank conservador do Judiciário. Agora, Gilmar
quer entrar na máquina administrativa. Recentemente relançou um livro de
Hely Lopes Meireles --- uma das referências do direito administrativo
--- com prefácio de Michel Temer.

Sua falta de limites tem incomodado muitos setores. Além disso, tem
inúmeras vulnerabilidades, como os patrocínios do \versal{IDP}, de empresas e
associações com interesses diretos no \versal{STF}.

A qualquer momento, poderá levar uma denúncia pelas costas. E~não virá
pela esquerda.

 

\section{Cena 2 --- como as quizilas entre Janot e Teori deram sobrevida
a Eduardo Cunha}

A partir de determinado momento, na Lava Jato, ocorreu um estranhamento
entre o Ministro Teori Zavascki, relator do processo, e o Procurador
Geral da República Rodrigo Janot.

Teori é um juiz rigoroso, técnico, discreto, que se escandaliza até com
o estrelismo de colegas. Começaram a incomodá"-lo os constantes
vazamentos de documentos que eram restritos à \versal{PGR} e ao \versal{STF}.

A estratégia de vazamentos se baseia na existência de dois ou mais
pontos de conhecimento dos documentos vazados, porque não permite a
identificação cabal do vazador. No caso das matérias afeitas ao Supremo,
só havia dois pontos: o \versal{PGR} ou o Supremo. Teori sabia que não partiam
dele os vazamentos e fez chegar a Janot seu descontentamento.

Janot devolveu a estocada no caso Eduardo Cunha. Poderia ter encaminhado
o pedido de afastamento em novembro de 2015, quando chegaram várias
denúncias contra ele. Teori precisaria de um tempo para consultar os
colegas. Não poderia dar uma decisão monocrática e, depois, ser
derrubado pelo pleno do Supremo. Seria desmoralização dele e consagração
de Cunha. Se a denúncia fosse apresentada em novembro, teria tempo para
sondar os colegas.

No entanto, Janot pediu o afastamento de Cunha no primeiro dia do
recesso do Supremo. A única razão para tal era a de colocar Teori contra
a parede. Teori só afastaria monocraticamente se tivesse certeza de que
o pleno endossaria. Mas, na sua avaliação, a denúncia ainda não tinha
consistência.

A partir de fevereiro e março, com Cunha articulando abertamente o
impeachment, foram aparecendo mais provas e foi sendo criado consenso
dentro do \versal{STF}. Mas a questão política falou mais alto. Muitos Ministros
temeram que, tirando Cunha, poderia parecer que o Supremo estaria
tomando partido no impeachment.

Só após o impeachment, o clima ficou a favor da saída de Cunha.

\section{Cena 3 -- o vazamento da delação de Delcídio}

O clima entre Teori e Janot explodiu no vazamento da delação do
ex"-senador Delcídio do Amaral, em março de 2016. Uma cópia da delação
estava no cofre do \versal{PGR}; a outra, no processo no Supremo.

O vazamento ocorreu na revista IstoÉ, pela mesma repórter das relações
pessoais do então Ministro José Eduardo Cardozo -- que, até então, era
companheiro inseparável de Janot. Para disfarçar a origem, a repórter --
que era de Brasília -- dava como origem da notícia Curitiba.

Quando ficou claro que não tinha vazado por lá, houve um movimento da
imprensa para descobrir quem vazara a delação.

Janot teria insinuado que o vazamento teria sido do gabinete de Teori.
Apresentou como prova a marca d'água do Supremo no \versal{PDF} baixado pelo
repórter. Rapidamente foi desmascarado. Se o Procurador faz o upload
para o banco de dados do Supremo e, em seguida, faz o download, o
documento já virá com a marca d'água do Supremo.

A prisão de Delcídio foi a primeira das atitudes inconstitucionais da
Corte. Prisão só em caso de flagrante delito ou sentença transitado em
julgado.

O Supremo acabou aprovando devido aos trechos da gravação nos quais
Delcídio detonava cada um dos Ministros. Teori chamou"-os com urgência
para ouvir as acusações. A~prisão de Delcídio foi uma reação de egos
feridos e abriu espaço, pela primeira vez, para que o guardião da
Constituição passasse a desconstruir a própria Constituição.

Foi apenas aí que caiu a ficha de Dilma sobre o papel de Janot.

\section{Cena 4 --- a tentativa de prisão de Jucá, Renan e Sarney}

Teori quebrou definitivamente as pernas de Janot no episódio do pedido
de prisão de Romero Jucá, Renan Calheiros e José Sarney como decorrência
das gravações de Sérgio Machado. E~não foi por vingança, mas por análise
técnica.

Há diversas gradações no planejamento do crime.

O primeiro passo é a chamada cogitação de crime. A~pessoa passa em
frente uma joalheria e cogita de assaltá"-la. Até aí, não é crime.

O segundo passo são os chamados atos preparatórios. Se forem típicos,
passa a ser crime.

O terceiro ato é o assalto propriamente dito.

No máximo, as conversas se enquadravam na cogitação de crime.

Mais ainda, um senador da República tem imunidade da palavra, que não se
resume ao que é dito na tribuna. E~seria legítimo se preocupar com uma
operação que estava destruindo a economia do país.

Todos esses argumentos conspiravam contra o pedido de prisão. Sabe"-se lá
a razão de Janot ter feito essa aposta. Na ocasião, inclusive, aqui no
\versal{GGN} julgávamos que ele teria outros trunfos na manga, tamanha era a
desproporção entre as conversas gravadas e o pedido de prisão.

A autorização foi negada e, ali, Janot se recolhe definitivamente.

\section{Cena 5 --- as nomeações de Ministros do Supremo}

Para ser nomeado, um candidato a Ministro do Supremo tem que se submeter
a uma maratona humilhante que, em grande parte, explica seu
comportamento pós"-indicação, muitas vezes de profunda arrogância.

Por exemplo, o ex"-Ministro Joaquim Barbosa enlouqueceu, quando percebeu
que Lula iria nomear um negro. Não saia mais das~ de José Dirceu, então
Ministro"-Chefe da Casa Civil, e a de Dias Toffoli, então na \versal{AGU}
(Advocacia Geral da União).

Depois, vendeu a versão de que estava nos Estados Unidos quando,
inesperadamente, recebeu telefonema de Lula perguntando se ele queria
ser Ministro.

Uma cena entre ele e Lula ajudou a espicaçar seu comportamento na \versal{AP}
470.

Um dia Joaquim Barbosa foi a Lula se queixar de que estava sendo
discriminado no Supremo. Lula tratou na gozação.

-- Ô Joaquim, você vem dizer para mim que está sendo discriminado? Vai
lá e lute, como eu lutei.

Já a nomeação de Luiz Fux foi um caso intrincado.

Lula garante que não partiram dele as pressões pela nomeação de Fux. A~um interlocutor disse que Fux chegou recomendado por Delfim Netto e por
João Pedro Stédile, do \versal{MST} (Movimento dos Sem Terra). E~ele, Lula,
jamais teria confiança em quem recebia recomendações de pessoas tão
opostas.

Na verdade, Fux foi uma concessão de Dilma a José Dirceu, que telefonou
pessoalmente pedindo a indicação de Fux como única maneira de livrá"-lo
da prisão. Embora discordasse em quase tudo de Dirceu, Dilma resolveu
ser solidária.

A intenção de Fux de ``matar no peito'' durou até a primeira conversa
com Gilmar Mendes.

\section{Cena 6 -- o estranho voto de Luiz Fachin}

Um dos julgamentos mais estranhos do Supremo foi a votação sobre os
ritos do impeachment.

A Câmara definira um rito quase sumário. O~\versal{PC}doB entrou com uma \versal{ADPF}
(Ação de Descumprimento de Preceito Constitucional), para que o tribunal
obedecesse ao mesmo roteiro do impeachment de Fernando Collor.

O relator era o Ministro Luiz Edson Fachin. Todos apostavam que seu voto
seria à favor da \versal{ADPF}. Surpreendentemente, votou contra. Coube ao
Ministro Luís Roberto Barroso dominar a sessão, com um voto considerado
unanimemente brilhante a favor da \versal{ADPF}. ``O papel do Supremo é o de
preservar as instituições, promover a Justiça e resguardar a segurança
jurídica. O~que liberta o tribunal é que ele seguiu seus próprios
precedentes'' disse Barroso.

Ali o respeitado Fachin sepultou sua reputação.

Nos bastidores, a história que se conta mostra um notável desprendimento
de Fachin.

No último momento, convocou alguns colegas, dizendo"-se vítima de uma
pressão que ele não poderia enfrentar. Nunca disse do que se tratava,
mas ficava claro que estava sendo chantageado.

Para contornar a chantagem, queria se assegurar de que, mesmo votando
contra a \versal{ADPF}, haveria maioria suficiente para aprová"-la. E~encaminhou a
Barroso os estudos que fundamentariam o seu parecer. Sacrificou"-se, mas
cometendo um dos mais dramáticos (e desconhecidos) gestos de
desprendimento de um juiz da Suprema Corte.

Ali, o Supremo teve seu último momento de grandeza.

Pouco depois, o próprio Barroso também foi alvo de chantagens. E~o
Supremo como um todo soçobrou ao clima que tomou conta do país.
