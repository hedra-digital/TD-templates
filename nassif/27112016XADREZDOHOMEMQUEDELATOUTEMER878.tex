\chapterspecial{27/\allowbreak{}11/\allowbreak{}2016 Xadrez do homem que delatou Temer}{}{}
 

\section{Introdução -- características das grandes conspirações}

Conspirações políticas não se montam com o controle completo e acabado
de todas as variáveis, obedecendo a um manual previamente definido.

Quando atua sobre realidades complexas, como o cenário
sócio"-político"-econômico de um país, não há controle sobre todas as
variáveis nem clareza sobre os desdobramentos dos grandes lances.

Jogam"-se os dados, então, em cima das circunstâncias do momento, tendo
apenas uma expectativa sobre seus desdobramentos.

Digo isso, para tentar avançar um pouco no Xadrez de Marcelo Calero, o
ex"-Ministro da Cultura que denunciou Michel Temer de pressioná"-lo em
favor de benefícios pessoais a Geddel Viria Lima.

\section{Peça 1 -- jabuti não sobe em árvore}

Em 2010 Calero foi candidato a deputado federal pelo \versal{PSDB} do Rio. Aluno
de Direito da \versal{UERJ} (Universidade Estadual do Rio de Janeiro) sempre
chamou a atenção pela extravagância, mas jamais pela vocação do suicídio
político. Fazia parte do time de yuppies que ascendeu na gestão Eduardo
Paes.

Reagiu contra as jogadas de Geddel Vieira Lima e, provavelmente, se
assustou quando este passou a jogar balões de ensaio para setoristas
palacianos. Em tempos de grampo e de prisões indiscriminadas, jacaré
nada de costas. E~aí resolveu pedir demissão e denunciar as pressões.

Até aí, se tinha um Ministro neófito resistindo às investidas de boca de
jacaré e gerando uma crise política restrita. Mas o inacreditável
amadorismo político de Michel Temer transformou em início de incêndio,
ao não tomar a decisão óbvia e imediata de demitir Geddel.

Aí ocorre o lance seguinte, a denúncia contra o próprio Temer na Polícia
Federal, com o depoimento vazando para a empresa mal chegou no \versal{STF}
(Supremo Tribunal Federal). Imaginar espontaneidade em lance dessa
amplitude é tão improvável quanto acreditar que jabuti sobe em árvore.

O desafio é saber quem pendurou o jabuti na árvore.

\section{Peça 2 -- o lance do partido da mídia}

A melhor maneira de garimpar os antecedentes é através de um balanço
rápido da repercussão:

\begin{itemize}
\itemsep1pt\parskip0pt\parsep0pt
\item
  O Jornal Nacional investiu contra Michel Temer com a mesma gana com
  que atacava Lula.
\item
  Época, o braço mais manipulável das Organizações Globo, depois da
  Globonews, registrou Calero na capa, o sir Galahad do novo, em
  contraposição ao velho Geddel Vieira Lima.
\end{itemize}

Folha e Estadão vão a reboque. E~todos trataram de poupar Eliseu
Padilha, principal avalista do pacote de apoio à mídia.

No mesmo dia, em que o escândalo Geddel expunha o vácuo Temer, \versal{FHC}
aparecia nos jornais online -- e no Jornal Nacional -- falando do
orgulho de ser \versal{PSDB}, o \versal{PSDB} representando o novo etc. E, como bom
malaco, afiançando, com ar confidente, que a presidência seria de alguma
das lideranças presentes. Mas não dele, é claro.

É difícil uma conspiração discreta com \versal{FHC} porque ele não consegue
conter a euforia nos momentos que antecedem o desfecho.

A delação de Calero serve, portanto, para dois objetivos:

\textbf{Objetivo 1}~-- enfraquecer substancialmente a camarilha de
Temer.

\textbf{Objetivo 2}~-- recolocar \versal{FHC} no centro das articulações, como a
alternativa para a travessia até 2018.

As circunstâncias ditarão os próximos lances, que poderá ser um Temer
sem camarilha, sendo tutelado pelo \versal{FHC}; ou um \versal{FHC} assumindo a
presidência para tocar o barco até 2018, tendo dois trabalhos a
entregar:

\begin{enumerate}
\itemsep1pt\parskip0pt\parsep0pt
\item
  Completar o desmonte da Constituição de 1988, conquistando o limite de
  despesas e a reforma trabalhista e da Previdência.
\item
  Implantar o parlamentarismo, ou outras alternativas de esvaziamento do
  poder Executivo e de poder do voto popular.
\end{enumerate}

\section{Peça 3 -- uma explicação para a capa de Veja}

Há dúvidas de monta sobre a capa de Veja, com a chamada superior
informando sobre as acusações contra José Serra e Geraldo Alckmin nas
delações da Odebrecht.

Três hipóteses foram aventadas:

\begin{enumerate}
\itemsep1pt\parskip0pt\parsep0pt
\item
  \textbf{Veja}~começou a fazer jornalismo.
\item
  Não bate com a insuficiência de dados da reportagem. A~rigor, há uma
  única informação, sobre a maneira como a Odebrecht repassava o
  dinheiro do caixa 2 para Serra através do banqueiro Ronaldo César
  Coelho.
\item
  Dar um chega"-prá"-lá nos três presidenciáveis atuais do \versal{PSDB},.
\item
  Para deixar claro que o momento não é de disputa, mas de coesão em
  torno de \versal{FHC}.
\item
  Arrumar um álibi para os três.
\end{enumerate}

É uma teoria um pouco mais complexa, mas que faz sentido.

Sabia"-se que haveria dois tipos de delação das empreiteiras. A~Odebrecht
se concentraria nos financiamentos de campanha; a \versal{OAS} nos casos de
corrupção para enriquecimento pessoal.

Aì houve a intervenção preciosa do Procurador Geral da República Rodrigo
Janot, cancelando as negociações com a \versal{OAS} e provocando um enorme alívio
nos advogados de Serra.

Com exceção de Geraldo Alckmin, há indícios robustos de que houve
enriquecimento pessoal tanto de Serra quanto Aécio. Focando"-se nos
pecados menores, confere"-se um álibi de isenção à Lava Jato e à mídia e,
ao mesmo tempo, desvia"-se o foco das investigações dos crimes mais
graves.

\begin{center}~\end{center}


\section{Peça 4 -- o enfrentamento da crise e o fator \versal{FHC}}

O quadro que se apresenta, hoje em dia, é ameaçador.

Na base, o agravamento da crise econômica, expandindo"-se por estados e
municípios. Os cortes nas políticas sociais, criando situação de fome
para parcela expressiva dos beneficiários do Bolsa Família. Um
endividamento circular das empresas, travando os negócios. E~os grandes
investimentos públicos paralisados.

Em cima desse quadro, um conjunto de medidas pró"-cíclicas, agravando a
crise econômica.

a.~~~~ O arrocho fiscal, aprofundando a recessão e ampliando o déficit
fiscal pela redução da receita.

b.~~~~ A política monetária com taxa básica a 14\%, inviabilizando
qualquer possibilidade de novos investimentos.

c.~~~~ A política cambial provocando a apreciação do real e abortando a
recuperação das exportações.

d.~~~~ O trancamento do crédito nos bancos comerciais. Não há crédito
mais e trabalha"-se com extrema cautela a rolagem das dívidas das
empresas inadimplentes.

e.~~~~ A retirada de R\$ 100 bilhões do \versal{BNDES}, em um momento em que o
endividamento circular das empresas paralisa a economia.

Em um ponto qualquer do futuro, haverá a necessidade de uma mudança de
180\textsuperscript{o}~na política econômica, com um choque fiscal --
ampliando despesas e investimentos públicos \mbox{---,} flexibilização das
políticas monetária e creditícia.

Um trabalho de recuperação da economia exigiria um conjunto de
qualidades que falta a \versal{FHC}:

1.~~~~ A proatividade para conduzir os diversos instrumentos de
recuperação da economia.

Nos seus 8 anos jamais se envolveu no dia"-a-dia da gestão política e
econômica.

2.~~~~ Habilidade política para recompor a base de apoio.

Em seu tempo de presidência, o varejo da política era garantido
justamente pelo quarteto que compõem a camarilha de Temer: Geddel,
Padilha, Moreira Franco e o próprio Temer. No Congresso, o \versal{PSDB} atual
regurgita ódio e, no campo das ideias, é um mero cavalo das ideias
mercadistas.

3.~~~~ Credibilidade para conduzir um pacto nacional.

Em todo o período de conspiração, \versal{FHC} sempre estimulou a radicalização e
o golpe. Jamais conseguiu entender que o único papel engrandecedor que
lhe caberia seria o de um futuro mediador, no caso de recrudescimento da
crise política. Pensa pequeno.

\section{Peça 5 -- o jogo de forças pós"-Temer}

Leve"-se em conta que a fritura de Temer promoverá um racha na frente
golpista.

A frente é composta pelo \versal{PMDB} de Temer, \versal{PMDB} dos caciques nordestinos,
\versal{PSDB}, centrão, \versal{PGR}- Lava Jato e mídia.

A implosão do governo Temer significará restringir o grupo vitorioso e
enfrentar, no Congresso, a reação do \versal{PMDB} e do chamado centrão, além da
oposição da esquerda.

Os 200 e tantos nomes da delação da Odebrecht não reporão de forma
alguma a isonomia nas investigações da Lava Jato, pelo fato de incluir
políticos tucanos nas delações. Pois a escolha dos investigados caberá
exclusivamente a Janot.

O movimento de fritura de Temer acirrará mais as contradições do golpe,
até que o aprofundamento da crise promova ou a conciliação ou o caos.

E aí será possível um pacto entre \versal{FHC} e Lula.
