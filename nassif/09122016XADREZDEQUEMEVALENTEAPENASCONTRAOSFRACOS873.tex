\chapterspecial{09/\allowbreak{}12/\allowbreak{}2016 Xadrez de quem é valente apenas contra os fracos}{}{}
 

\section{Peça 1 -- a Previdência e o pacto democrático}

Os grupos que se uniram para assaltar o poder romperam o pacto
democrático e, agora, há uma disputa feroz pelo controle do aparelho do
Estado. É~nítido na maneira como se pretender desconstruir o modelo de
Estado que surge da Constituição de 1988.

A Previdência Social, com o Regime Geral e com a Seguridade Social
significaram um avanço civilizatório gigantesco e foram fruto do pacto,
da luta democrática e social que floresce em 1988, no pós"-ditadura, como
lembra o ex"-Ministro Miguel Rossetto.

Como proposta, a reforma da Previdência e o desmonte da Seguridade
Social é a barbárie, a própria negação do direito previdenciário
construído pela democracia brasileira.

Os problemas enfrentados agora pela Previdência não decorrem do
descontrole das despesas, mas da brutal queda da arrecadação
previdenciária, por conta da recessão, desemprego e renúncias
previdenciárias legais, acentuadas no governo Dilma.

O que se pretende não é a correção de distorções, mas o desmonte do
modelo previdenciário.

Ao igualar homens e mulheres, as novas regras desconsideram a condição
feminina, a maternidade ou o fato de, nos lares mais humildes, a mulher
sempre receber o encargo de administrar os problemas familiares, um
conjunto de responsabilidades que impede o cumprimento integral do tempo
de trabalho.

Além disso, na realidade do mercado de trabalho brasileiro, com a
rotatividade periódica do emprego, é muito difícil mesmo para o homem
cumprir integralmente o tempo de trabalho. Sua renda não permite juntar
uma poupança para pagar a Previdência nos tempos de desemprego.

Finalmente, é criminoso tirar o salário mínimo do \versal{BPC} (Benefícios de
Prestação Continuada), muitas delas famílias extremamente pobres, com
pessoas com deficiência. Tenta"-se tirar o único benefício dos pobres
entre pobres, famílias em situação de miséria.

As distorções não param por aí.

Retirar o tempo de trabalho e de contribuição é um enorme erro, por
conta do perfil de trabalho do brasileiro: pobres começam a trabalhar
formalmente aos 16 anos; os mais ricos, aos 25 anos. No campo, começa"-se
a trabalhar muito mais cedo e em muito piores condições de trabalho.

Países que têm como critério a aposentadoria por tempo de trabalho são
aqueles de mercado de trabalho mais homogêneo.

\section{Peça 2 -- os problemas reais da Previdência}

Há um problema geracional em todos os sistemas de previdência, na medida
em que aumenta a expectativa de vida das pessoas e aumenta também a
idade média da população. Cria"-se a situação em que o tempo de
aposentadoria passa a ser maior que o tempo de contribuição.

São problemas de longo prazo que exigem acordos no presente, para
implementação gradativa, sem grandes traumas.

Por ser tema estrutural, as soluções têm que ser negociadas, pactuadas,
aceitas pelos cidadãos. Pela fórmula atual, a pessoa só conseguirá se
aposentar com a aposentadoria integral com 45 anos de contribuição.

No governo Dilma, adotou"-se inicialmente a fórmula 85/\allowbreak{}95 -- a soma do
tempo de contribuição com idade respectivamente para mulheres e homens.
Permitia uma aposentadoria superior ao fator previdenciário, mas que
iria se ajustando ano a ano.

Não resolvia o longo prazo, mas valia como como início de discussão. O~passo seguinte seria convocar as centrais sindicais, o Congresso e os
empresários para discutir as saídas e acertar o grande acordo.

\section{Peça 3 -- a impossibilidade de um acordo}

Pela abrangência, pelos impactos sobre as próximas décadas, propostas
dessa natureza não podem provir de um governo ilegítimo, empurrando
goela abaixo da população, baseando"-se na máquina de moer consciências
das Organizações Globo.

Trata"-se da consolidação de um modelo infame que, em outros tempos,
envergonharia até mentes mais insensíveis, como o Ministro Luís Roberto
Barroso -- que se jacta de falar em nome do futuro, contra o atraso.

A insensibilidade de Barroso, a maneira como atropela princípios básicos
de um regime democrático -- como endossar as medidas autoritárias em
cima do orçamento \mbox{---,} o pensamento autocrático~ são típicos de uma
burguesia da República Velha, pré direitos sociais básicos.

E, se insisto em mencioná"-lo, é por ser o representante máximo do que
mais anacrônico a elite brasileira produziu, o intelectual falsamente
sofisticado, que trata as ideias com a mesma superficialidade com que
busca bens de consumo da moda, sem jamais se introjetar princípios
básicos constituintes de uma sociedade civilizada.

A agenda proposta inviabiliza o direito previdenciário.

É indecente num sistema tributário em que a maior parte dos tributos é
arrecadado de contribuintes de baixa renda, cortar os benefícios dos
mais pobres mantendo as brechas que permitem aos de maior renda
beneficiar"-se do planejamento fiscal e dos juros praticados pelo Banco
Central.

Tenho sido, em geral, contrário às medidas radicais, à intolerância
política, ao radicalismo estéril. Mas endosso a posição de Rosseto:
qualquer tentativa de negociar esse pacote será um endosso à liquidação
do direito previdenciário.

\section{Peça 4 -- o país da bazófia}

A exacerbação em torno do golpe, a falsa frente majoritária que derrubou
o governo criou em muitos de seus protagonistas a falsa sensação de
segurança, típica das torcidas organizadas, que transforma os
pusilânimes em valentes, os fracos em truculentos, os tímidos em
arrogantes.

Passado o instante de catarse, muitos são apanhados em flagrante, com
seus medos e fragilidades, como as cortinas do teatro que se abrem e
flagram o ator urinando no vaso de flores na frente de uma plateia
lotada.

\paragraph{\textbf{Valente 1}~~}

a Ministra Carmen Lúcia declara guerra aos políticos, dá palavras de
ordem, pratica quatro frases de efeito e arregimenta a categoria e o
populacho. Aí, no meio do caminho tinha um presidente do Senado, e ela
recua tão ostensivamente que faz questão de registrar seu voto de
presidente favorável a Renan. Nem seguiu a máxima do recuo: não seja tão
lento que pareça provocação, nem tão rápido que sugira medo.

\paragraph{\textbf{Valente 2}~}

O valente Eliseu Padilha anuncia a reforma da Previdência e retira dela
os militares e a Polícia Militar -- que supunha os únicos poderes
armados. Aí a policia civil e os bombeiros fazem o maior pampeiro,
inclusive enfrentando a \versal{PM} do Rio de Janeiro. Depois, 190 representantes
de sindicatos de policiais, bombeiros e agentes penitenciários invadem o
Ministério da Justiça. O~Ministro Alexandre de Moraes, depois de sua
notável performance carpindo pés de maconha no Paraguai, ~nem ousa
aparecer.

No final do dia, vem o recado ligeiro: policiais e bombeiros estão fora
da reforma.

\paragraph{\textbf{Valente 3}~}

Com a queda de Geddel Vieira Lima, depois de dias e dias ganhando
coragem, Michel Temer nomeia o tucano Antônio Imbassahy como Secretário
de Governo. O~centrão bate o pé e Imbassahy sai de cena, antes mesmo de
entrar.

\paragraph{\textbf{Valente 4}~}

O douto representante de Eduardo Cunha e Eliseu Padilha na \versal{EBC}, Laerte
Rímoli, manda embora a apresentadora Leda Nagle. Os protestos invadem a
rede e merecem home de O Globo Online. Imediatamente, recua, antes mesmo
de pensar em uma saída estratégica.

\paragraph{\textbf{Valentes 5}~}

Tal e qual um Aníbal, o Cartaginês, Geddel e Padilha irromperam nas
hostes adversárias indo direto ao butim, mandando demitir servidores,
criando um crivo ideológico nas verbas da Secom. Com os primeiros
petardos, Geddel é flagrado aos prantos, pelos corredores do Planalto, e
Padilha internado com pressão alta. Nesse ínterim, a postura altiva de
Dilma Rousseff é reconhecida pelo Financial Time, que a premia como uma
das mulheres de 2016 por seu comportamento na tragédia do impeachment.

Duas conclusões reiteradas:

\begin{enumerate}
\itemsep1pt\parskip0pt\parsep0pt
\item
  O governo Temer morreu.
\item
  Não há saída fora do grande pacto em torno das eleições diretas.
\end{enumerate}
