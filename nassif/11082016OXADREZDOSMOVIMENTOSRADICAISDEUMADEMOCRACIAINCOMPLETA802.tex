\chapterspecial{11/\allowbreak{}08/\allowbreak{}2016 O Xadrez dos movimentos radicais de uma democracia incompleta}{}{}
 

O Brasil é o país dos extremos, vítima de movimentos pendulares
radicais.

Determinadas tendências vão se radicalizando pela inércia, sem que sejam
contidas por fatores moderadores. Quando assumem proporções
intoleráveis, são sucedidas por movimentos contrários que, primeiro
corrigem os excessos anteriores para, depois, promoverem sua própria
radicalização. E~não há freios, amortecedores para reduzir a intensidade
desses movimentos.

Se alguém afirmar que o governo Dilma foi dos mais desastrosos da
história, não vou discutir. Mas um sistema institucional robusto teria
que dispor de instrumentos para passar incólume pelo desafio Dilma,
permitir ajustes sem abrir espaço para aventuras golpistas. E~o golpismo
impediu os movimentos corretivos de Dilma.

A crise atual lança luzes sobre um conjunto de vulnerabilidade o ~da
sociedade brasileira, permite identificar as correções a serem feitas,
mas não se vislumbram agentes econômicos, sociais ou políticos para
cumprir a função moderadora.

O subdesenvolvimento é uma construção de gerações, já se dizia.

\section{Agentes moderadores das políticas públicas}

Os movimentos de política econômica costumam ser pendulares. A~oposição
torna"-se governo criticando os exageros da política anterior. Há um
movimento inicial, virtuoso, de correção de rumos, de trazer o pêndulo
para o centro. Na medida em que se tem sucesso, o movimento tende a
radicalizar para o outro extremo. Ou seja, o próprio sucesso do modelo
planta as sementes dos exageros posteriores.

Com a eleição de Dilma Rousseff, após as ações anticíclicas vitoriosas
de 2008, havia a esperança de que o país estaria imunizado contra
movimentos radicais voluntaristas.

O que se viu foi o poder solitário de uma presidente produzindo um
conjunto de medidas voluntaristas não tão drásticas quanto os vizinhos,
mas suficiente para desmontar a economia, expondo o governo a uma
oposição destrutiva.

Como conseguir o equilíbrio? A imprensa não tem capacidade ou maturidade
para exercer esse papel moderador. Há décadas é presa ao refrão único
dos juros altos, livre fluxo de capitais, Estado mínimo, alergia a
qualquer forma de aprofundamento da democracia. É~uma imprensa do nível
da venezuelana.

Um Conselho Superior de Economia não só coibiria os exageros, como
qualquer mudança de rota. Portanto, não seria aconselhável.

O grande problema do presidencialismo brasileiro não é apenas a
dispersão de partidos. É~também o poder absoluto do presidente. Quase
tão absurdo quanto o golpe foi a atuação individual da presidente,
inibindo a atuação de conselhos populares, de fóruns empresariais, não
concedendo audiências a representantes de outros poderes e sequer se
alinhando com seu próprio partido.

O ideal seria partidos políticos programáticos, com ideias claras sobre
a economia e, principalmente, instrumentos para conter ímpetos
voluntaristas dos seus candidatos eleitos.

O mínimo que se espera é que os atos do presidente sejam analisados,
avalizados ou não, pelo seu partido ou base de apoio. Hoje em dia, nem
partidos há.

A grande dicotomia a ser vencida é, de um lado, criar ferramentas que
subordinem o presidente ao programa do partido e canais de participação
técnica e popular. De outro, não inibir seu protagonismo.

\section{O papel desestabilizador das corporações públicas.}

O grupo que se apossou do poder --- Michel Temer, Eliseu Padilha, Romero
Jucá, Moreira Franco, Geddel Vieira Lima e Eduardo Cunha --- deve sua
vitória ao Procurador Geral da República Rodrigo Janot, ao Ministério
Público Federal em geral e ao Tribunal de Contas da União. Eles foram os
agentes finais, que ajudaram a desequilibrar o jogo, que colocaram a
caneta mais poderosa da República nas mãos de Temer e Padilha, com um
protagonismo político inaceitável em qualquer país civilizado.

Pior, o corporativismo impediu o \versal{CNMP} (Conselho Nacional do Ministério
Público) de exercer o papel moderador. A~procuradoria de coalizão ---
fruto da escolha do procurador mais votado para a \versal{PGR} --- faz com que os
candidatos cada vez mais se afastem dos valores constitucionais do
Ministério Público e se aproximem da ansiedade por poder da massa da
corporação.

Por outro lado, sem o mecanismo da eleição direta para a lista tríplice,
corre"-se o risco de se voltar ao tempo do Ministério Público
engavetador.

De alguma forma, se terá que encontrar o meio"-termo, ou através da
formação de um Conselho de Notáveis, com as figuras referenciais do
próprio Ministério Público que, mesmo não tendo poder de veto, possa
exercer moralmente um papel moderador.

É inacreditável que um poder, a \versal{PGR}, que se vangloriava de contar com
altos conselhos técnicos para qualquer tema, não tenha conseguido montar
uma identidade simples:

Poder Executivo --- Dilma Rousseff = Michel Temer + Eliseu Padilha +
Geddel Vieira Lima + Moreira Franco + Eduardo Cunha = --- Poder do \versal{MPF}

\section{A leniência com a ilegalidade}

O jogo anterior à Lava Jato estimulava o malfeito. Apelações infinitas,
uso indiscriminado do fruto podre para anular inquéritos, sentenças
jamais cumpridas.

Aí o movimento pendular se inverte.

Os vazamentos de inquéritos sigilosos, com propósitos políticos,
tornam"-se uma constante. O~uso de inquéritos policiais para represálias
políticas, um novo normal. O~uso abusivo de poder de Estado de qualquer
procurador iniciante, representando contra grupos políticos, solicitando
prisões midiáticas, vazando informações para a imprensa passam a ser
aceitos como normal. A~incapacidade do \versal{STF} de confrontar os abusos,
infelizmente, tornou"-se uma constante.

Ao tolerar vazamentos, o \versal{PGR} Rodrigo Janot ajudou a criar um poder
paralelo incontrolável, na parceria política mídia"-procuradores. O~que
era uma prática coibida, considerada abusiva, torna"-se o novo normal,
inclusive na \versal{PGR}.

Ao aceitar as gravações contra Delcídio do Amaral, o Ministro Teori
Zavascki convalidou o grampo ilegal. E~a falta de providências contra os
vazamentos de escutas ilegais, no episódio dos diálogos da presidente,
comprovou a subversão no sistema de hierarquia do Judiciário.

Os ataques montados pela parceria mídia"-procuradores contra o Ministro
Marcelo Navarro Ribeiro do \versal{STJ} (Superior Tribunal de Justiça), são de
natureza pior do que os ataques apócrifos perpetrados contra a esposa do
Ministro Luís Roberto Barroso, ou com a possível intimidação do Ministro
Luiz Facchin, que praticamente imobilizaram o Supremo.

A história do ``não é comigo'' não cola. Esse quadro é de
responsabilidade direta de Rodrigo Janot, Teori Zavascki, Ricardo
Lewandowski, Luís Roberto Barroso, que permitiram que o \versal{STF} passasse a
ter a cara de Gilmar Mendes e a se deixar conduzir pela Lava Jato. E~menciono apenas aqueles dos quais se esperava algo.

Tem"-se agora uma re"-centralização política similar ao período da
ditadura. Os estamentos brasilienses -- Congresso, corporações públicas,
\versal{MPF}, Judiciário -- avançando sobre o orçamento público, ao preço de
arrebentar com a estrutura de despesas federais, saúde, educação,
segurança, Previdência e outras funções de Estado.

\section{O papel desestabilizador da Globo}

Desde as campanhas de 2006 e 2010, ampliada pela campanha do mensalão,
observou"-se o papel deletério do cartelização da mídia. A~cartelização
produziu dois fenômenos opostos, mas correlatos. De um lado, a plena
liberdade dos grupos oficiais de mídia para assassinar reputações,
adulterar notícias, jogar vergonhosamente com a autoestima nacional, em
episódios inaceitáveis para qualquer sociedade minimamente civilizada.
De outro, uma ação pertinaz de esmagamento do discurso contrário,
através de ações judiciais contra blogs e sites independentes..

A tendência dominante é o de enfraquecimento gradativo da mídia e
aumento da atoarda representada pelas redes sociais. Mas o papel das
Organizações Globo tornou"-se uma questão de Estado. Sua influência sobre
a opinião pública, o Judiciário e o Ministério Público criou um
território indevassável, que conseguiu bloquear até a cooperação
internacional do \versal{MPF} com o \versal{FBI}, nas investigações do caso \versal{FIFA}, ou das
contas no escritório Mossak Fonseca.

A regulação econômica da mídia e o uso correto das concessões públicas
tornaram"-se uma questão de sobrevivência da democracia brasileira.~

\versal{PS} --- Agora à noite, ao solicitar que o \versal{TSE} (Tribunal Superior
Eleitoral) também investigue as contas de Aécio Neves, ``por uma questão
de isonomia'', a Ministra Maria Thereza de Assis Moura comprova a
superioridade de gênero: mulher, fez o que Ministro nenhum tem ousado
fazer, com receio da agressividade inaudita e do uso da mídia por Gilmar
Mendes, deixando o Judiciário refém da falta de limites.~
