\chapterspecial{25/\allowbreak{}06/\allowbreak{}2016 Impeachment é a aposta do Brasil improdutivo}{}{}
 

Está na hora de dar um intervalo no golpismo, um arrefecimento nessa
disputa ideológica anacrônica e se começar a pensar seriamente no
próximo tempo do jogo.

A insistência no impeachment, por parte de Gilmar Mendes e de setores do
\versal{PSDB} já ultrapassou os limites de qualquer razoabilidade. A~intervenção
do \versal{STF} (Supremo Tribunal Federal) e as manifestações gerais de
condenação ao impeachment, o racha no \versal{PMDB}, o desmonte da imagem de
Michel Temer, comprovam que o impeachment é o caminho mais traumático
para o país.

A insistência na tese parte de um tipo específico de pessoa: 1) a que
vai obter ganhos pessoais e políticos com o grupo que ascender e que 2)
faz parte do país improdutivo, não afetado por crises econômicas.

Integra esse grupo de privilegiados"-a-salvo"-de"-crises inclusive o
presidente da \versal{FIESP} (Federação das Indústrias do Estado de São Paulo)
Paulo Skaf cuja fonte de receita são aluguéis e a pilotagem da \versal{FIESP}.
Verdadeiros industriais, como os representados pela \versal{ABIMAQ} (Associação
Brasileira de Máquinas e Equipamentos) e \versal{IEDI} (Instituto de Estudos de
Desenvolvimento da Indústria) defendem a estabilização rápida do jogo
político para permitir à economia respirar.

\asterisc{}

A disputa para enquadrar o novo Ministro da Fazenda Nelson Barbosa na
esquerda ou na nova matriz econômica (como se os erros do período Guido
Mantega fossem fruto de qualquer matriz) é ridícula.

Com Joaquim Levy ou Nelson Barbosa, o trabalho é o mesmo, de reduzir na
medida do possível os gastos de custeio, aprovar a \versal{CPMF} e dar sequencias
às propostas de reforma fiscal. Não é 0,2 ponto percentual a mais ou a
menos no superávit primário que define a ideologia de um ou outro.

\asterisc{}

Nesses momentos de mudanças, o mercado age sempre com comportamento de
manada. Não significa que a sabedoria esteja com a tendência
majoritária. Pelo contrário, os verdadeiros campeões são os que sabem
jogar no contra fluxo.

Quem sabe das coisas conhece o pensamento de Nelson Barbosa. Sabe que
ele foi o principal comandante das medidas anticíclicas de 2008 -- que
impediram que o país afundasse com a crise global -- e não teve
participação nos desastres dos dois últimos anos do governo Dilma. Sabe
que tem um pensamento articulado e pouco propenso a aventuras.

Nos primeiros momentos, no entanto, a visão dominante era a de um
aloprado fiscal no comando da Fazenda. A~Bolsa cai, o dólar sobe e os
profissionais realizam lucro.

Depois, haverá um momento de calmaria com o dólar refluindo.

\asterisc{}

As análises iniciais do mercado e de jornalistas financeiros sobre os
desafios de Nelson Barbosa são típicas de quem não consegue ir além dos
limites da planilha. São críticas sem nenhum realismo.

Uma estratégia bem"-sucedida precisa ser consistente do ponto de vista
macroeconômico, e factível, do ponto de vista político e social.

Por exemplo:

1.~~~~ Se o maior fator de desequilíbrio é a queda de receitas, que
ameaça inviabilizar União e estados, é evidente que estanca"-la é
prioridade número 1.

2.~~~~ É evidente que não há o menos espaço para aventuras fiscais e nem
a menor possibilidade de sair do embrulho fiscal sem uma \versal{CPMF}.

3.~~~~ Propostas de fim das transferências constitucionais são inviáveis
politicamente, selvagens e política e juridicamente inviáveis, no atual
estágio de desenvolvimento nacional. A~não ser que as Forças Armadas
concordem em voltar ao poder.~
