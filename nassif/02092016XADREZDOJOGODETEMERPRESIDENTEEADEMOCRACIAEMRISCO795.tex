\chapterspecial{02/\allowbreak{}09/\allowbreak{}2016 Xadrez do jogo de Temer presidente e a democracia em risco}{}{}
 

O cenário estratégico do governo Temer dependerá dos seguintes
desdobramentos:

\begin{enumerate}
\itemsep1pt\parskip0pt\parsep0pt
\item
  A questão econômica.
\item
  A questão política.
\item
  O desdobramento de ambas no campo da aliança que levou ao golpe
\end{enumerate}

\section{Peça 1 --- primeiro tempo da economia}

O quadro que se tem hoje é de uma enorme liquidez internacional.
Internamente, uma enorme taxa de juros internas, com a Selic a 14,25\% e
a garantia do Banco Central de mantê"-la elevada por bom tempo. E~também
um governo novo, o que garante estabilidade política pelo menos até o
final do ano.

Tudo isso combinado leva a um duplo movimento de apreciação dos ativos
internos --- especialmente a Bolsa e o real. Ou seja, Bolsa sobe, dólar
desce impulsionados pelo capital de curto prazo. Não se espere nenhum
investimento de longo prazo em um país em que nem o contrato dos
contratos -- o voto para presidente -- é respeitado.

As raposas de mercado sabem disso e acentuarão esse movimento altista de
curto prazo em parceria com a mídia, celebrando cada pequeno refresco
econômico como se fosse o fim da crise e cada avanço parlamentar como se
fosse o desfecho das reformas.

Vale até dezembro.

\section{Peça 2 --- segundo tempo da economia}

À medida em que se aproxime o final do ano, a realidade começará a se
impor:

\begin{enumerate}
\itemsep1pt\parskip0pt\parsep0pt
\item
  A constatação de que \versal{NÃO} haverá reformas.
\item
  A constatação de que a camarilha dos 6 e o baixo clero do Congresso
  --- e do Ministério Temer --- não trabalham estratégias de longo
  prazo: querem o seu à vista e imediatamente. Em parte, pela ausência
  total de conteúdo programático. Em parte, porque pairam sobre grande
  parte deles processos judiciais e ameaças de prisão. Apenas Michel
  Temer ganhou salvo"-conduto enquanto permanecer presidente.
\item
  Da base pode"-se esperar leis contra direitos trabalhistas. Ela
  representa os cafundós, o país pré"-industrializado, sem sindicatos nem
  direitos trabalhistas garantidos. Não se dá o mesmo com a Previdência
  Social e os gastos com saúde e educação. A~característica maior do
  baixo clero é seu caráter municipalista. E é nos municípios que irão
  bater os cortes orçamentários.
\item
  Finalmente, haverá duas obstruções importantes nos caminhos do
  desmonte: da oposição a Temer e da base aliada, que se alinhará para
  impedir a cassação e os processos contra Eduardo Cunha.
\end{enumerate}

Se terá, de um lado, o governo Temer sem condições de gerar fumaça e
expectativas positivas. De outro, o peso da realidade, do vencimento de
um caminhão de bônus externos afetando todos os grandes grupos
nacionais. Eles se prevaleceram da queda dos juros internacionais para
um pesadíssimo programa de investimentos, com os preços das commodities
ainda em alta. A~conta é impagável, exigindo praticamente um Proer para
esses grupos, com recursos do \versal{BNDES} e do Tesouro.

Qual a legitimidade do governo Temer para bancar essa operação?

Portanto, a partir do final do ano invertem"-se os movimentos: a bolsa
começará a cair, o dólar a subir e a blindagem irá se diluindo.

E aí se chegará na hora da verdade.

\section{Peça 3 -- a guerra dos porões}

Assim como nos estertores do regime militar, o fim do inimigo comum
promoverá uma guerra surda entre os diversos grupos que compõem a
repressão.

De um lado, Gilmar Mendes tratará de montar estratégias de contenção da
Lava Jato. Os grupos de mídia começarão lentamente a tirar os microfones
da operação. Polícia Federal e procuradores já estão em guerra aberta. O~mesmo acontece entre investigadores e delegados da \versal{PF}.

No Ministério Público Federal, a demissão de Ela Wiecko da
vice"-procuradoria da República teve desdobramentos muito mais profundos
dos que aqueles divulgados pela mídia. Ela não pediu demissão
simplesmente porque afirmou inadvertidamente para um repórter que havia
delações contra Michel Temer. Antes disso, ela pediu demissão e disse
pessoalmente ao Procurador Geral da República (\versal{PGR}) Rodrigo Janot seu
desconforto com o fato de estar segurando uma enorme quantidade de
evidências contra Temer e sua turma, permitindo que se blindassem no
poder. Refletia o estado de espírito de muitos procuradores
insatisfeitos com a condução das investigações da Lava Jato.

O \versal{MPF} é uma corporação disciplinada, na qual pouquíssimos procuradores
têm a coragem pessoal de manifestar discordância -- mesmo sabendo que a
atuação atual do \versal{PGR} compromete a imagem do \versal{MPF} junto a círculos
influentes dos organismos de direitos humanos internacionais e
nacionais. A~atitude de Ela pela primeira vez expos, para fora, os
rachas internos.

Nos próximos meses crescerá a disputa interna pela sucessão de Janot,
provavelmente entre Nicolau Dino, candidato de Janot (é um dos
formuladores dos Dez Mandamentos do \versal{MPF}), Mário Bonsaglia, de São Paulo,
e a própria Ela, representando os segmentos mais legalistas. Só que,
desta vez, não se terá um presidente disposto a nomear o mais votado.
Certamente Michel Temer escolherá um \versal{PGR} da sua estrita confiança.

Como ficará, então, o jogo de poder no \versal{MPF} e na própria Lava Jato?

A Lava Jato criou ilhas de privilégio dentro das duas corporações. A~força tarefa é premiada com diárias cumulativas, e, agora, com essa
medida imprudente de ficar com percentuais das quantias recuperadas.
Essas vantagens fazem com que se apegue cada vez mais ao cargo -- da
mesma maneira que os porões na ditadura -- e crie ilhas de excelência em
um poder que, como toda burocracia, tem apego ao formalismo e à
igualdade hierarquizada.

Na ditadura, a guerra dos porões resultou em bombas na \versal{OAB} e em bancas
de jornais. Obviamente os tempos são ~outros e há bombas políticas mais
sutis e de maior octanagem, que será a caça ao grupo de Temer --
devidamente aparada por Gilmar no \versal{STF}, e Janot na \versal{PGR}. Não disponho de
nenhuma informação maior sobre o novo comportamento de Janot. É~apenas
uma suposição levando em conta seu histórico, seu caráter adaptativo,
ainda mais depois de ter se tornado o principal articulador das
operações que levaram Temer ao poder.

Mas como ficarão as relações com o Supremo -- com Gilmar matando no
peito -- e nos tribunais superiores? Ontem, na sua posse como presidente
do Superior Tribunal de Justiça, Laurita Vaz, ex"-procuradora, fez um
candente discurso contra a corrupção. Na solenidade, foi muito aplaudida
pelos representantes do governo Temer, Eliseu Padilha, Romero Jucá,
Eunício Oliveira. Eduardo Cunha não pôde comparecer por problemas de
ordem maior.

A sustentação desse jogo hipócrita se dará apenas com uma melhoria
considerável da economia -- cenário que não está no horizonte.

A alternativa, então, será aumentar a repressão.

\section{Peça 4 -- a repressão}

Ontem as redes sociais divulgaram um vídeo de um advogado sendo vilmente
espancado por três \versal{PM}s gaúches. Hoje a informação de que o advogado foi
indiciado e nada foi aberto contra os \versal{PM}s. Alguma idiota da objetividade
ainda dirá que a divulgação do vídeo é prova maior de que estamos em um
regime democrático.

A repressão generalizada das Polícias Militares aos protestos contra o
impeachment tem dois organizadores principais. Um deles, o Ministro da
Justiça Alexandre de Moraes, através da Secretaria Nacional de Segurança
Pública -- que coordena as \versal{PM}s -- o outro, o general Sergio Etchegoyen
que, através do \versal{GSI} (Gabinete de Segurança Institucional) passou a
controlar o Sisbin (Sistema Brasileiro de Inteligência).

A segurança das Olimpíadas deveria ter sido confiada ao comando do
Estado Maior das Forças Armadas. Ou ao Ministério da Defesa. Temer
entregou à \versal{GSI}. O~envio de tropas do Exército para ocupar a avenida
Paulista, no próximo domingo -- a pretexto de escoltar a tocha da
Paraolimpíada -- faz parte dessa estratégia de endurecimento e de
entorpecimento gradativo da consciência jurídica do país. A~cada
manifestação é maior a violência das \versal{PM}s. Mas a mídia estava preocupada
em discutir a constitucionalidade do fatiamento do julgamento de Dilma.

À medida em que a legitimidade de Temer for sendo corroída junto ao
mercado, aumentará a escalada repressiva. Vai ser curioso porque
acelerará o desfecho de um jogo hipócrita. Cada passo a mais da
repressão tornará mais evidente a ficha suja dos novos donos de poder.
Analisarei esse quadro em um Xadrez próximo.

Aí, sim, se verá se há procuradores e juízes na República.
