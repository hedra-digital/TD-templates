\chapterspecial{05/\allowbreak{}12/\allowbreak{}2016 Xadrez do desmonte da democracia}{}{}
 

\section{Peça 1 -- os referenciais para analisar a crise}

Os referenciais em torno dos quais montaremos nossos cenários:

\begin{enumerate}
\itemsep1pt\parskip0pt\parsep0pt
\item
  O maior agente político continua sendo a massa dos bestificados que
  saem às ruas impulsionados pelo ódio e pela intolerância exarados pela
  mídia e pela Lava Jato.
\item
  Quase todas palavras de ordem pré"-impeachment se esvaziaram. Agora, o
  alvo da mobilização é o Congresso, com todos seus defeitos, o último
  setor de manifestação do voto popular. E~a turba sendo engrossada por
  procuradores e juízes, em uma nítida perda de rumo das instituições.
\item
  Agora, se tem um Judiciário brigando com o Legislativo, procuradores
  de Força"-Tarefa assumindo a liderança da classe, se sobrepondo ao
  Procurador Geral, em um quadro de indisciplina generalizada e
  crescente.
\item
  Esse clímax se dará com a revelação das delações da Odebrecht,
  tornando mais aguda a crise, a desmoralização da política e a busca de
  saídas milagrosas.
\item
  Se terá então a crise econômica se ampliando, o vácuo político se
  acentuando, e massas raivosas atrás de qualquer solução, por mais
  ilusória que seja, como esse cavalo de batalha contra a Lei
  Anti"-abusos.
\end{enumerate}

Vamos montar, por partes, esse mapa do inferno.

\section{Peça 2 -- o fim de Temer, o breve}

A economia se moverá seguindo o roteiro abaixo:

\begin{enumerate}
\itemsep1pt\parskip0pt\parsep0pt
\item
  O governo Michel Temer acabou. Trata"-se de um político menor e pior do
  que as piores avaliações sobre ele.
\item
  A era Henrique Meirelles também acabou.
\item
  O país está à beira de uma depressão, com convulsão social e com um
  governo sem diagnóstico e sem condição de comandar a recuperação.~~Mas
  o mercado insistirá em uma última tentativa, seguindo o jogo das
  expectativas sucessivas, conforme você poderá conferir no artigo
  ``Como o marketing reduziu a economia a um produto de
  boutique''(~\url{https:/\allowbreak{}/\allowbreak{}is.gd/\allowbreak{}\versal{WXB}q\versal{JW}}).
\end{enumerate}

Henrique Meirelles e sua tropa deixarão de ser a equipe brilhante que
salvaria a economia. Daqui para a frente, serão colocados no limbo, e a
nova equipe brilhante será a do ex"-presidente do Banco Central Armínio
Fraga, que é um Meirelles elevado à tríplice potência.

O problema da equipe econômica que assumiu as rédeas é que o seu
objetivo não é o de recuperação da economia, impedindo um desastre
social, mas o de destruir qualquer vestígio do modelo anterior, um
ideologismo barato e cego, marca, aliás, de boa parte do pensamento
econômico brasileiro.

\section{Peça 3 -- o governo de transição}

Com o fim do governo Temer, aventa"-se uma eleição indireta com Fernando
Henrique Cardoso, trazendo Armínio Fraga para aprofundar o ajuste
fiscal.

Aparentemente, essa loucura não se consumará por dois motivos.

\section{Motivo 1 -- \versal{FHC} refugou.}

Em duas manifestações seguidas, \versal{FHC} admitiu o óbvio: sem a recuperação
do voto, através de novas eleições diretas, será impossível a
implementação de qualquer programa econômico minimamente consistente. Na
verdade, \versal{FHC} tem noção de suas próprias limitações. Em momentos menos
graves -- como no processo inicial de consolidação do Real e no início
do segundo mandato -- \versal{FHC} foi incapaz de uma ação proativa sequer.
Limitou"-se a seguir o receituário de seus economistas, de um enorme
aperto fiscal, que contribuiu, nos dois casos, para uma economia
estagnada durante seus dois mandatos.

\section{Motivo 2 -- a aposta errada no aperto}

Além disso, caiu a ficha da classe empresarial sobre a loucura de
persistir nessa política suicida. Mesmo no mercado, a sensação é que a
persistência do quadro recessivo não permite ganhos a ninguém, mesmo ao
mercado. E~abre o risco de algum populismo de direita, que transforme o
mercado no bode expiatório.

A discussão que se iniciará agora é sobre o momento e a oportunidade das
novas eleições diretas, uma discussão que levará em conta o potencial
eleitoral de Lula e do \versal{PT} e as alternativas do atual grupo de poder.

\section{O fator Nelson Jobim}

Com o \versal{PSDB} pedindo para afastar de si este cálice, o nome mais forte
aventado -- lembrado pelo Xadrez de algumas semanas atrás -- é do
ex"-Ministro da Defesa e ex"-Ministro do Supremo Nelson Jobim. Tem bom
trânsito junto ao \versal{PSDB} e ao \versal{PT} e familiaridade com as Forças Armadas,
pela condução do Plano Nacional de Defesa.

Como presidente, será uma incógnita. Como candidato potencial, é a
melhor aposta até agora.

Mas todas essas alternativas caminham sobre o pântano, representado pelo
estímulo fascista às manifestações de rua. Abriu"-se nova temporada de
estímulo à violência, mostrando que a marcha da insensatez se abateu
também sobre os operadores da lei.

\section{Peça 4 --- sobre a irresponsabilidade dos golpistas}

Não era surpresa para quem tem um mínimo de visão e de responsabilidade
institucional. O~golpe desmontou definitivamente a democracia
brasileira, o modelo que garantiu o equilíbrio político do país desde a
Constituição de 1988. Uma mescla de aventureirismo, oportunismo,
despreparo, covardia promoveu a abertura da Caixa de Pandora.

Agora, a democracia está desmontada, a economia caminhando para uma
depressão. E, no momento, o que se tem é o seguinte:

\begin{itemize}
\itemsep1pt\parskip0pt\parsep0pt
\item
  O Executivo liquidado.
\item
  Uma campanha pesada visando inviabilizar o Congresso.
\item
  Uma briga de foice entre instituições, com uma cegueira generalizada
  sobre a gravidade do atual momento.
\item
  E a ultradireita sendo definitivamente bancada pela parceria Lava
  Jato"-Globo.
\end{itemize}

O fato de um mero procurador regional ousar afrontar o Congresso em nome
pessoal, ameaçando ``pedir demissão'' de uma força"-tarefa para o qual
ele foi indicado, mostra a desmoralização institucional do país e a
quebra total de hierarquia no próprio Ministério Público Federal.
Qualquer deslumbrado, com um metro e meio de autoridade, e uma tonelada
de atrevimento, coloca em corner não apenas o Congresso, mas o próprio
Procurador Geral.

Até onde irá esse clima? Difícil saber.

Com a delação da Odebrecht, os procuradores da Lava Jato insuflando as
manifestações, a crise se aprofundando, o caldeirão das ruas entrará
novamente em ebulição, sem que haja uma saída institucional à vista.

A crise começou seu trabalho de espalhar um pouco de bom senso. Mas
ainda é uma gota em um oceano de insensatez.
