\chapterspecial{18/\allowbreak{}07/\allowbreak{}2016 Xadrez dos novos tempos, da democracia em risco}{}{}
 

A democracia brasileira finalmente encontrou seu mais perfeito tradutor:
o espanhol Lluis Barba. Conforme explicam os críticos, Barba é conhecido
por ``ter desenvolvido uma reinterpretação da fragilidade da memória
histórica e sistemas de poder, juntando obras de artes de grandes
mestres com elementos contemporâneos''.

Nada melhor para definir a maneira como o país está encarando a
destruição sistemática dos mecanismos democráticos.~

\section{Peça 1 -- montagem de um pacto de governabilidade com Temer}

A receita está dada. Juntam"-se grupos econômicos, \versal{PSDB}"-\versal{DEM}, estamento
Jurídico, em torno de três objetivos claros:

\begin{enumerate}
\itemsep1pt\parskip0pt\parsep0pt
\item
  Desmanche das conquistas sociais.
\item
  Abertura integral da economia.
\item
  Recuperação pontual da economia garantindo as eleições de 2018.
\end{enumerate}

Tem que combinar com os russos.

Como dificilmente o modelo Temer será eleitoralmente competitivo, haverá
a tentativa de melar as eleições. Tudo obviamente dependendo da
viabilidade política do governo Temer perante os grupos de poder.

Aí entra o imprevisível: as delações do grupo Odebrecht.

\section{Peça 2 -- o fator Odebrecht}

A delação da Odebrecht significará a bomba de nêutron, o fato que zerará
de vez o jogo político. Serão poupados apenas membros do Judiciário. No
campo político, não restará pedra sobre pedra, cerca de 2/\allowbreak{}3 do
Congresso, todas as lideranças partidárias, todos os presidenciáveis,
incluindo os interinos que tomaram o poder de assalto.

Será um terremoto tão devastador que levará a um pacto inevitável.

\section{Peça 3 -- o pacto político da anistia}

O caminho mais viável será o de separar as doações em dois campos:
aquelas destinadas ao financiamento de campanha; a as que foram para
enriquecimento pessoal. Trata"-se de uma divisão tácita, que o meio
político utiliza internamente para não misturar alhos com bugalhos.

Saliente"-se que sempre houve uma prática generalizada do político
utilizar sobras de campanha em proveito pessoal, mesmo em pessoas de
reputação ilibada como André Franco Montoro (que adquiriu um apartamento
com as sobras de campanha para governador), Fernando Henrique Cardoso e
sua fazenda mineira, Tancredo Neves quando presidente eleito.

A proposta provável será votar uma anistia geral para o primeiro grupo e
jogar o segundo grupo aos leões.

Esbarra, no entanto em alguns problemas políticos.

A Lava Jato e a Procuradoria Geral da República têm lado, o \versal{PSDB}. E~seu
lado seria apanhado em cheio por uma peneira isenta, da qual não
escapariam José Serra e os fundos de investimento de sua filha Verônica;
Aécio Neves e Michel Temer, com elementos de enriquecimento pessoal
muito mais consistentes que sítios em Atibaia.

\section{Peça 4 -- os desafios à normalidade democrática.}

Ao zerar o jogo político, o novo redesenho comporta de tudo: até modelos
de democracia mitigada. Especialmente porque o país já não vive tempos
de normalidade democrática. E~não apenas pelo golpe parlamentar.

Os sinais são cada vez mais evidentes. Não se avançou mais porque houve
um refluxo das manifestações anti"-\versal{PT}, depois do susto de assistir à
ascensão da camarilha dos 6: Temer, Jucá, Padilha, Geddel, Moreira Fraco
e Cunha.

Mas é apenas uma pausa. Os sintomas do endurecimento político estão
nítidos:

\subsection{Sinal 1 -- o medo das palavras}

Na ditadura, jornalistas acostumaram"-se a escrever nas entrelinhas, pela
dificuldade de se abordar temas mais delicados. Voltaram"-se às mesmas
práticas:

\textbf{Exemplo 1}~-- para desgosto de Madame Natasha, o colunista Elio
Gaspari inventou o termo ``golpe vocabular'' para poder expressar sua
opinião sobre o golpe do impeachment.

\textbf{Exemplo 2}~-- o Ministro do \versal{STF} (Supremo Tribunal Federal) Luís
Roberto Barroso, explicando a não"-tomada de decisão no caso do golpe: um
lado pensa uma coisa e tem argumentos; o outro pensa outra e tem outros
argumentos; como é uma discussão política, não cabe ao \versal{STF} dizer quem
tem razão. Foi o dia em que se igualou a Ayres Brito.

\textbf{Exemplo 3~}-- o historiador José Murilo de Carvalho analisando
todos os golpes da República e, na hora de analisar o atual,
limitando"-se a dizer que, na oposição, o \versal{PT} também tentou o mesmo,
demonstração cabal de que o historiador é um personagem da história que
ele mesmo relatou: golpe é apenas o impeachment que os do outro lado
planejam.

\textbf{Exemplo 4~}-- a completa subversão de conceitos. A~Lava Jato
transformou em suspeitos qualquer forma de negócio ou de acordo político
ou comercial. Esse clima persecutório atingiu financiamentos à
exportação, atuação de diplomacia comercial. E~qualquer tentativa de
moderar abusos é imediatamente demonizada com o bordão de que ``querem
enfraquecer a Lava Jato''.

Ou seja, em vários dos campos do establishment, já se abdicou da defesa
ampla da democracia, ou por temor ou por falta de convicções
democráticas,

\subsection{\textbf{Sinal 2~}-- o dedurismo}

Desde o primeiro momento, o governo Temer mostrou características de
dedurismo como forma de retaliação contra adversários. Consiste em
membros do governo levantar informações do Estado, dos quais eles se
tornaram guardiões, vazando"-as seletivamente e dando"-lhes um sentido
criminoso. Os precursores da volta do dedurismo foram Laerte Rimoli e os
contratos da \versal{EBC}; José Márcio Freitas e a publicidade pública; e o
general Sérgio Ethchegoyen e as informações do Gabinete de Segurança
Institucional sobre a segurança da família de Dilma.

\subsection{Sinal 3 -- o uso da Lava Jato para represálias}

Gilberto Carvalho ousou criticar a Lava Jato e foi enfiado de
contrabando em uma das ações. O~mesmo vem ocorrendo com advogados e
blogs que critiquem a operação. A~difamação através da mídia tornou"-se
exercício banal: procuradores induzem delatores a inserir insinuações em
uma delação para que jornalistas policiais transformem em manchetes.

Não há coincidência desses diversos sintomas com os períodos de
intolerância que precederam guinadas autoritárias de governos e países.
Hoje em dia, há um alinhamento de parte de alguns juízes e procuradores,
visando atuar politicamente no cargo. E~um forte grupo de homens de
Estado trabalhando em direção ao endurecimento político, especialmente à
medida em que o \versal{PSDB} naufraga definitivamente como alternativa política.

\section{Cenários possíveis}

São muitas as variáveis e se tem um país relativamente complexo. Daí a
dificuldade de cravar as fichas em um cenário apenas. Prefiro relacionar
as forças em marcha para que vocês ajudem a desenhar o cenário final:

\textbf{Jogo político}~-- na medida em que o governo interino se
consolide, haverá uma tendência de acomodamento político, com os
governistas de sempre aderindo e a oposição parlamentar aceitando o jogo
a fim de preservar algum espaço. E, com isso, permitindo o crescimento
de partidos alternativos fora do parlamento e de movimentos sociais fora
da tutela dos partidos.

\textbf{Reformas~}-- haverá uma disputa dura de informações. De um lado,
os críticos tentando demonstrar que limite de despesas significará o
desmonte do \versal{SUS} e da educação. De outro, o establishment vendendo o
peixe de que esses cortes serão a salvação da lavoura.

\textbf{Jogo econômico}~-- o interino envereda de novo pelo populismo
cambial. O~aumento dos limites de endividamento e a apreciação cambial
darão um breve respiro à economia que eles esperam que dure até 2018.
Daqui até a votação do impeachment o mercado reagirá positivamente. No
dia da votação, despencará de novo, para realização de lucros.

\textbf{Jogo jurídico~}-- depois de inúmeras pressões contra sua
parcialidade, o Procurador Geral da República (\versal{PGR}) Rodrigo Janot
esboçou alguma tentativa de isonomia, abrindo ações contra Aécio Neves.
Cumpridas as formalidades, nada mais se sabe das investigações. No
momento, ele está preocupado com os presentes que Lula e Dilma receberam
de dignitários estrangeiros. A~delação da Odebrecht exigirá outras
estratégias evasivas.
