\chapterspecial{22/\allowbreak{}09/\allowbreak{}2016 Xadrez do \versal{MPF} como ameaça à democracia}{}{}
 

Ontem de manhã fiz uma palestra no encontro do Instituto Ethos sob o
tema ``Operação Lava"-Jato: como equacionar a relação entre
desenvolvimento econômico e combate à corrupção''. Era para contar com a
participação de um membro do Ministério Público Federal (\versal{MPF}). Nenhum
dos convites foi aceito.

O Ethos lançou uma bela carta sobre o tema
(\url{migre.me/\allowbreak{}v2n\versal{WL}}), com um conjunto de princípios ideais,
entre os quais:

\begin{itemize}
\itemsep1pt\parskip0pt\parsep0pt
\item
  ·~~~~~~ Apoiamos o avanço da operação no âmbito dos marcos
  constitucionais, sem foco partidário, vazamentos seletivos ou qualquer
  tipo de influência de interesses alheios às suas metas.
\item
  ·~~~~~~ Ela tem de ser ampla e irrestrita, devendo prosseguir enquanto
  houver irregularidades a apurar, independentemente de quem atingir,
  esteja essa pessoa no poder ou não.
\item
  ·~~~~~~ Hoje, somente 5\% dos condenados na Operação Lava"-Jato são
  políticos. Sabemos que há foro privilegiado, mas é necessário obter,
  de fato, progressos na celeridade e na efetivação dos processos que
  envolvem a classe política.
\end{itemize}

A operação que o Ethos apoia seguramente não é a que estamos
testemunhando.

Na minha apresentação, procurei demonstrar que essa operação ideal é
improvável na conjuntura política atual.

Meu xadrez é o seguinte:

\begin{enumerate}
\itemsep1pt\parskip0pt\parsep0pt
\item
  1.~~~ A nova jurisprudência penal, a ampliação do poder de
  investigação do Ministério Público Federal, inclusive com o acesso a
  dados internacionais, conferiu poder enorme à corporação.
\item
  2.~~~ Não existe superpoder que possa depender exclusivamente dos
  princípios éticos e valores morais individuais de seus membros. Com a
  ausência de um sistema de freios e contrapesos, a lógica do \versal{MPF} será
  cada vez mais de tentar ampliar o espaço, até bater no muro de um
  pacto entre os demais poderes.
\item
  3.~~~ A ocupação do espaço pelo \versal{MPF} passou pela parceria com a mídia e
  pelo apoio da classe média ascendente, com a qual a corporação é mais
  identificada. O~pacto se deu em torno do combate ao inimigo comum, o
  \versal{PT}. Sem a figura do inimigo e a prática do direito penal do inimigo, a
  aliança não se sustenta.
\item
  4.~~~ O primeiro uso da força pelo \versal{MPF} foi na \versal{AP} 470, que
  desequilibrou o jogo político do nosso precário presidencialismo de
  coalizão, empurrando o governo Lula para os braços do \versal{PMDB}, usando a
  Petrobras como moeda de troca, conforme se conferiu na delação do
  ex"-senador Delcídio do Amaral.
\item
  5.~~~ O segundo movimento foi com a Lava Jato explorando as
  vulnerabilidades criadas pelo primeiro movimento, e levando à queda do
  governo.
\end{enumerate}

Portanto, fez"-se uma campanha moralista, fundada na luta anticorrupção e
o resultado final foi o desmantelamento do sistema partidário e a
entrega do comando do país ao grupo político mais suspeito das últimas
décadas, que mais cedo ou mais tarde utilizará o poder do qual se viu
revestido para para interromper a Lava Jato e enquadrar o \versal{MPF}.

Ontem, o \versal{CNMP} (Conselho Nacional do Ministério Público) premiou a Lava
Jato com destaque do ano. Prova maior de que a miopia política não
acometeu apenas os governos Lula e Dilma e o \versal{PT}. É~processo
generalizado.

\section{Peça 1 -- o processo judicial e a busca da verdade}

Primeiro, vamos entender como analisar um procedimento jurídico.

Meu primeiro desafio jornalístico em temas jurídicos foi uma denúncia
que fiz contra o então Consultor Geral da República Saulo Ramos, devido
a um decreto, logo após o Plano Cruzado, que recriava a indústria de
liquidação extrajudicial.

Saulo manobrava conceitos jurídicos, que eu desconhecia.

No meio do debate, consegui uma fonte especialíssima, um Ministro do \versal{STF}
(Supremo Tribunal Federal), que me passou uma lição básica para me
livrar do jugo do especialista:

-- O processo judicial tem que levar à justiça. Analise a realidade e
veja o resultado da decisão tomada. Se levar a um resultado injusto, ou
a lei é injusta ou a interpretação dela está errada.

Lembro essa história para nos debruçarmos sobre os resultados dessa
metodologia do \versal{MPF} de colheita de provas -- explicada em um livro de
Deltan Dallagnol muito elogiado --- sobre a construção das provas
através do levantamento de indícios. Ele leva à verdade e, através dela,
à justiça? Ou o excesso de poder desequilibrou o jogo a tal ponto que a
lógica do acusador se impõe por si, sem poder ser retificada pelos
argumentos da defesa?

A prova do pudim consiste em confrontar essa metodologia com os
resultados alcançados. Levou à justiça, ou foi apenas a
instrumentalização do combate ao escândalo, para benefício de um grupo
político e de uma corporação? Levou ao aprimoramento das instituições
ou, pela desorganização da política, à criação de uma realidade pior?

O primeiro passo é entender a conjuntura que levou à consolidação desse
novo modelo de operar a lei.

\section{Peça 2- a punição dos chefes das organizações criminosas}

Me deparei com essa questão pela primeira vez na cobertura do golpe
aplicado no Banco do Comércio e Indústria de São Paulo, o Comind, ainda
nos anos 80. Era voz corrente que dificilmente os chefes de golpe seriam
apanhados porque não deixavam vestígios, assinaturas, documentos.
Simplesmente davam ordens verbais. Havia um nítido desequilíbrio em
favor do crime organizado.

Com a expansão internacional do crime organizado, com a captura de
muitos Estados nacionais pelo crime, houve mudanças também na
jurisprudência sobre o tema, aceitando que um conjunto robusto de
indícios poderia ser tratado como prova, mesmo que não houvesse as
impressões digitais do chefe no cometimento do crime.

Essa jurisprudência surgiu a partir, principalmente, da luta contra o
tráfico de droga e contra o terrorismo. Entende"-se, daí, seu caráter
draconiano.

Os indícios vão da identificação do comando hierárquico da organização,
a provas testemunhais --- em geral, de pouco valor nos processos penais.
Passaram a ser aceitos também outros instrumentos jurídicos, como o da
delação premiada, que veio se somar à quebra de sigilo telefônico,
fiscal e bancário.

Flexibilizou"-se radicalmente o processo de obtenção de provas. Aí o
pêndulo se inverteu completamente e o poder acabou centralizado nos
acusadores. E, como tal, sujeito às suas idiossincrasias e preferências
políticas e ideológicas.

Para não incorrer em abusos, com enorme poder recebido, havia a
necessidade do chamado intérprete da lei ter conhecimento e observância
de princípios de direitos humanos aceitos internacionalmente -- entre os
quais os valores democráticos e a relevância central do voto.

Mas não apenas isso. Não existe instituição cuja idoneidade dependa
exclusivamente dos valores individuais de cada membro. O~modelo exige os
chamados freios e balanços para coibir abusos.

Não é o caso do Brasil. As corporações se apropriaram dos órgãos de
controle, que passaram a responder às demandas corporativas.

Nos tribunais de primeira instância, as provas indiciárias se voltam
preferencialmente contra os \versal{PPP}s (preto, pobre e puta). Servem para
enviar ``mulas'' para os presídios, mas não alcançam os chefes do
tráfico.

Na área política, em muitos países de democracia precária -- como
Portugal e Brasil -- o modelo agregou o quarto P, de petista ou popular.
E~aí, introduziu"-se no processo democrático um enorme fator de
desestabilização, no qual as armas conquistadas pelo \versal{MP}, pela lógica de
poder, são colocadas a serviço de grupos políticos e ideológicos aos
quais ele se aliou estrategicamente para ampliar seu poder.

Provavelmente a maior ameaça à democracia, hoje em dia, seja a
interferência do Ministério Público e da Justiça no jogo político. O~século do Judiciário -- na celebração infeliz do Ministro Ricardo
Lewandowski -- de certo modo é similar às \versal{UPP}s (Unidades de Policia
Pacificadora) nas favelas. A~pretexto de coibir o crime, apossam"-se de
todo o território e criam um poder paralelo muito mais letal.

\section{Peça 3 -- o teste da \versal{AP} 470, o ``mensalão''}

O ``mensalão'' foi o primeiro grande processo de impacto político a
testar as tais provas indiciárias. A~celebérrima frase de Rosa Weber
(apud Sérgio Moro) de que ``não tenho provas (contra Dirceu) mas a
jurisprudência me autoriza a condenar'', celebrava o ``abre"-te Sésamo''
do Judiciário para abrir a caverna onde se encontravam as capas de Super
Homem, os novos superpoderes que conquistaram.

O que havia -- e isso era do conhecimento de qualquer analista político
--- era o pagamento de despesas de campanha dos pequenos partidos que
passaram a fazer parte da base aliada. A~acusação defendeu a tese de que
havia uma mesada intermitente para garantir a aprovação de leis de
interesse do governo.

Mais do que isso, procedeu a enormes malabarismos para casar data de
pagamento com aprovação de leis, , inclusive para parlamentares
petistas, forçando relações de causalidade inexistentes, da maneira como
descrevo no ``Xadrez do não temos prova, mas temos convicção''
(\url{migre.me/\allowbreak{}v2mmk)}. Quem acompanhava o jogo político sabia
que era uma narrativa falsa. Mas passou.

A maneira como costuraram essa narrativa era da modalidade de ``enfiar
argumentos na tese a marteladas''.

\begin{enumerate}
\itemsep1pt\parskip0pt\parsep0pt
\item
  1.~~~~ A história do suposto desvio da Visanet, quando se sabia que o
  grande financiador de Marcos Valério era o banqueiro Daniel Dantas. A
  razão era simples. Para caracterizar corrupção, o dinheiro teria que
  ser proveniente de ente público. Tratava"-se o dinheiro de Dantas como
  privado; e o da Visanet como público (embora não fosse), devido à
  participação do Banco do Brasil no capital da empresa. Sem a Visanet,
  portanto, a tese da \versal{PGR} não se sustentaria. Não só trataram a Visanet
  como empresa pública, não sendo, como denunciaram um desvio que jamais
  houve, ignorando laudos de auditorias e da própria Polícia Federal.
\item
  2.~~~~ A história da ida de políticos do \versal{PTB} a Portugal com Marcos
  Valério negociar com a Portugal Telecom a venda da Telemig. Atribuíam
  ao \versal{PT}. Eu tinha informações seguríssimas --- inclusive após conversas
  com executivos da Portugal Telecom \mbox{---,} que a ida foi bancada por
  Daniel Dantas, que ainda mantinha o controle da Telemig e para quem
  Valério trabalhava.
\item
  3.~~~~ A inclusão de José Genoíno no inquérito. O~alvo era José
  Dirceu, então Ministro"-Chefe da Casa Civil, já que o inquérito nasceu
  das denúncias de Roberto Jefferson. Mas como pegar Dirceu sem envolver
  o presidente do \versal{PT}, José Genoíno? Havia a necessidade desse elo na
  corrente (\url{migre.me/\allowbreak{}v2smK}).
\end{enumerate}

A primeira e a segunda questão beneficiaram diretamente Daniel Dantas.

Como foi possível que um erro desse tamanho passasse pelo filtro da
Procuradoria Geral da República, com a \versal{AP} 470 sendo analisada por
diversos procuradores, depois pelo relator, Ministro Joaquim Barbosa e,
finalmente, pelo pleno do \versal{STF}?

Mas passou.

Havia indícios de corrupção na decisão de Antônio Fernando de poupar
Daniel Dantas (logo depois aposentou"-se e seu escritório ganhou enorme
contrato da Brasil Telecom, controlada por Dantas). Mas seria
impossível, mesmo para alguém do alto do cargo de \versal{PGR}, impor uma tese
dessas a todo uma equipe, se não houvesse outros ingredientes no jogo.

O endosso às teses de Antônio Fernando foi fruto da grande celebração do
\versal{MPF}, ante a possibilidade de usar pela primeira vez os superpoderes e
balançar a República, a possibilidade de impor a narrativa que quisesse,
desde que escudada em campanhas massacrantes de mídia. Foi um porre
geral. E~a mítica da narrativa exigia que se concentrasse no \versal{PT} todas as
acusações de corrupção, transformado na fonte de toda a corrupção. É~por
ali que se consolidaria a aliança com a mídia e a identificação com os
anseios da classe média.

A parceria do \versal{MPF} com a mídia esvaziou a \versal{CPMI} de Cachoeira -- que estava
prestes a convocar Roberto Civita, da Abril. No mesmo período, o
processo sobre o ``mensalão do \versal{PSDB}'' foi interrompido da maneira mais
canhestra possível. O~Ministro Ayres Britto deveria relatá"-lo em uma
sessão do \versal{STF}. Houve o intervalo, ele saiu para o café, voltou e passou
por cima da pauta. Simples, assim, sem nenhuma cobrança da parte
acusadora -- justamente o Ministério Público Federal.

Uma das regras básicas do presidencialismo de coalizão é que, quanto
mais fraco o governo, maiores as concessões à fisiologia. Ocorreu com o
governo \versal{FHC}, após a maxidesvalorização de 1999; e com o governo Lula,
devido à \versal{AP} 470.

O resultado dessa primeira intervenção do \versal{MPF} no jogo político foi o
seguinte:

\begin{enumerate}
\itemsep1pt\parskip0pt\parsep0pt
\item
  1. O~abandono da estratégia de Lula de montar uma base com os pequenos
  partidos e o fechamento do acordo com o \versal{PMDB}.
\item
  2. Com o risco concreto de impeachment, uma dependência cada vez maior
  do \versal{PMDB}.
\item
  3. Uma arquitetura política que só se sustentaria com economia em
  crescimento.
\end{enumerate}

O sucesso da economia nos anos seguintes inibiu por algum tempo sua
atuação. E~a razia promovida pela \versal{AP} 470 nas lideranças petistas
históricas, deixou o partido sem nenhuma capacidade de formulação
estratégica.

A última trégua, antes do embate final, foi desperdiçada por Lula,
embalado pelos feitos que o deixaram na posição de internacionalmente
mais celebrado presidente brasileiro da história.

Dormiu em berço esplêndido. Acordou quando a serpente já dera o bote
final.

\section{Peça 4 -- os desdobramentos da Lava Jato}

É evidente que há problemas estruturais nesse presidencialismo de
coalizão e circunstâncias políticas que levaram os partidos aliados e o
próprio \versal{PT} a se lambuzarem. É~evidente também que se desperdiçou o
momento de enorme popularidade de Lula para se proceder a uma reforma
política radical. Não adianta: apenas os problemas que afetam o dia a
dia merecem prioridade.

No entanto, em vez de um trabalho isento contra a corrupção, o que se
viu da parte do \versal{MPF} foi uma ação seletiva, com nítido propósito
partidário, de consolidação do poder corporação, e uma perseguição
implacável a Lula, ao mesmo tempo que se blindavam as principais
lideranças da oposição.

Nesse período, a publicidade opressiva alimentada pelo \versal{MPF}, ajudou a
fomentar movimentos de manada instituindo um clima de vale"-tudo no país,
exacerbando o que de pior existe no imaginário popular: violência,
preconceito, caça às bruxas, queda da autoestima nacional.

Os resultados estão aí:

\begin{enumerate}
\itemsep1pt\parskip0pt\parsep0pt
\item
  1.~~~~ Insegurança jurídica, com a entrada em um período de exceção,
  na qual nenhuma pessoa que se oponha à Lava Jato ou ao novo governo
  pode se considerar juridicamente segura.
\item
  2.~~~~ Insegurança jurídica nos negócios, à medida que qualquer
  procurador idiossincrático poderá invocar como suspeitos até
  financiamentos do \versal{BNDES}. Perdeu"-se o referencial, a divisória entre
  operações legais e as criminosas.
\item
  3.~~~~ Insegurança política para qualquer governador, já que as tais
  provas indiciárias podem tentar casar qualquer ato de governo com
  contribuições de campanha.
\item
  4.~~~~ Insegurança física, com o país rachado em dois e a montagem de
  um sistema de repressão, e um liberou geral para as Polícias
  Militares. Em São Paulo há notícias da \versal{P}2 (o serviço secreto da \versal{PM})
  monitorando jovens secundaristas que participaram da ocupação das
  escolas estaduais no ano passado. A~tentativa da \versal{PFDC} (Procuradoria
  Federal dos Direitos do Cidadão) de monitorar a \versal{PM} foi rechaçada pelo
  Ministério Público de São Paulo e pelo Ministro da Justiça sem um
  posicionamento sequer do \versal{PGR} em defesa da sua Procuradoria.
\item
  5.~~~~ Insegurança política, com enorme leque de possibilidades, fruto
  dos arreglos políticos e dos interesses dos grupos que se apoderaram
  do poder, nenhum dos quais contemplando eleições diretas. E~o país
  entregue a uma camarilha de políticos suspeitos, com o fim da bazófia
  do Procurador Geral da República (\versal{PGR}) de avançar sobre as lideranças
  políticas que assumiram o poder, deixando"-as à vontade para o
  exercício do arbítrio e dos negócios.
\item
  6.~~~~ Insegurança social, com a perspectiva de retrocessos em todas
  as áreas, especialmente saúde e educação, pela imposição dos tais
  tetos nominais de despesa, tudo feito ao largo do voto popular.
\item
  7.~~~~ Queima de ativos nacionais, com a venda de empresas e reservas
  petrolíferas na bacia das almas.
\item
  8.~~~~ Desmontagem de setores inteiros da economia
\item
  9.~~~~ Consolidação da ideia de parcialidade do \versal{MPF}, com as manobras
  sucessivas para invalidar o depoimento de Léo Pinheiro e livrar Aécio
  Neves e José Serra.
\end{enumerate}

O \versal{MPF} importou a tese da supremacia das provas indiciárias e está
aplicando. E~vai exportar um caso que será analisado por todos os
centros especializados no estudo do crime organizado: as
vulnerabilidades da tese e o risco que trouxe para a estabilidade
democrática em países de democracia não consolidada, como é o caso do
Brasil.

\subsection{Esclarecimento do Ministro Lewandowski}

 

O Ministro Ricardo Lewandowski entra em contato para alguns
esclarecimentos relevantes.

 

Questiona a inclusão do Judiciário no superdimensionamento do sistema
repressor. O~Judiciário são milhares de juizes com posições plurais e,
muitas vezes, servindo de barragem contra os exageros do aparelho
repressor.

 

Reclama da maneira como foi citada sua frase de que ``o século 21 é do
Judiciário''. Em nenhum momento defendeu o Judiciário substituindo a
política, muito pelo contrário. Assim como sempre atuou como garantista.
A~frase ele retirou dos estudos de Norberto Bobbio explicando que o
avanço da cidadania faz com que a sociedade evolua dos direitos básicos
para os direitos específicos, de minorias. Esse foi o tema de sua tese
de doutorado. E~esse papel cabe ao Judiciário, como defensor central dos
direitos das minorias.

 

Lembra as sucessivas votações no \versal{STF} (Supremo Tribunal Federal) em favor
da lei das cotas, da união homoafetiva e de tantos outros avanços
civilizatórios. Ontem mesmo, o \versal{STF} aprovou a lei responsabilizando o pai
biológico pela manutenção de pensão aos filhos criados por terceiros.

 

O Ministro Lewandowski tem razão. Fica aqui, no mesmo texto, os devidos
esclarecimentos. E~o pedido de desculpa pela interpretação de uma frase
sua, fora do contexto
