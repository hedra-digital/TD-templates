\chapterspecial{19/\allowbreak{}09/\allowbreak{}2016 Xadrez do não temos provas, mas temos convicção}{}{}
 

Nos anos 90, entrei em uma lista de procuradores da República, para me
defender de críticas pelo fato de ter condenado, na época, o uso abusivo
da prisão preventiva. Uma subprocuradora me enviou um e"-mail pessoal,
contando as agruras dela com alguns colegas. Dois procuradores
adversários enviaram uma denúncia em off para o Correio Braziliense.
 Com base na denúncia, fizeram uma representação contra ela na
corregedoria.

Li o relato, e, embora nada soubesse da subprocuradora, confirmei e
denunciei a manobra, por abusiva.

Tempos depois, participei de um programa do Observatório da Imprensa,
sobre os abusos do \versal{MP}, com um procurador penal que atuara do caso
Banestado. No ar, ao vivo, ele dispara:

-- Este jornalista defendeu uma subprocuradora que foi a última pessoa
que autorizou o repasse para as obras do \versal{TRT} em São Paulo. E~ele atacou
o pedido de prisão preventiva dos empresários suspeitos. Logo, é
cúmplice do golpe do \versal{TRT}.

Agradeci ao Alberto Dines a oportunidade de uma análise de caso, ao
vivo. Ao vivo, mostrei o ridículo das ilações e ponderei: se eu não
estivesse ali, para rebater na hora as insinuações do procurador, qual
seria o resultado final? Ficaria a suspeita lançada.

Do início do Sisbin (Sistema Brasileiro de Inteligência) para cá, tanto
a \versal{PF} quanto o \versal{MPF} receberam instrumentos poderosíssimos de investigação:

\begin{itemize}
\itemsep1pt\parskip0pt\parsep0pt
\item
  ·~~~~~~ A integração com a fiscalização do \versal{BC}, \versal{COAF}, \versal{TCU} e Receita.
\item
  ·~~~~~~ A cooperação internacional, com acesso a contas em paraísos
  fiscais.
\item
  ·~~~~~~ O Instituto da delação premiada.
\item
  ·~~~~~~ A tecnologia de trabalhar com grandes massas de informação.
\item
  ·~~~~~~ Aprenderam a usar o Power Point.
\end{itemize}

Falta apenas um dado para completar o quadro: capacidade analítica, que,
pelo visto, pouco mudou.

O que se vê, hoje em dia, além do partidarismo político, é muito
amadorismo nas investigações, falta de conhecimento sobre o objeto
investigado, com delegados e promotores enfiando provas à martelada,
transformando saladas de indícios em provas, para conseguir montar
relações de causalidade entre fatos muitas vezes completamente
distintos.

Desmoralizam os suspeitos junto ao público leigo, e as investigações
junto ao público bem informado. Pior, criam o efeito manada graças a uma
imprensa igualmente sem filtros e sem isenção, deixando o Judiciário
entre ou analisar tecnicamente as provas apresentadas ou se submeter ao
clamor da besta, a multidão bestializada querendo linchamentos a partir
das análises disseminadas.

Hoje em dia, a investigação penal se resume aos meios:

\begin{enumerate}
\itemsep1pt\parskip0pt\parsep0pt
\item
  1.~~~~ Escuta telefônica.
\item
  2.~~~~ Delação pressionada.
\item
  3.~~~~ Quebra de sigilos fiscal e bancário.
\end{enumerate}

É uma montanha de dados fornecidos por diversos órgãos, e que são
submetidos -- pelos exemplos tornados públicos -- a análises simplórias,
sem nenhuma sofisticação analítica, tipo pão"-pão queijo"-queijo, ou
pão"-trigo, queijo"-cabra, ~forçando ilações inexistentes.

Fazendo blague: o sujeito disse pão; se disse pão, quis dizer trigo que
é a matéria prima do pão; se quis dizer trigo, disse Rio Grande do Sul,
que é o maior produtor de trigo; logo, o suspeito é um gaúcho.

Para entender melhor esse método de investigação, vamos dissecar de novo
a ``teoria do fato'', que está na base das investigações do \versal{MPF}, e já
foi analisada em outro Xadrez.

\section{Passo 1 -- a teoria do fato}

Trata"-se de um método -- a ``teoria do fato'' (não confundir com a
teoria do domínio do fato) -- que, intuitivamente, sempre utilizei na
investigação de fatos complexos -- aqueles que geram uma grande
quantidade de informações. Descrevo ela no livro ``O Jornalismo dos anos
90'', no capítulo referente à \versal{CPI} dos Precatórios.

Quando já se tem uma quantidade suficiente de indícios, monta"-se uma
hipótese de trabalho, o que facilita colocar ordem no raciocínio e, a
partir daí, a seleção das informações a serem filtradas.

Se a teoria não consegue enquadrar determinada informação, muda"-se a
teoria, é claro.

Não parece ser o método do \versal{MPF} e da \versal{PF}.

Antes de começar a investigar os investigadores montam uma hipótese de
partida, e apegam"-se a ela como se fosse matéria de fé. Julgam que
qualquer correção da rota inicial poderá ser interpretada como erro.
Insistindo na narrativa inicial, acabam forçando a barra, minimizando
informações que não a confirmem, e forçando a busca de evidências que a
reforcem. É~daí que decorre esse bordão de ``não temos provas, mas temos
convicção'' que, ao contrário do que parece, não é uma postura isolada,
mas uma posição recorrente em investigações apressadas ou
pré"-endereçadas.

\emph{Leia mais sobre isso em
``{Interrogatório
de Delcídio mostra como a Lava Jato forma sua convicção}".}

Por exemplo, no escândalo dos precatórios (que descrevo no livro), a
hipótese de uma certa cobertura pavloviana de Brasília, era a de que a
prefeitura de São Paulo, na gestão Paulo Maluf, montou a primeira
operação. Um pequeno banco do Rio teria aprendido a fórmula, cooptado um
funcionário da Secretaria das Finanças e replicado em diversos outros
estados.

Era uma teoria do fato que livrava Maluf e deixava a bomba explodir no
colo do banco e do funcionário. Mas era claramente insuficiente para
explicar alguns pontos.

Como um pequeno banco do Rio, sem nenhuma expressão, conseguia que a \versal{CAF}
(Comissão de Assuntos Financeiros do Senado), presidida pelo notório
senador Gilberto Miranda, aprovasse o aumento do endividamento de
estados e municípios (base do golpe), chegando a movimentar R\$ 5
bilhões? E, depois, fosse bater nos principais estados do país?

À medida que surgiram novos fatos, consegui formular uma nova narrativa.

O golpe foi planejado por empreiteiros com o prefeito de São Paulo,
Paulo Maluf, valendo"-se de uma interpretação forçada de dispositivo
transitório da Constituição de 1988. Maluf montou a jogada em conluio
com o presidente da Comissão de Finanças do Senado, Gilberto Miranda,
com quem já havia aplicado outros golpes em parceria (como o dossiê
Cayman). Vendo que dava certo, terceirizou a venda para outros estados
através do banco Vetor, do Rio de Janeiro, e colocou o tal funcionário
da Prefeitura como intermediário dele, Maluf, junto ao banco.

Pouco tempo depois, o advogado Márcio Thomas Bastos me apresentou seu
cliente, justamente o funcionário de carreira da Prefeitura que estava
pagando o pato. Ele me confirmou que fazia o meio campo entre Maluf e o
banco, tratando diretamente com Maluf e passando por cima do Secretário
de Finanças Celso Pitta -- outro para quem quiserem empurrar a culpa.

Era uma nova teoria do fato que abarcava toda a quadrilha. Se aparecesse
nova evidência, mostrando outra realidade, se mudaria a hipótese
novamente, não os fatos.

Um dia, ainda, o senador Roberto Requião vai me contar o que o levou a
desconsiderar essa conclusão no relatório final da \versal{CPI}, a quem coube a
ele relatar.

Essa inflexibilidade de ficar preso à primeira versão explica os grandes
crimes de imprensa que narro no livro. ~Escola Base, Bar Bodega e tantos
outros ocorreram porque os editores se apegaram à primeira versão e não
quiseram corrigir o rumo, à medida em que apareciam outros dados.

No entanto, esse vício de investigação tornou"-se padrão no \versal{MPF} e na \versal{PF},
especialmente em investigações sobre corrupção política, por um
desconhecimento rotundo do objeto investigado e também para abrir espaço
para jogadas políticas.

\section{Peça 2 -- as formas de parceria política}

Em qualquer democracia, coalizão é elemento essencial de governabilidade
. E~consiste em montar alianças com outros partidos, oferecendo"-lhes
participação no governo, com cargos no Ministério ou em autarquias,
empresas e entidades públicas.

Em momentos de fragilidade política -- \versal{FHC} pós maxidesvalorização e Lula
pós"-mensalão \mbox{---,} o Executivo fica mais vulnerável às chantagens
políticas. E é obrigado a fazer mais concessões.

Grosso modo, há duas utilizações possíveis: uma legal, embora
transitando na zona cinzenta; e outra claramente criminosa.

\subsection{\textbf{Fórmula A}~-- as parcerias institucionais}

A empresa A faz doações ao partido B -- esteja ele na Presidência ou à
frente de um Ministério ou uma estatal --- para obter sua boa vontade de
maneira geral e ajudar a influir em políticas públicas do seu interesse.

Não se pense em pactos em torno das excelsas virtudes públicas. São
jogos de interesse, parte integrante de todas as democracias, que
garantem a eleição dos políticos e a governabilidade. O~papel das
empreiteiras norte"-americanas no Iraque, o complexo industrial"-militar
por trás das guerras, a influência dos grupos econômicos no Departamento
de Estado ou na \versal{CIA}, para competir com adversários estrangeiros, tudo
isso faz parte desse modelo de atuação.

Quando se entrega um cargo para alguém indicado por político, a
responsabilidade é do político e do seu indicado.

Embora transite em uma zona cinzenta, este é o modelo usual de praticar
pactos políticos. É~intrínseco a qualquer democracia.

\subsection{\textbf{Fórmula B}~- a propina}

Aqui se entra no terreno penal propriamente dito.

A Empresa A paga uma propina ao Político ou Partido B em troca de um
contrato específico. Nesse caso, tem"-se um crime comum caracterizado: a
figura do subornado, do subornador e do objeto de suborno, o contrato
firmado. É~o que está nítido na ação política na Petrobras, envolvendo
vários partidos, entre os quais o \versal{PT}, o \versal{PP} e o \versal{PMDB}.

\section{Peça 3 --- a jabuticaba da \versal{PF} e do \versal{MPF}}

Procuradores e delegados da Lava Jato misturam os dois fenômenos com
cabeça de burocrata seguidor de manual, que precisa de qualquer jeito
encontrar as figuras clássicas que caracterizam o ato de corrupção: o
subornador, o subornado e o objeto alvo do suborno.

Fazem assim:

\begin{enumerate}
\itemsep1pt\parskip0pt\parsep0pt
\item
  1.~~~~ Levantam doação eleitoral de uma determinada empresa a um
  determinado candidato.
\item
  2.~~~~ Os nossos Sherlocks vão atrás de qualquer medida que possa ter
  beneficiado a empresa. Pouco importa se são medidas genéricas,
  beneficiando todo um setor, se passaram por várias instâncias de
  aprovação, se as decisões não estão sob o controle direto do acusado.
  Tem que encontrar qualquer uma para completar a lição de casa. E, como
  usam muito pouco a capacidade analítica, e têm escasso conhecimento de
  mecanismos financeiros ou de políticas públicas, o que cai na rede é
  peixe.
\end{enumerate}

É por aí que se entende tentar criminalizar financiamentos de exportação
de serviços do \versal{BNDES}, a diplomacia comercial de Lula na África, e até o
fato de Lula ter intercedido junto a um presidente mexicano por uma
empresa brasileira e outras bobagens memoráveis -- mas de óbvio impacto
político.

Se Lula era presidente, a empreiteira recebeu financiamento do \versal{BNDES} e
contribuiu com a campanha eleitoral, logo se tem o subornado (Lula), o
subornador (a empreiteira) e o objeto de suborno (o financiamento). E~essa a lógica obtusa vale para toda a cadeia de comando. Serve para
criminalizar o presidente, a quem o Ministro está subordinado; o
Ministro a quem o presidente do \versal{BNDES} está subordinado; o presidente do
\versal{BNDES}, a quem os diretores de área estáo subordinados; o diretor de
área. Mesmo que seja uma operação absolutamente legal e prevista nas
linhas de financiamento do banco.

Nenhum juiz sério endossará essas ilações. Mas o objetivo político é
atingido, através do pacto com a mídia e da escandalização de factoides.

Vamos analisar dois episódios da Operação Acrônimo, que move perseguição
implacável contra o governador de Minas Gerais Fernando Pimentel.

\textbf{Caso 1}~- a \versal{CAOA}, fabricante de automóveis, contribuiu para sua
campanha para governador de Minas. Como provavelmente contribuiu para
candidatos de outros partidos. Aí, delegados vão atrás de alguma medida
que tenha beneficiado a \versal{CAOA} enquanto Pimentel era Ministro do
Ministério do Desenvolvimento, Indústria e Comércio Exterior (\versal{MDIC}).
Encontram uma medida provisória assinada pela presidente com benefícios
para montadoras fora do circuito São Paulo"-Minas. A~tal \versal{MP} beneficiou
todas as montadoras que se enquadravam, foi aprovada por todas as
lideranças do Senado --- incluindo os senadores da oposição \mbox{---,} tem uma
justificativa de ordem regional. Pouco importa: encontraram a
causa"-efeito para fechar o raciocínio e é o que basta. É~a lógica do
pertinaz procurador que participou do Observatório da Imprensa.

\textbf{Caso 2}~- A Odebrecht contribuiu para a campanha de Pimentel,
assim como para metade do mundo político. O~que levou em troca? Segundo
a Acrônimo, financiamento do \versal{BNDES} para obras na África. Era uma linha
disponível para todas as empreiteiras, dentro da política de exportação
de serviços. O~financiamento passa por diversas instâncias de análise,
por quadros técnicos de carreira, tem sua sistemática, sua dinâmica,
seus procedimentos impessoais. A~Odebrecht teria conseguido o
financiamento mesmo que tivesse bancado apenas a campanha do candidato
do \versal{PSDB}. Pouco importa: o \versal{BNDES} estava sob o comando do \versal{MDIC}; Pimentel
era Ministro do \versal{MDIC}; logo, foi ele quem deu ordem para o \versal{BNDES} aprovar
o financiamento. E~levam essa bobagem para o tribunal da mídia. Afinal,
como diz o Procurador Geral da República (\versal{PGR}) Rodrigo Janot, pau que
bate em Lula, bate também em Pimentel. Ops, enganei"-me?

\section{Peça 4 -- o jogo da convicção política}

Não é apenas falta de raciocínio investigatório que explica esse
primarismo das denúncias por ``convicção''.

Quando se obriga a denúncia a ser embasada em provas, não têm discussão,
há pouca margem para o arbítrio. A~prova é a comprovação do crime. Tendo
a comprovação, denuncia"-se o suspeito. Não tendo, não se denuncia.

Quando a denúncia se baseia na ``convicção'' do acusador, ele passa a
ser o dono do julgamento. Utiliza a ``convicção'' como quiser, para
denunciar inimigos e para absolver os aliados. Monta"-se a salada de
indícios para dar impressão de volume de provas e dispara"-se o pastel de
vento para a mídia.

Não se venha com a história de que o sistema judicial é tão perfeito que
os erros do \versal{MPF} serão corrigidos pelo Judiciário. Está"-se falando em
episódios com efeitos políticos óbvios e imediatos, com tratamento
escandaloso por uma mídia que, assim como o \versal{PGR}, também tem lado.

Comparem os argumentos da Lava Jato contra Lula com os argumentos
invocados por Rodrigo Janot para não denunciar Aécio Neves na primeira
leva de denúncias da Lava Jato que chegou ao \versal{STF}.

O delator"-mor, Alberto Yousseff, contou que o deputado José Janene ---
para quem ele trabalhava --- declarou que recebia \versal{US}\$ 100 mil por mês
de Furnas e Aécio outros \versal{US}\$ 100 mil. Disse o nome da empresa por onde
transitava o dinheiro, a Bauruense. Declarou o destino final, Andrea
Neves.

Havia todos os ingredientes: o suspeito de ser subornador, o subornado,
o objeto do suborno e, ainda, o nome da lavanderia e do destinatário
final, uma pessoa física, não um partido. E~tudo devidamente informado
por um dos operadores centrais do esquema.

O que disse Janot:

\begin{itemize}
\itemsep1pt\parskip0pt\parsep0pt
\item
  ·~~~~~~~\emph{Prefacialmente (}\versal{NR}: primeiramente em
  jurisdiquês\emph{), há se ver que os fatos referidos são totalmente
  dissociados da investigação central em voga, relacionada à apuração
  dos fatos que ensejaram notadamente desvios de recursos da Petrobras.
  A referência que se fez ao Senador \versal{AÉCIO} \versal{NEVES} diz
  com~\textbf{supostos~}fatos no âmbito da administração de \versal{FURNAS}.}
\item
  Equivale a um delegado que vai investigar receptação de bens roubados,
  chega na casa do receptador e encontra provas de diversos outros
  assaltos, praticados por outros ladrões, mas deixa de lado porque seu
  trabalho é investigar apenas um determinado roubo. Um trabalho sério
  no mínimo abriria um inquérito à parte para investigar a denúncia.
\item
  ·~~~~~~~\emph{O~caso em tela tem uma caraterística fundamental que
  merece o~\textbf{devido e prudente sopesamento no presente momento}. É
  que as afirmativas de Alberto Youssef são muito vagas e, sobretudo,
  assentadas em circunstancias de ter ouvido os supostos fatos por
  intermédio de terceiros (um deles, inclusive, já falecido: José
  Janene).}
\item
  A denúncia sobre o suposto tríplex se baseou no depoimento de um
  porteiro que viu Lula visitando o apartamento ao lado de Léo Pinheiro.
  A de Yousseff partiu de um dos operadores centrais dos esquemas de
  corrupção das empreiteiras, principal operador do deputado que dividia
  as propinas com Aécio e veio acompanhada das informações fundamentais
  para uma investigação rápida. Janot tinha todos os dados, mas não
  tinha a convicção.
\item
  ·~~~~~~~\emph{Outro detalhe relevante: a referência de que existia uma
  suposta ``divisão'' na diretoria de Furnas entre o \versal{PP} e o \versal{PSDB} -- o
  que poderia ensejar a suposição de uma ilegítima repartição de valores
  entre as duas agremiações -- não conta com nenhuma indicação, na
  presente investigação, de outro elemento que a corrobore"}.
\item
  Há um inquérito correndo em Minas Gerais, conduzido pelo próprio
  Ministério Público Federal, com enorme profusão de dados sobre a
  corrupção em Furnas. Janot sequer levantou as informações já apuradas
  pelo inquérito. Baseou"-se no fato de que não havia nenhuma indicação
  ``na presente investigação''. Ou seja, mentiu sem faltar com a
  verdade.
\item
  ·~~~~~~~\emph{No entender do Procurador"-Geral da Republica,
  \redondo{[…]} não há sustentação mínima para requerimento
  de~\textbf{formal investigação.}}
\end{itemize}

Janot tem alguma dúvida de que o modelo aplicado pelo governo Lula
também o foi em Minas de Aécio Neves, em São Paulo de José Serra e
Geraldo Alckmin, em, Furnas, no \versal{DER}? Evidente que não. Ao contrário dos
jovens curitibanos, Janot goza da proximidade do poder e é ``malaco''
(designação das pessoas sabidas).

Daqui a dez anos, quando Janot decidir denunciar Aécio, qual o efeito
prático da medida? A manipulação consiste justamente em administrar o
prazo, é óbvio.

\section{Peça 5 -- a manipulação das delações}

Vamos sair um pouco da Lava Jato e analisar o que ocorreu com o Maestro
John Neschling, reputado internacionalmente.

Ele identificou irregularidades na fundação que administra o Teatro
Municipal. Comunicou ao prefeito Fernando Haddad. Foi acionada a
corregedoria do município que descobriu desvios pesados.

Foram detidos dois suspeitos, José Luiz Herencia, diretor da fundação
\versal{IBGC} (Instituto Brasileiro de Gestão Cultural) e William Nacked,
acusados de desvios de R\$ 15 milhões.

Não havia nenhuma dúvida de que eram os autores do golpe. Tanto que o
Ministério Público Estadual aceitou uma delação premiada. O~que fazem os
dois? Denunciam supostas irregularidades de Neschling e do porta"-voz do
Prefeito, Nunzio Briguglio.

Aí o \versal{MPE} começa a montar suas peças de convicção, a partir das delações
do cabeça da organização criminosa, Herencia:

\begin{enumerate}
\itemsep1pt\parskip0pt\parsep0pt
\item
  1~~~~~~ O maestro trabalha com alguns produtores na Europa, que têm
  autorização para agenciar apresentações suas.
\item
  2~~~~~~ São ~poucos os produtores mundiais que agenciam os grandes
  espetáculos.
\item
  3~~~~~~ O Municipal contratou uma infinidade de espetáculos, alguns
  dos quais de um desses produtores.
\end{enumerate}

Como há interesses políticos em jogo, imediatamente dois picaretas
confessos se tornam os acusadores, mediante meras ilações sem poder de
prova, tendo direito a eventual redução da pena. Seus nomes desaparecem
do noticiário e o foco passa a ser Neschling. E~o prefeito Fernando
Haddad, em um momento lamentável de fraqueza, demite Neschling com
receio de que sua candidatura seja prejudicada pela dobradinha
\versal{MPE}"-imprensa.

Esse tipo de episódio foi utilizado abundantemente pela Lava Jato e pelo
\versal{PGR}. Hoje em dia, restam poucas dúvidas de que o objetivo maior das
delações não é punir os corruptos, mas torna"-las instrumento de jogo
político.

Pergunto: é para reforçar esse jogo que se pretende aprovar as tais Dez
Medidas Contra a Corrupção?
