\chapterspecial{01/\allowbreak{}11/\allowbreak{}2016 Xadrez de um Supremo que se apequenou—\versal{II}}{}{}
 

Um livro pequeno, do jurista italiano Luigi Ferrajoli explica bem porque
os Ministros do \versal{STF} (Supremo Tribunal Federal) abdicaram de sua posição
de defensores da Constituição e o Brasil caminha a passos largos para um
modelo político próximo ao do fascismo.

O livro é ``Poderes Selvagens --- a crise da democracia italiana''.
Analisa a Itália pós Mãos Limpas e sob Berlusconi. A~descrição do
processo italiano, sob Berlusconi, é em tudo semelhante ao brasileiro,
sob Michel Temer.

O livro está esgotado nas principais livrarias. Alguns Ministros do
Supremo escanearam para oferecer o \versal{PDF} a um público restrito.

Não é pouca coisa.

Ferrajoli é um dos ideólogos na formação de toda uma geração de juristas
garantistas brasileiros -- os que consideram que a garantia prevista na
Constituição deve prevalecer sobre as leis ordinárias. É~uma linha à
qual se filiavam, em tempos idos, de Carmen Lúcia a Luís Roberto
Barroso, de Ricardo Lewandowski e Luiz Edson Fachin.~

Mais do que descrever a Itália de Berlusconi, o livro é quase um réquiem
do que está se tornando o Estado de Direito por aqui. Embora não se
refira ao Brasil, realça de maneira ampla as contradições do Supremo e o
processo de abandono de seu papel constitucional para se curvar ao
clamor das turbas que invadiu o país através dos grupos de mídia.

\section{Peça 1 --- Como o fascismo de baseou nas maiorias}

A parte central do livro é explicar o contraste entre o poder político e
o poder constitucional, e de como as Constituições foram sendo
reelaboradas para permitir a consolidação democrática -- isto é, a
expressão de todos os setores, das maiorias às minorias \mbox{---,} definindo
freios às maiorias políticas ocasionais, que poderiam interromper as
garantias civis.

Para os que saúdam o golpe brasileiro como uma demonstração de
vitalidade democrática, um trecho de Ferrajoli:

\emph{``O fascismo afundou a democracia e as liberdades fundamentais sem
um formal golpe de Estado: tratou"-se, de fato, de um golpe na
substância, mas não em suas formas, pois as leis fascistas, que fizeram
em pedaços o Estado de direito e a representação parlamentar, eram
consideradas formalmente válidas, pois votadas pela maioria dos
deputados segundo os cânones da democracia política ou formal''}

\section{Peça 2 --- As Constituições como defesa da democracia}

Foi só após a catástrofe do fascismo que houve uma completa revisão dos
instrumentos democráticos. E~o caminho encontrado foi a Constituição
detalhada, ficando acima da legislação ordinária, como forma de garantir
a separação dos poderes, a paz, a igualdade e a garantia dos direitos
fundamentais, impedindo que maiorias ocasionais colocassem em risco as
liberdades democráticas.

\begin{quote}
 \emph{``Foi propriamente por causa dessas trágicas experiências
(nazismo, integralismo) que se produziu na Europa, logo após a Segunda
Guerra Mundial, uma mudança de paradigma tanto do direito quanto da
democracia, por intermédio da constitucionalização daquele e deste. Essa
mudança consistiu na sujeição da inteira produção do direito, incluída a
legislação, a normas constitucionais rigidamente sobrepostas a todos os
poderes normativos''.}
\end{quote}

Houve uma mudança na relação da política e do direito. Diz ele:

\begin{quote}
\emph{``O direito não ficou mais subordinado à política como instrumento
desta, mas é a política que se torna instrumento de atuação do direito,
submetidos a vínculos por ela impostos por princípios constitucionais:
vínculos negativos como são aqueles gerados pelos direitos de liberdade
que não podem ser violados; e vínculos positivos, como são aqueles
gerados pelos direitos sociais que devem ser satisfeitos''.}
\end{quote}

O papel da Constituição, portanto, é garantir os direitos fundamentais.
Justamente por isso ela está acima das leis --- que necessitam se
subordinar aos seus princípios. O~país não fica mais à mercê das
maiorias ocasionais que, com seu poder de voto, podem desconstruir todos
os avanços democráticos.

Ferrajoli separa a dimensão constitucional da dimensão política, criando
duas esferas de decisão.

Há a chamada esfera do~\textbf{indecidível}, daquilo que não pode ser
objeto de deliberação porque incluído nos direitos fundamentais; e a
esfera~\textbf{daquilo que não pode não ser decidido}, que deve ser
objeto de deliberação a partir dos direitos sociais previstos na
Constituição.

Se o Congresso não aprova leis que deveriam dar substância aos direitos
previstos na Constituição, caberá ao Judiciário -- através do Supremo --
atuar, preenchendo os vazios.

\begin{quote}
\emph{``As democracias passam a se caracterizar não apenas pela sua
dimensão política ou formal, mas também pela dimensão substancial,
relativa aos conteúdos das decisões''.}
\end{quote}

É aqui -- e só aqui -- que se admite o ativismo do Supremo. É~para dar
consistência legal aos direitos previstos na Constituição: jamais para
revoga"-los.

Compare"-se esse princípio com a decisão do Supremo de não analisar a
substância do impeachment, mas só a forma, para se conferir como os
Ministros abdicaram de serem os defensores da Constituição. Não apenas
isso, mas permitindo que as leis se sobreponham aos direitos (caso do
direito de greve).

\section{Peça 3 --- Onde se justifica a judicialização da política}

Sob o comando da Constituição, articulam"-se as diversas formas de
democracia:~ a democracia política, assegurada pelas garantias dos
direitos políticos; a democracia civil, assegurada pelas garantias dos
direitos civis; a democracia liberal, assegurada pelas garantias dos
direitos de liberdade e a democracia social, assegurada pelas garantias
dos direitos sociais.

\begin{quote}
\emph{``A violação das garantias primárias positivas --- mais graves
porque estruturais consistindo na falta de instituição, por exemplo, da
escola pública ou da assistência à saúde gratuita como garantias dos
respectivos direitos sociais --- dá lugar a lacunas, isto é, à ausência
indevida de leis de atuação, igualmente em contraste com a Constituição
e, contudo, não reparável pela via judiciária, mas só pela via
legislativa''.}
\end{quote}

O ativismo do Supremo só se justifica, portanto, na defesa dos direitos
fundamentais previstos na Constituição. Daí o erro de imaginar que o
ativismo em defesa dos direitos fundamentais abriria espaço para o
ativismo político contra determinações da Constituição.

A inação do Supremo no episódio do golpe foi uma deformação do seu
papel.

\section{Peça 4 -- a desconstitucionalização e a volta da barbárie}

Depois que se abriram as comportas da desconstitucionalização, a Itália
mergulhou em uma orgia política sustentada pela mídia de Berlusconi e
pela maioria política.

Ferrajoli descreve leis que deram anistia ao primeiro ministro; leis que
negaram aos imigrantes os direitos elementares à saúde, à moradia e à
reunião familiar; as medidas demagógicas relativas à segurança, que
militarizaram o território, legitimaram as rondas, previram o registro
dos sem"-tetos e agravaram as penas para a pequena criminalidade de rua;
a redução das garantias jurisdicionais dos direitos dos trabalhadores;
controle político da informação e da mídia, sobretudo o da televisão.

Completa ele:

\begin{quote}
\emph{``E ainda os cortes na despesa pública relativamente à saúde e
educação, a agressão aos sindicatos, precarização do trabalho e,
portanto, das condições de vida de milhões de pessoas. Enfim, o projeto
de desconstituição, manifestou"-se nas propostas de lei destinadas a
reduzir a liberdade de imprensa em matéria de interceptações e de
direito de greve''.}
\end{quote}

Alguma semelhança com o Supremo e com o Brasil de Temer?

\section{Peça 5 -- a desconstrução da política}

Nos antecedentes, os mesmos fatores brasileiros.

A desmoralização da política pelos conflitos de interresse, as formas de
corrupção, dos tráficos de influência da política com o mundo das
finanças, os lobbies corporativos e --- sobretudo --- com a grande mídia
são fenômenos endêmicos em todos os ordenamentos democráticos, nos quais
se torna cada vez mais estreita a relação entre dinheiro, informação e
política: dinheiro para fazer política e informação; informação para
fazer dinheiro e política; política para fazer dinheiro e informação.

Os partidos se transformaram em entidades feudais, com leis entregando
aos seus chefes a indicação de candidatos. Se tornaram instituições
paraestatais, gerenciando informalmente a distribuição e o exercício das
funções públicas.

Outro fator de crise foi a dissolução da representação com o fim da
separação entre partidos e instituições e do papel daqueles como
instrumento de mediação representativas das instituições com a
sociedade.~ Os partidos se tornaram ``oligarquias custosas estavelmente
colocadas nas instituições representativas e expostas ao máximo à
corrupção''.

Esse processo desaguou na operação Mãos Limpas.~

\section{Peça 6 -- a liberdade de expressão como direito do jornalista}

A parte mais substanciosa do livro é sobre o papel da mídia, a liberdade
de imprensa e o direito à informação.

Diz ele:

\begin{quote}
\emph{``Não são mais a informação e a opinião pública que controlam o
poder político, mas é o poder político, e ao mesmo tempo econômico, que
controla a informação e a formação da opinião pública''.}
\end{quote}

Nesse ponto, Ferrajoli define uma diferença fundamental entre liberdade
de imprensa e liberdade de informação.

\begin{quote}
\emph{``Completamente ignorada e removida, inclusive pelo pensamento
liberal, da subordinação da liberdade de informação à propriedade dos
meios de comunicação. Mesmo os pronunciamentos mais avançados da Corte
constitucional e as posições mais críticas do atual aparato informativo
restringem"-se a reivindicar limites mais rígidos à concentração dos
meios de comunicação, em favor da garantia do pluralismo e da
concorrência. Mas a questão é muito mais radical: a liberdade de
imprensa e de informação é uma variável dependendo do mercado, ou é um
direito fundamental constitucionalmente estabelecido?''}
\end{quote}

Ferrajoli enxerga dois direitos radicalmente distintos.

Um, é o direito fundamental de todos à informação correta. O~outro é um
direito patrimonial pertencente somente aos proprietários dos meios de
comunicação.

\begin{quote}
\emph{``O que é um poder patrimonial, o poder empreendedor da
propriedade, vem a se sobrepor a um direito de liberdade de nível
constitucional, a liberdade de imprensa e de informação, diz ele, e,
portanto, a englobá"-lo e a eliminá"-lo''.}
\end{quote}

A Constituição italiana garante apenas o primeiro, mas não o segundo. O~artigo 21 diz que ``todos têm o direito de manifestar o livre
pensamento''. Diz Fajoli que se refere ``evidentemente'' ao pensamento
dos jornalistas, não dos proprietários. Por isso mesmo, liberdade de
expressão é um direito para garantir a independência das redações em
relação aos proprietários, diz ele.

Mas a relação entre os dois direitos se inverteu: os direitos de
liberdades, antes de operarem como limites ao poder, são por este
limitados.
