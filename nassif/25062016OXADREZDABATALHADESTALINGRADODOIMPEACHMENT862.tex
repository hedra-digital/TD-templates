\chapterspecial{25/\allowbreak{}06/\allowbreak{}2016 O xadrez da batalha de Stalingrado do impeachment}{}{}
 

Os alemães montaram uma blitzkrieg contra Stalingrado. Precisavam vencer
rapidamente, caso contrário o inverno rigoroso jogaria contra a
ocupação. Houve uma resistência heróica que segurou as tropas alemãs,
expondo"-as ao inverno russo. O~fator tempo decidiu a batalha.

É um quadro muito similar ao brasileiro.

O inverno rigoroso é o aprofundamento da crise, podendo chegar a um
ponto crítico em meados do próximo semestre. Os grandes grupos
econômicos apoiarão qualquer acordo que coloque fim à guerra política.
Poderá ser saída com Michel Temer, novas eleições ou mesmo saída com
Dilma.

Instalada a Comissão do Impeachment, no início a saída de Dilma
afigura"-se a solução mais rápida. Serão 15 sessões, 48 horas de
Congresso e o desfecho da votação na Câmara no final de abril.

Nesse período, o consórcio Lava Jato"-mídia jogará com tudo.

Por outro lado, o governo Dilma contará agora com Lula na articulação
política. Serão necessários 171 votos para matar o impeachment.
Passando, ainda haverá disputa no Senado.

No Congresso há dois grupos definidos, contra e a favor do governo. E~um
meio campo indefinido. Com Lula entrando no jogo, aumentam as
possibilidades de ampliar o número de parlamentares contrários ao
impeachment.

Se o governo conseguir os 171 votos, mata o impeachment. Aí, como em
Stalingrado, o inverno passa a jogar contra as tropas nazistas. Vencido
o desafio do impeachment, haverá condições de um novo pacto político
estabilizando a crise política, para começar a atacar a crise econômica.

\section{A grande ópera da Lava Jato}

Desde o ano passado, a Lava Jato segue um roteiro profissional, de casar
seu tempo com o tempo político.

O governo Dilma entrou 2016 mais animado.

De 16 a 20 de dezembro do ano passado, uma sucessão de fatos abriu algum
espaço para respirar. O~\versal{STF} derrotou o surpreendente voto do Ministro
Luiz Fachin sobre o rito do impeachment. Eduardo Cunha foi
responsabilizado no Conselho de Ética da Câmara e o \versal{PGR} entrou com uma
ação contra ele. Finalmente, houve a queda do Ministro Joaquim Levy e a
chegada de Ricardo Berzoini e Jacques Wagner, abrindo espaço para uma
agenda de recomposição política.

Dilma se abriu finalmente para a sociedade civil, em uma reunião exitosa
do \versal{CDES} (Conselho de Desenvolvimento Econômico e Social) com
pronunciamentos importantes de Roberto Setúbal, presidente do
Itaú"-Unibanco, e Luiz Trabuco, presidente do Bradesco. Os movimentos de
rua pareciam esgotados.

Pensava"-se que a Lava Jato tivesse esbanjado todo seu estoque de fatos e
factoides e o governo ganhasse algum fôlego para enfrentar a crise.

Mal abriu o ano, antes do Congresso começar a atuar, a Lava Jato apertou
o passo visando recriar o clima de catarse que sensibilizasse novamente
as ruas.

A corrida contra o tempo fez com que rapidamente se despisse do manto da
isenção. Começou com Sérgio Moro oferecendo seus delatores para instruir
as ações no \versal{TSE} (Tribunal Superior Eleitoral) contra Dilma. Depois a
ofensiva sobre o tríplex, rapidamente voltando"-se para o sítio em
Atibaia quando apareceu a offshore que poderia ter ligações com a Globo.
Alimentou por semanas o noticiário com pedalinhos, barquinhos de
alumínio, estátuas de Cristo Redentor.

Finalmente, entrou no jogo pesado da condução coercitiva de Lula,
culminando com a divulgação de todos os grampos coletados de pessoas no
entorno de Lula. E~chegou ao ápice com a divulgação de um grampo ilegal
na própria presidente da República.

\section{Um rescaldo do nosso xadrez}

No dia 13 de março tracei o cenário possível no dia da grande
manifestação pró"-impeachment, ``O xadrez da política no dia D''
(\url{migre.me/\allowbreak{}thPrk)}.

Nele, juntávamos as seguintes peças e hipóteses:

1.~~~~ Os três grupos protagonistas da crise eram os Lulistas, os
Parlamentaristas (Renan à frente) e o Alto Comando (a Procuradoria Geral
da República).

2.~~~~ Só há duas saídas negociadas possíveis: o semiparlamentarismo de
direito com Dilma ou semiparlamentarismo de fato com Lula assumindo a
articulação do lado do governo.

3.~~~~ Há duas forças conflitantes: forças moderadoras percebendo o
risco de uma guerra selvagem, entre Lulistas e Parlamentaristas; e o
Alto Comando apostando tudo no confronto.

4.~~~~ Por enquanto, o cenário mais provável é o do pacto \versal{PMDB}"-\versal{PSDB}
visando apoiar ao impeachment.

De lá para cá ocorreram novos lances, uma montanha russa fantástica, na
qual os dois lados jogam suas peças no tabuleiro visando controlar o
estado de ânimo das suas respectivas tropas.

\section{A guerra de nervos}

Ponto importante é a guerra de informações, com um componente
psicológico dos mais relevantes. Daqui até a votação do impeachment
haverá uma sucessão de fatos, de lado a lado, visando derrubar o ânimo
dos adversários e animar as próprias hostes.

Não se trata de uma mera briga de torcidas, mas da criação de um clima
psicológico que influenciará os parlamentares na hora de se colocarem
ante o impeachment e o próprio ânimo do \versal{STF} para atuar como moderador e
legalista.

É um caso clássico em que o clima do ``já ganhou'' pode influir na
vitória. Portanto, os que sugerem asilo político para Lula ou entram em
pânico, sugere"-se escalda"-pés, chá de limão e remédios naturebas contra
a ansiedade.

Esse ping pong começou lá atrás e ganhou ímpeto a partir do dia 13. É~uma verdadeira montanha russa de alternativas políticas, de ataques e
contra"-ataques.

Lance 0~-- Oposição

A super"-manifestação do dia 13, pró"-impeachment,

Lance 1~-- Governo

Lula aceitando o Ministério e reanimando as forças anti"-impeachment, um
reforço considerável, a última chance do governo Dilma.

Lance 2~-- Oposição

Com autorização da \versal{PGR}, a Lava Jato torna públicos todos os grampos,
conseguindo abafar as repercussões positivas da decisão de Lula, mas
avançando perigosamente nos limites de atuação, incorrendo em suspeita
de crime, tanto os policiais federais (por terem aceito um grampo
efetuado fora do prazo legal), quanto o juiz Sérgio Moro (que admitiu a
divulgação do grampo).

Lance 3~-- Governo.

Assim que a poeira assentou, os abusos da Lava Jato provocaram uma série
de manifestações indignadas, inclusive de jornais. E~uma série de
manifestações públicas de advogados, intelectuais, estudantes e
militantes. A~presunção de crime abriu espaço, de um lado, para
enquadramento da Polícia Federal -- o que foi feito rapidamente pelo
novo Ministro da Justiça, subprocurador Eugênio Aragão, em seu primeiro
pronunciamento -- e para enquadramento da própria Lava Jato.

Segundo o jurista Luiz Flávio Gomes, houve crime no vazamento

\emph{\redondo{[…]} As críticas duras também dizem respeito a ter
divulgado tudo, sem ``selecionar'' o que era pertinente para a
investigação (conversas que não têm nada a ver com a investigação não
podem ser publicadas -- é crime essa divulgação);}

\emph{\redondo{[…]} Por força do direito vigente não pode ser quebrado
o sigilo telefônico de advogado, enquanto advogado (havendo suspeita
contra ele, sim, pode haver interceptação);}

\emph{\redondo{[…]} Ponto que será discutido é o seguinte: na hora da
interceptação que captou a fala da Dilma (13:32h) a autorização do Moro
já não existia; nesse caso a prova pode ser considerada ilegal pelo \versal{STF}
(por ter sido colhida no ``diley'');}

\emph{\redondo{[…]} Moro não apontou em sua decisão os artigos legais e
constitucionais do seu ato de divulgação de ``tudo'' (há déficit de
fundamentação); invocar o interesse público não vale quando o conteúdo,
por lei, não pode ser divulgado (somente o \versal{STF} poderia ter divulgado,
por razões de segurança nacional, diz Dilma).}

Ouvidos pela Folha em Curitiba, investigadores informaram ter obtido
autorização do \versal{PGR}.

Lance 4~-- Oposição

Da Europa, Janot confirmou a autorização, embora ressalvando não saber
sobre a última gravação. Mas sabia obviamente que a divulgação do
conteúdo das demais visaria meramente espalhar intrigas e fortalecer o
clima favorável ao impeachment.

Para reforçar a estratégia, o \versal{PGR} e o decano do \versal{STF}, Ministro Celso de
Mello, inverteram o crime: não mais a divulgação de um grampo na própria
presidente da República, mas as frases proferidas por Lula em conversas
informais que, em um país sob o comando das leis, jamais poderiam ter
sido divulgadas.

Lance 4~-- Governo.

A super"-manifestação do dia 19, contra o impeachment repõe a bola com o
governo, aumentando substancialmente o cacife político de Lula.

Lance 5~- Oposição

A decisão do Ministro Gilmar Mendes, assim que confirmado o sucesso das
manifestações, de anular a posse de Lula e devolver o inquérito para
Curitiba.

Nos próximos dias o governo terá que agir rapidamente para reverter o
voto de Gilmar.

\section{Próximos lances}

Daqui até o fim de abril a disputa será atrás dos votos do Parlamento.

A divulgação dos grampos pela Lava Jato teve dois efeitos:

1.~~~~ Criou álibis para Gilmar Mendes atuar no \versal{STF} e criar amarras
jurídicas para a movimentação de Lula.

2.~~~~ Intimidou os críticos e o grupo de Lula. Todos passaram a fugir
dos telefones -- em um momento que exige muito contato e muita conversa.

Por outro lado, é possível que o enquadramento dos policiais federais
pelo novo Ministro da Justiça Eugênio Aragão reduza os vazamentos da
operação.

Mas os dois campos de disputa serão efetivamente o \versal{STF} e o Congresso.
Tem jogo ainda pela frente.
