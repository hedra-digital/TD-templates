\chapterspecial{10/\allowbreak{}09/\allowbreak{}2016 Xadrez da volta da comunidade de informações}{}{}
 

Não é surpresa que as duas notícias mais relevantes sobre as
manifestações anti"-Temer tenham sido de veículos fora
do~\emph{mainstream}.

No dia 8 de setembro, o blog Ponte divulgou a presença de um agente
infiltrado nas redes de namoro Tinder e Facebook, com participação ativa
na prisão dos 26 manifestantes (\url{ponte.org/\allowbreak{}?p=17404})
mantidos em isolamento.

Sob o codinome de Balta, o agente infiltrou"-se também em um grupo de
WhatsApp. Nas mensagens do grupo, a Ponte não encontrou nada que
sugerisse atos de violência. Coube a Balta sugerir o encontro no Centro
Cultural São Paulo. Lá, deu"-se a batida da \versal{PM} e a prisão. Enquanto os
detidos eram levados para uma viatura, Beta era isolado do grupo.

Entrevistado, o coronel Dimitrios Fyskatoris, comandante do Comando de
Policiamento da Capital (\versal{CPC}), afirmou que a detenção do grupo ocorreu
por acaso.

No dia 9, El País identificou o agente infiltrado
(\url{https:/\allowbreak{}/\allowbreak{}is.gd/\allowbreak{}e7\versal{ZZ}oW}). Tratava"-se do capitão do Exército Willian
Pina Botelho, da comunidade de inteligência.

\section{Peça 1 -- as duas pinças do aparelho repressor}

Nesses tempos de informação online, confirma"-se o que o~Xadrez~já vinha
prevendo pelo menos desde~7 de maio.

No ``Xadrez do governo Temer e o fator militar''
(\url{https:/\allowbreak{}/\allowbreak{}is.gd/\allowbreak{}4nh\versal{JKW}}) mostramos as duas pinças que estavam sendo
montadas para devolver protagonismo aos militares: a recriação do \versal{GSI},
entregue ao general Sérgio Etchegoyen, e a entrega do Ministério da
Justiça a Alexandre Moraes.

\emph{``Do lado de Temer, uma das maneiras de desviar o foco das
críticas seria a criação do inimigo interno. Nos últimos anos, uma certa
imprensa de ultradireita recriou versões tupininquins da Guerra Fria,
com pirações de toda ordem -- como a invasão das \versal{FARC}s, a aliança com as
forças bolivarianas. A~tentativa de recriação da legitimidade política
das Forças Armadas passa por aí''.}

\emph{\redondo{[…]} A maneira dos militares voltarem para a política
seria através da recriação de uma estrutura militar de controle no
governo federal, mas diferente do extinto \versal{GSI} (Gabinete de Segurança
Institucional da Presidência da República) e mais próximo do \versal{SNI}
(Serviço Nacional de Informações) e da segurança presidencial".}

Os movimentos do jogo estão aí:

Movimento 1~- a recriação da \versal{GSI}, colocando embaixo dela a Abin (Agência
Brasileira de Inteligência e o Sisbin (Sistema Brasileiro de
Inteligência)~ com as comunidades de inteligência, com vistas a envolver
as Forças Armadas na repressão interna. A~aprovação da Lei Nacional de
Inteligência (\url{https:/\allowbreak{}/\allowbreak{}is.gd/\allowbreak{}n5mj1P}) consolidou o novo modelo.

Movimento 2~- o carnaval em torno dos supostos terroristas islâmicos, a
fim de recriar o mito do inimigo externo e justificar a entrada das
Forças Armadas no jogo.

Movimento 3~- a coordenação da repressão às manifestações pelas Polícias
Militares. Há sinais de participação direta do Ministro Alexandre de
Moraes nesse jogo
(\url{https:/\allowbreak{}/\allowbreak{}is.gd/\allowbreak{}pIispG}~e~\url{https:/\allowbreak{}/\allowbreak{}is.gd/\allowbreak{}\versal{C}4gsGj}). Desde maio,
aliás, Moraes já tentava recriar o clima pós-64 equiparando os protestos
pró"-Dilma a atos de guerrilha (\url{https:/\allowbreak{}/\allowbreak{}is.gd/\allowbreak{}9lgLvC)}.

\section{Peça 2 -- a generalização do conceito de inimigo}

A repressão de domingo traz um dado assustador.

Os ``inimigos'' identificados eram jovens, adolescentes em sua maioria,
a maioria deles ``socorristas'' -- preparadas para socorrer
manifestantes vítimas de gás lacrimogênio. Todos foram levados para a
delegacia e mantidos incomunicáveis por várias horas e ameaçados de
enquadramento como ``organização criminosa''. Para sua sorte, o caso
caiu com um juiz legalista e corajoso que mandou solt;a"-los.

A pesada e injustificada repressão aos manifestantes mereceu cobertura
das reportagens online -- devido à óbvia dificuldade em enquadrar os
repórteres na hora. No dia seguinte, meras reportagens burocráticas dos
jornalões. Deu"-se mais espaço a \versal{UM} fotógrafo da \versal{UOL} que teria sido
agredido por \versal{UM} manifestante. E~retomou"-se o caso do cinegrafista morto
por um rojão de um black bloc em manifestação do Rio de anos atrás.
Escondeu"-se a informação de que, no domingo, coube a seguranças da \versal{CUT}
-- e não à \versal{PM} -- conter os black blocs nas manifestações.

Pior: a primeira prova de envolvimento das Forças Armadas na repressão
-- o caso do capitão -- foi ignorado pelos jornais. Chegou"-se ao cúmulo
de tirar lições políticas positivas do ``look'' da primeira dama, e
ignorar"-se a participação de um capitão da ativa na repressão.

\section{Peça 3 -- a radicalização da repressão}

O vácuo político abriu espaço para as corporações de Estado.

Os setores técnicos das Forças Armadas e os combatentes na ponta não são
chegados à política. O~envolvimento das Forças Armadas está sendo
comandado pelo General Sérgio Etchegoyen e por setores da burocracia
brasiliense. As comunidades de inteligência voltam ao primeiro plano. Em
vez de se dedicarem à guerra cibernética, à segurança das pesquisas
brasileiras ou do pré"-sal, sua missão será recriar a figura do inimigo.

No início, foi o carnaval em torno dos tais terroristas do Estado
islâmico. Depois, dos movimentos populares. Agora, chegaram aos filhos
da classe média paulistana. Entre a rapaziada que se organizou para as
manifestações poderiam estar filhos de procuradores ``coxinhas'', de
juízes, de jornalistas, armados de gaze, vinagre e outros elementos para
socorrer pessoas sufocadas por gás ou pimenta.

Atravessaram muito rapidamente as fronteiras sociais. Imagine"-se o que
não sucederá na base da pirâmide. Na divisão de trabalhos, caberá à
dobradinha \versal{GSI}"-\versal{PM} a repressão a qualquer forma de protesto de rua. E~aos
procuradores da Lava Jato~ a repressão continuada aos quadros políticos.

\section{Peça 4 -- as forças anti"-repressão}

Aí se entra em um quadro complicado: institucionalmente, não há forças
capazes de se contrapor a essa escalada do arbítrio.

\textbf{Mídia}~-- não se espere defesa de movimentos populares ou de
pobres da periferia: não são seu público. Os manifestantes de domingo,
sim. A~indignação com a violência da \versal{PM} transbordou dos canais
político"-partidários. Era a oportunidade de expressarem a indignação,
até como maneira de legitimar suas campanhas pró"-golpe. O~fato de não se
manifestarem, ou se manifestarem timidamente, sobre os abusos da \versal{PM}
paulista, e de esconderem a participação do capitão do Exército na
criminalização de um grupo de jovens, é significativo: jogaram a toalha.
De um lado, devido à pesada crise financeira por que passam. De outro,
pelo uso continuado do cachimbo da direita, que os deixou de boca torta.
Estão sendo tratados com cenoura e chicote. Na semana passada, a Folha
recebeu em almoço a nova diretoria da Caixa Econômica Federal. Ontem,
comandantes do Exército.

\textbf{Judiciário}~-- dependerá cada vez mais da ação individual de
juízes. À~esta altura, nada se pode esperar do \versal{STF} (Supremo Tribunal
Federal) ou dos tribunais superiores. Os bravos Ministros do Supremo
abriram mão de analisar a ilegalidade do golpe, foram cedendo a cada
passo, imaginando ser o último, pretendendo afastar de si o cálice da
resistência à atual onda de arbítrio, julgando que, alcançados os
objetivos de derrubar o governo, a repressão cessaria com em um passe de
mágica. A~cada dia será mais caro o preço do silêncio. É~pungente
assistir o Ministro Luís Roberto Barroso derramando diariamente
declarações repletas de boas intenções por todos os lados e todos os
temas… menos os essenciais. Quando Dilma caiu, suspirou fundo e
pensou: poderei voltar à minha doutrinação civilizatória, o discurso do
politicamente correto para os salões, sem ser incomodado. Prezado
Ministro, lamento decepcioná"-lo, mas voces chocaram o ovo da serpente

\textbf{Ministério Público Federal}~-- No \versal{MPF} há um conjunto de bravos
procuradores, alguns dos quais correram às delegacias no domingo, em
defesa dos meninos presos, ao contrário dos procuradores estaduais. A~própria \versal{PFDC} (Procuradoria Federal dos Direitos do Cidadão)
manifestou"-se bravamente em defesa do direito de manifestação. Mas no
topo do \versal{MPF}, o Procurador Geral da República (\versal{PGR}) Rodrigo Janot já
escolheu lado. E, no presidencialismo do \versal{MPF}, é o \versal{PGR} quem dá as cartas.
Nada se espere dele, a não ser o recrudescimento das ações contra
políticos adversários, cumprindo o dignificante trabalho de fuzilar
prisioneiros no campo de batalha e blindar os aliados.

A receita está dada: Etchegoyen montando a espionagem; Moraes
articulando a repressão às manifestações; Janot coordenando a repressão
política. E~o sentimento democrático brotando em sucessivas
manifestações, tanto de manifestantes nas ruas, como de personalidades
em abaixo"-assinados.
