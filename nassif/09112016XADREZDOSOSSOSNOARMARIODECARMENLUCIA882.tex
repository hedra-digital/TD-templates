\chapterspecial{09/\allowbreak{}11/\allowbreak{}2016 Xadrez dos ossos no armário de Carmen Lúcia}{}{}
 

\section{Peça 1 -- sobre as vulnerabilidades dos magistrados}

No processo que culminou no afastamento da presidente Dilma Rousseff,
foram divulgadas pressões e chantagens sobre Ministros da Corte. Em dois
casos, pelo menos, eram denúncias vazias, uma delas de atitude até
questionável do ponto de visita ético, mas longe de qualquer tipificação
de ilícito.

Mesmo assim, os Ministros cederam, mostrando como são vulneráveis a
ataques contra a honra, mesmo de embasamento precário.

Quando o Ministro acumula uma série de decisões polêmicas,
definitivamente ele se torna refém de pessoas -- ou instituições -- em
condições de escandalizar os fatos.

O caso Ayres Brito é exemplar. Seu genro foi flagrado oferecendo
serviços do sogro para o ex"-governador do Distrito Federal Joaquim
Roriz. O~episódio ganhou as manchetes. Ayres Brito era cúmplice do genro
ou apenas vítima? Pouco importa. No momento seguinte, ele se curvou
amplamente às demandas da imprensa, liquidou com a regulamentação do
direito de resposta e se tornou o campeão nacional na defesa dos
interesses dos grupos de mídia.

A denúncia desapareceu do noticiário.

Dessa mesma vulnerabilidade padece a nova presidente do \versal{STF} (Supremo
Tribunal Federal) Carmen Lúcia. É~o que poderia explicar a mudança
ocorrida em suas posições ao longo dos tempos e a maneira como, ao
estilo Ayres Britto, passou a se compor com todas as demandas do
noticiário e a jogar sempre para as manchetes?

Não se sabe. Mas a cidadania -- tão prezada por Carmen Lúcia -- ficaria
mais tranquila se a Ministra apresentasse explicações claras para pelo
menos três episódios polêmicos em sua carreira no Supremo.

\section{Peça 2 -- sobre o lado público de Carmen Lúcia}

A presidente do \versal{STF} (Supremo Tribunal Federal) se manifestou a favor da
transparência, da honestidade, da probidade e de outras virtudes
cívicas.

No dia 22 de maio de 2012, um dia após o \versal{STF} decidir divulgar os
salários de todos os ministros e funcionários, Carmen Lúcia foi a
primeira a divulgar o seu. Embora todos passassem a divulgar seus
contracheques, Carmen Lúcia saiu na frente, chamou a atenção da mídia e
se tornou a musa da transparência (\url{https:/\allowbreak{}/\allowbreak{}goo.gl/\allowbreak{}oy9vgx}).

Na votação sobre julgamentos em segunda instância, votou a favor da tese
de que o réu deve cumprir pena após a sentença ser confirmada pelo
tribunal regional, demonstrando assim sua posição intransigente contra
os malfeitos e a favor da punição célere.

Ao assumir a presidência do Conselho Nacional de Justiça (\versal{CNJ})~ anunciou
seu projeto com duas palavrinhas mágicas: transparência e eficiência
(\url{https:/\allowbreak{}/\allowbreak{}goo.gl/\allowbreak{}\versal{E}8\versal{TAL}w}).

Fez mais, no \versal{STF} e no \versal{CNJ} prometeu cumprir o regimento e definir prazos
para devolução de processos com pedidos de vista
(\url{https:/\allowbreak{}/\allowbreak{}goo.gl/\allowbreak{}7\versal{AX}hB9}).

No entanto, mesmo nas matérias laudatórias há alguns senões que
preocupam. Mesmo sendo ``simplinha'', guiando seu próprio carro, sendo
contra os salamaleques da corte, vozes em off falam de sua incoerência,
de seu fascínio pelos holofotes, de decisões suspeitas.

\section{Peça 3 -- quando o adiamento é manobra}

Antes, algumas explicações sobre maneiras sorrateiras de juízes atuarem.

O Supremo é uma corte colegiada. Um de seus pressupostos --- e da
própria democracia --- é não permitir decisões monocráticas, aquelas em
que um juiz define sozinho o destino do processo.

Uma das maneiras de burlar é o pedido de vista. A~corte já havia
definido maioria contra o financiamento privado de campanha. Bastou um
pedido de vista de Gilmar Mendes para segurar a decisão por meses e
meses. Ou, agora, a decisão da corte, por maioria absoluta, de não
permitir que políticos envolvidos em processos assumam a presidência da
República. Todos os votos foram a favor quando entrou em cena o Ministro
Dias Toffoli, com pedido de vista. Vai devolver o processo sabe"-se lá
quando.

Há outras formas de ludibriar os colegas e as partes.

Outra delas, similar ao pedido de vista, ~é o uso da gaveta.

Anos atrás, o Ministro Ayres Britto iria apresentar seu relatório sobre
o mensalão do \versal{PSDB}. Tudo devidamente pautado, houve o intervalo da
sessão para o café. Na volta, Ayres simplesmente não apresentou o
relatório, nem lhe foi cobrado. Segurou por anos e anos o processo. Foi
uma decisão neutra do ponto de vista processual? Evidentemente, não. Foi
um esquecimento que beneficiou profundamente a parte condenada.

O que leva um Ministro a engavetar um processo?

Hipótese 1~- não quer entrar em dividida, devido à complexidade do
problema.

Nesse caso, incorre no crime de prevaricação, que é quando o funcionário
público retarda ou deixa de praticar ato de ofício, visando satisfazer
seu interesse pessoal. No caso, o interesse de não se expor com as
partes.

Hipótese 2 -~excesso de trabalho na corte.

O meritíssimo terá que comprovar que a ação adiada era menos relevante
do que aquelas que foram julgadas no período.

Hipótese 3~- ~engaveta por razões pessoais, de benefícios ou de
influência externa.

Trata"-se de um caso de corrupção passiva, crime que só pode ser
praticado por funcionário público, de acordo com o Código Penal.

\section{Peça 4 --- a gaveta de Carmen Lúcia}

Ontem relatei o caso da \versal{ADIN} sobre pipeline, o pedido da Procuradoria
Geral da República para que caíssem as patentes pipeline --- um tipo
jurídico criado para reconhecer patentes não englobadas no acordo de
patentes.

Havia dois grupos interessados. Em um deles, a poderosa Interfarma,
representante dos laboratórios internacionais, que se apresentou
como~\emph{amicus curiae}~. Na outra, laboratórios nacionais, \versal{PGR},
Procons.

Carmen Lúcia tinha acesso, assim, a todos os argumentos de todas as
partes interessadas, em tema da mais alta relevância, porque envolvendo
custo de medicamentos para milhões de brasileiros. Mas decidiu não
decidir. Trancou a \versal{ADIN} em sua gaveta, de onde não mais saiu.

Ao não decidir, ela decidiu em favor dos laboratórios multinacionais,
contra o \versal{SUS} --- que paga muito mais caro pelos remédios patenteados ---
e laboratórios nacionais, prontos a produzir genéricos e contra milhões
de consumidores que deixaram de ter acesso à redução de preços desses
medicamentos.

Não cabe o argumento da insignificância. Pelo contrário, pelos valores
envolvidos e pelos beneficiários potenciais -- a população carente do
país -- deveria ser prioridade absoluta.

Seria devido à complexidade do tema? Um Ministro que teme a complexidade
dos temas a serem julgados não está à altura do Supremo.

Seria por alguma razão extra"-processo?

Seja qual for o motivo, a Ministra Carmen Lúcia certamente esclarecerá a
razão de jamais ter levado o caso a julgamento, com a mesma presteza com
que divulgou seu contracheque.

\section{Peça 5 --- os julgamentos de crimes de corrupção}

O voto de Carmen Lúcia a favor do cumprimento da pena após condenação em
segunda instância, mostra a preocupação com os processos -- e,
principalmente, com as punições. A ínclita Carmen Lúcia quer celeridade
na punição.

Segundo o sóbrio site Jota, há dois episódios que não batem com as
preocupações externadas pela Ministra (\url{migre.me/\allowbreak{}vrVd7}):

\subsection{Caso 1 -- Natan Donadon}

Em outubro de 2010 Carmen Lúcia levou a plenário o julgamento da Ação
Penal 396, que julgava o deputado Natan Donadon, condenado a 13 anos de
reclusão. Era o primeiro caso de prisão de parlamentar sob a
Constituição de 1988. A~data da condenação foi 28 de outubro de 2010.

O acórdão só foi publicado seis meses depois. Nesse ínterim, Donadon
continuou solto. A~defesa de Donadon embargou a condenação. No entanto,
Carmen Lúcia só liberou o recurso para julgamento ~um ano depois da
condenação. Logo em seguida, percebeu um erro qualquer e pediu a
retirada do processo.

Segundo o Jota, ``publicamente, a ministra deixou consignado que
solicitou à Presidência do Supremo que desse prioridade ao julgamento.
No entanto, em reservado a realidade era outra''. Foram tantos os
pedidos de Carmen Lúcia, em particular, ~para que o projeto fosse
retirado de pauta, diz o Jota, que o então presidente do Supremo,
Ministro Ayres Brito, exigiu que fizesse o pedido publicamente, diante
da \versal{TV} Justiça. Foi obrigada a faze"-lo.

Só em junho de 2013 -- três anos após a condenação -- o processo chegou
ao fim e a pena foi finalmente cumprida.

O que levou Carmen Lúcia a protelar de tal maneira o cumprimento da
sentença de Donadon, a atuar nos bastidores para tirar o julgamento de
pauta, enquanto fingia, de público, pretender prioridade para o
julgamento?

Em qual das três hipóteses se enquadra esse movimento?

\subsection{Caso 2 -- Ivo Cassol}

Ex"-governador de Rondônia, em agosto de 2013 foi condenado a 4 anos de
detenção. O~processo foi relatado também por Carmen Lúcia. Cassol
continuou solto devido à demora no julgamento do recurso. O~acórdão só
foi publicado 9 meses após o julgamento. E~era um mero acórdão.

A defesa entrou com embargos, que foram rejeitados. Opôs novos embargos
em dezembro de 2014. Segundo o Jota, ``em várias oportunidades, no ano
de 2015, jornalistas perguntaram à ministra quando ela levaria a
julgamento os últimos recursos do Senador.''. E~Carmen Lúcia deixava a
pergunta sem resposta e o processo sem julgamento..

Somente em abril os recursos foram levados a plenário. Porém o
julgamento foi interrompido por um pedido de vistas de… Dias
Toffoli. Condenado por fraudar licitações, Cassol permanece livre e no
exercício do mandato, diz o Jota.

Qual o seu compromisso efetivo com a transparência?

\section{Peça 6 --- a frasista Carmen Lúcia}

Fica evidente que Carmen Lúcia está muito mais próxima do, digamos,
estilo de um Gilmar Mendes e Dias Toffoli, do que de um Teori Zavascki
ou Celso de Mello.

Durante o julgamento do impeachment de Dilma Rousseff, pelo menos uma
vez Carmen Lúcia se viu ante uma representação contra o Ministro Gilmar
Mendes. E~refugou, demonstrando receio de entrar em área de alto risco.

Há um profundo desvio no sistema de Justiça, de só convalidar denúncias
provenientes da velha mídia. Cria"-se um desequilíbrio monumental,
beneficiando grupos de interesse em temas políticas, empresariais ou
penais. Caso tivesse manifestado o menor pendor de votar contra o
impeachment de Dilma, esse conjunto de fatos, divulgado no Jornal
Nacional, teria sido mais que suficiente para detonar Carmen Lúcia.

Como descrever suas contradições? Talvez recorrendo às suas próprias
palavras.

\begin{quote}
\emph{``Na história recente da nossa pátria, houve um momento em que a
maioria de nós, brasileiros, acreditou no mote segundo o qual uma
esperança tinha vencido o medo. Depois \redondo{[…]} descobrimos que o
cinismo tinha vencido aquela esperança. Agora parece se constatar que o
escárnio venceu o cinismo.''}
\end{quote}

 

Mas, nesses tempos de redes sociais, de construção superficial de
imagens, o que importa não é conferir a história, os trabalhos e
decisões dos personagens. Bastam gestos populistas, de quem ambiciona o
``curtir'' do Facebook e a blindagem da mídia.

Independentemente de jogos de palavras, fique registrado: Carmen Lúcia é
uma ministra com muitas vulnerabilidades. E~como não se pode duvidar nem
da mulher de César, muito menos de uma presidente do Supremo, deve
explicações por esses e outros episódios controversos em sua carreira de
Ministra do Supremo.

Caso contrário, todas as sentenças que proferir, decisões que tomar,
lançarão dúvidas sobre suas reais motivações.

E o escárnio, definitivamente, vencerá o cinismo.
