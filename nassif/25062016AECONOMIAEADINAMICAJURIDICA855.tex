\chapterspecial{25/\allowbreak{}06/\allowbreak{}2016 A economia e a dinâmica jurídica}{}{}
 

Os economistas discutem se a economia está ou não sob a dominância
fiscal -- termo para descrever situações em que o quadro fiscal assume
tal risco que a economia não responde mais aos impulsos monetários.

Objetivamente, a economia brasileira está sob a dominância jurídica.

Durante boas décadas o país foi governado pela teocracia dos
economistas. A~inflação elevada, mais a globalização dos mercados,
conferiu"-lhes poder político acima dos partidos.

A perda do discurso econômico e a erosão da credibilidade presidencial
provocaram um vácuo na opinião pública, uma perda de rumo, do discurso e
das propostas aglutinadoras, enfim, de um projeto de país. E~a
dominância econômica cedeu lugar á dominância jurídica.

\asterisc{}

É nesse vácuo que a besta foi liberada --- o sentimento de manada
irracional, alimentado pelo ódio e pela intolerância, que acomete países
que perderam o rumo.

Quem consegue cavalgar a besta, assume o controle do discurso público.
Torna"-se um deus ex"-machina.

A besta foi alimentada com um foco claro: a corrupção política.

De repente, a opinião pública perdida encontra um discurso unificador, a
enorme vendetta política contra a corrupção dos outros, como se todos os
problemas do país fossem resolvidos meramente com uma gigantesca caça às
bruxas.

\asterisc{}

A besta traz consigo a balbúrdia de informações. Os alertas sobre a
necessidade de prender os culpados sem desmontar a economia são
ignorados. Basta os novos oráculos brandirem frases de senso comum.
Tipo, basta limpar a corrupção para a economia ficar saudável. Ou, se
quebrar uma empresa, outra surgirá no lugar.

Na era das banalidades, dos factoides das redes sociais, e do
empobrecimento do discurso econômico, a descrença em relação às chamadas
``opiniões técnicas'' permite toda sorte de demagogia do senso comum.

\asterisc{}

Com a falta de ação do Executivo, a sombra da besta vai se impondo sobre
todos os setores. Há a banalização das prisões e o recuo das figuras
referenciais. Por receio de enfrentar a besta, Ministros do \versal{STF} (Supremo
Tribunal Federal), que se supunha guardiões impávidos dos direitos civis
-- abolem a apelação à Terceira Instância.

Para prender o vilão símbolo -- Luiz Estevão \mbox{---,} em vez de atuar sobre
outros fatores de protelação, liquidam com a possibilidade da terceira
instância, sujeitando outros tipos de réus às idiossincrasias e
influências políticas nos tribunais estaduais.

É só acompanhar o que ocorreu no Maranhão e no Amapá de José Sarney, com
políticos eleitos perdendo o mandato em cima de acusações risíveis:
compra de um voto por R\$ 15,00. Ou no Rio de Janeiro, com as ações da
Globo contra jornalistas críticos.

A caça às bruxas atual será refresco perto do que virá pela frente.

\asterisc{}

Nem se pense em uma dominância jurídica impessoal pairando acima das
paixões políticas. A~besta tem lado e se prevalece das imperfeições
jurídicas e políticas.

O modelo político universalizou as práticas ilegais. Todos os partidos
se valeram disso. O~jogo político consiste em investir contra um lado
apenas -- blindando e fortalecendo o outro. Ou então, valer"-se das
dificuldades processuais para proteger aliados. Jackson Lago foi deposto
acusado de ter comprado um eleitor com R\$ 20,00.

É nítida a aliança entre Ministério Público Federal e grupos de mídia. O~bater bumbo da mídia ajuda a superar obstáculos, quando os alvos são
adversários da mídia. Quando o suspeito é a própria mídia, obviamente
não há bumbo, e não há vontade política de avançar.

É o que explica o caso das informações sobre a Globo, enviadas pelo \versal{FBI},
estarem paradas há um ano nas mãos de uma juíza de primeira instância.

Culpa das imperfeições jurídicas, é claro.

\asterisc{}

Quando a besta sai às ruas, os valentes tremem, os crentes abjuram, os
referenciais se escondem.

A besta reescreve biografias, reputações, reavalia caráteres, pois é um
teste de estresse, no qual muitos grandes se apequenam, e alguns
pequenos se agigantam.

A besta passa imperial, despejando fogos pelas ventas e farejando de
longe o cheiro do medo. Os medrosos, ela espanta com seus uivos. Os que
resistem, ela ataca, rasga reputações, destrói histórias, promove
grandes orgias públicas expondo os recalcitrantes em praça pública.

É só conferir o que está acontecendo com advogados que ousam criticar a
operação. Ou o que irá acontecer com a esposa do publicitário João
Santana que, algemada, ousou dizer que não baixará a cabeça. Como não?
Será isolada, a prisão será prorrogada, mensagens pessoais serão
divulgadas, sua vida será devassada até que baixe a cabeça. Esse Deus
vingador não admite esses arroubos.

No Brasil, a besta intimidou até donos de grandes biografias, como o
Ministro Luís ~Roberto Barroso.

\asterisc{}

A pior parte da história é que a besta não resolve problemas econômicos.
E~não quer abrir mão do protagonismo político. Há uma crise
perigosíssima no horizonte. A~cada tentativa da presidente de avançar em
um acordo com o Congresso ou com a sociedade civil, a besta irrompe de
Curitiba e promove um novo festival de factoides, paralisando
completamente o discurso público.

\asterisc{}

Nesse clima irracional, há a desmoralização total dos partidos
políticos, do \versal{PT} ao \versal{PSDB}. Nas últimas pesquisas de opinião, o único
segmento que cresceu foi o dos anti"-petistas -- denominação para os que
querem a volta dos militares e acham que os terroristas árabes vão
invadir o Brasil. Hoje em dia, tem 18\% da opinião pública, mais do que
qualquer outro partido.

\asterisc{}

Não se sabe até onde irá essa loucura coletiva.

Resta o consolo de saber que, mesmo impotentes, ainda existem juízes que
colocam suas convicções acima do medo.

Salve Celso de Mello, Ricardo Lewandowski e Marco Aurélio de Mello.
