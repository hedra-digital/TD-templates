\chapterspecial{12/\allowbreak{}08/\allowbreak{}2016 Xadrez de como o \versal{PT} ajudou o \versal{MPF} se tornar partido político}{}{}
 

\section{1\textsuperscript{o}~Movimento -- o nascimento do ativismo
judicial}

A Constituinte de 1988 foi montada em cima de um retrovisor: o regime
militar que se encerrava.

A relatoria da Constituinte tinha subcomissões. A~do Ministério Público
e do Judiciário foi relatada por Plinio de Arruda Sampaio, do \versal{PT}.~ Do
lado do \versal{MP}, recebia assessoria de Álvaro Augusto Ribeiro Costa,
Sepúlveda Pertence, Eugenio Aragão, Wagner Gonçalves e Aristides
Junqueira, futuro Procurador Geral da República de Itamar Franco.

O Judiciário estava na defensiva. O~\versal{PT} questionou a presidência da
Constituinte por Moreira Alves, presidente do \versal{STF} (Supremo Tribunal
Federal) e foi apoiado por Ulisses Guimarães;

Mesmo assim, o Judiciário derrotou Plinio impedindo a criação do \versal{CNJ}
(Conselho Nacional de Justiça), derrubando a proposta de mandatos por
tempo limitado para Ministros do \versal{STF}, criando o \versal{STJ} (Superior Tribunal
de Justiça) e diminuindo para 11 o número de Ministros do Supremo.

Em seguida à promulgação, o Judiciário tomou decisão de recepcionar na
nova Constituição as demais normas do Poder Judiciário, como a Loman
(Lei Orgânica da Magistratura Nacional), que vinha da época de Geisel.

Ficaram vazios, de leis regulatórias que precisariam ser votadas, porque
nenhum partido tinha maioria parlamentar. Foi o caso dos capítulos da
Comunicação, Segurança Pública, Reforma Agrária. Para preencher o vácuo,
o Judiciário foi interpretando e cada vez mais ocupando o espaço das
decisões políticas.

No debate sobre a regulamentação do capítulo da Comunicação, o Supremo,
através de \versal{ADIN}s aprovou que concessões de \versal{TV} não envolviam sinais
digitais, só as \versal{TV}s abertas. Abriu"-se uma avenida pelo Ministro das
Comunicações Antônio Carlos Magalhães. Só no período Sarney o número de
novas concessões foi similar a tudo que foi concedido de 1946 a 1964.

E aí a esquerda começou a chocar o ovo da serpente.

Derrotada nas eleições de 1989, minoria no Congresso, perdia votação e
recorria ao Judiciário, através de \versal{ADIN}s (Ação Direta de
Inconstitucionalidade). E~passou a recorrer a denúncias sistemáticas ao
\versal{MPF} contra adversários políticos.

\section{2\textsuperscript{o}~Movimento: o pacto com as corporações
públicas}

A Constituição foi produto de vários relatórios e projetos.

Ao fim do muro de Berlim e da União Soviética, se somou a crise do \versal{ABC},
com desemprego trazido pela crise econômica. A~base social do \versal{PT}
deslocou"-se para o funcionalismo público. E~o partido começou a
construir pontes com a elite do funcionalismo, Ministério Público,
Tribunal de Contas, Polícia Federal, juízes de primeira instância,
conferindo"-lhes os três Ps: privilégios, prerrogativas e promoção, junto
com autonomia funcional e administrativa.

Era um relacionamento em clima de companheirismo total. Daí os relatos
de que, antes da Lava Jato, Sérgio Moro era eleitor do \versal{PT}, assim como
Rodrigo Janot.

O mesmo aconteceu com a Polícia Federal. Desaparelhada no período \versal{FHC},
quando entrou o governo Lula, a \versal{PF} foi buscar alianças com a esquerda
para se aparelhar. Quando Lula foi eleito, a \versal{PF} queria se
responsabilizar até pela segurança presidencial, ocupando o lugar das
Forças Armadas.

A partir dessa parceria, o \versal{PT} adotou várias posições pró"-\versal{MP} e
pró"-corporações públicas.

Ajudou a derrubar projeto do deputado Paulo Maluf, que penalizava
procuradores em ações consideradas ineptas pelo Judiciário -- projeto
similar ao do atual presidente do Senado Renan Calheiros, visando coibir
abusos de poder.

Na Lei Orgânica do \versal{MP}, os procuradores começaram a pressionar para
incluir a ideia da lista tríplice. Era um discurso de fácil assimilação
pela esquerda, pelo poder dado à corporação e por ser eleição direta. E~essa aliança foi fundamental para consolidar o poder do \versal{MPF} sobre os
demais \versal{MP}s.

O Ministério Público é composto pelo \versal{MPF} (Ministério Público Federal),
integrado pelos chamados procuradores da República, o da União (os \versal{MP}s
estaduais), o do Trabalho e o Militar. O~lobby do \versal{MP} da República,
associado ao \versal{PT}, possibilitou que a escolha do \versal{PGR} ocorresse apenas
dentro do \versal{MPF}, o menor de todos. Aliás, Lula vivia se vangloriando de
sempre indicar o primeiro da lista tríplice para \versal{PGR}.

Mais que isso, a Constituição conferiu uma série de competências
inéditas ao \versal{MP}, como o de representar, propor \versal{ADIN}s e ``exercer outras
funções que lhe forem conferidas por lei, desde que compatíveis com sua
finalidade, sendo vedada a representação judicial e a consultoria''.

O lobby do \versal{MPF} retirou do texto o trecho ``que lhe forem conferidas por
lei''. Com isso, o próprio Ministério Público passou a interpretar o que
seriam as outras funções.

Hoje em dia, o \versal{PGR} tem poder para definir tudo da carreira, arbitrar
valor das vantagens devidas aos membros do \versal{MP}, férias, gratificações.
Esse poder criou um vício eleitoral similar ao presidencialismo de
coalizão: quem está no poder negocia as vantagens e dificilmente perde
eleições.

\section{3\textsuperscript{o}~Movimento: dos direitos humanos ao padrão
\versal{OBAN}}

A ideia inicial na Constituição era a de que o \versal{MP} trataria
fundamentalmente das questões sociais. Gradativamente, no entanto, a
maior parte do \versal{MP} passou a privilegiar as operações que davam mídia.

Três ações históricas garantiram o prestigio do \versal{MP} na questão penal:

\begin{enumerate}
\itemsep1pt\parskip0pt\parsep0pt
\item
  Acre: o trabalho do procurador. Luiz Francisco conseguindo a
  condenação e prisão do parlamentar que matava a os adversários com
  motosserra.
\item
  Espírito Santo, contra o ex"-presidente da Assembleia Legislativa José
  Carlos Gratz.
\item
  No Rio de Janeiro com o procurador Antônio Carlos Biscaia e a juíza
  Denise Frossard contra os banqueiros do bicho.
\end{enumerate}

Dali em diante, ganhavam espaço as operações em que a imprensa batia
bumbo. Procuradores passaram a fazer media training e a buscar a
parceria com repórteres policiais nas suas operações, a se valer dos
vazamentos como armas contra a defesa e contra juízes garantistas. Tudo
com a contribuição do \versal{PT}, embarcando na onda das táticas moralistas para
provocar investigações contra inimigos do partido.

O Centro de Inteligência do Exército é o único que funciona bem, dentro
da máquina pública. Nem todos os coronéis do \versal{CIEX} passam no funil para
se tornarem generais.

Mesmo não tendo poder de investigar, o \versal{MPF} buscou 13 deles na área de
inteligência para assessorar nos grampos, no Guardião, na tecnologia de
infiltração e de interrogatório.

Aos poucos, a área penal do \versal{MPF} passou a incorporar todas as
características da \versal{OBAN}, a Operação Bandeirantes, que marcou o período
de maior repressão do governo militar. A~\versal{OBAN} foi responsável por um
conjunto de inovações, posteriormente adotada pelo \versal{MPF} e pela \versal{PF}.

A primeira delas foi a constituição de forças"-tarefas para operações
específicas. E~toda operação tem que ter uma narrativa, uma marca, uma
versão dos fatos que seja verossímil, embora não necessariamente
verdadeira, para orientar os trabalhos. Foi assim com a guerrilha, com
os dominicanos e no episódio Vladimir Herzog. Herzog morreu porque
submetido a torturas para que delatasse suposto conluio do governador
Paulo Egydio Martins com o \versal{PCB}.

A segunda, o princípio básico da guerra revolucionária -- bem detalhado
pelo jornalista Antônio Carlos Godoy no livro ``A Casa de Vovó'' \mbox{---,} de
que quem prende, quem investiga, quem denuncia e quem julga não podem
ter contradições entre si. Tem que haver consenso para impor a versão à
mídia e à Justiça. E~deve se valer da mídia para espalhar versões ou
declarações de arrependimento, dentro do conceito da guerra psicológica
adversa.

A Lava Jato recorreu a métodos similares, devidamente aplainados pelos
novos tempos: em vez de pau de arara, outras formas de tortura moral,
como as prisões preventivas sem prazo para acabar, isolados do mundo e
da família.

Não apenas isso.

Criou o conceito do inimigo interno, através da chamada teoria dos
fatos, colocando como narrativa central a existência de uma organização
criminosa, comandada pelo \versal{PT} e por Lula, composta de um núcleo
dirigente, um núcleo político e um núcleo de operadores.

Foi essa narrativa que permitiu focar as investigações em Lula e \versal{PT},
deixando de lado o \versal{PSDB} e outros partidos de fora da base.

Se a narrativa fosse de um conluio de empreiteiras atuando em todos os
níveis de poder, as investigações chegariam a Minas Gerais, São Paulo e
demais estados. Portanto, a seleção da narrativa não foi aleatória.

Na Força Tarefa há identidade absoluta entre Policiais Federais,
procuradores e juiz, atropelando um modelo clássico do liberalismo,
segundo o qual quem investiga não denuncia; quem denuncia não julga.
Cabe ao procurador fiscalizar o policial e o juiz impedir abusos de
ambos. No caso do juiz de instrução acusador, aceito por alguns países,
o julgador final precisa ser outro juiz, sem envolvimento direto com o
caso.

Prevaleceu o padrão \versal{OBAN}.

\section{4\textsuperscript{o}~Movimento: o pacto eleitoral para a \versal{PGR}}

Cada vez mais, o \versal{MPF} passou a atuar como partido político, a começar da
eleição para \versal{PGR}.

As eleições para a \versal{PGR} obedeceram às receitas padrão do presidencialismo
de coalizão. Cada candidato atua politicamente, aproximando"-se de
líderes do governo, de deputados, de senadores e cativando a base. Foi o
caso de Janot, levado por Aragão a visitar José Dirceu, mesmo após o
julgamento da \versal{AP} 470, ou oferecendo jantares a José Genoíno em sua casa,
comparecendo a jantares com políticos petistas.

Como em todo arranjo político, havia um pacto entre as lideranças do
\versal{MPF}, de ninguém se candidatar à reeleição.

Cláudio Fonteles permaneceu \versal{PGR} por um mandato. Passou o bastão a
Antônio Fernando de Souza. Este pegou a \versal{AP} 470 pela frente, e pressionou
Lula: se não fosse reconduzido poderia parecer que o governo pretendia
varrer o mensalão para baixo do tapete. Atropelou o acordo.

Na rodada seguinte, o favorito era Wagner Gonçalves, ex"-presidente da
\versal{ANPR} (Associação Nacional dos Procuradores da República), apoiado por
Eugênio Aragão. Foi vetado por Gilmar Mendes, na época presidindo o \versal{STF},
e inimigo declarado de Wagner e Eugênio. Foi uma das muitas vitórias que
Gilmar conseguiu contra Lula, meramente blefando: a outra foi o
afastamento de Paulo Lacerda, que liberou o jogo político"-partidário na
\versal{PF}.

Restou a Lula indicar Roberto Gurgel que imediatamente transformou a
denúncia de Antônio Fernando em representação no \versal{STF}. E~coube o papel de
verdugo ao ex"-procurador Joaquim Barbosa, do círculo mais próximo dos
procuradores progressistas.

Inquéritos volumosos têm muitas razões que os leitores desconhecem ou
não conseguem captar. Mas, ali, havia duas informações indesmentíveis:

\begin{enumerate}
\itemsep1pt\parskip0pt\parsep0pt
\item
  A \versal{AP} 470 foi montada totalmente em cima da suposição do desvio de R\$
  75 milhões da Visanet.
\item
  O desvio não ocorreu.
\end{enumerate}

O estratagema serviu para que o \versal{PGR} livrasse Daniel Dantas da operação,
mesmo dispondo de um laudo técnico da \versal{PF} mostrando o pagamento prometido
de R\$ 50 milhões a Marcos Valério sem contrapartida de serviços.

Ao lado o episódio da helicoca, trata"-se de um dos grandes mistérios
nesses tempos em que se supunha que nenhuma informação ficasse
escondida. Não se tratava de um ato unilateral de um \versal{PGR}, mas de um
artifício endossado por várias instâncias.

O \versal{MPF} passava a atuar como partido político.

\section{5\textsuperscript{o}~Movimento: o \versal{MPF} de Rodrigo Janot}

Rodrigo Janot sempre pertenceu ao chamado grupo progressista do
Ministério Público. Foi subprocurador de Cláudio Fonteles e candidato a
vice de Wagner Gonçalves.

Quando dirigia a Escola Nacional do Ministério Público, montou reuniões
periódicas para discutir temas nacionais, das quais eram participantes
ativos Eugênio, Wagner, o advogado ativista Luiz Moreira, Álvaro Ribeiro
da Costa, para o qual eram convidados dirigentes petistas de maior
preparo, como José Genoíno. Nesses encontros, discutia"-se de
nanotecnologia ao papel das Forças Armadas.

Em todo esse período, Janot preparou"-se para ser \versal{PGR}. Valeu"-se para tal
do estreito conhecimento que tinha da máquina do \versal{MPF}, como segundo de
Fonteles e presidente da \versal{ANPR}.

Nas eleições, atropelou duas procuradores símbolos do \versal{MPF}, Ela Wiecko e
Deborah Duprat -- a quem acusou de ser ligada a José Serra. Computados
os votos de todos os Ministérios Públicos, Deborah foi a mais votada.
Mas Janot foi o mais votado dentre os procuradores da República.

A esta altura, um novo fenômeno alterava a natureza do \versal{MPF}, com a
ampliação dos quadros e o advento da era dos concurseiros, jovens
preparados, com recursos para se dedicar por anos para se preparar para
os concursos, não necessariamente com vocação pública, mas encantados
pela possibilidade de altos salários iniciais e do poder de
``autoridade''.

É nesse momento que Janot passa por uma um processo de conversão ao
status quo. Percebendo os novos tempos, foca sua campanha eleitoral em
temas de gestão e de atendimento às demandas corporativas da classe. E~dando"-se conta dos impactos da \versal{AP} 470 sobre a classe e sobre a opinião
pública, preparou"-se para transformar a Lava Jato no passaporte final do
\versal{MPF} para o centro do poder.

Reconduzido ao cargo por Dilma, uma de suas primeiras atitudes foi abrir
uma ação contra ela, em um gesto considerado desleal por seus antigos
companheiros.

Respondeu a uma fala de Lula -- em uma escuta ilegalmente divulgada --
afirmando que devia tudo ao concurso público e não a ele, Lula. A~declaração era inoportuna, visto que respondendo a uma conversa informal
ilegalmente divulgada. Mas mostrava que Janot já vestira o avental
asséptico do concurso público para se alinhar com a nova clientela,
mesmo tendo pavimentado sua carreira na \versal{PGR} por confabulações políticas
de praxe.

Só após muita pressão dos amigos ousou avançar sobre Aécio Neves, o
filho dileto do status quo.

Teve papel central na deposição de uma presidente eleita e na condução
ao centro de poder de figuras do naipe de Michel Temer, Eliseu Padilha,
Geddel Vieira Lima e Moreira Franco. E~sua agenda continua em sintonia
com a agenda política, ampliando a ofensiva contra o antigo governo
sempre que se aproximam datas relevantes, como o da votação do
impeachment.

\section{6\textsuperscript{o}~Movimento --- Próximos passos}

Hoje em dia, em que pesem tantos bravos procuradores de direitos da
cidadania, o \versal{MPF} tornou"-se peça central no desmonte do estado de
bem"-estar social.

Desde o início da crise política, sabia"-se que não se tratava apenas da
disputa entre uma presidente atabalhoada e políticos barras"-pesadas, mas
de concepções de Estado.

Bem antes da votação do impeachment se sabia que o novo governo entraria
ungido pela promessa de limitar as despesas públicas, definindo limites
nominais para gastos voltados para os interesses difusos, saúde,
educação, sem definir limites para os gastos com juros. Para um leigo,
parece medida disciplinadora de gastos. Para quem é do ramo, significará
o desmonte do \versal{SUS} e do sistema educacional público.

Derrubada Dilma, a primeira atitude da \versal{ANPR} (Associação Nacional dos
Procuradores da República) foi procurar o interino, não para garantir a
manutenção dos direitos sociais, mas para assegurar vantagens
corporativas já prometidas.

Hoje em dia, o \versal{MPF} atua como verdadeiro partido político, com assessoria
de imprensa, estratégias de marketing, iniciativas parlamentares,
discursos políticos, parceria com a mídia e ações sincronizadas com
movimentos da política. Como tal, monta alianças, joga de olho na
opinião pública, sujeita"-se às pressões da mídia. Da mesma maneira que
um partido convencional.

Deborah, Ela, Eugênio, Fonteles, Álvaro Augusto, Eugênia, e outros
procuradores símbolos de um Ministério Público que não mais há. O~atual
\versal{MPF}, do dr. Janot, tornou"-se peça central do maior ataque aos direitos
sociais desde o regime militar. E~poderá se tornar o coveiro da
democracia.

No âmbito do governo Temer há um movimento nítido de devolver às Forças
Armadas o papel de gendarme. A~segurança nas Olimpíadas ficou sob as
ordens do Gabinete de Segurança Institucional (\versal{GSI}), do general Sérgio
Etchgoyen. Não apenas isso. Ele passa a comandar também o Sisbin
(Sistema Brasileiro de Inteligência), que tem sob seu guarda"-chuva as
áreas de fiscalização da Receita, Banco Central, Abin (Agência
Brasileira de Inteligência). Terá, ao alcance do seu computador, a ficha
de qualquer cidadão brasileiro com registro civil.

Conseguirá dispor de um poder de intimidação similar ao aparato do qual
se vale hoje em dia o \versal{PGR}.

No entanto, nos próximos meses, haverá um movimento similar ao que
marcou o fim da guerrilha.

Inicialmente, havia uma coesão dentre todos os aparelhos de repressão
contra o inimigo comum, a guerrilha. Vencida a guerra, observou"-se uma
luta intestina dentre eles. Prisioneiros de um aparelho eram ameaçados
de tortura se passassem informações quando interrogados por outros
aparelhos.

Consumada a vitória final sobre Lula e o \versal{PT}, provavelmente haverá
embates similares entre \versal{PF} e \versal{MPF}, entre \versal{GSI} e Forças Armadas.

O país de hoje não se assemelha ao do início da ditadura. Mas há no ar
as mesmas jogadas oportunistas em cima do vácuo político que marcou a
agonia do regime militar.

Por enquanto, a melhor tradução da \versal{PGR} é a imensa catedral brasiliense
onde está alojada a sua sede, redonda, permitindo a todos se verem
internamente. Mas de vidros indevassáveis, que permitem enxergar tudo o
que ocorre lá fora; mas impedem que se veja de fora o que acontece lá
dentro.
