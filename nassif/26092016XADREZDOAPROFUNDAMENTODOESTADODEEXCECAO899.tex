\chapterspecial{26/\allowbreak{}09/\allowbreak{}2016 Xadrez do aprofundamento do Estado de Exceção}{}{}
 

\section{Peça 1 -- o cenário provável}

Traçar cenários é tarefa complexa.

O ponto inicial é identificar a tendência da onda do momento e o que
poderá acontecer se não surgir nenhum elemento novo, anticíclico, capaz
de contê"-la. Em geral, esse tipo de cenário serve de alerta, ajudando a
estimular forças contracíclicas quando se quer prevenir desastres. Mesmo
assim, nações entram na onda fatal, no que cientistas sociais
denominaram de ``era da insensatez'' e vão para o buraco, sem que
nenhuma força contracíclica consiga segurar a queda.

Neste momento, há duas tendências se consolidando, uma de forma mais
evidente, outra de forma mais tênue.

Tendência 1~- ~o aprofundamento do estado de exceção.

Tendência 2~-- o início do processo de fritura do governo Michel Temer
pela aliança Globo"-Lava Jato"-\versal{PSDB}.

A eventual queda da camarilha dos 6, ao contrário das visões mais
otimistas, significará um aprofundamento da repressão.

Vamos por partes.

\section{Peça 2 -- o aprofundamento do Estado de Exceção}

Ao contrário dos golpes militares, nos golpes judiciais o estado de
exceção se impõe por etapas. Muda"-se o patamar da legalidade aos poucos,
através de episódios centrais, que muitas vezes passam ao largo da
opinião pública.

No caso brasileiro, as etapas do estado de exceção são nítidas:

\textbf{1\textsuperscript{o}~Passo}~-- a condução coercitiva de Lula e o
vazamento dos grampos entre ele e Dilma.

\textbf{2\textsuperscript{o}~Passo}~-- o fato do Ministro Teori
Zavascki, do \versal{STF} (Supremo Tribunal Federal), ter reconhecido a
ilegalidade dos vazamentos, mas não adotado nenhuma sanção contra o juiz
Sérgio Moro.

\textbf{3\textsuperscript{o}~Passo}~-- a profusão de prisões
preventivas, culminando com os argumentos invocados para a prisão do
ex"-Ministro Guido Mantega, sem que se vislumbre nenhuma força capaz de
impedir a escalada de arbítrio.

\textbf{4\textsuperscript{o}~Passo}~-- a decisão do \versal{TRF}4 de apoiar Moro,
consagrando o estado de exceção, sob a alegação de que a Lava Jato
enfrenta inimigos poderosos e, portanto, não pode ser tratada de maneira
convencional. Segundo o relator, desembargador federal Rômulo Pizollati,
``o Supremo Tribunal Federal perdoa esse tipo de desvio de conduta
quando é para um bem maior'' (\url{migre.me/\allowbreak{}v3Wqj)}. É~a primeira
vez que um tribunal, em país democrático, valida expressamente o Estado
de Exceção em tempos de paz, após o ``patriot act'' dos Estados Unidos
contra o terrorismo.

\textbf{5\textsuperscript{o}~Passo}~- Gradativamente as Forças Armadas
estão sendo direcionadas para o combate ao inimigo interno, o Ministério
da Justiça passou a articular a repressão nas Polícias Militares e
assumiu o comando da Polícia Federal. Ontem, em um palanque do \versal{PSDB}, o
Ministro Alexandre de Moraes anunciou novas operações da Lava Jato para
esta semana, ao lado de um acusado pela máfia das merendas.

\subsection{A defesa do Estado de Exceção pelo \versal{TRF}4}

Até agora, este foi o sinal mais grave do ingresso no Estado de Exceção.

No \versal{TRF}4, a consagração do estado de exceção foi apoiada por 13
desembargadores, como apenas um voto contrário.

Voz isolada, o desembargador Rogério Favreto, alertou em seu voto
(\url{migre.me/\allowbreak{}v4sbG}):

\begin{itemize}
\itemsep1pt\parskip0pt\parsep0pt
\item
  ·~~~~~~ O entendimento, amplamente consolidado, é de que o magistrado
  incorreu em transgressão à literalidade da lei, ao determinar o
  levantamento do sigilo de conversas captadas em interceptações
  telefônicas.
\item
  ·~~~~~~ Também descumpriu normativa do Conselho Nacional de Justiça
  (\versal{CNJ}), ao fornecer para a mídia elementos contidos em processos ou
  inquéritos sigilosos.
\item
  ·~~~~~~ Diante de tal arcabouço, não vislumbro hipótese de
  relativização do sigilo, direito fundamental do cidadão inscrito na
  Carta Federal.
\end{itemize}

Recentemente, o jurista Pedro Estevam Serrano lançou o livro
``Autoritarismo e golpe na América Latina''
(\url{migre.me/\allowbreak{}v4rzJ)}~analisando o fascismo judicial através do
uso do chamado Estado de Exceção.

A ideia de exceção é que o direito é uma boa forma de administrar as
sociedades em tempos de paz. Mas quando a sociedade é ameaçada por
inimigos ou grandes desastres, podem se aceitar Estados de Exceção para
garantir o país. É~uma lógica que se aplica às guerras externas. Em
muitos momentos, houve a tendência de trazer a lógica da guerra para a
lógica interna. Se fulano é inimigo, não deve ter os mesmos direitos dos
demais cidadãos. É~a convalidação do chamado direito penal do inimigo.

Segundo Serrano, as concessões jurídicas ao Estado de Exceção sempre
foram identificadas em sentenças pontuais. Em nenhum país democrático
houve uma explicitação tão nítida quanto na sentença dos 13
desembargadores do \versal{TRF}4 sobre Sérgio Moro.

Na sentença do \versal{TRF}4 invoca"-se um voto o ex"-Ministro Eros Grau e trechos
de Giorgio Agamben, jurista italiano que estudou o Estado de Exceção. A~sentença do \versal{TRF}4 parte de uma leitura incorreta de Agambem, que
analisava o Estado de Exceção para criticá"-lo, não para endossá"-lo, como
fizeram os desembargadores. Mesmo porque, segundo Serrano, exceção é
fascismo. O~estado de exceção foi o argumento utilizado por Hitler para
instaurar a ditadura nazista.

Algumas das ideias de Agambem (\url{migre.me/\allowbreak{}v4s14)}

\begin{itemize}
\itemsep1pt\parskip0pt\parsep0pt
\item
  ·~~~~~ As democracias são muito preocupadas: de que outra forma se
  poderia explicar que elas têm uma política de segurança duas vezes
  pior do que o fascismo italiano teve?~Aos olhos do poder, cada cidadão
  é um terrorista em potencial.
\item
  ·~~~~~ A crise está continuamente em curso, uma vez que, assim como
  outros mecanismos de exceção, permite que as autoridades imponham
  medidas que nunca seriam capazes de fazer funcionar em um período
  normal.
\end{itemize}

O cenário atual indica um gradativo endurecimento político. Atualmente
está em curso uma guerra de extermínio com a aplicação do direito penal
do inimigo contra o \versal{PT}, visando não apenas as eleições de 2018, que só
ocorrerão se a oposição não mostrar nenhum sinal de vida. Nessa
escalada, em breve se chegará a críticos da Lava Jato, independentemente
de cor política.

\subsection{A prisão de Guido Mantega}

Autorizada pelo juiz Sérgio Moro, a prisão do ex"-Ministro Guido Mantega
é o caso mais clamoroso, até agora, dessa manipulação dos indícios na
investigação penal.

Ela se baseou em dois elementos frágeis, pequenos.

\begin{enumerate}
\itemsep1pt\parskip0pt\parsep0pt
\item
  1. Afirmação do empresário Eike Baptista de que Mantega solicitou
  apoio para o \versal{PT} cobrir dívidas de campanha.
\item
  2. O~fato de, um mês depois, ter havido o recebimento de um pagamento
  pela Mendes Jr de obra na Petrobras da qual uma empresa de Eike
  participavam, como parceiro menor do consórcio.
\end{enumerate}

E só. Bastou para mandar um ex"-Ministro para a cadeia, de onde foi solto
algumas horas depois, por ``razões humanitárias'', e também -- segundo
alegação do juiz -- porque já tinha sido feita a coleta de equipamentos
e documentos, e portanto não haveria riscos de atrapalhar as provas.
Ora, se não havia riscos, não havia motivos para a detenção,
independentemente dos fatores humanitários.

Na delação espontânea de Eike, ele afirma taxativamente que o apoio não
estava vinculado a nenhuma obra da Petrobras; que Mantega limitou"-se a
ser o veículo do pedido de apoio. De seu lado, Mantega negou qualquer
pedido. Um caso de palavra contra palavra.

Não adiantou. Para reforçar a suspeita, o juiz Sérgio Moro ainda jogou
datas, dentro da estratégia banalizada de manipulação de indícios.

Segundo o relato que me foi enviado por um observador:

\begin{itemize}
\itemsep1pt\parskip0pt\parsep0pt
\item
  a)~~ Em sua decisão, Sérgio Moro confunde"-se sobre a data da suposta
  reunião entre o ex"-Ministro da Fazenda e o empresário Eike Batista. De
  acordo com a denúncia do \versal{MPF} e com o depoimento do empresário a
  reunião teria ocorrido em 1/\allowbreak{}11. Moro, no entanto, ora menciona que a
  reunião teria ocorrido de fato em 1/\allowbreak{}11 (p\,12), data em que o
  contrato entre a Petrobras e o consórcio Integra formado pelas
  empresas Mendes Jr e \versal{OXZ} teria sido firmado, ora no dia 1/\allowbreak{}12/\allowbreak{}2012 (pgs
  13 e 14), uma das datas em que haveria telefonemas da assessoria
  próxima do Ministro da Fazenda para as agências de João Santana.
\item
  b)~ Não há, contudo, na tabela anexada pelo \versal{MPF} em sua denúncia
  (páginas 38 e 39), nenhuma menção a ligações efetuadas do Ministério
  da Fazenda para as agências de João Santana no dia 1/\allowbreak{}12/\allowbreak{}2012. As
  ligações mais próximas são entre 15/\allowbreak{}10/\allowbreak{}2012 (15 dias antes da data
  mais provável da suposta reunião) e 01/\allowbreak{}03/\allowbreak{}2013 (meses após a data da
  reunião).
\end{itemize}

\section{Cena 3 -- os atores e a repressão}

Nesse exato momento, o comportamento de alguns atores centrais não
permite visões otimistas em relação à democracia.

\subsection{Supremo Tribunal Federal}

 Indagado sobre o fato de Mantega ter sido detido em um hospital, aonde
estava acompanhando uma cirurgia no cérebro de sua esposa, o decano do
\versal{STF} (Supremo Tribunal Federal) Celso de Mello, excelso garantista,
poderia ter opinado sobre diversos ângulos:

\begin{itemize}
\itemsep1pt\parskip0pt\parsep0pt
\item
  ·~~~~~~ O uso abusivo da prisão preventiva.
\item
  ·~~~~~~ A insensibilidade da Polícia Federal de tê"-lo detido no
  hospital (o juiz não poderia saber).
\item
  ·~~~~~~ O show midiático expondo os réus antes do julgamento.
\end{itemize}

Preferiu, acacianamente, dizer que o Código Penal autoriza a detenção de
pessoas em hospitais. E~nada mais não disse nem lhe foi perguntado.

Dos demais Ministros, Marco Aurélio Mello se manifesta de vez em quando,
assim como Gilmar Mendes. Mas nenhum ousa qualquer ação para deter a
escalada do regime de exceção.

É uma desmoralização tão grande para as instituições brasileiras que,
recentemente, um veículo norte"-americano incluiu Sérgio Moro entre os
dez líderes mais poderosos do planeta. Ora, o poder de um juiz de
primeira instância é diretamente proporcional à fraqueza dos tribunais
superiores. O~prêmio a Moro é a confirmação do fracasso do sistema
judiciário brasileiro, impotente para impedir a escalada de
arbitrariedades do juiz.

\subsection{Ministério Público Federal}

Em debate na Folha sobre os abusos da Lava Jato, o procurador Jefferson
Dias apelou para o teorema da isonomia no desrespeito aos direitos
básicos, uma falácia indesculpável em operadores do direito, aliás ponto
central na psicologia de massa do fascismo: a ideia de que direitos são
privilégios dos mais favorecidos. Acerca da superexposição dos réus, seu
argumento foi: ``Sempre houve isso com os menos favorecidos. Mas aí,
quando acontece com pessoas mais ilustres, eles reclamam.''

Em recente sessão da Comissão de Segurança da Câmara, o Procurador Geral
da República (\versal{PGR}) Rodrigo Janot enviou como representante o procurador
Rafael Perissé. A~audiência foi convocada para desagravar militares
envolvidos em grupos de extermínio, e sob investigação.

Em nome do \versal{MPF}, Perissé declarou que o aumento da letalidade, em
operações da polícia, era resultado do trabalho deletério de \versal{ONG}s e de
procuradores criticando a polícia. Como resultado, a polícia ficou mais
enfraquecida e os bandidos mais atrevidos. O~aumento da letalidade,
portanto, foi decorrência do maior atrevimento dos bandidos.

 

Nas redes sociais, o argumento central de procuradores é o uso
recorrente da visão do inimigo externo. Qualquer crítica à Lava Jato é
enquadrada como defesa dos corruptores contra os mocinhos.

\subsection{Mídia}

A crise da mídia provocou dois efeitos: submissão dos três grupos
jornalísticos (Folha, Estadão e Abril) ao governo; e dos jornalistas em
relação aos jornais e ao governo.

Do lado dos jornais, há um silêncio obsequioso em relação aos abusos,
uma tentativa de criar um clima positivo, todos aguardando a bolsa mídia
em gestação.

Nas redações, colunistas sob ameaça de desemprego, ou esperando surfar
na onda do governo, ingressaram na era da infâmia. O~momento atual,
aliás, tem proporcionado um amplo desnudamento de caráter, especialmente
quando se tem em conta que o golpe já venceu a guerra e está em
andamento uma caça aos ``inimigos''. À~esta altura, colocar lenha na
fogueira inquisitorial significa expor setores cada vez mais amplos à
caça aos inimigos, que poderão ser seus colegas de ofício.

Cronistas com projetos na rádio \versal{MEC}, colunistas beneficiários da Bolsa
Ditadura, blogueiros de grandes grupos, colunista econômica, estão
atuando como soldados incumbidos de executar o inimigo ferido no campo
de batalha. Deveriam pensar melhor na sua biografia em um momento em que
o país está prestes a cruzar de forma decisiva o Rubicão da democracia.
A~ficha caiu até para Fernando Henrique Cardoso.

\section{Cena 4 -- a reorganização das alianças}

O golpe foi desfechado por uma coalizão composta principalmente ~pelos
seguintes elementos:

\begin{enumerate}
\itemsep1pt\parskip0pt\parsep0pt
\item
  1.~~~~ A camarilha dos 6 (Temer, Moreira Franco, Padilha, Geddel, Jucá
  e Cunha) cujo maior estrategista era Eduardo Cunha.
\item
  2.~~~~ A mídia, liderada pela Globo.
\item
  3.~~~~ O \versal{PSDB} como agente secundário, tentando ser o legítimo
  representante do mercado.
\item
  4.~~~~ A Procuradoria Geral da República, como agente operador da
  repressão.
\end{enumerate}

As afinidades maiores são entre os três últimos grupos. O~grupo 1, no
entanto, enfeixou o poder, mas não cumpre com alguns requisitos básicos:

\textbf{Falta de legitimidade}~-- o Estadão tratando Eliseu Padilha como
grande agente público é jornalismo de alto risco
(\url{migre.me/\allowbreak{}v4s\versal{TL})}. Há limites para o jornalismo chapa
branca. E~os jornais sabem que o custo para tentar legitimar o governo
Temer é excessivamente alto. Como montar um regime fundado na
anticorrupção tendo na cabeça o mais suspeito grupo político pós
redemocratização?

\textbf{Incapacidade de conduzir reformas}~-- chegando ao poder, Temer
passou a distribuir benesses entre os vitoriosos. Ampliou desmedidamente
os gastos públicos em troca da promessa de reformar radicalmente a
Previdência e impor teto aos gastos públicos. A~cada dia que passa, fica
mais distante da promessa de promover os cortes radicais e entregar o
produto prometido. Aliás, em sua primeira atitude legítima em muito
tempo, Rodrigo Janot ingressou com uma ação no \versal{STF} visando impedir os
danos às políticas sociais com os tais limites orçamentários
(\url{migre.me/\allowbreak{}v4s\versal{WN})}.

\textbf{Aumento da impopularidade}~-- o ``Fora Temer'' tornou"-se um
bordão irresistível. Não haverá parceria capaz de conferir a Temer
senioridade no cargo.

\subsection{A bolsa mídia}

A reação do governo Temer está no preparo da bolsa mídia. É~o que tem
segurado as críticas dos jornais.

Algumas deduções sobre a tal bolsa mídia:

\begin{itemize}
\itemsep1pt\parskip0pt\parsep0pt
\item
  ·~~~~~~ Três dos quatro grupos mais influentes -- Folha, Estadão e
  Abril -- padecem de problemas de geração de caixa. Portanto, apenas
  uma operação hospital pelo \versal{BNDES} (Banco Nacional de Desenvolvimento
  Econômico e Social) --- como ocorreu com a Globo no início dos anos
  2.000 -- seria insuficiente.
\item
  ·~~~~~~ Também não será um pacote exclusivamente publicitário. A~não
  ser os grandes eventos de \versal{TV} aberta, não haveria maneira de carrear
  para os veículos valores para equilibrar o fluxo de caixa.
\item
  ·~~~~~~ Provavelmente o pacote envolverá \versal{BNDES} + publicidade +
  projetos especiais com Ministérios + alguma operação cinzenta com o
  \versal{MEC} (Ministério da Educação).
\end{itemize}

No domingo, a surpreendente crítica de Faustão ao projeto de reforma da
educação traz elemento novo desse jogo, que precisará ser colocado sob
análise. Aparentemente, a paciência da Globo -- a principal agente
midiática do golpe -- está se esgotando. Mais provável ser mais um
capítulo de guerra comercial em torno da fatia da Globo na bolsa"-mídia.

O tempo de vida útil do governo Temer é até o final do ano. Se até lá
não conseguir reverter o jogo, provavelmente não passará pelo teste do
\versal{TSE} (Tribunal Superior Eleitoral). Sendo apeado do poder, não haverá
mais a possibilidade de novas eleições diretas. E~aí se tentaria
reeditar o pacto conservador com o presidente da Câmara Rodrigo Maia.

Mas pairam dúvidas de monta no ar. Como passar por cima das suspeitas
envolvendo lideranças expressivas do \versal{PSDB}? Ou a ideia seria jogá"-los ao
mar, refazendo as alianças com os remanescentes?

\section{Cena 5 -- as forças contra"-cíclicas}

Nas últimas décadas houve uma modernização no país, com temas
civilizatórios, direitos de minorias, cotas raciais, o surgimento de um
empresariado moderno -- aquilo que o Jessé de Souza chama de o Estocolmo
de São Paulo.

À medida em que se aprofunda o Estado de Exceção, tenderão a se
posicionar como força contra"-cíclica.

A dúvida é sobre o efeito"-demonstração das manifestações democráticas.
Daqui para frente, os abusos da Lava Jato serão cada vez maiores,
estimulando mais manifestações de resistência.

Mas há dúvidas de monta sobre essa resistência. O~governo e a Lava Jato
jogam com a tática da intimidação. O~país moderno está suficientemente
consolidado para rebater essa ampliação do Estado de Exceção?
