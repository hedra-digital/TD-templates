\chapterspecial{25/\allowbreak{}06/\allowbreak{}2016 O xadrez do estado policial, no dia D~}{}{}
 

Vamos ao nosso balanço pós"-crise

A rigor, manifestações servem apenas para o exercício da catarse e para
expor o verdadeiro tamanho de alguns personagens públicos, jornalistas,
políticos.

O efeito de fato é sobre a perspectiva futura do eleitorado. Nesse
sentido, as manifestações são eficientes por demonstrar capacidade de
mobilização contra o governo e as esquerdas em geral.

Mas é apenas um fator.

O que se discute hoje em Brasília é mais amplo do que a votação em 2018:
é disputa de poder real, atual, do momento e as expectativas de poder a
partir de 2018.

Vamos a um pequeno rescaldo dos fatos do momento.

\subsection{\textbf{Fato 1}~-- O D+ do impeachment.}

Não há consenso sobre o dia seguinte de um eventual impeachment.

Há um enorme conjunto de interrogações no ar, o que dificulta o pacto
pró"-impeachment.

A maior parte das alternativas degola Dilma e leva junto o
vice"-presidente Michel Temer. Se Lula conseguir apresentar uma
alternativa minimamente viável, ainda pode segurar o \versal{PMDB}.

Chama a atenção o fato de, na delação do senador Delcídio do Amaral,
nenhum procurador ou delegado ter manifestado a mais remota curiosidade
sobre o papel de Gregório Preciado -- ligado ao senador José Serra --
nas relações com empreiteiros. Como se recorda, Preciado foi
expressamente mencionado por Delcídio no grampo armado por Bernardo
Cerveró -- o filho de Nestor -- que serviu de base para a prisão do
senador.

\subsection{\textbf{Fa}to 2 -- Lava Jato e o estado policial.}

É ilusória a ideia de que o afastamento de Dilma ou a prisão de Lula
faria cessar o jogo político da Lava Jato. Pelo contrário, se derrubarem
a Bastilha do poder presidencial, conseguirão implantar definitivamente
o estado policial no país. Pois não haverá mais força institucional
capaz de detê"-la. Se ousarem recuar, perdem toda a base de apoio da
malta.

Os episódios das últimas semanas revelaram que intenção da Lava Jato é
implodir qualquer forma de acordo político, mesmo que signifique jogar o
país em uma guerra interna e em uma depressão econômica.

São vários os indícios.

\textbf{Indício 1}~- As manifestações de domingo corroborando a ideia de
que o movimento atual é contra o sistema político em geral. E~a visão
redentorista de que resolvendo a questão da corrupção, todas as soluções
aparecerão por si.

\textbf{Indício 2}~-- as pregações do procurador Deltan Dallagnol, dando
outra leitura para a Itália pós"-Mãos Limpas. Segundo ele, houve a
eleição de Berlusconi e a paralisação da operação não porque jogou a
economia da Itália no fundo do poço, mas porque a Mãos Limpas não cuidou
de impedir os acertos políticos posteriores. Ou seja, para limpar
definitivamente o país da corrupção e implantar a paz dos cemitérios, a
Lava Jato tem que ir além da derrubada do atual governo. ~O adversário é
toda a classe política.

\textbf{Indício 3}~-- O acirramento da perseguição a Lula, pela Lava
Jato, que está em Curitiba; o vazamento do grampo em Aloizio Mercadante
e da delação de Delcídio, que estão em Brasília. Alguma dúvida?

\textbf{Indício 4}~-- em maio do ano passado, o Ministro Teori Zavascki
suspendeu os mandados de prisão determinados por Sérgio Moro e
requisitou os processos para analisa"-los. No dia seguinte, o site da
revista Veja divulgou matéria na qual dizia que um investigado da Lava
Jato tinha apoiado Zavascki nas eleições para a diretoria do Grêmio.
Naquela mesma tarde, Zavascki reviu sua posição (veja nos anexos pdf da
matéria de Marcelo Auler da Carta Capital). O~mesmo ocorreu com o
Ministro Luís Roberto Barroso. Após um voto corajoso sobre os ritos do
impeachment, foi alvo de ataques através das redes sociais, sobre um
apartamento adquirido por sua esposa em Miami, em operação perfeitamente
legal. No momento seguinte, reescreveu sua biografia voltando pela
supressão da terceira instância nos processos.

A estratégia de grampear um Ministro de Estado é apenas mais um passo na
escalada da Lava Jato, depois que atravessou o Rubicão. Lateralmente, há
uma estratégia de intimidação de quem ousar ficar na frente.

\subsection{\textbf{Fator 3}~- Lula.~}

Se fosse pensar exclusivamente no seu processo, seria mais prudente a
Lula não aceitar o cargo. Sendo Ministro, o caso vai para o \versal{STF} (Supremo
Tribunal Federal) e não haverá instâncias de apelação.

Politicamente, não há outra saída.

Não se tem um desafio fácil pela frente.

No formato mais rápido, o rito do impeachment leva 45 dias. Nesse
período, Lula precisará apresentar ao \versal{PMDB} um cenário minimamente
confortável, que precisa incluir uma estratégia econômica, que não pode
ser conservadora a ponto de agravar a recessão, nem fiscalmente
imprudente.

O Ministro da Fazenda, Nelson Barbosa, tem em mãos algumas alternativas
interessantes de retomar investimentos em saneamento com participação da
iniciativa privada, em cima do encontro de contas com Estados e
Municípios. O~padrão Dilma, no entanto, é o receio de que um respiro
mais forte e a água entrar pelo nariz. Talvez com Lula chegando, possa
ser rompida a inércia.

O acordo com o \versal{PMDB} só será selado se Lula demonstrar que o governo
Dilma se tornou economicamente viável.

Há alguns fatos a favor. Nas últimas semanas surgiram no ar indícios de
recuperação da economia e de redução do ímpeto inflacionário. Não se
sabe até que ponto a Lava Jato conseguirá abortar essa recuperação, com
as bombas dos últimos dias.

\subsection{\textbf{Fator 4}~-- Eugênio Aragão e o \versal{STF}.}

Se não pararem com a escalada da Lava Jato, não haverá estabilização
possível.

Tem"-se um quadro curioso.

O \versal{PGR} Rodrigo Janot montou uma Força Tarefa absolutamente confiável, com
total integração e afinidade política entre os procuradores escolhidos e
os delegados da Polícia Federal.

Fora da Lava Jato, no entanto, há uma animosidade crescente entre
procuradores e policiais, devido à tentativa da Polícia Federal em
conquistar autonomia financeira e administrativa.

Aragão é um procurador eminentemente legalista. Mesmo tendo assumido a
defesa do governo, em algumas ocasiões, conta com o amplo respaldo das
associações de procuradores. E, na condição de Ministro da Justiça,
passa a ser o comandante de fato da Polícia Federal.

A total perda de controle na Lava Jato --- com grampos em Ministros do
governo, ataques a Ministro do \versal{STF}, truculência na invasão de
residências e na decretação de prisões --- é um fator potencial a mais
para reagrupar forças jurídicas e políticas, antes que sobrevenha o
caos.

Não tenho elementos suficientes sobre a habilidade política de Aragão,
sobre o nível atual das suas relações com Janot, ou mesmo sobre sua
posição em relação à Lava Jato para formular qualquer prognóstico. Nos
debates e na defesa de ideias, mostrou"-se uma fortaleza de coerência e
de responsabilidade institucional. O~jogo político exige mais,
habilidade, visão estratégica. Aí, é esperar para ver.

A questão central é que sem uma ação orquestrada e responsável, passando
pelo \versal{STF}, que limite as estripulias da Lava Jato, não haverá
normalização política e econômica possíveis.

\subsection{Cenário de estabilização}

O cenário de estabilização política e econômica passa pelas seguintes
etapas, de difícil execução:

1.~~ Lula conseguindo reagrupar a base de apoio com o \versal{PMDB}. O~\versal{PSDB}
caindo na real sobre os riscos do estado policial.

2.~~ O \versal{STF} colocando um freio na escalada da Lava Jato. Nos próximos
dias a Lava Jato, juntamente com a Globo, jogará tudo -- inclusive
ameaças à família de Lula -- para abortar qualquer tentativa de acordo.

3.~~ A perspectiva de recuperação da economia não ser novamente abortada
pela estratégia da Lava Jato.

Há uma possibilidade, ainda que pequena, de reverter o jogo.
