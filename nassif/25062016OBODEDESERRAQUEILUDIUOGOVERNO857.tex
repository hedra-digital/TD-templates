\chapterspecial{25/\allowbreak{}06/\allowbreak{}2016 O bode de Serra que iludiu o governo}{}{}
 

No início de 2015, quando a base de apoio à Dilma Rousseff erodiu,
iniciou"-se imediatamente uma caça ao petróleo, digna dos pioneiros
texanos. Três craques saíram na frente tentando perfurar o primeiro
poço: o presidente da Câmara Eduardo Cunha e os senadores José Serra e
Renan Calheiros.

Serra e Calheiros acabaram se aliando em seus trabalhos pioneiros.

Nas votações de ontem conseguiram a adesão do governo com uma versão
muito simples da estratégia do bode na sala.

Consistiu no seguinte.

A Petrobras, de fato, tem problemas imediatos para manter o ritmo de
investimentos no pré"-sal. Está com um alto grau de endividamento
agravado pela queda nos preços do petróleo.

Serra apresentou um projeto que tirava da Petrobras a obrigatoriedade e
a preferência de ficar com os 30\% de cada exploração. Teve início as
negociações, e a base aliada foi convencida de que, dando à Petrobras o
direito de preferência, tudo estaria resolvido.

Ou seja, em cada leilão, a Petrobras terá direito de preferência sobre
seus 30\%. Só se abrir mão dele, o leilão será estendido às demais
petroleiras.

Resolvido. A~Petrobras optará apenas pelos campos que forem vantajosos e
empurrará os demais para outras petroleiras --- como sustentou Serra e
outros senadores. Não explicaram por que petroleiras competentes
aceitariam ficar com campos desinteressantes.

A questão central é que em 2018 haverá novas eleições presidenciais. E~há enorme possibilidade de entrar um presidente que não tenha o pré"-sal
em suas prioridades.

Entrando, indicará uma nova diretoria da Petrobras. Para alijar a
Petrobras do pré"-sal, bastará a nova diretoria não exercer nenhum de
seus direitos de preferência. Tudo de acordo com o projeto de lei
aprovado ontem no Senado.
