\chapterspecial{24/\allowbreak{}06/\allowbreak{}2016 Atualizado: Xadrez do golpe aperfeiçoado}{}{}
 

Em uma das colunas passadas, descrevi a chamada subversão das palavras e
dos conceitos. É~quando o clima persecutório se infiltra por todos os
poros do universo da informação, adultera fatos, conceitos e princípios
e bate no coração do Judiciário. E~aí se tem a subversão final, do
magistrado que primeiro define o alvo para só depois ir atrás da
justificativa.

\section{Peça 1 -- o novo normal entre os juízes de
1\textsuperscript{a}~instância}

Em novembro do ano passado, o jovem juiz Paulo Bueno de Azevedo, da 6a
Vara Criminal Federal de São Paulo, afirmava ser partidário da
discrição. A~6a Vara abrigou as operações Satiagraha e Castelo de Areia.

Mais do que a discrição, o juiz fazia questão de definir melhor o papel
do juiz no julgamento.~\emph{``O juiz não pode assumir uma posição de
combate ao crime, eis que, nesse caso, estaria no mínimo, se colocando
como um potencial adversário do réu, papel que deve ser, quando muito,
do Ministério Público ou, em alguns casos, do querelante''.}

Até ontem, Bueno de Azevedo era considerado um juiz criterioso e pouco
propenso a shows midiáticos.

Na edição de julho/\allowbreak{}setembro de 2013 da Revista do \versal{TRF}3 (Tribunal
Regional Federal da 3a Região) publicou o artigo ``Contra um processo
penal ideológico'' (\url{migre.me/\allowbreak{}ubD4G}), insurgindo"-se contra
as generalizações de fundo ideológico praticadas por parte da
magistratura.

Dizia ele:

Sobre o ``Direito Penal do Inimigo''

Trata"-se de teoria desenvolvida na Alemanha que pontifica que a pessoa
que comete ilícitos perde as garantias de cidadão.

\emph{O~Direito Penal do Inimigo é mais um exemplo de direito penal
ideológico. Inimigos não devem ser tratados como pessoas, sob pena de se
colocar em risco a sociedade. É~a argumentação ad terrorem.}

\emph{\redondo{[…]}. Nenhum réu deve ser considerado inimigo. A~justificativa filosófica do Direito Penal do Inimigo é frágil e, pior,
oferece o imenso perigo de formação de um processo penal ideológico. Por
isso, tal visão deve ser combatida.}

Sobre as pressões da mídia

\emph{A~mídia costuma apontar como um dos problemas brasileiros a
impunidade, seja de modo geral seja em relação a setores específicos,
como os crimes de colarinho branco, delitos cometidos por políticos etc.
Trata"-se de um justo anseio da sociedade, o combate à impunidade.
Todavia, esse combate precisa ser encarado como uma pretensão moral ou
social, e nunca como uma obrigação jurídica do Estado"-Juiz, encarnado
pelo Poder Judiciário. Se existe a obrigação de punir, o resultado do
processo é previamente conhecido. Assim, novamente o processo penal
torna"-se uma farsa}

\subsection{Sobre o direito de defesa}

\emph{Há, também, outras garantias, como o estabelecimento de prazos
prescricionais, a proibição de provas ilícitas etc. Recorde"-se que os
princípios e garantias beneficiam a todos os réus e, porventura, podem
impedir a condenação de culpados. Exatamente por isso, não há falar"-se
numa obrigação estatal de punir, porém tão"-somente na obrigação de
propiciar um julgamento justo. Para todo e qualquer réu}

\subsection{Sobre o maniqueísmo político}

\emph{Os cegos pela ideologia, de esquerda ou de direita, praticamente
dividem o mundo entre o Bem e o Mal. Tudo o que os de direita fazem é
mal, segundo um esquerdista. Se a mesma coisa é feita por esquerdistas,
aí tudo é justificado. E~vice"-versa.}

Na época, Bueno de Azevedo fazia doutorado na \versal{USP}, sob orientação da
professora Janaina Conceição Paschoal.

Ontem, ele autorizou a invasão da casa de uma senadora, levou preso seu
marido, o ex"-Ministro Paulo Bernardo, mais duas pessoas e ordenou a
invasão da sede do \versal{PT}.

Se Paulo Bernardo for considerado culpado, que seja condenado e pague
por seus crimes. Mas qual a razão da prisão preventiva que, segundo o
juiz, ``não significa antecipação de juízo de culpabilidade. Ela é
decorrente .de uma combinação de indícios suficiente de materialidade e
autoria delitiva e da presença dos requisitos cautelares, acima
expostos''.

Por seu histórico de sentenças de bom senso, colegas de Bueno de Azevedo
chegaram a supor que a prisão de Bernardo estivesse fundamentada em
argumentos fortes e irrefutáveis.

Mas o que diz o juiz?

\emph{``Risco à ordem pública existe também quando, em tese, desviados
milhões de reais dos cofres públicos, máxime na situação conhecida de
nosso País, que enfrenta grave crise financeira e cogita aumento de
impostos e diminuição de gastos sociais. \redondo{[…]} O\,risco de que
tal dinheiro desviado não seja recuperado também representa perigo
concreto à aplicação da lei penal''.}

Continua o juiz Bueno de Azevedo, que já foi crítico do ``processo penal
ideológico''.

\emph{A~autoridade policial sustenta o pedido de prisão preventiva com
base no risco à instrução criminal, baseando"-se na colaboração de
Delcídio do Amaral, segundo a qual Paulo Bernardo seria pessoa muito
influente, ``com muita força política'' e ``poder de decisão'', tendo
muita ``facilidade de contato com empresários e com o próprio
governo''.}

Agora Paulo Bernardo é ex"-Ministro e é de conhecimento geral seu
isolamento:

\emph{"O fato de \versal{PAULO} \versal{BERNARDO} não ser mais Ministro também não
elidiria o risco de influência negativa para a instrução criminal nem a
prática de novos delitos, citando argumentação do Juiz Federal Sêrgio
Moro em situação semelhante, referente a um ex"-parlamentar (11. 330).}

Fez mais: autorizou a invasão da casa de uma senadora da República,
atropelando o Supremo Tribunal Federal, mediante o argumento de que a
Polícia Federal deveria recolher apenas as provas do marido, Paulo
Bernardo, e não mexer nas da esposa, senadora.

É evidente que, pela frente, o governo interino vai hipotecar apoio a
qualquer arbitrariedade. Por trás, a retaliação virá na forma de cortes
orçamentários e outros recursos.

Cada tentativa de juízes de 1\textsuperscript{a}~instância e
procuradores de açambarcar o protagonismo político -- com operações
midiáticas -- é um tiro a mais no pé do Judiciário e do Ministério
Público.

Mas esse protagonismo aumentará à medida em que se aproxima a data final
da votação do impeachment. Sem terem sido eleitos, muitos juízes se
colocam no papel de eleitores preferenciais do impeachment.

\section{Parte 2 -- o pacto da Lava Jato com Alexandre Moraes}

Nos últimos tempos, em função de questionamentos sofridos, o Procurador
Geral da República Rodrigo Janot deu"-se conta da desmoralização que o
Ministério Público estava enfrentando pela parcialidade com que conduzia
a Lava Jato. Acelerou, então, um conjunto de denúncias contra líderes do
\versal{PSDB} e contra o grupo que tomou o poder.

Imediatamente, veio o contra"-ataque da primeira instância, repondo o \versal{PT}
no centro das atenções.

Aliás, um levantamento dos vazamentos registrados nos últimos tempos
mostrará que, de Curitiba, vazam apenas informações contra Dilma, Lula e
o \versal{PT}. De Brasília, amplia"-se um pouco mais o leque dos vazados.

A reunião entre o Ministro da Justiça Alexandre Moraes e os integrantes
da Lava Jato, incluindo o juiz Sérgio Moro, juntou pessoas politicamente
alinhadas.

A agenda da Lava Jato continua sincronizada com a pauta política.

\section{Peça 3 -- a arte de cortar na carne dos outros}

O interino pretende avançar sobre os gastos de educação, saúde e sobre o
Regime Geral da Previdência. Não há verbas para pesquisas, há atrasos
para bolsas do \versal{CNP}q (Conselho Nacional de Pesquisa) e impediu"-se o
reajuste para o Bolsa Família.

No entanto, aprovou"-se um déficit de R\$ 170 bilhões no orçamento, uma
autorização para gastar que está alocando recursos nos Ministérios para
as demandas fisiológicas de cada Ministro, com vistas às eleições deste
ano.

Verbas destinadas a Secretarias extintas (como a das Mulheres) foram
realocadas na própria presidência.

O interino deu uma bola dentro com a indicação de Wilson Ferreira Jr. --
da \versal{CPFL} -- para a Eletrobrás. São poucas indicações técnicas visando
legitimar um arranjo no qual pontifica o que de mais fisiológico a
político brasileira gerou no país pós"-ditadura.

No livro de memórias de Fernando Henrique Cardoso estão registradas as
raízes da formação do atual grupo de poder. Coube a Michel Temer
apadrinhas as indicações de Eliseu Padilha e Geddel Vieira Lima para seu
Ministério.

Há uma torcida enorme da mídia, tentando encontrar aperto fiscal na
sangria. Será necessário muito esforço para construir essa narrativa.

\section{Peça 4 -- o papel de Dilma}

A bola do impeachment está com Dilma. Se conseguir desenvolver uma Carta
à Nação coerente, fundada em princípios, montar um arco de alianças mais
amplo e deixar mais clara a proposta de plebiscito seguido de
eleições,Dilma terá condições de derrubar o impeachment.

Por enquanto, o que se ouve são endossos vagos a ideias vagas. A última
tacada de Dilma será a Carta aos Brasileiros. Poderá ser uma bomba. Mas
poderá ser um track.

Peça 5 --- o aprimoramento do golpe

O golpe está sendo aperfeiócado para contornar o incômodo de senadores e
de Ministros do \versal{STF} com o caráter golpista explícito do atual processo
de impeachment. Há que se sofisticar.

Aproveitando a não ocupação do espaço político por Dilma, o interino
está prometendo aos senadores que ficará só até janeiro, deixando para o
Congresso escolher o sucessor.

O Senado aprovaria o impeachment. Em dezembro, o \versal{TSE} (Tribunal Superior
Eleitoral) cassaria a chapa Dilma"-Temer, mas ressalvando a elegibilidade
de Temer. Ele renunciaria e em janeiro seria eleito indiretamente pelo
Congresso para os dois anos seguintes. Como diz um especialista
eleitoral de Brasilia, ``consuma"-me o jogo elegantemente''.

Segundo o especialista, ``isto tem o dedo de Gilmar: é muito sofisticado
para a cabeça de jaca dos golpistas''.

Seria a saída jurídica perfeita e a comprovação de que a teoria do
impeachment constitucional não pegou: foi golpe mesmo.
