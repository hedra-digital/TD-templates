\chapterspecial{25/\allowbreak{}06/\allowbreak{}2016 O xadrez de uma semana que o Supremo será o ator}{}{}
 

Vamos ao nosso xadrez quase diário.

A rigor, existem dois grupos se articulando:

\textbf{Grupo 1}~-- o governo, agora sob a coordenação de Lula.

\textbf{Grupo 2}~-- o eixo Temer"-\versal{PSDB}, agora depurado da interferência
amadora de Aécio Neves e da resistência de Geraldo Alckmin.

O fiel da balança continua sendo o presidente do Senado Renan Calheiros
e o grande desafio consiste em conter o êxodo da base parlamentar que
Lula legou para o governo Dilma.

Como já tinha alertado, daqui até a votação do impeachment haverá uma
guerra de informações de lado a lado, com inúmeros factoides,
informações distorcidas, visando desmontar o ânimo do adversário.

Ontem, a home do Estadão manteve durante quase todo o dia a manchete de
que o presidente do Senado Renan Calheiros teria afirmado que não ter
condições de segurar o impeachment. Nem se pode dizer que a manutenção
da manchete seja uma estratégia porque o site do Estadão tem trocado no
máximo três vezes por dia a manchete principal.

O que Renan teria dito é que se o impeachment passasse pela Câmara, ele
não conseguiria barrar no Senado. E~que Lula precisaria entrar nas
negociações para barrar o impeachment na Câmara.

Enfim, reforça o que vimos explicando aqui: a grande batalha de
Stalingrado será na Câmara no mês de abril e os instrumentos da
contrainformação estão atuando a mil por hora.

Há um jogo político"-jurídico que passa pelas duas casas: \versal{STF} e
Congresso. E~no qual poderá haver interferência de dois atores políticos
constantes: o Alto Comando da Lava Jato, tendo o controle do ritmo das
denúncias midiáticas e respectivos alvos, e o Ministro Gilmar Mendes,
com seu estoque de ousadias.

As batalhas que se prenunciam são as seguintes:

\textbf{Frente Câmara Federal}

Começou a contagem regressiva para a votação do impeachment.

De um lado se tentará de todas as maneiras restringir a ação de Lula,
para impedi"-lo de formar uma base que segure o impeachment.

A oposição tem munição para, através do Alto Comando, atirar em Renan
Calheiros e parlamentares com foro privilegiado, que venham a ter papel
relevante na montagem da base de apoio. Mas não tem mais o espaço de que
dispunha. Depois da autorização para a divulgação de conversas pessoais
pela Lava Jato, o primado da isenção saiu pelo fio do telefone. Avançar
mais implicaria em comprometer irreversivelmente a imagem da \versal{PGR}.

\textbf{Frente \versal{STF}}

O silêncio do \versal{STF} (Supremo Tribunal Federal) está ficando ensurdecedor.
Não haverá mais como se eximir do que está ocorrendo.

Nas decisões mais simples, há duas ações procurando barrar Lula. Uma,
devolvendo a ação contra ele para o juiz Sérgio Moro. Outra, outra, a
série de ações para impedi"-lo de assumir a Casa Civil.

No primeiro caso, o juiz natural é Teori Zavascki. No segundo, o juiz
prevento (aquele que já apreciou ação parecida) é Marco Aurélio de Mello
Mello que já julgou ação similar no \versal{STF} negando uma ação cautelar
similar (\url{migre.me/\allowbreak{}tizGb)}.

Segundo explicação na Jusbrasil (\url{migre.me/\allowbreak{}tizDi}):

\emph{Prevenção é um critério de confirmação e manutenção da competência
do juiz que conheceu a causa em primeiro lugar, perpetuando a sua
jurisdição e excluindo possíveis competências concorrentes de outros
juízos.}

No domingo foi encaminhado ao \versal{STF} um pedido de habeas corpus assinado
por um conjunto de juristas dos mais respeitáveis, solicitando a
suspensão da decisão do Ministro Gilmar Mendes de remeter o processo de
volta para Curitiba. Não haverá como o \versal{STF} se eximir mais.

Se demorar, a ponto de Moro ousar aplicar uma prisão em Lula, o \versal{STF} será
responsável por todas as tragédias que poderão advir.

A parte mais complicada será a de encarar os crimes e abusos cometidos
pela Lava Jato e pelo juiz Sérgio Moro. Há juristas de peso -- incluindo
o vice"-decano do \versal{STF} Marco Aurélio de Mello -- apontando claramente para
crimes cometidos, ao vazar os diálogos em geral e o último grampo em
particular, envolvendo a própria presidente da República. O~decano Celso
de Mello terá que comprovar que sua indignação não é seletiva. E~o \versal{STF}
terá que escolher entre conter a escalada de abusos ou abrir mão de vez
de conter o Estado policial.

\textbf{Frente Polícia Federal}

Depois de um início indignado, o novo Ministro da Justiça Eugênio Aragão
amenizou um pouco o discurso, mas continua firme no foco do respeito à
legalidade.

Não será tarefa fácil disciplinar a tropa. A~autonomia da Polícia
Federal é um dos pontos centrais da balbúrdia institucional que tomou
conta do país. Só faltava dar autonomia a uma corporação armada, com as
prerrogativas de Estado.

No seu período de Ministro, José Eduardo Cardozo conseguiu o impossível.
Primeiro, provocou uma revolta interna na \versal{PF}, com o abandono a que foi
relegada em sua gestão. Conseguiu transformar uma força orgulhosa em uma
tropa indignada. Depois, para contemporizar, abriu mão de qualquer
controle sobre ela e liberou geral.

Devolver o profissionalismo, impedir a politização escandalosa -- de
delegados fazendo campanhas políticas e participando de vazamentos de
informações -- será um trabalho que exigirá muita estratégia e firmeza.

Ontem, no Twitter, um procurador regional chegou a ameaçar o Ministro
--- que é subprocurador -- com uma sanção do Conselho Superior do
Ministério Público, ao qual ele, procurador, jamais pertenceu.

\textbf{Frente midiática}

Não se surpreenda se os grupos de mídia conseguirem se rebaixar mais
ainda e desenterrar o estilo abjeto de 2010, com a fabricação de todo
tipo de factoide ou de falsos escândalos. Serão vinte dias que valerão
dez anos.

É bem possível que a Lava Jato se aproveite da demora do Supremo em
decidir qualquer coisa, para soltar seus últimos foguetes.

\textbf{Manifestações}

Continuará a guerra de manifestações populares. Do lado dos
pró"-impeachment, certamente estimuladas pela parceria Lava Jato"-mídia.
Do lado anti"-impeachment, o estímulo serão eventuais arbitrariedades que
voltarem a ser cometidas.

Na noite de domingo, por exemplo, havia convocação de militantes para
uma vigília na frente da casa de Lula, para impedir qualquer tentativa
de prisão.
