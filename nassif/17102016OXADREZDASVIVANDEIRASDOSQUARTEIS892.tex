\chapterspecial{17/\allowbreak{}10/\allowbreak{}2016 O Xadrez das vivandeiras dos quartéis}{}{}
 

Cenários não são previsões taxativas. Consistem na análise de um
conjunto de variáveis que apontam uma tendência --- que poderá se
confirmar ou ser modificada por novas variáveis.

Os fatos apontam para uma tendência cada vez maior de intervenção dos
militares na vida nacional e, ao mesmo tempo, um desprestígio cada vez
maior do poder civil.

Sinais recentes:

\begin{itemize}
\itemsep1pt\parskip0pt\parsep0pt
\item
  ·~~~~~ A entrega do Gabinete de Segurança Institucional (\versal{GSI}) a um
  militar da ativa, que passa a frequentar o coração do governo.
\item
  ·~~~~~ A tentativa do Ministro da Justiça de criar a figura do inimigo
  interno nas manifestações e em factoides sobre o Islã e colocar as
  \versal{FFAA}s na repressão interna.
\item
  ·~~~~~ O convite da presidente do \versal{STF} (Supremo Tribunal Federal)
  Carmen Lúcia, para que as Forças Armadas ampliem sua participação na
  segurança nacional.
\item
  ·~~~~~ A criação e utilização da Força Nacional de Segurança para
  outros propósitos.
\end{itemize}

Vamos juntar algumas peças do momento, para avaliar até onde poderá ir
esse flerte com as Forças Amadas.

\section{Peça 1 --- o chamamento aos militares}

Saliente"-se que, até agora, o comportamento das Forças Armadas tem sido
irrepreensível, profissional, sem interferências no jogo político.

No governo Lula e em parte do governo Dilma Rousseff, houve ênfase na
indústria da defesa, com as Forças Armadas envolvidas em temas nacionais
de peso: defesa da Amazônia, da Amazônia azul, guerras telemáticas,
setor aeroespacial, eletronuclear.

Por outro lado, blindou"-se os militares na Comissão Nacional da Verdade,
mas se reservou o Ministério da Defesa para os civis. Esse modelo foi
mantido por dois superministros da pasta, Nelson Jobim e Celso Amorim.

O modelo começou a degringolar quando, ainda no governo Dilma, se
entregou a Defesa ao deputado Aldo Rebelo, do \versal{PC}doB.

Na discussão do Código Florestal Brasileiro, Rebelo já dera sinais de
seu estilo político, demasiadamente contemporizador, de aproximação
excessivamente estreita com setores avessos ao partido e às convicções
que aparentava defender. Pode ser virtude política, se bem utilizado e
não se avançar além dos limites das convicções políticas que se defende.
E~Aldo usou bem esses dons na organização da base de apoio ao governo.

No caso do Código Florestal, no entanto, demonstrou uma preocupante
simbiose política com os ruralistas. E, na Defesa, escancarou~de forma
temerária o Ministério aos militares. A~ponto de, no curto período em
que assumiu a Casa Civil, o ex"-governador da Bahia Jacques Wagner ter se
dado conta dos riscos implícitos nessa ação e convencido a presidente
Dilma a assinar um decreto devolvendo funções de Defesa aos civis.

Assim que assumiu, um dos primeiros atos de Michel Temer foi revogar
esse decreto.

\section{Peça 2 --- a dispersão de poder dos golpistas}

O golpe parlamentar aplicado no país contou com a participação de
inúmeros atores, mas nenhum se impondo politicamente. A~saber:

\begin{itemize}
\itemsep1pt\parskip0pt\parsep0pt
\item
  ·~~~~~ O grupo de Eduardo Cunha, que logrou colocar Michel Temer na
  presidência.
\item
  ·~~~~~ O \versal{PMDB} liderado por Renan Calheiros e José Sarney.
\item
  ·~~~~~ O \versal{PSDB}, especialmente o de Aécio Neves.
\item
  ·~~~~~ O Pode Judiciário comandado por Gilmar Mendes.
\item
  ·~~~~~ Os grupos de mídia.
\item
  ·~~~~~ A Procuradoria Geral da República.
\item
  ·~~~~~ A Lava Jato, colocada como um poder a parte da \versal{PGR}.
\item
  ·~~~~~ Grupos da Polícia Federal
\end{itemize}

Consumado o golpe, há um jogo de espera, uma aliança mal costurada,
cheia de desconfianças, com cada lado tentando se fortalecer para
preservar o poder ou para derrubar os ex"-aliados do poder em que se
encontram.

Além disso, o jogo de cena da Lava Jato está se esgotando com a caça aos
petistas e ao ex"-presidente Lula.

Em breve se entrará em novo capítulo político, no qual os antigos
aliados se digladiarão pensando em 2018. É~nesse quadro surgem as
chamadas vivandeiras, buscando aproximação com o poder militar.

\section{Peça 3 -- a situação dos atores políticos}

Vamos a uma análise da situação de cada um desses braços do golpe.

\subsection{Grupo de Eduardo Cunha}

A tentativa de criar um personagem para Michel Temer esbarra, sempre, no
perfil medíocre do presidente. Não tem carisma, ideias, informação, nem
reputação ilibada. E, no seu entorno, abriga a truculência de um Geddel
Vieira Lima e de um Eliseu Padilha e o estilo escorregadio"-gosmento de
Moreira Franco.

Se a \versal{PEC} 241 não for aprovada, morre no início do ano. Aprovada,
inviabilizará qualquer campanha situacionista em 2018, tal o arrocho
fiscal que implantará.

A criação da figura do inimigo externo é a saída mais óbvia para o
governo Temer, especialmente se a \versal{PEC} 241 for derrotada. Não terá nenhum
prurido em invocar o poder militar, em caso de enfraquecimento maior de
sua base.

\subsection{A mídia}

A Globo é poder dominante. Estadão e Folha não conduzem: são conduzidos.
Por conta de seus problemas financeiros, estão de joelhos ante o governo
Temer e a própria Lava Jato.

A Folha se amedrontou no episódio Savonalora, ao responder a Sérgio
Moro, que sugeriu explicitamente o que o jornal deveria ou não publicar.
A~resposta foi tíbia: ``A opinião do jornal está na página 2, em artigos
não assinados''.

Foi necessária a~Veja, da moribunda Editora Abril, fazer valer o poder
da mídia, enquadrando Sérgio Moro e seu~\emph{imprimatur}~ e alertando
para os riscos do excesso de poder e de abusos de Moro e de diversos
procuradores, como os que tentaram proibir manifestações políticas em
Universidades ou o ``Fora Temer'' no Colégio Pedro 2o. O~que a despertou
foi a tentativa de quebra judicial do sigilo telefônico de um repórter
de outra publicação.

Aliás, por incrível que pareça, os únicos assomos de jornalismo
ultimamente são de~Veja, com as denúncias contra a camarilha dos 6
(Temer, Cunha, Geddel, Padilha, Moreira Franco e Jucá). Finalmente,
colocou um jornalista como diretor de redação, mas não a tempo de
salvá"-la. É~possível que haja a mão de José Serra por trás desses
ataques.

\subsection{A Procuradoria Geral da República}

Assim que começou a Lava Jato, Dilma incumbiu seu Ministro da Justiça
José Eduardo Cardoso, de procurar o \versal{PGR} Rodrigo Janot para viabilizar um
acordo de leniência que salvasse as empresas envolvidas com a Lava Jato
e não machucasse tanto a economia.

O texto do acordo chegou a ser esboçado. Os procuradores da Lava Jato
peitaram a decisão: se o acordo fosse assinado, denunciaram Janot e
pediriam demissão. Janot perdeu a batalha contra os de baixo.

Procurou, então, recuperar a autoridade tentando assumir a liderança dos
rebelados, em gesto muito comum em comandantes frágeis. Tornou"-se ele
próprio um conspirador, como ficou evidente no episódio do vazamento dos
grampos de Lula e Dilma.

Quando o golpe se consumou, sem noção de~\emph{timing}~pediu a prisão de
três senadores baseando"-se exclusivamente em um grampo ilegal e
inconclusivo do ex"-senador Sérgio Machado. Era tão desproporcional o
pedido que se imaginou que teria outros trunfos na manga. Não tinha. O~pedido foi liminarmente rejeitado pelo Ministro Teori Zavascki, Janot
ficou sem espaço e perdeu a batalha contra os de cima.

Hoje em dia, está paralisado, não consegue conter os abusos individuais
de procuradores e perdeu o apoio quase unânime com que a categoria o
sufragou nas últimas eleições. Foi um dos coordenadores do golpe, mas
não conseguiu manter o comando.

Tudo isso induz a acreditar que empurrará com a barriga o cargo até o
final do seu mandato. Pelo desenrolar do jogo, no entanto, uma eventual
reação de seus aliados do \versal{PSDB} contra a camarilha dos 6 poderá ensejar
nova tentativa de protagonismo de Janot, acelerando as denúncias contra
o grupo de Temer.

\subsection{O Judiciário~}

As intervenções militares exigem algum grau de legitimação jurídica. Até
o golpe de 64 necessitou dos tais Atos Institucionais.

É por aí que se torna perigosa a iniciativa da Ministra Carmen Lúcia de
convocar militares para discutir segurança nacional. No plano
institucional, a Ministra é tão despreparada que não se sabe se sua
iniciativa foi apenas uma demonstração de ignorância sobre o papel dos
militares, uma tentativa canhestra de se colocar no centro dos
acontecimentos, de ``causar'', ou se de fato foi mordida pela mosca azul
de Brasília.

De qualquer modo, com ela na presidência do \versal{STF} e Gilmar Mendes na do
\versal{TSE} (Tribunal Superior Eleitoral), há jogo, caso de pretenda
incompatibilizar Temer para abrir espaço para um presidenciável do \versal{PSDB}.

\subsection{O \versal{PSDB} e o \versal{PMDB} de Renan}

Mantém o casamento por conveniência com a camarilha dos 6. ~É um partido
sem coluna vertebral, que se sustenta apenas com a blindagem que lhe é
conferida por Janot e pela Lava Jato.

\section{Peça 4 -- as probabilidades políticas}

Como se percebe, fora a aliança contra o inimigo comum -- o \versal{PT} e Lula --
não há um grupo hegemônico nem uma pauta capaz de unir os mentores do
golpe. Por outro lado, mesmo com o \versal{PGR} não tomando medidas mais efetivas
contra o \versal{PSDB} e setores aliados do \versal{PMDB}, há a percepção generalizada da
falta de legitimidade de todos esses grupos civis, ainda mais em uma
empreitada em que a palavra de ordem foi o combate à corrupção.

O único fator a uni"-los seria a perspectiva de perder as eleições de
2018.

Por tudo isso, as perspectivas atuais são as seguintes:

\begin{enumerate}
\itemsep1pt\parskip0pt\parsep0pt
\item
  1.~~~~ Permanece o risco da prisão de Lula, visando promover agitações
  populares que justifiquem o endurecimento do regime.
\item
  2.~~~~ Continua baixa a probabilidade de recuperação da economia,
  ainda mais com a combinação de ajuste fiscal rigoroso e ritmo lento de
  queda dos juros.
\item
  3.~~~~ Há uma probabilidade não desprezível de Temer ser despojado do
  cargo por conta dos julgamentos do \versal{TSE} e pela desmoralização contínua
  de seu governo.
\item
  4.~~~~ Persistirá a tendência de ampliação da presença dos militares
  no governo, ao mesmo tempo em que se aprofunda a desmoralização do
  poder civil.
\item
  5.~~~~ Mesmo assim, qualquer ampliação da intervenção militar viria
  como retaguarda para um governo civil.
\end{enumerate}
