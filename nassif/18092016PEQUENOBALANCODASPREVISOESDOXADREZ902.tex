\chapterspecial{18/\allowbreak{}09/\allowbreak{}2016 Pequeno balanço das previsões do Xadrez}{}{}
 

Como alertamos, será cada vez mais difícil o sistema de poder conviver
com a hipocrisia escancarada, desvendada pelas redes sociais, Internet e
por essa maldição chamada de liberdade de imprensa --- ainda que
restrita aos veículos alternativos.

À medida em que seus passos vão sendo desvendados, por vazamentos
imprevistos, notícias escondidas em pé de página por antecipação de
jogadas óbvias, cria"-se a necessidade de explicações, satisfações,
aumentando a confusão.

É a velha máxima de que tudo que começa com uma mentira fica prisioneiro
dela, precisando sucessivamente criar novas mentiras para sustentar a
inicial.

Vamos a um breve rescaldo dos últimos capítulos do Xadrez.

\section{Gilmar x Lava Fato}

É o caso do Ministro Gilmar Mendes, acusado aqui de ter atiçado seus
blogueiros contra a apresentação da Lava Jato, sobre a denúncia de Lula,
com o propósito de preparar o terreno para futuras investidas.

No dia seguinte, o blogueiro em questão escreveu caudaloso artigo
mostrando como ele era dono se seu nariz e não dava satisfações a
ninguém. E~Gilmar, sem que nada lhe fosse perguntado, correu a elogiar a
decisão da denúncia, ainda que de forma dúbia: a denúncia é boa porque,
finalmente, vai permitir a Lula se defender. O~que não deixa de ser uma
verdade. Mas passou recibo.

No \versal{STF}, não foi bem"-sucedida a tentativa de levar o Ministro Luiz Fux
para a Segunda Turma -- que julga a Lava Jato \mbox{---,} em lugar de Carmen
Lúcia. Quem assumirá será o ex"-presidente Ricardo Lewandowski.

Não será garantia de penas mais brandas, mas de isonomia e defesa da
legalidade dos procedimentos. Aliás, agora que as chamas da Lava Jato se
voltam contra o \versal{PSDB}, veremos o reaparecimento do grande garantista
Gilmar Mendes.

Chama atenção um paradoxo intrigante, neste país de paradoxos. Na
despedida, o legalista Lewandowski fez um discurso saudando o século 21
como o século do Judiciário -- aliás, no mesmo tom da palestra que deu
em seminário do Brasilianas de fins de 2013. Trata"-se de uma tendência
perigosíssima, como se está vendo hoje em dia na atuação desmedida do
Ministério Público e da Polícia Federal.

A crítica à tese partiu de Dias Toffoli, que sabiamente alertou para o
risco representado por uma república de juízes, mais perigoso que uma
república de militares, como foi em 1964,

\section{Lava Jato x Procurador Geral da República}

Monta"-se o jogo de cena com a capa de~\textbf{Veja}, da suposta delação
de Léo Pinheiro, presidente da \versal{OAS}, a um não"-crime do Ministro do \versal{STF}
Dias Toffoli. Imediatamente, deflagram"-se três movimentos:

\begin{enumerate}
\itemsep1pt\parskip0pt\parsep0pt
\item
  1. O~\versal{PGR} Janot ordena o cancelamento das negociações da delação de
  Pinheiro, que estava pronto para denunciar Aecio Neves e José Serra
  por crime de corrupção~
\item
  2. Gilmar Mendes sai a campo, atacando a Lava Jato.
\item
  3. O~ataque de Gilmar provoca um sentimento de autodefesa na Lava
  Jato, do qual Janot se vale para arrancar uma nota conjunta, assinada
  por todos os procuradores, endossando a decisão de suspender a delação
  de Pinheiro.
\end{enumerate}

Passam alguns dias, o jogo vai clareando, e o que ocorre?

\begin{enumerate}
\itemsep1pt\parskip0pt\parsep0pt
\item
  1. Sérgio Moro convoca Léo Pinheiro para mais uma rodada de
  depoimento.
\item
  2. Pinheiro fala nas propinas pagas a José Serra e Geraldo Alckmin, e
  a delação é vazada para O Globo por integrantes da Lava Jato.
\end{enumerate}

São cutucadas óbvias no \versal{PGR} já que Serra, por ser senador e Ministro,
tem foro privilegiado.

Mas Janot continua a passos de tartaruga pingando em conta"-gotas
denúncias contra nomes de menor peso. Na semana passada, denunciou o
apagadíssimo senador Valdir Raupp (\versal{PMDB}"-\versal{RO}).

A suposição de que talvez esteja conduzindo investigações sigilosas
sobre Aécio Neves e José Serra esbarra em uma evidência: até agora não
há nenhuma informação de que Dimas Toledo (homem de Aécio em Furnas) e
Paulo Preto (homem de Serra no \versal{DNER}) tenham sido sequer convocados para
prestar depoimento.

Ora, eles são tão importantes para os esquemas Serra"-Aécio quanto Paulo
Roberto Costa para o \versal{PMDB} e \versal{PP} e Nelson Duque para o \versal{PT}. Qual a razão
desse esquecimento?

Os vazamentos contra Serra e Aécio estão chegando por tabela, como
informação subsidiária de delatores pressionados a delatar Lula e o \versal{PT}.
E~até agora Janot não reviu sua posição de voltar a negociar a delação
de Pinheiro. Provavelmente porque ficou muito óbvio que o ponto central
seria convencer Pinheiro a livrar Serra e Aécio de acusações de
enriquecimento pessoal.

Dois fatos adicionais comprovam a luta pessoal de Janot na Lava Jato:

\begin{enumerate}
\itemsep1pt\parskip0pt\parsep0pt
\item
  1. A~mesquinharia de manter a acusação de ``obstrução da Justiça'' aos
  questionamentos da defesa de Lula, mesmo após o \versal{MI}nistro Teori
  Zavascki ter voltado atrás e retirado a expressão da sua sentença.
\item
  2. A~informação da Folha de hoje (18.09.2016) de que a Lava Jato
  aproveitou pedaços da delação anulada de Léo Pinheiro contra Lula. Se
  a delação era necessária, qual a razão de Janot em procurar anulã-la?
  Obviamente blindar Aécio Neves.
\end{enumerate}

Eduardo Cunha x rapa

A entrevista de Eduardo Cunha ao Estadão comprova algumas previsões do
Xadrez:

\begin{enumerate}
\itemsep1pt\parskip0pt\parsep0pt
\item
  1.~~~~ Já começou a antropofagia no grupo golpista.
\item
  2.~~~~ A aliança Rodrigo Maia + \versal{PSDB} é uma das hipóteses em jogo.
  Cunha acrescentou nela Moreira Franco, sogro de Maia.
\item
  3.~~~~ O \versal{TSE} (Tribunal Superior Eleitoral), através de seu presidente
  Gilmar Mendes, está jogando com prazos de julgamento da chapa
  Dilma"-Temer para impedir eleições diretas, caso a condenação se dê
  ainda em 2016. E~também para manter Temer sob rédea curta, com a
  possibilidade de ser condenado em 2017. Nesse caso, assumiria o
  presidente da Câmara.
\item
  4. Janot aumentará seu protagonismo no jogo político, podendo desovar
  denúncias contra o \versal{PMDB}, dentro de sua aliança tácita com Aécio e o
  \versal{PSDB}.
\end{enumerate}
