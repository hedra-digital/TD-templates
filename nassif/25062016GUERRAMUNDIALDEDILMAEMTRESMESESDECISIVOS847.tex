\chapterspecial{25/\allowbreak{}06/\allowbreak{}2016 guerra mundial de Dilma em três meses decisivos}{}{Luis Nassif}
 

A primeira rodada do golpe paraguaio foi a tentativa de glosar a
campanha de Dilma Rousseff no \versal{TSE} (Tribunal Superior Eleitoral), uma
manobra envolvendo o presidente Dias Toffoli e seu líder Gilmar Mendes.
Falhou no último instante graças ao recuo do Ministro Luiz Fux.

A segunda foi a de abrir o ritual do impeachment com base nas pedaladas,
obra do presidente da Câmara Eduardo Cunha.

Com os índices de popularidade de Dilma em patamares mínimos, pensava"-se
que a mera abertura do rito do impeachment seria suficiente para
derrubar a presidente.

A manobra expôs de modo imprudente o perfil da frente de conspiradores:
Aécio Neves, José Serra, \versal{FHC}, Gilmar Mendes, Paulinho da Força, Michel
Temer, Eduardo Cunha, Paulo Skaf, Ronaldo Caiado, Agripino Maia. ~Pela
primeira vez se via, de forma panorâmica, o que seria a partilha do bolo
e quem seriam os novos vitoriosos.

Pelo efeito comparação, estão criando um movimento de reversão da imagem
negativa de Dilma.

O segundo movimento termina com alterações em relação ao cenário de
2015:

1.~~~~ O \versal{STF} projeta"-se, de fato, como poder mediador, pouco propenso a
endossar aventuras caudilhescas.

2.~~~~ A reação geral comprovou que não existirá impeachment indolor,
como foi no caso de Fernando Collor. A~gana com que os conspiradores se
atiraram ao pote despertou um movimento de proteção à presidente. Ficou
claro quem representava os vícios do modelo político vigente.

3.~~~~ A crise interna do \versal{PMDB} demonstrou também que a frente \versal{PMDB}"-\versal{PSDB}
não seria nenhuma garantia de aglutinação e estabilização política.

A segunda rodada termina com a alternativa Michel Temer queimada e com o
impeachment via Câmara inviabilizado. Todo o trabalho de construir a
imagem de Temer mediador virou pó com alguns dias de exposição ao sol.

A próxima rodada será novamente no \versal{TSE}. Com as últimas revelações da
parceria Camargo Correia"-Cunha"-Temer, o vice"-presidente será jogado ao
mar pela mídia. Restará ao \versal{PSDB} -- liderado por Gilmar Mendes -- o
protagonismo solitário da próxima tentativa de golpe. O ápice do jogo
acontecerá ainda no primeiro trimestre.

\section{O cenário do golpe paraguaio}

A substituição de Joaquim Levy por Nelson Barbosa permitirá um recomeço
na política econômica. Ambos têm como meta a recomposição fiscal. Mas há
diferenças radicais nos estilos e propósitos.

Levy tem a mentalidade do Tesouro, de analisar os gastos apenas do lado
quantitativo, sem se preocupar com os efeitos sobre a economia. Para
ele, bastaria um superávit consistente para imediatamente os
investimentos voltarem. Nada de pensar em criar demanda.

Barbosa tem uma visão sistêmica. Sabe que o ajuste não pode ser o único
fator motivador dos investimentos. Há desafios na manutenção da demanda,
no destravamento de setores baleados.

Essa diferença de visão manifestava"-se nas conversas com parlamentares.
Levy limitava"-se a repetir a retórica do fim do mundo para sensibilizar
os parlamentares. Barbosa mostra um todo lógico e acena com a volta do
desenvolvimento calçada em uma série mais ampla de fatores: das
concessões à recuperação da capacidade de investimento do Estado.

Finalmente, há uma diferença crucial na maneira de analisar o orçamento.
Levy preferiria que as vinculações orçamentárias desaparecessem e que oi
superávit surgisse da redução dos gastos com educação e saúde. Já
Barbosa considera os gastos sociais como indissociáveis com o atual
nível de avanço político do país. Aliás, esta é a posição de Dilma, que
não permitiu que avançassem as ações visando a desvinculação.

\section{O trimestre decisivo}

Mesmo assim, Barbosa enfrentará desafios de monta pela frente.

No primeiro trimestre a crise econômica estará no auge. Haverá ainda
pressões inflacionárias e um aumento substancial das taxas de
desemprego. A~sensação de mal"-estar chegará ao auge.

Barbosa terá que atravessar esse período tendo que administrar duas
expectativas até certo ponto conflitantes.

Do lado direito, o mercado, que terá que ser convencido de sua
responsabilidade fiscal. Do lado esquerdo, os desenvolvimentistas e
movimentos sociais terão que conter a impaciência.

O primeiro bicho a ser domado será o mercado.

A ideia de que o mercado é inimigo de Barbosa só se sustenta nas
manchetes de jornais. O~mercado quer regras claras e previsibilidade. E~previsibilidade se consegue com um programa factível que exponha
claramente os custos da travessia e o cenário a ser perseguido.

A grande bronca do mercado com o primeiro governo Dilma não foi a queda
da taxa de juros. Naquele primeiro momento o mercado acreditou e iniciou
uma realocação de recursos para o longo prazo, em infraestrutura. Quando
Dilma inverteu a mão e passou a aumentar os juros, quebrou as pernas dos
que apostaram no novo ciclo.

No intervalo entre um golpe paraguaio e outro, o desafio do governo
será:

1.~~~~ Acertar o ajuste fiscal com a aprovação da \versal{CPMF} e demais
micro"-reformas fiscais anunciadas.

2.~~~~ Destravar o setor de petróleo e gás em cima da nova Lei de
Leniência.

3.~~~~ Acelerar as concessões. Nesse campo, a indicação de Valdir Simão
para o Planejamento é boa escolha. Trata"-se do melhor gestor do governo.

4.~~~~ Convencer o Banco Central a amenizar essa política monetária
suicida.

 Serão três meses de chumbo grosso. Sabendo que, completada a travessia,
a economia terá condições de reagir no segundo semestre, o \versal{PSDB} junto
com parte da mídia apostará todas as fichas na estratégia da terra
arrasada.

Mas, ao contrário de 2015, desta vez não haverá a mesma complacência com
que suas diatribes foram tratadas pela opinião pública.

Na medida em que Barbosa consiga definir com clareza sua estratégia, e
coloca"-la em prática, haverá uma reação cada vez maior da opinião
pública contra o oportunismo dos incendiários.
