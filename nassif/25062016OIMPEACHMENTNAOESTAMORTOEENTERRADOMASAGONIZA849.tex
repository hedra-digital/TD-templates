\chapterspecial{25/\allowbreak{}06/\allowbreak{}2016 O impeachment não está morto e enterrado, mas agoniza}{}{}
 

Foi um ano duríssimo e um fim de ano muito melhor que o esperado. O~desabafo é de Jacques Wagner, Ministro"-Chefe da Casa Civil.

Estava pensando no esvaziamento do impeachment.

De fato, como esperado, ontem o \versal{PSDB} de Aécio jogou Michel Temer ao mar.
Ao mesmo tempo, a ala fluminense do \versal{PMDB} procurou Temer pretendendo
apaziguar a luta e dar tranquilidade para enfrentar uma crise brava. Só
com a volta do recesso haverá mais clareza sobre a nova composição de
forças. Mas Wagner garante que a posição majoritária do \versal{PMDB} é contra a
proposta de impeachment.

Por outro lado, o voto do relator das contas presidenciais de 2014,
senador Acir Gurgacz (\versal{PDT} --\versal{RO}) foi pela aprovação das contas, com
ressalvas. E~afastou as acusações de ilegalidade nas ``pedaladas'',
defendendo que têm previsão na legislação orçamentária.

\asterisc{}

Na votação do \versal{STF}, dois pontos liquidaram com as pretensões dos
defensores do impeachment. O~primeiro, o fato do afastamento da
Presidente só se dar após decisão do Senado. O~segundo, o da votação em
aberto, que impedirá os votos de traição. Assim, a favor do impeachment
ficará apenas a oposição.

O \versal{STF} acabou com a banalização do impeachment e isso terá impacto na
economia na medida em que reduz as indefinições políticas, diz Jacques
Wagner ao \versal{GGN}.

\asterisc{}

As apostas se concentram, agora, no \versal{TSE} (Tribunal Superior Eleitoral),
onde atuam com desenvoltura a dupla Gilmar Mendes e Dias Toffoli.

Por lá, o embate é relativamente tranquilo.

A presidência e a vice"-presidência estão com dois militantes políticos,
Dias Toffoli e Gilmar Mendes.

Mas há a contraposição equilibrada dos Ministros Antônio Herman de
Vasconcellos e Benjamin, Henrique Neves da Silva e Maria Thereza Rocha
de Assis Moura. E~Luiz Fux tem se comportado com responsabilidade
institucional.

Ainda assim tem"-se a rede de segurança do \versal{STF} (Supremo Tribunal
Federal), onde irão cair todas as tentativas.

Tudo acaba no \versal{STF}, diz o Jacques Wagner. É~o \versal{STF} irá analisar as razões
jurídicas do impeachment. No caso de Fernando Collor, o impeachment foi
decorrência de uma \versal{CPI} que revelou inúmeros malfeitos e, no final, do
tal Fiat Elba.

\asterisc{}

Para Wagner, \versal{STF} e Congresso deveriam ter adiado por alguns dias o
recesso, para resolver essas pendências e trazer tranquilidade ao país.

Por exemplo, há o Projeto de Lei de Leniência aprovado no Senado e
pronto para ser aprovado na Câmara. É~uma peça central para a
recuperação da economia. Deveria ter sido aprovada antes do recesso para
permitir ao governo correr contra o relógio e recuperar setores
baleados.

Se a Sete Brasil quebrar, vai estourar nos bancos, admite Wagner. ``Uma
coisa importante é a transparência. Outra é destruir tudo o que foi
construído nas últimas décadas'', diz ele. ``Empresas são um somatório
do empresário, dos trabalhadores, técnicos e engenheiros e fazem a
síntese da inteligência nacional. São as marcas de um país, assim como a
\versal{IBM}, a Microsoft, a Volvo''. A~Medida Provisória assinado, do acordo de
Leniência, seguiu a tendência das economias mais amadurecidas.

\asterisc{}

Agora, trata"-se de começar o governo. 2015 foi inteiramente desperdiçado
na luta sem quartel pelo impeachment. A~rigor, o segundo governo Dilma
começará agora.
