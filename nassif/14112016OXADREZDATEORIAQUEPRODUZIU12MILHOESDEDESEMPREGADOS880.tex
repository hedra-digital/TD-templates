\chapterspecial{14/\allowbreak{}11/\allowbreak{}2016 O Xadrez da teoria que produziu 12 milhões de desempregados}{}{}
 

\section{Peça 1 -- ~do plano Joaquim Levy à \versal{PEC} 241}

Em 2015, mal assumiu o segundo governo, a presidente Dilma Rousseff
anunciou o plano Joaquim Levy, um enorme aperto fiscal que, segundo ela,
ajudaria a tirar o país rapidamente da crise. Em março daquele ano,
baseada nos estudos de Levy, Dilma sustentava que o pior da crise já
havia passado. Nem havia começado.

Em 2016, Michel Temer e o seu Ministro da Fazenda --- e o editorialista
da Folha -prometem que, depois da \versal{PEC} 241 virá o paraíso do crescimento
porque, graças aos cortes fiscais, haverá a redução dos juros e a
retomada do crescimento.

Sem consumo de governo (por conta da \versal{PEC} 55), sem consumo das famílias
(por conta do desemprego) e sem o impulso das exportações (por conta da
apreciação cambial), de onde viria o crescimento? Da fé cega e da faca
amolada dos cortes. Será um desastre continuado, fazendo a economia
regredir uma década.

No primeiro semestre de 2017 dirão que o pacote não deu certo porque não
foi duro o suficiente. Os crentes aceitarão que a culpa foi da sua falta
de fé. E~toca sacrificar mais empregos, produção e riqueza para seus
experimentos.

\section{Peça 2 -- a teoria que legitimou os desastres}

Em ambos os casos, de Dilma"-Levy e Temer"-Meirelles, houve a obediência
cega a teorias que surgiram nos anos 80 e 90 visando demonstrar a pouca
eficácia das políticas fiscais.

Nos anos 90, duas duplas de autores -- Giovani"-Pagano e Alesina"-Perotti
-- sistematizaram os estudos, querendo provar que aumento dos gastos
públicos não tinha nenhum efeito sobre a demanda agregada. ~Portanto, a
melhor alternativa seria efetuar grandes cortes -- com baixo impacto no
produto -- e, com isso, recuperar a confiança empresarial, despertando o
espírito animal do empresário. Tornou"-se o cabo de guerra do
neoliberalismo.

A teoria estimava os multiplicadores (o cálculo do efeito de cada
unidade gasta) para subsídios, gastos sociais, compra de ativos etc.,
com impacto aparecendo de 3 a 10 meses depois:

\begin{itemize}
\itemsep1pt\parskip0pt\parsep0pt
\item
  ·~~~~~ Benefícios Sociais: 0,8416
\item
  ·~~~~~ Ativos Fixos: 0,414
\item
  ·~~~~~ Subsídios: 1,5013
\item
  ·~~~~~ Gasto de pessoal: 0,6055
\end{itemize}

Eram esses estudos que lhe davam confiança para afirmar, em março de
2015, que o pior da crise já havia passado. Ou, então, nos anos
anteriores, para investir tão pesadamente nos subsídios. Afinal, para
cada 1 de subsídios haveria um efeito de 1,5013 no produto, em um prazo
de 3 a 10 meses. E~com cortes fiscais, haveria impacto mínimo sobre o
produto.

Seria como jogar na Loto sabendo os resultados antecipadamente.

E de nada adiantavam os alertas dos que dispõem de conhecimento empírico
da realidade econômica, que conseguem prever a rota de desastre de
teorias que ignoram a realidade econômica. Serão considerados meros
palpiteiros até que, com o desastre consumado, algum economista
consolide os erros cometidos em um paper.

\section{Peça 3 -- a identificação dos erros na teoria}

A Secretaria do Tesouro Nacional (\versal{STN}) acaba de premiar o trabalho
``Política Fiscal e Ciclo Econômico: uma análise baseada em
multiplicadores de gastos públicos'' -- de autoria de Rodrigo Octávio
Orais, Fernando de Faria Siqueira e Sergio Wulf Gobetti \mbox{---,} de onde
foram tirados os dados acima, apontando um erro crucial nos trabalhos
originais de Giovani"-Pagano e Alesina"-Perotti .

Os autores dos trabalhos iniciais montaram uma metodologia analisando a
média histórica dos indicadores. E~não se deram conta de que havia
variações fundamentais dependendo dos ciclos econômicos: quando a
economia está em expansão, o impacto dos cortes fiscais é mínimo; mas
com a economia em recessão, o impacto é significativo.

Os brasileiros refizeram, então, as séries, mas separando os resultados
da média (levantada de acordo com a metodologia em vigor), e dos
multiplicadores com a economia em expansão e em recessão. Abaixo, se tem
o raio"-x dos desastres econômicos produzidos pelo uso acrítico da
teoria.

[\versal{INCLUIR} \versal{TABELA} \versal{AQUI}]

Com a economia em expansão, há a garantia de demanda que leva o
empresário a investir. O~subsídio barateia o investimento ou o custo de
produção e ele consegue ampliar sua produção. Na recessão, sem garantia
de mercado, o empresário aproveitará os subsídios para melhorar sua
margem e fazer caixa, não para ampliar os investimentos.Dilma havia lido
apenas o trabalho anterior. O~multiplicador para subsídio era de 1,5013
na média, porque de 4,7338 em períodos de expansão. Na recessão, no
entanto, caía para 0,5972. Foi esse o resultado que explicou a falta de
impacto dos subsídios no produto em 2013 e 2014.

O segundo macro"-erro foi no pacote Levy.

Do mesmo modo, na recessão o multiplicador para benefícios sociais é de
1,5065 -- expressivo. Para compra de ativos, é mais ainda: 1,6803. Dilma
imaginava que para cada unidade de gasto em benefícios sociais, o
retorno seria de 0,8417, inferior, portanto ,ao que foi gasto. O~mesmo
para investimento em ativos fixos. Baseou"-se em dados errados.

Repare que, depois de afastada no cargo, nas sessões históricas do
Senado, Dilma invocou várias vezes o \versal{FMI} para sustentar a importância
dos gastos públicos. Ou seja, só depois de apeada do poder, tomou
conhecimento dos estudos confirmando o que os críticos diziam sobre o
desastre do plano Levy. E~Henrique Meirelles nem chegou lá ainda.

De fato, segundo os autores do estudo do \versal{STN}, o \versal{FMI} estimulou um debate
público entre 2011 e 2012 -- três a quatro anos antes do desastre do
pacote Levy --- sobre os rumos da política fiscal nas economias
avançadas e em desenvolvimento, em cima dos motes ``O que nós pensávamos
que sabíamos'' e ``O que nós aprendemos com a crise''.

O estudo do \versal{FMI}, de autoria de Blanchard, Dell'Ariccia e Mauro (2010)
sustenta que ``a política fiscal anticíclica é um importante instrumento
na conjuntura atual, dada a durabilidade esperada da recessão e o
escasso espaço de ação para a política monetária''.

As conclusões são diametralmente opostas aos enunciados do período
Levy"-Dilma e Meirelles"-Temer. Concluem que se vive um período
extraordinário no qual o gasto público tem efeitos multiplicadores
significativos e no qual ajustes fiscais convencionais podem ter efeitos
contraproducentes para o próprio objetivo de consolidação fiscal e
redução do endividamento (Romer, 2012; De Long e Summers, 2012), segundo
dados que constam do trabalho premiado.~

Concluem os autores:

\emph{``A~ luz~ desses~ parâmetros,~ por~ exemplo,~ é totalmente
inapropriado o corte de investimentos~ públicos~ realizado~ em~ 2015~ e
mantido em 2016.~ Diante~ disso,~ constituiu se~ um~ consenso~ no
mainstream, principalmente~ acadêmico, ~de que o foco da política fiscal
deveria se concentrar na sustentabilidade do endividamento público e em
regras fiscais voltadas a limitar a discricionariedade dos governos,
deixando preferencialmente para a política monetária o papel
estabilizador da demanda agregada.}

O pesado manto ideológico de que se revestiu a teoria econômica impediu
qualquer questionamento a essas supostas verdades estabelecidas. A~fé
cega nesses estudos derrubou a economia sob Dilma, contribuiu para
derrubar seu próprio governo, e continuará derrubando a economia sob
Temer. Milhões de empregos perdidos, riqueza transformada em pó, dívida
pública explodindo, receitas fiscais caindo, tudo com base na fé cega
nesses estudos.

Agora, os grandes gurus da ortodoxia -- como os economistas Afonso Celso
Pastore e Armínio Fraga -- já começam a preparar terreno, buscando
explicações antecipadas para o fato da economia não se recuperar no
próximo ano.

\section{Peça 4 -- os abusos do experimentalismo econômico}

A economia não é nem ciência exata nem universal. Mais ainda que na
medicina, exige o conhecimento teórico, mas associado à sensibilidade
para analisar as condições do paciente.

No entanto, há uma ignorância ampla e generalizada do mainstream
econômico em relação ao mundo real. Como se o conhecimento da economia
real fosse uma extravagância, acientífica, uma forma menor de
conhecimento.

Nesse mesmo período, o pacote Levy promoveu um superchoque tarifário e
cambial, simultaneamente a problemas internos de seca impactando os
alimentos. ~Imediatamente explodiu a inflação. Ao choque inicial
sucedem"-se ondas inflacionárias em diversos setores. A~lógica dizia que
bastaria os meses do choque saírem da contagem da inflação anual, para
os preços irem se acomodando e a inflação refluir.

No entanto, a visão do cabeça de planilha é incapaz de ir além da
planilha. Não entende a economia real, os impactos dos choques nas
diversas cadeias produtivas, as maneiras como ada setor reage, para
poder chegar a uma conclusão sobre a melhor posologia.

Substituem esse amplo desconhecimento pela análise exclusiva dos grandes
agregados.

É o caso da economista Mônica de Bolle, analisando a demora da inflação
em refluir. Segundo ela, o país estaria entrando na fase da dominância
fiscal, na qual os instrumentos monetários e fiscais não produzem mais
efeito deflacionário. A única saída seria vender reservas cambiais para
montar uma âncora cambial. Não dispensou um parágrafo sequer analisando
os impactos da queda de reservas na volatilidade cambial ou ao menos
estimando o que aconteceria com a inflação quando o impacto dos choques
tarifário e cambial saíssem da contagem anual.

No fim, a inflação está refluindo sem nenhuma atitude heroica.

Pior é a questão das metas inflacionárias, um sistema que drenou para os
rentistas a maior parte do orçamento público. Provavelmente, o excedente
dos juros pagos no período daria para prover toda a malha ferroviária
brasileira e grande parte do sistema de saneamento.

\section{Peça 5 -- os limites Constitucionais.}

Por todos esses fatores, o ideal seria que a sede de participação do
Judiciário o levasse a pensar em limites constitucionais para a política
econômica.

Tome"-se o caso do Banco Central. Nos Estados Unidos, o \versal{FED} é obrigado a
seguir dois objetivos: controle da inflação e preservação do emprego. No
Brasil, apenas o controle da inflação.

Como não tem em suas mãos os instrumentos fiscais, o \versal{BC} joga todo o peso
em juros estratosféricos, que arrebentam com a atividade econômica, sem
nenhuma preocupação com os impactos sobre o produto e o emprego.

Para fazer demagogia de baixo risco, a presidente do \versal{STF} (Supremo
Tribunal Federal) Carmen Lúcia afirmou que não é Ministra da Fazenda,
para avaliar o impacto de medidas judiciais na economia.

Seria mais consistente se, junto com seus colegas, definissem limites
constitucionais ao experimentalismo da política econômica e aos abusos
das politicas fiscal e monetária.
