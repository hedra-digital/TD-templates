\chapterspecial{20/\allowbreak{}06/\allowbreak{}2016 Xadrez de Temer e o caso dos 10 negrinhos}{}{}
 

Lembrando o poema ``O caso dos dez negrinhos''

 

\epigraph{Cinco homens no comando\ Irmanados em um trato\ Levaram Eduardo
Cunha\ Restaram apenas quatro.\ Quatro homens planejando\ A grande
jogada da vez\ Levaram Romero Jucá\ Agora, só restam três.\ Três
homens bem assustados\ Com o que a Justiça expôs\ Chegou a vez do
Padilha?\ Restarão apenas dois.}{} 

 

 

No site de Eliseu Padilha, um rock mambembe conta a vida do ``Padilha
subindo a ladeira'', com ``a chama no peito e de mãos dadas com a
esperança brasileira''.

A ladeira rima com um cinco meia meia, seu número de candidato a
deputado federal. No meio da música, relatos de Padilha menino
``descendo a ladeira'', procurando ``faturar um trocado para aumentar o
quinhão'' e ``querendo subir na vida''.

Subiu. Tornou"-se Ministro"-Chefe interino da Casa Civil e homem mais
poderoso da República, posto que o braço direito de um presidente
vacilante.

Na condição de Ministro"-Chefe da Casa Civil, Eliseu Padilha mapeou os
três pilares da frente que construiu o impeachment: o Judiciário, a base
política e a imprensa.

Para o Judiciário, ofereceu aumentos de proventos. Para a base política,
cargos e lambança. Para a mídia, toda a publicidade da Secom (Secretaria
de Comunicação). E, para o país, ofereceu uma das biografias mais
polêmicas da República.

Nos governos anteriores, a Secom servia para filtrar a publicidade,
especialmente a que era direcionada para veículos fora do circuito da
grande mídia. Pelo menos até o ano passado, trabalhou muito mais para
podar campanhas do que para estimular. Mas a decisão da publicidade
ficava com cada empresa pública.

Com o interino, a Secom passa a centralizar toda a publicidade.

Isso lhe confere um poder de pressão inédito sobre os veículos de mídia.
Nenhum grupo privado conseguirá fazer frente à soma combinada da
publicidade da Petrobras, Banco do Brasil, Caixa Econômica Federal,
\versal{BNDES}, Ministérios. Em um momento de crise dos grupos de mídia, trata"-se
de uma ameaça direta à liberdade de expressão, na medida em que
constrangerá inclusive os grandes veículos.

Neste fim de semana, alguns veículos noticiaram a condenação de Padilha
a devolver R\$ 300 mil ao Tesouro, devido à contratação de uma servidora
fantasma.

O caso da funcionária fantasma é o menor envolvendo Padilha.

A seguir, um breve resumo da carreira pública de Padilha.

\section{Caso \versal{DNER}}

O presidente Fernando Henrique Cardoso devia dois favores a Eliseu
Padilha. O~primeiro, a operação empreendida por ele e por Gedel Vieira
Lima, visando impedir a candidatura de Itamar Franco pelo \versal{PMDB}. A~segunda, seu papel na compra de votos para a emenda da reeleição.

Como pagamento, Padilha foi nomeado Ministro dos Transportes em maio de
1997, permanecendo até novembro de 2001.

Saiu no meio de um mega escândalo de corrupção no \versal{DNER} (Departamento
Nacional de Estradas de Rodagem) denunciado pelo então senador Antônio
Carlos Magalhães. Ficou conhecido como o escândalo da Máfia dos
Precatórios. Consistia em furar a fila dos precatórios do \versal{DNER} e
superfaturar os valores devidos, mediante o pagamento de propinas.

O caso foi investigado internamente, tanto pela Secretaria Federal de
Controle Interno como pelo Tribunal de Contas da União. Mas \versal{FHC} abafou o
escândalo. Suas únicas providências consistiram em demitir Padilha e
acabar com o \versal{DNER}, substituído pelo \versal{DNIT} (Departamento Nacional de
Infraestrutura de Transportes) que permaneceu no governo \versal{FHC}, nos de
Lula e Dilma o maior centro de corrupção da União -- segundo o
depoimento insuspeito de Sérgio Machado.

Em 2002 o Ministério Público Federal abriu inquérito para apurar
irregularidades no Ministério dos Transportes. Chegou"-se até uma certa
Nova Agência de Automóveis Ltda.

Segundo o processo (\url{migre.me/\allowbreak{}u9h8g)}, Ulisses José Ferreira
Leite ~recebeu mais de dez milhões em suas contas pessoais, cerca de um
décimo dos desvios do \versal{DNER}. Ao lado de Geraldo Nóbrega e de Olívio
Moacir Padilha tornou"-se sócio da firma Nova Agência de Automóveis Ltda.
Sua parte no negócio consistia em presentear funcionários e o próprio
Ministro com automóveis de alto valor.

Nos depoimentos, constatou"-se que a própria esposa de Padilha foi uma
das presenteadas com automóvel de luxo.

Foi apenas uma das evidências da participação estreita de Padilha no
escândalo.

Em 2003 o \versal{MPF} ajuizou a ação contra Padilha. Apenas em 2013 a ação foi
aceita pela 6\textsuperscript{a}~Vara Federal do \versal{DF}. A~demora se deveu à
questão de competência: a Justiça federal remetendo ao Supremo que
devolveu à 6\textsuperscript{a}~Vara (\url{migre.me/\allowbreak{}u9h\versal{SH})}.

Na ação, o \versal{MPF} apontou Padilha como lobista, que usava o cargo para
atender ``pleitos políticos para pagamentos absolutamente ilícitos e
ainda por cima superfaturado.

Em um dos casos, o \versal{DNER} usou por 82 dias o prédio da empresa Comércio,
Importação e Exportação 3 Irmãos. Pelo aluguel, a empresa teria direito
a R\$ 185 mil. O~\versal{DNER} acertou um acordo extrajudicial elevando o valor
para R\$ 2,3 milhões. ~``Bastaria o bom senso para compreender que 82
dias de uso de um prédio \redondo{[…]} não poderia custar quase o valor
do prédio'', segundo o procurador Luiz Francisco Fernandes de Souza,
autor da ação. (\url{migre.me/\allowbreak{}u9ihE})

Em documentos e depoimentos, Padilha foi apontado como mandante, a
pedido do ex"-deputado Álvaro Gaudêncio Neto, cujo pleito foi encaminhado
ao Ministério por Eduardo Jorge, o influente assessor especial de \versal{FHC}.

O procurador do \versal{DNER} Pedro Elói Soares~denunciou Padilha pelas
falcatruas. E~a própria Advocacia Geral da União (\versal{AGU}) divulgou
relatório apontando prejuízo de R\$ 122,9 milhões com as fraudes. O~documento afirmava expressamente que Padilha tinha conhecimento das
irregularidades, assim como o consultor Arnoldo Braga Filho.

A ação do \versal{MPF} apresentou ofício do assessor especial de Padilha, Marcos
Antônio Tozzatti, pedindo ``a maior brevidade possível'', por ``ordem do
excelentíssimo senhor ministro dos Transportes, Eliseu Padilha''.

Em sua defesa, Padilha afirmou ter sido vítima de irregularidades
cometidas por funcionários do terceiro escalão.

As denúncias de \versal{ACM} levaram a recém"-criada Corregedoria Geral da União à
sua primeira investigação: justamente as aventuras de Padilha nos
Transportes.

O governo havia instalado uma comissão para analisar as irregularidades
no \versal{DNER}. Mas, das 46 irregularidades constatadas, apenas uma foi
analisada, justamente para suspender as punições contra dois
funcionários.

Em vista disso, a corregedora Anadyr de Mendonça Rodrigues levantou
irregularidades na desapropriação de uma área em Sinop (\versal{MT}) e produziu
um documento com 14 páginas, enviado pessoalmente a Fernando Henrique.
Segundo Anadyr, a comissão de inquérito passou por cima de 46 processos
conexos que tratavam de desapropriação (\url{migre.me/\allowbreak{}u9ize})

E aí \versal{FHC} não teve como se abster.

O presidente chamou Padilha em seu gabinete e ordenou"-lhe que reabrisse
as investigações.

Em vez de punir os responsáveis, \versal{FHC} tirou o sofá da sala: extinguiu o
\versal{DNER} e substituiu"-o pelo \versal{DNIT}

A decisão de \versal{FHC} provocou forte reação de \versal{ACM}

``Tenta"-se calar a grande imprensa, mas V.Exa. bem sabe que não se pode
silenciar por todo o tempo a consciência cívica do país. A~pura e
simples extinção do \versal{DNER}, como foi a da Sudam e a da Sudene, é
insuficiente, pois o necessário é pegar os ladrões do erário''.

Nos anexos está a Ação Cautelar de Improbidade do \versal{MPF} contra Padilha e
outros para ser consultado por vocês, para me ajudarem a identificar
pessoas e parcerias.

\section{A Operação Solidária}

Em 2007 a Polícia Federal decidiu investigar a terceirização do
fornecimento de merenda escolas em Canoas \versal{RS}), na administração Marcos
Ronchetti (\versal{PSDB}). O~seu lema de campanha era ``administração
solidária'', vindo daí o nome da operação.

Descobriu fraudes em licitações para obras de saneamento, construção de
estradas e de sistemas de irrigação (\url{migre.me/\allowbreak{}u9dwc)}

A empresa"-chave na operação era a \versal{MAC} Engenharia, de Marco Antônio
Camino, apontado como o operador do esquema. As escutas identificaram
várias chamadas de Carmino para Padilha. Investigações da \versal{PF} e do \versal{MPF}
indicaram depósito de R\$ 267 mil da \versal{MAC} na conta da Fonte Consultoria
Empresarial, de Padilha e de sua esposa.

O inquérito total tinha sete volumes, dos quais seis baseados na
operação de escuta. A~Operação Solidária apontou indícios de fraude que
chegaram a R\$ 300 milhões, em valores da época.

As investigações envolveram ainda o ex"-presidente da Assembleia
Legislativa, Alceu Moreira (\versal{PMDB}) e o Secretário Estadual de Habitação,
Saneamento e Desenvolvimento Urbano (\versal{PMDB}), todos da gestão Yeda
Crusius.

Todas as licitações importantes de Canoas passavam por Chico Fraga,
Secretário"-Geral da Prefeitura. A~\versal{PF} apurou ligações estreitas entre
Chico a Padilha. Ambos tiveram grande influência no governo Yeda
Crusius. E~Padilha empregou em seu gabinete, mas lotada em Porto Alegre,
a esposa de Chico Fraga, Maria Dolores Fraga,

Essa contratação originou outro inquérito, sobre a contratação da
servidora fantasma -- divulgada esta semana pela imprensa.

A quadrilha atuou também em obras do Estado. Segundo matéria do jornal
Zero Hora, a Secretária Adjunta da Secretaria Estadual de Obras
Públicas, Rosi Bernardes, foi apontada como suspeita de repassar
informações privilegiadas sobre licitações. O~projeto era a principal
obra do programa de irrigação do governo do Estado.

\section{A Operação Rodin}

Não se ficou no roubo de merenda escolar. As investigações constataram o
elo entre a Operação Solidária e a Operação Rodin, que desviou R\$ 44
milhões do \versal{DETRAN} gaúcho.

De acordo com a \versal{PF}, Padilha e o também deputado Otávio Germano (\versal{PP}"-\versal{RS})
passaram a Carmino informações privilegiadas sobre recursos do \versal{FAT},
\versal{BNDES} e \versal{DNIT}. E~montaram um esquema para desviar recursos das obras do
\versal{PAC}.

Reportagem da IstoÉ, em 25/\allowbreak{}03/\allowbreak{}2009, relatava que em uma das conversas
Camino dizia a Padilha: ``Aquele assunto que nós tratamos na terça feira
será viabilizado 100, tá?''

Segundo a \versal{PF}, a quadrilha se valia de códigos nas licitações para
direcionar as obras para empresas ligadas ao grupo.

Em outra conversa, Camino convida Padilha a visitar a empresa e
manifesta interesse em licitação na Secretaria de Irrigação.

Segundo alegações dos advogados de Padilha, os R\$ 100 mil recebidos
seriam provenientes da compra de uma casa e os R\$ 267 mil não seriam da
\versal{MAC} mas da Magna (\url{migre.me/\allowbreak{}u9dcx)}. O~argumento era frouxo.
A~Magna Engenharia também havia sido indiciada pela \versal{PF} por participação
no esquema.

Os diálogos gravados mostravam acertos de Padilha com autoridades
estaduais, para direcionar os editais de duas barragens para a quadrilha
(\url{migre.me/\allowbreak{}u9dRq}).

No dia 14 de fevereiro de 2011 Padilha foi indiciado por crime em
licitações e formação de quadrilha, após prestar depoimentos na sede da
\versal{PF} em Porto Alegre (\url{migre.me/\allowbreak{}u9dG6})

\section{O arquivamento das ações}

O inquérito terminou anulado no \versal{STF} (Supremo Tribunal Federal) em agosto
de 2014 porque, ao pedir autorização para o grampo, na 1a Instância, a
\versal{PF} descuidou"-se em relação ao foro de Padilha, no cargo de deputado
federal.

Padilha assumiu o cargo de deputado em fevereiro de 2007, mas a primeira
instância só remeteu o caso ao Supremo em junho de 2008. Devido a esse
detalhe, todas as provas colhidas foram anuladas. A~autorização teria
que ter sido concedida pelo próprio \versal{STF}. Marco Aurélio considerou que o
foro deve ser utilizado inclusive na fase do inquérito
(\url{migre.me/\allowbreak{}u9d7X)}.

Em dezembro de 2014 foi arquivado o segundo inquérito, da contratação da
funcionária fantasma.

Uma das medidas propostas pelo Ministério Público Federal visa
justamente evitar a anulação completa de processos devido ao chamado
``fruto da árvore proibida''.

\section{Consequências}

Padilha se safou dos processos meramente por erros processuais.

Mas como fica politicamente? Todas as informações acima foram levantadas
da Internet. São informações públicas, que constam de inquéritos da \versal{PF},
processos do \versal{MPF}, sentenças do \versal{STF}, relatórios da \versal{AGU}.

Padilha não é mais o deputado obscuro montando jogadas e safando"-se
milagrosamente de processos. Agora, é o homem forte da República. No
Executivo, acima dele há apenas o Presidente interino Michel Temer, de
quem é carne e unha.

Por suas mãos passam, agora, todas as demandas políticas e ele é a voz
de comando sobre todo o Ministério.
