\chapterspecial{06/\allowbreak{}11/\allowbreak{}2016 Xadrez da reconstrução do Brasil}{}{}
 

O \versal{PSDB} de Sérgio Motta e Mário Covas tinha um projeto de poder de 20
anos, porque pensava em um projeto de Nação. Apostava que a estabilidade
econômica e a pacificação política pós"-real permitiriam desenvolver um
conjunto de políticas legitimadoras que garantissem o poder do partido.

O projeto dançou com a inoperância do governo Fernando Henrique Cardoso
e com sua falta de visão política sobre as novas bandeiras a serem
içadas.

O \versal{PT} de Lula entrou, então, com um plano de 20 anos, porque ancorado em
um projeto de Nação. Desenvolveu políticas sociais, políticas
industriais, avançou nas políticas educacionais e
científico"-tecnológicas.

Dançou ao não perceber os novos tempos de militância em redes sociais,
ao não conseguir se desvencilhar das políticas tradicionais e não dispor
de uma estratégia para o segundo governo Dilma.

O grupo que assumiu o poder não tem projeto algum.

Na ponta do \versal{PMDB} montou o maior assalto ao poder desde o governo Sarney.

Na ponta do mercado, um grupo de economistas que se move exclusivamente
por bordões ideológicos e visão de curtíssimo prazo. São a contrapartida
do mercado à superficialidade desenvolvimentista de um Guido Mantega.

Já o ex"-\versal{PSDB} tornou"-se um partido de direita tosca, perdendo qualquer
capacidade de planejamento do futuro. Seus formuladores ou se
aposentaram ou desistiram do partido. As lideranças atuais só conseguem
articular o discurso da intolerância e do antipetismo.

Não há poder que sobreviva sem um projeto de Nação.

Esse quadro lança o seguinte quebra"-cabeças para os próximos anos.

\section{Peça 1 -- a crise e um governo sem projeto}

Há muito tempo, um dos principais déficits do país é no estudo da
economia.

Para manter os privilégios do rentismo, o establishment econômico trocou
as visões sistêmicas, a análise de realidades complexas, por
simplificações absurdas e visões de curtíssimo prazo, atropelando
princípios básicos de economia. Como a ideia de que um violento choque
fiscal pró"-cíclico, em meio a um quadro recessivo, permitiria em poucos
meses a volta do crescimento -- brandido por Joaquim Levy, o coveiro de
Dilma. Ou, agora, a superstição que bastará um teto nas despesas
primárias, deixando de lado os juros, a demanda e as relações sociais e
políticas, para devolver a confiança aos investidores.

No médio prazo, é tão draconiano que não se sustentará. É~um pacote que
não tem viabilidade econômica, nem apelo eleitoral.

No curto prazo, ajuste fiscal e política monetária restritiva, em uma
economia em depressão, apenas produzirão mais depressão.

O mercado tem as seguintes fontes de dinamismo:

·~~~~~ Gastos públicos.

·~~~~~ Demanda externa.

·~~~~~ Novos investimentos.

Com a \versal{PEC} 241, esqueçam"-se os gastos públicos.

Com a taxa de juros puxando o câmbio para baixo, mata"-se a possibilidade
de estimular a economia via exportações.

Com os juros reais em alta (devido à queda da inflação), e com a
capacidade ociosa da economia (em função da recessão) o único
investimento que acontecerá será na compra de ativos públicos e privados
depreciados pela crise.

O país entrará em 2018 com a economia se arrastando, sendo tocado por um
governo ilegítimo, sem dispor de nenhum projeto de país.

Portanto, reduzam as estimativas de dez anos de predomínio da direita.
Ela fenece antes disso.

\section{\textbf{Peça 2 -- o acervo de projetos da era Lula"-Dilma}}

Por outro lado, em que pese os desastres dos últimos anos, os governos
Lula e Dilma lograram desenvolver um conjunto de novas políticas
públicas exitosas, industriais, de parcerias com confederações para
estímulo à inovação, de avanços na educação, de estratégias
diplomáticas, além das políticas sociais.

Como já foram utilizadas com sucesso, bastará a recuperação da história
recente do país para se voltar a apresentar um projeto de país ao
eleitorado. E, daqui a dois anos, o bicho"-papão não será mais o \versal{PT}, mas
a camarilha de Temer.

Algumas das ideias do período que necessitarão ser recuperadas:

Novas políticas industriais

A ideia de juntar políticas de bem estar -- como educação e saúde -- com
compras públicas que estimulem um complexo industrial, cujo momento
maior foram as negociações com multinacionais para a transferência de
tecnologia de medicamentos adquiridos pelo \versal{SUS} para laboratórios
públicos e nacionais privados,.

As parcerias com a \versal{CNI} para aparelhamento de laboratórios de
universidades federais, em torno de projetos factíveis de
desenvolvimento de novos produtos e de apoio a setores competitivos.

A tentativa de transformar a Finep em articuladora das Fundações de
Amparo à Pesquisa para financiamento de startups e de setores de
tecnologia de ponta, mapeados pelos conselhos empresariais reunidos na
\versal{ABDI} (Agência Brasileira de Desenvolvimento Industrial), infelizmente
abandonados na gestão Dilma.

A recuperação do projeto da Petrobras como centro de um complexo
químico"-industrial, um dos grandes feitos de Dilma.

Novas políticas educacionais

As mudanças com a nova base curricular, os avanços nas conferências
nacionais, no desenho de um Plano Nacional de Educação acordado com
professores, secretarias de educação e \versal{ONG}s privadas, o Fundeb, a
identificação de projetos bem"-sucedidos para disseminação pelas escolas,
tudo isso significaram décadas de avanço sobre as baboseiras de
pretensos especialistas pretendendo tratar a educação como uma
Olimpíada, com cada escola disputando isoladamente os indicadores de
avaliação.

A integração de políticas

A parceria interministerial para políticas sociais, como o Brasil
Carinhoso, o embricamento do Bolsa Família com o Ministério da Educação.

Os modelos de participação

Embora deixados de lado por Dilma, a figura dos conselhos da sociedade
civil -- tanto na área social quanto empresarial -- são figuras
jurídicas maduras, testadas, e que deverão constar da plataforma de
qualquer candidatura progressista.

\section{Peça 3 --- Os avanços sociais e o governo sem projeto}

Há um país moderno, contemporâneo, que transcende o \versal{PT} e penetra também
nos setores mais arejados das grandes metrópoles e no próprio
empresariado mais antenado com a contemporaneidade.

Essas bandeiras estão a quilômetros de distância do eixo \versal{PSDB}"-\versal{PMDB}, que
definitivamente se associou ao que existe de mais anacrônico no campo
dos costumes.

A ideia de um poder de direita se sustentando por décadas não resiste à
falta de projetos do grupo. Com Lula ou sem Lula, dificilmente terá uma
plataforma competitiva para 2018.

A arma à qual recorrerá o grupo de poder será a radicalização política,
a tentativa de aprofundar o estado de exceção.

A resposta política não poderá ser mais radicalização, mas o exercício
diuturno da política, buscando um arco amplo de alianças, que não fique
restrito aos grupos de esquerda, mas a todo um espectro de forças
modernas, antenadas com a contemporaneidade.

\section{Peça 4 -- a construção da oposição}

O caminho passará pela construção de pontes com grupos internos
comprometidos com os valores democráticos em cada área de poder, que
possam exercer papel de liderança e de influência no seu meio.

As mudanças no Judiciário, Ministério Público, Polícia Federal só se
concretizarão com alianças com os setores internos de cada instituição,
hoje calados pela maré obscurantista que tomou conta do país.

Há um conjunto de temas a ser trabalhado, dentro da reconquista de um
novo projeto de país, aprendendo em cima dos erros cometidos:

Gestão pública

Na ponta da gestão, a responsabilidade maior caberá ao conjunto de
governadores progressistas -- de Minas, Bahia, Piauí, Maranhão, Ceará e
Acre \mbox{---,} estimulando práticas participativas, recuperando os conceitos
de políticas de desenvolvimento. E, principalmente, disseminando os
conceitos de forma competente nas redes sociais.

Papel relevante pode ser assumido pelos técnicos que ajudaram a
implementar um conjunto amplo de políticas sociais inovadoras no governo
Fernando Haddad.

O governo Dilma e, principalmente, o de Haddad, comprovam que sem uma
venda política eficiente de um projeto de governo, as melhores
administrações tendem a naufragar.

Economia e política

No campo econômico, o questionamento radical -- e didático --- da
política monetária do Banco Central, assim como da liberdade absoluta
dos fluxos de capital, algo que nenhuma administração do \versal{PT} ousou
afrontar. Principalmente após a \versal{PEC} 241, ficará cada vez mais claro que
o problema econômico central do Brasil é a carga de juros, sustentada
pela falsa ciência das metas inflacionárias.

Em algum momento, será necessário algum consenso entre as principais
escolas de pensamento heterodoxo, Institutos de Economia da Unicamp e da
Universidade Federal do Rio de Janeiro, Escola de Economia da Fundação
Getúlio Vargas, mas em um plano interdisciplinar, com os institutos de
análise política, como o \versal{IUPERJ}, a \versal{UFMG} e outros, para consolidar no
plano teórico o verdadeiro normal da economia.

É um desafio para a academia.

Mídia

No campo da comunicação, o questionamento amplo da cartelização da
mídia, do sistema de concessões públicas.

Mais do que nunca, haverá a necessidade de uma regulação da mídia, não
apenas no campo econômico, mas também em temas diretamente ligados aos
direitos sociais e individuais: como o direito de resposta e a
necessidade de pluralidade nas \versal{TV}s abertas e rádios.

Não será desafio fácil. Não poderá ser uma agência centralizada que
possa, mais à frente, instituir formas de censura.

O caminho passa pela diversificação das fontes de notícias, da entrada
dos grupos estrangeiros -- que hoje em dia já produzem o melhor
jornalismo no Brasil -- à organização de alternativas internas das
centrais sindicais, Igreja, coletivos, conselhos, que terão que se
organizar para se tornarem produtores de informação.

Segurança pública

No campo da segurança pública, o enquadramento dos órgãos de repressão,
especialmente a Polícia Federal e as Polícias Militares, devolvendo às
Forças Armadas seu papel de defesa contra o inimigo externo,
envolvendo"-a prioritariamente com tecnologia, defesa aeroespacial e
marítima.

Trata"-se de um desafio que passa pela formação profissional, pelas
formas de ingresso e de ascensão na carreira. Terá que haver uma
aproximação com setores democráticos dentro dessas corporações

Sistema judicial

No campo institucional, uma discussão ponderada e firme sobre os órgãos
de controle, acabando com a ampla subversão de poderes, nos quais se tem
um Tribunal de Contas opinando sobre políticas fiscais, procuradores
avançando sem controle algum (nem externo nem interno) sobre todos os
temas, tribunais superiores sem~\emph{accountability.}

Não se pode coibir os poderes, mas submetê"-los a formas de prestação de
contas para a sociedade civil, em parceria com os grupos internos mais
arejados.

Sistemas participativos

Há toda uma estrutura institucionalizada de organização da sociedade
civil, através dos conselhos sociais representando os diversos setores,
desde os temáticos -- saúde, educação, segurança -- até os de minorias
-- negros, idosos, juventude, criança e adolescente, criança com
deficiência.

Especialmente, há as universidades, institutos de pesquisa e a extensa
rede de escolas secundárias, como centros de discussão e de formação
gradativa de consensos. E~há as redes sociais e a Internet
possibilitando fóruns de discussão nacionais.

\section{Peça 5 -- desenho do futuro}

Falta um partido ou frente política dando consistência a esse
arquipélago de movimentos modernizadores, dessa musculatura da sociedade
civil.

Mesmo assim, há quantidade e variedade de grupos de interesse,
espalhados por todo o país, se constituirão em trincheiras contra os
esbirros autoritários da camarilha de Temer apoiada pela mídia.
Dificilmente esses atentados ao Estado de Direito conseguirão prosperar.

Há instrumentistas disponíveis em todos os campos sociais, políticos e
econômicos, jurídicos e policiais, uma massa crítica considerável. Falta
um maestro ou estrutura que dê organicidade à orquestra.

Como se trata de uma sociedade viva, nos próximos anos se verão
movimentos de aproximação entre esses grupos, de busca de identidades
comuns, até haver massa crítica para a constituição de um partido ou
frente que acomode todas essas forças modernizantes.

Mesmo que a colheita dure um pouco, o plantio está começando de forma
vibrante.
