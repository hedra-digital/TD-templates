\chapterspecial{25/\allowbreak{}06/\allowbreak{}2016 O xadrez da política no dia D}{}{}
 

Vamos ao novo xadrez da crise.

\textbf{Peça 1}~-- a crise ficou grande demais para Dilma

É a única certeza nesse oceano de imprevisibilidades que caracteriza a
crise atual. Dilma não tem fôlego político nem para lançar planos mais
audaciosos nem para recompor sua base política. Mantido o quadro atual,
se não cair por impeachment, cai pela crise.

\textbf{Peça 2}~-- qualquer solução de conflito mergulha o país em uma
crise imprevisível.

Essa premissa é central para todo o raciocínio posterior. Não significa
que, automaticamente, conduzirá as discussões para a racionalidade. Mas
será um fator relevante a estimular algumas lideranças mais responsáveis
na busca do entendimento.

Grosso modo, há dois grupos trabalhando em saídas mais articuladas para
a crise.

No Senado, o grupo formado por Renan Calheiros, José Serra e Romero
Jucá, articulando alguma forma de semiparlamentarismo que mantenha Dilma
Rousseff na presidência, mas sem governar. Vamos trata"-lo de
os~\textbf{Parlamentaristas}~para facilitar a leitura.

Ao largo, o grupo que cerca Lula, insistindo para que assuma um cargo de
coordenação no Palácio, mesmo sem ser formal, mas que o transforme em um
primeiro"-ministro de fato. Chamemos de~\textbf{Lulistas}.

Correndo por fora, o grupo do impeachment, com Aécio Neves na ponta.
Seriam os~\textbf{Jacobinos}.

Finalmente, o grupo do Ministério Público Federal diretamente liderado
pelo Procurador Geral Rodrigo Janot. Vamos batizar de~\textbf{Alto
Comando}, para fugir da confusão corriqueira, de considerar que o
comando e a estratégia da Lava Jato estão em Curitiba.

São esses personagens que jogam o jogo atual, cujo ápice serão as
manifestações deste domingo.

\section{As formas de jogo político}

Para acompanhar o jogo é preciso entender melhor sua natureza.

Não se trata de uma conspiração palaciana, com um comando organizando
todas as ações.

Movimentos de opinião pública são operações muito mais fluidas, mais
amplas, nas quais se escolhe o momento adequado -- o mal"-estar econômico
-- e, se deflagra um conjunto de ações visando estimular as reações
populares. A~denúncia da corrupção é o mote mais eficaz.

Aberta a porteira, provoca"-se o estouro da boiada e abre"-se a caixa de
Pandora. Há uma sucessão de eventos, alguns aleatórios, outros
planejados. A~arte da conspiração consiste em controlar os bois guias,
os que vão na frente da boiada conduzindo"-a. Mas o final sempre é
imprevisível, daí a preocupação de Fernando Henrique Cardoso e de
quadros do \versal{PSDB}, recuando na radicalização.

O estouro da boiada foi possível com a parceria montada pelo~Alto
Comando~com a mídia, a entrada dos novos grupos que se apossaram das
manifestações (Movimento Brasil Livre, Revoltados Online, provavelmente
bancados de fora), e um investigação capaz de gerar fatos jornalísticos
diários.

Hoje em dia, quem controla os bois guias é o~Alto Comando, através da
usina de geração de fatos da Lava Jato, sincronizando com os movimentos
da oposição.

Os protagonistas a serem acompanhados são, portanto,
os~Parlamentaristas, os~Lulistas~e oAlto Comando. Os~Jacobinos~de Aécio
e a~mídia~são agentes acessórios -- no caso da mídia, fundamental para o
sucesso da operação, mas vindo a reboque, sem papel na formulação
estratégica,

A dificuldade de definição de estratégias se deve à extrema habilidade
de um jogador essencial, oAlto Comando,~que conseguiu jogar xadrez
escondendo o rei. É~uma velha gíria do xadrez: como a vitória consiste
no xeque"-mate ao rei, se você esconde o seu no tabuleiro, não tem como
levar xeque.

Quando os demais personagens entenderem adequadamente o papel do~Alto
Comando, os erros de estratégia serão minimizados.

\section{Como se organiza o jogo}

Se consumado o impeachment de Dilma Rousseff, será
um~\emph{case}~mundial, provavelmente a mais bem"-sucedida estratégia de
golpe político das últimas décadas.

Não é o caso de voltar ao tema da geopolítica norte"-americana na quadra
atual. Maiores dados vocês poderão ler aqui
(\url{migre.me/\allowbreak{}tdbtp)}. A~estratégia de desmonte dos grandes
grupos nacionais que poderiam se habilitar a algum protagonismo externo
pode ser lida aqui (\url{migre.me/\allowbreak{}tdb\versal{AZ})}.

Há duas vertentes para dobrar a espinha do país.

A primeira, que dá o start, é a política de depreciação continuada de
tudo que possa despertar o orgulho nacional. Esse trabalho ficou nítido
na Copa do Mundo, um exercício tão funesto de derrubar a autoestima que
conseguiu espantar das ruas até o orgulho de vestir camisa da Seleção. E~isso antes do 7 x 1 e pouco tempo depois do país ter atingido o momento
mais alto do seu orgulho, respeitado mundialmente pelos avanços sociais
registrados e pela forma como superou a crise de 2008.

A segunda vertente foi o papel do~Alto Comando~como estrategista central
da Lava Jato.

Do lado jurídico, a maneira como a Lava Jato foi montada foi bem
explicada pelo advogado Juarez Cirino dos Santos no site Jota
(\url{migre.me/\allowbreak{}td3\versal{XB}}).

\emph{4. Além de constrangimentos e humilhações aos adversários
políticos, a Operação Lava Jato apresenta inúmeras vantagens
\redondo{[…]}:}

\emph{- primeiro, os procedimentos investigatórios e os processos
criminais são seletivos e sigilosos: seletivos, porque dirigidos contra
líderes do \versal{PT} ou pessoas/\allowbreak{}empresas relacionadas ao Governo do \versal{PT} -- por
motivos ideológicos ou não; sigilosos, porque não permitem conhecer a
natureza real ou hipotética dos fatos imputados, fazendo prevalecer a
versão oficial desses fatos, verdadeiros ou não;}

\emph{- segundo, os nomes dos investigados são revelados ao público
externo, como autores ou partícipes (por ação ou omissão) das hipóteses
criminais imputadas, mediante programados vazamentos de informações
(sigilosas) aos meios de comunicação de massa, com efeitos sociais e
eleitorais devastadores sobre os adversários políticos dos grupos
conservadores;}

\emph{- terceiro, o espetáculo de buscas e apreensões violentas e de
condução coercitiva ilegal de investigados (o ex"-Presidente Lula, por
exemplo), ou as ilegais quebras de sigilo (telefônico, bancário e
fiscal) seguidas de espalhafatosas prisões preventivas (Zé Dirceu ou
João Vaccari Neto, por exemplo), geram convenientes presunções de
veracidade e de legitimidade da ação repressiva oficial perante a
opinião pública.}

\emph{5. Nesse contexto, a contribuição objetiva da Operação Lava Jato--
voluntária ou não, mas essencial para os fins político"-eleitorais das
classes hegemônicas organizadas no \versal{PSDB}, no \versal{PPS}, no \versal{DEM} e outras siglas
-- ocorre na forma de contínua violação do devido processo legal, com o
espetacular cancelamento dos princípios do contraditório, da ampla
defesa, da proteção contra a autoincriminação, da presunção de inocência
e outras conquistas históricas da civilização -- apesar da reconhecida
competência técnico"-jurídica de seus protagonistas. A~justiça criminal
no âmbito da Operação Lava Jato produz a sensação perturbadora de que o
processo penal brasileiro não é o que diz a lei processual, nem o que
afirmam os Tribunais, menos ainda o que ensina a teoria jurídica, mas
apenas e somente o que os dignos Procuradores da República e o ilustre
Juiz Sérgio Moro imaginam que deve ser o processo penal. A~insegurança
jurídica e a falta de transparência dominante na justiça criminal da
Operação Lava Jato levou o Ministro Marco Aurélio, do Supremo Tribunal
Federal, a reproduzir antigo conceito de Rui Barbosa: ``a pior ditadura
é a ditadura do Poder Judiciário''.}

\emph{6. Então, entra em ação o grande parceiro da Operação Lava Jato:
os meios de comunicação de massa (\versal{TV}, jornais e rádios),~ com
informações baseadas nas evidências processuais ou no material
probatório obtido nas condições referidas, produzem um espetáculo
midiático para consumo popular -- e comícios diários de imagens virtuais
audiovisuais, impressas e sonoras tomam conta do País, com efeitos
psicossociais coletivos avassaladores. As versões, interpretações e
hipóteses da justiça criminal da Operação Lava Jato, difundidas pela
ação repressiva da Polícia Federal, pelas manifestações acusatórias dos
Procuradores da República e pelas decisões punitivas do Juiz Sérgio
Moro, produzem efeitos de lavagem cerebral e de condicionamento
progressivo da opinião pública, submetida ao processo de inculcação
diuturna de um discurso jurídico populista, com evidente significado
político"-partidário, mas apresentado sob aparência ilusória de uma
impossível neutralidade política.}

Quando os procuradores paulistas tentaram atropelar a agenda, coube ao
Procurador Geral da República Rodrigo Janot articular pessoalmente a
estratégia da Lava Jato em relação às trapalhadas cometidas
(\url{migre.me/\allowbreak{}tde\versal{QT})}. E~a toda imprensa vocalizar as críticas
contra quem poderia comprometer o script inicial, cuidadosamente
planejado para chegar a bom termo respeitando as aparências jurídicas.

Do lado político, o~Alto Comando~opera a partir de Brasília visando
criar toda a blindagem jurídica necessária, não apenas junto ao \versal{STF},
como ao próprio governo e nas redes sociais.

No Twitter, por exemplo, os principais lugares"-tenentes de Janot,
através de seus perfis pessoais, conduzem uma ampla campanha de
esclarecimento e de defesa da Lava Jato. Antes da constatação de que foi
um desastre, até as trapalhadas dos procuradores paulistas mereceram
esboços de defesa, por parte dos procuradores de Janot.

No \versal{STF} e no \versal{TSE} Janot não convalidou nenhuma tentativa de golpe branco.
 Consolidou a imagem de legalista junto ao \versal{STF} e à presidente da
República e, com isso, o espaço político para bancar a estratégia
central, a Lava Jato. Nenhuma outra iniciativa roubou"-lhe o
protagonismo. Escondeu o rei e iludiu a rainha quanto aos propósitos
republicanos da Lava Jato.~

A Lava Jato foi apenas o aríete, atrás do qual montou"-se um trabalho
sistemático de destruição de todos os símbolos de país.

Nas ruas, movimentos conduzidos pelo \versal{MBL} e outros vocalizando críticas
às políticas sociais.

Na Lava Jato, um trabalho sistemático de destruição das maiores empresas
nacionais, não apenas com inquéritos, mas com escracho. Recorreram ao
escracho, ao boicote a qualquer acordo de leniência, à perseguição
diuturna, com operações seguidas de invasão de sede, exposição de
mensagens -- até pessoais. A~ideia não é punir: é destruir.

O ápice tem sido a tentativa de destruição do símbolo Lula. Qualquer
compêndio futuro sobre a infâmia na vida nacional contemplará o que foi
feito, até acusações de furto de obras no Palácio.

O Ministério Público Federal é composto por procuradores preparados. Não
será necessário muito tempo para que, caindo a ficha do que fizeram,
venham à tona os bastidores da operação.

Como foi possível, no entanto, cooptar quase toda a corporação?

A campanha antinacional da Copa e, principalmente, a revelação da enorme
rede de corrupção da Petrobras, facilitaram a venda da ideia da
destruição da velha ordem, por uma nova ordem, liderada pelo trabalho
redentor do Ministério Público.

A velha ordem passou a se resumir a empreiteiras corruptas, cooptando o
sistema político e judiciário, e um governo populista que cooptou a
população com políticas sociais paternalistas. E~não a lenta
reconstrução democrática, os avanços civilizatórios (dos quais o próprio
\versal{MPF} foi agente importante), os avanços tecnológicos nas áreas do pré"-sal
e da defesa, o feito histórico de tirar milhões de pessoas da miséria e
reduzir graus históricos de desigualdade. A~corrupção foi o álibi para
apagar a história recente do país, até a luta pela redemocratização.

Principalmente pesou a visão redentorista de um novo poder se sobrepondo
aos demais e salvando o país.

Para avaliar os resultados do jogo, é fundamental esse entendimento
sobre a posição do~Alto Comando.~

\section{As próximas jogadas}

Sabendo"-se disso, fica mais claro o jogo, embora ainda seja difícil
antecipar o resultado final.

Há duas saídas negociadas possíveis, nenhuma tendo Dilma como
protagonista.

\textbf{Saída 1}~-- O semiparlamentarismo com o \versal{PMDB}, que tem várias
nuances. No regime parlamentarista, cabe ao presidente indicar o
primeiro ministro e o gabinete. E~ao Congresso aceitar ou rejeitar.
Pode"-se tentar um parlamentarismo goela abaixo, mas seria
regimentalmente complicado.

\textbf{Saída 2}~-- semiparlamentarismo com Lula assumindo o papel de
coordenador de governo, um primeiro"-ministro de fato.

\textbf{Impasse}~-- qualquer decisão de força, sem consenso, tenderá ao
fracasso. Sem um núcleo de poder, qualquer governo que assuma um país
dividido ficará refém das forças que o elegeram. Será um ataque ao butim
que inviabilizará qualquer tentativa de normalização econômica. Haverá
agitação, repressão aos movimentos sociais, caça às bruxas.

Independentemente de pecadilhos ou grandes pecados, um pacto entre
os~\textbf{Parlamentaristas}~e os~\textbf{Lulistas}~é o único sinal
visível de um polo racional na política.

Com Lula à frente, poderiam ser viabilizados acordos, através de uma
coordenação dele, como primeiro"-ministro de fato, ou em uma transição
com um primeiro"-ministro negociado entre ambos as partes.

Aí entram as jogadas do xadrez.

Antevendo essa possibilidade, o~\textbf{Alto Comando}~deflagrou novas
operações simultâneas: a ofensiva total contra Lula, o alarido em torno
dos presentes recebidos por Lula no exercício do poder; mais uma
denúncia contra Renan Calheiros; mais detalhes da delação do senador
Delcídio do Amaral, cujo conteúdo era conhecido apenas do~\textbf{Alto
Comando}~e do \versal{STF} (Supremo Tribunal Federal).

Tem"-se, então, duas forças conflitantes. De um lado o~\textbf{Alto
Comando~}apostando tudo no confronto; de outro, forças moderadoras
percebendo a possibilidade de uma guerra selvagem, se não se chegar a um
entendimento.

A tentativa de acordo passa por ambientes confusos, mas depende
fundamentalmente de Renan Calheiros e Lula.

\textbf{Fator 1}~-- o \versal{STF} e o fator Renan.

Os Ministros tendem a privilegiar a responsabilidade institucional. E~na
vitrine do Supremo, Janot tende a ter bom senso.

Nessa hipótese, Renan poderia ser poupado de atropelos imediatos, em
nome da estabilidade política. Aparentemente o foro privilegiado o
blindaria contra novas surpresas da Lava Jato. Mas não se descartam
vazamentos de delações visando comprometer sua atuação.

Além da nova investida de Janot contra Renan, na próxima
4\textsuperscript{a}~feira a oposição tentará pressionar o Ministro Luís
Roberto Barroso a rever seu voto em relação ao ritual do impeachment.

Desde que sua esposa foi alvo de ataques baixos, Barroso inibiu"-se. As
loucuras dos três procuradores paulistas estão diretamente ligadas ao
seu recuo na questão da Terceira Instância. Como explicou o promotor
Ricardo Blat, o pedido de prisão de Lula visou criar uma ``inovação
jurisprudencial'' depois que os garantistas do Supremo abriram a guarda
com a eliminação da terceira instância.

Espera"-se que Barroso e demais garantistas se sintam mais fortalecidos.
Mas ainda são uma incógnita.

Barroso terá um papel essencial. Se flexibiliza o impeachment, consolida
a parceria \versal{PSDB}"-\versal{PMDB} para derrubar a Dilma, pois nesse caso Michel Temer
seria poupado. Se resiste, obriga a um pacto mais amplo e à busca de
entendimento.

\textbf{Fator 2}~-- O fator \versal{PSDB}"-\versal{PMDB}.

O acordo semiparlamentarista prejudica Aécio e Alckmin para 2018

No momento, os~\textbf{Parlamentaristas}~confiam no indiciamento de
Aécio Neves para avançar nas tratativas.

Obviamente não levaram em conta o~\textbf{Alto Comando}. Se o nome de
Aécio não aparecer nas delações de executivos da Andrade Gutierrez,
aliás, consolidará a opinião geral sobre a proteção recebida. Mas há a
possibilidade de que a abundância de indícios obrigue Janot a mudar de
posição.

Alckmin se aproximou de Sérgio Moro através de seu candidato João Doria
Jr. Essa aproximação pode ser debitada na conta dos eventos aleatórios,
fora do script original. A~própria truculência do Secretário de
Segurança de São Paulo, Alexandre Moraes, colocando a \versal{PM} para reprimir
uma assembleia do Sindicato dos Metalúrgicos, é significativa dessa
reação a qualquer acordo.

\textbf{Fator 3}~- O fator Lula

Depois de sua fase classe média ascendente -- aceitando favores
descabidos de empreiteiras -- Lula vacila entre encarar a luta ou entrar
para a história, como um novo Mandela, preso pela direita. Ótimo! Salva
sua biografia à custa do comprometimento de todas as bandeiras que
representa.

Se Lula não assumir um protagonismo total no governo Dilma, sua queda
será questão de semanas.

\textbf{Fator 4}~- O Alto Comando.

O~\textbf{Alto Comando~}é integrado por procuradores probos, bem
intencionados e iludidos pela visão redentorista. Nâo se descarte a
possibilidade de um chamamento ao mundo real, quando avaliarem friamente
os desdobramentos da crise atual.

Por enquanto, o cenário mais provável será o do pacto \versal{PMDB}"-\versal{PSDB} visando
apoiar ao impeachment.

Caso fracassarem as saídas políticas, a primeira fase do golpe de1964
será café pequeno. Juízes e procuradores serão liberados para acabar com
a raça de tudo que cheire a esquerda.

O país será envolvido em uma guerra fratricida, com um novo governo
previamente enfraquecido pela falta de consenso e exposto a ataques ao
butim de todos os ``vencedores'', de grupos jornalísticos a líderes
empresariais e a impolutos de ordem geral que ajudaram a consumar o
golpe.

Neste domingo, joga"-se o último lance da guerra do impeachment. Se o
governo resistir por mais algum tempo e Lula entrar na linha de frente,
é possível alguma esperança de normalização democrática.

 

 
