\chapterspecial{05/\allowbreak{}08/\allowbreak{}2016 Xadrez da democracia em transe e dos aprendizes de feiticeiro}{}{}
 

\section{Caso 1 -- cenas de uma República risonha e franca}

O Ministro interino da Casa Civil, Eliseu Padilha, é acusado de ter
manobrado ilegalmente certificados de filantropia para uma universidade
privada em troca de bolsas para apaniguados e contratos para suas
empresas (\url{migre.me/\allowbreak{}uz6\versal{NF}}).

Também é réu por jogadas com precatórios envolvendo o \versal{DNER} (Departamento
Nacional de Estradas de Rodagem). Como Ministro dos Transportes,
valeu"-se de um acordo extrajudicial para repassar R\$ 2,3 milhões a uma
empresa gaúcha (\url{migre.me/\allowbreak{}uz6Vv}).

Foi condenado em R\$ 300 mil por manter um servidor fantasma
(\url{migre.me/\allowbreak{}uz6Sv}).

Em 2011 foi indiciado pela Polícia federal por formação de quadrilha na
construção das barragens Jaguari e Taquarengó
(\url{migre.me/\allowbreak{}uz711}).

É feliz proprietário de um terreno onde se instalou um parque eólico.
Padilha recebe R\$ 1,5 milhão por ano apenas por estar na corrente do
vento (\url{migre.me/\allowbreak{}uz7t8)}, embora vozes maliciosas sugerissem
se tratar de propina da \versal{EDP}.

Mas passou a controlar a Secom (Secretaria da Comunicação) e a inundar
sites e blogs da velha mídia com o controle centralizado da publicidade
de todas as estatais.

Com isso, transmudou---se. Eliseu Padilha aparece nos jornais com
aspecto grave, pontificando sobre reforma da Previdência, reforma
administrativa, diplomacia. Tem a última palavra para liberar verbas
milionárias para estados (\url{migre.me/\allowbreak{}uz7vQ)}. Tornou"-se o
segundo homem mais poderoso de um país continental, com mais de 200
milhões de habitantes. E~sob as vistas benevolentes dos mais intimoratos
defensores da moralidade pública que a República já conheceu: o
Procurador Geral da República Rodrigo Janot, a Força tarefa da Lava
Jato, colunistas moralistas da velha mídia.

\section{Cena 2 -- a destruição de direitos sociais}

Montou"-se o seguinte pacto dos justos:

\begin{enumerate}
\itemsep1pt\parskip0pt\parsep0pt
\item
  Em 2016 o orçamento federal será utilizado para pagar a conta do
  impeachment. Houve aumento para as emendas parlamentares, para as
  carreiras que mais contribuíram para a vitória e para os amigos dos
  reis. Ao mesmo tempo, manteve"-se no Banco Central a política de Selic
  a 14,15\% com apreciação cambial, propiciando enormes lucros para os
  especuladores que apostaram na queda do dólar pós"-golpe.
\item
  O enorme déficit contratado para 2016 e 2017 será compensado pela lei
  de limitação de gastos públicos. Se congelarão as despesas no menor
  nível da década (porque após dois anos de recessão). Depois de
  aprovada a lei, só poderão ser corrigidas pela inflação. Como os
  gastos com a Previdência continuarão aumentando, pelo envelhecimento
  da população, haverá cada vez menos recursos para saúde, educação,
  políticas assistenciais e todos os demais gastos públicos. E~a conta
  de juros continuará sem limitações.
\item
  A lambança fiscal deste ano ajudará de algum modo na recuperação
  moderada da economia. A~médio prazo, a situação não será sustentável,
  a menos que se coloque o Exército nas ruas para administrar a
  segurança.
\item
  A estratégia não é eleitoralmente viável para 2018. Mesmo esperando
  alguma recuperação da economia -- que parece ter batido no fundo do
  poço.
\end{enumerate}

Tem"-se, então, um modelo em xeque.

O atual grupo de poder conseguiu salvo"-conduto na condição de abrir o
saco de maldades e liquidar com direitos consagrados na Constituição de
1988.

Mas vai"-se chegar a 2018 com a dívida pública muito mais elevada --
devido às lambanças desses dois anos -- e sem a menor garantia de
manutenção desse pacote devido a essa inconveniência das democracias
chamada de voto popular.

Não é nenhum pouco factível a ideia de um grupo de desprendidos,
dispostos a passar dois anos praticando maldades para depois passar o
bastão para terceiros.

De duas, uma: ou prorrogarão a irresponsabilidade fiscal; ou tratarão de
se articular para impedir as eleições de 2018.

\section{Cena 3 -- os sinais da ditadura que se avizinha}

Conforme o ``Xadrez'' antecipou, o caminho natural será apostar na
figura do inimigo interno visando primeiro partir para uma democracia
mitigada e, mais à frente, para um endurecimento maior do regime,
provavelmente inviabilizando 2018.

O grupo que tomou o poder é alvo de muitas críticas. Mas há de se
reconhecer que, no exercício do poder, são profissionais,
incomparavelmente mais eficientes do que o amadorismo constrangedor do
governo deposto: sabe intimidar os que se intimidam; e a comprar os que
se vendem.

Basta ver a reação da cúpula do governo à informação de que Michel Temer
estaria inelegível pela Lei da Ficha Limpa: ``Não tem problema:
mudaremos a lei''.

Hoje em dia, todas as nomeações, inclusive de funcionários concursados,
passam pelo crivo do Ministro Padilha, com a assessoria do general
Sérgio Etchegoyen, do \versal{GSI} (Gabinete de Segurança Institucional). Há um
estímulo à delação, visando eliminar qualquer sinal de resistência na
máquina pública.

Multiplicam"-se as ações contra Lula do mesmo modo que desaparecem as
ações contra políticos do \versal{PSDB}. Tanto na \versal{PGR} quanto na Força Tarefa da
Lava Jato se mantém informações sob sigilo, só sendo vazadas informações
contra Lula e o \versal{PT}.

\section{Cena 4 -- os responsáveis}

Seria esse o objetivo final do Procurador Geral da República Rodrigo
Janot, da força"-tarefa da Lava Jato, do juiz Sérgio Moro, da Globo, de
entregar o país a um grupo tão polêmico? Tenho para mim que não.

Então o que leva forças tão díspares a montarem uma operação de tal
envergadura, derrubando uma presidente de forma ilegítima, para entregar
o poder a um grupo com tal biografia?

As ditaduras não nascem do nada. São semeadas, irrigadas, até ganhar
vida própria.

A base da psicologia de massa do fascismo, dos movimentos violentos
europeus dos anos 30, consistia em centrar a intolerância em uma figura
ou um grupo e ir alimentando gradativamente o ódio, até que a besta
saísse às ruas.

Criado o efeito manada, no início consegue"-se manipular a coesão contra
o inimigo comum, levantando o grande véu debaixo do qual misturam"-se do
exercício do preconceito mais odioso, o personalismo mais entranhado às
jogadas mais rasteiras, mas todos imbuídos da orientação divina de
erradicar o mal.

Destampada a caixa de Pandora, gradativamente a besta vai ganhando vida
própria e todos os atores, em volta, acabam sendo conduzidos pelo
turbilhão que pretendiam controlar.

Quando a grita contra Lula e o \versal{PT} tornou"-se hegemônica, Janot resolveu
surfar na onda. No início, dá uma indizível sensação de onipotência.
Provavelmente o que se passava na sua cabeça foi isso, um tremendo
sentimento de onipotência. Julgava que o primeiro passo seria
defenestrar Dilma e prender Lula. Ficaria titular de um poder tão
avassalador que o segundo passo seria pegar o \versal{PMDB}.

No momento do segundo passo, o aprendiz de feiticeiro deu uma tacada
altíssima: o pedido de prisão de vários senadores com base em grampos de
Sérgio Machado. O álibi do inimigo interno só se aplicava ao \versal{PT}, as
declarações grampeadas não eram fortes o suficiente para justificar tais
arroubos, o \versal{STF} impediu e a \versal{PGR} veio ao chão como um balão furado.

Imediatamente recuou. Não se ouviu falar mais em ofensivas contra o
\versal{PMDB}, manteve a blindagem a Aécio Neves. As únicas duas missões do \versal{PGR} e
do \versal{MPF}, assim como dos juízes de 1\textsuperscript{a~}instância, são
continuar atacando os bichos de sempre -- \versal{PT} e Lula --- ~e, no andar de
cima, Eduardo Cunha, e apenas ele, já que se tornou pato manco. Tudo sem
risco.

\section{Cena 5 -- os desdobramentos}

Nesses momentos de selvageria, o país não dispõe de forças contra
cíclicas. O~conceito de democracia e estado de direito não é
suficientemente forte nem entre os seus operadores.

Ante a onda formada, o Supremo apequenou"-se. No \versal{MPF}, qualquer ato contra
a onda recebe críticas da própria corporação. Hoje em dia há um exército
de procuradores espalhados por todas as comarcas do país, indo à forra
contra os advogados graças à Lava Jato.

Qualquer medida contra Lula ou Dilma é aplaudida. E~o \versal{PGR} não conduz,
pelo contrário, é conduzido por esse sentimento porque vive"-se a
Procuradoria de coalizão, desde que o \versal{PT} inventou a história de indicar
automaticamente o Procurador mais votado -- que obviamente vai ser
descontinuado pelos novos donos do poder.

Criou"-se um movimento inercial irresistível. Numa ponta, o desmonte
total do precário Estado de bem"-estar que o país ousou levantar nos
últimos anos, somado à venda indiscriminada de estatais na bacia das
almas. Na outra, a condescendência com o grupo de Temer, desde que
entregue o combinado. Alimentando essa fuzarca, a intolerância, a
radicalização do Judiciário, o delenda Lula. E, a cada momento, mais
incursões sobre o estado de direito.

E como impedir essa avalanche? O que aconteceria se o \versal{PGR} ganhasse
coragem cívica e tentasse enquadrar Temer e sua turma? Pagaria o preço
de uma segunda crise institucional sem dispor da unanimidade em que se
calçou para derrubar a presidente? A caixa de mágicas do \versal{PGR} só tinha
combustível para um impeachment.

A caminhada rumo ao arbítrio acontecerá grotescamente, mas passo a
passo. Cada novo ato de arbítrio será minimizado, cada novo cerco aos
recalcitrantes ignorado. Até que o país passe da democracia mitigada ao
autoritarismo.

E não haverá forças contra cíclicas. Hoje em dia, em lugar de Raimundo
Faoro, Luís Roberto Barroso; em lugar de Waldir Pires, Rodrigo Janot; em
lugar de Pedro Aleixo, Michel Temer; em lugar de Paulo Brossard,
Cristovam Buarque.

Enfim, o país é a resultante do nível de suas elites.
