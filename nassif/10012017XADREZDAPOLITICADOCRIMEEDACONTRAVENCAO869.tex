\chapterspecial{10/\allowbreak{}01/\allowbreak{}2017 Xadrez da política, do crime e da contravenção}{}{}
 

Os massacres em presídios são apenas o desfecho de um amplo processo
nacional de convivência com o crime, de aceitação social, de parceria
política e de negócios entre o país formal e a criminalidade.

O escravagismo e o bicho foram as primeiras organizações criminosas. Da
estrutura do bicho nasceu o narcotráfico. Com seu conhecimento da arte
de corromper autoridades públicas, o bicho saltou das delegacias
municipais para as Secretarias de Segurança estaduais, e de lá para toda
a máquina pública, para todos os poros da administração pública,
entrando nas licitações municipais, estaduais, controlando o lixo, os
transportes urbanos, da mesma maneira que a máfia na Itália, conforme
revelou a \versal{CPMI} de Carlinhos Cachoeira.

No mundo oficial, na base do Judiciário, há a generalização da prisão
provisória de preso pobre. Para os tubarões, há a sucessão de recursos,
a anulação de inquéritos por ninharias. Cada aumento das penalidades
pega apenas os de baixo. Cada flexibilização nas penas, beneficia apenas
os de cima.

Historicamente, política e crime -- perdão, contravenção -- sempre
caminharam de mãos dadas. Na época da Proclamação da República, por
exemplo, o jogo do bicho já dominava a Câmara Municipal do Rio de
Janeiro (\url{https:/\allowbreak{}/\allowbreak{}goo.gl/\allowbreak{}a3YqrK}), através do Barão de Drummond.

Nenhum partido saiu imune dessas alianças, do \versal{PDT} de Brizola ao \versal{PMDB} de
Quércia e Michel Temer, ao
\versal{PT}~{(clique
aqui}). Desde o momento em que o jogo internacional entrou na Caixa
Econômica Federal, no governo Itamar Franco -- vendendo sistemas para as
loterias \mbox{---,} uma sucessão de escândalos abalou vários governos e abriu
a primeira brecha no governo Lula com o caso Valdomiro -- do qual se
valeu a ala do Ministério Público Federal ligada a José Serra.

Vamos a um pequeno histórico das relações com a contravenção de dois
personagens"-símbolos do Brasil atual: o presidente Michel Temer e o
Ministro da Justiça Alexandre de Morais.

\section{Peça 1 --- Michel Temer e a primeira pax paulista}

As ligações de Temer com o jogo nasceram com sua própria carreira
política. De advogado, tornou"-se procurador. De procurador, Secretário
de Segurança em São Paulo na gestão Franco Montoro. Assumiu com Montoro
acossado, com manifestantes derrubando as grades do Palácio
Bandeirantes, com a incumbência de montar a pax paulista. Empossado
Secretário, sua primeira declaração foi pela legalização do jogo do
bicho
({Estadão02}).

Quando saiu da Secretaria, estouraram denúncias de que sua campanha para
deputado Constituinte foi bancada pelo jogo"-de"-bicho
({clique
aqui)}. O~deputado estadual santista Del Bosco do Amaral (\versal{PMDB}) acusou
Temer de ter se apoiado nos ``piores setores policiais, inclusive
aqueles ligados ao jogo de bicho''. Acusava"-o também de ter afrouxado a
repressão ao jogo em troca da ``corretagem zoológica''
({clique
aqui}).

Houve uma \versal{CPI} na Assembleia Legislativa, na qual o chefe de polícia de
Temer, Álvaro Luz, afirmou ter sido orientado a reprimir apenas os
bicheiros que atuassem de modo ``ostensivo''. No caso, pequenos
bicheiros que ousavam montar seus próprios negócios, competindo com Ivo
Noal, Marechal, os donos do bicho em São Paulo.

As relações de Temer com o jogo não pararam aí.

O pacto do jogo com o governo do Estado durou até o caso Carandiru, no
governo Fleury. Desde fins dos anos 80, o tráfico começava a invadir o
estado, ameaçando o reinado dos bicheiros. Montou"-se uma operação da
Polícia Militar destinada a atingir alguns traficantes presos no
Carandiru. Um acidente no caminho -- um aparelho de \versal{TV} arremessado na
cabeça do comandante do efetivo, e o boato de que tinha morrido --
resultou no estouro da boiada e no massacre de Carandiru.

Temer foi rapidamente convocado por Fleury a reassumir a Secretaria de
Segurança. No período em que se manteve Secretário, o número de
flagrantes contra o bicho caiu de 1.006 em 1990 (gestão anterior), para
746 em 1992 e 624 em 1993
({Folha17021997}).
Como sempre, as autuações eram em cima de pequenos bicheiros, que
ousavam voos independentes. E~já eram conhecidas as ligações dos
bicheiros com o tráfico de cocaína (Angerami).

Um novo poder se sobrepunha ao bicho, o do \versal{PCC} que rapidamente conseguiu
a adesão das populações carcerárias, como efeito da profunda insegurança
que se seguiu ao massacre caso Carandiru.

A ligação de Temer com o jogo era tão conhecida que, na \versal{CPI} do Bingo, na
Câmara Federal, incumbiu o senador Garibaldi Alves (\versal{PMDB}"-\versal{RN}), relator da
Comissão, de mantê"-lo informado sobre as pessoas que seriam convocadas
ou investigadas (\url{https:/\allowbreak{}/\allowbreak{}goo.gl/\allowbreak{}39aySf)}.

Recentemente, o presidente da Comissão de Turismo Herculano Passos
(\versal{PSD}"-\versal{SP}), informou que avançará na proposta de legalização do jogo,
depois de ter conversado com Temer e ``ele disse que pretende tocar para
frente a proposta porque é bom para a economia e é bom para o país''.

\section{Peça 2 --- Alexandre de Morais e a segunda pax paulista}

Do mesmo modo que Temer, a estreia de Morais no campo político"-jurídico
foi um livro sobre constitucionalismo. E~seu princípio de vida talvez
possa ser sintetizado no discurso que, como Ministro, fez a uma plateia
lotada de estudantes: defendeu que os futuros advogados se preocupem em
ganhar dinheiro, ``porque não sou comunista nem socialista, muito pelo
contrário'' (\url{https:/\allowbreak{}/\allowbreak{}goo.gl/\allowbreak{}eHu3\versal{MK}}).

Se a gestão Temer foi no auge do poder do bicho, a de Morais se deu no
auge do poder do \versal{PCC}. O~embate maior foi em 2006, durante a campanha
presidencial de Geraldo Alckmin. O~\versal{PCC} invadiu a cidade e executou
diversos agentes públicos.

Celebrou"-se um acordo (\url{https:/\allowbreak{}/\allowbreak{}goo.gl/\allowbreak{}Nz8lzc)}, do qual Morais não
participou.

Depois disso, a paz voltou a reinar -- e o \versal{PCC} ganhou espaço para
crescer. Em troca da liberdade de ação, o \versal{PCC} ajudou a reduzir os crimes
violentos na periferia, um varejo que tinha o inconveniente de chamar a
atenção da opinião pública, obrigando a polícia a intervir.

Dessa convivência pacífica se prevaleceu Alexandre de Morais. Quando
Gilberto Kassab assumiu a prefeitura de São Paulo, levou Morais como seu
homem forte, iludido por sua retórica de gestor.

Tornou"-se um super"-secretário acumulando as pastas de Transportes e de
Serviços, presidindo o Serviço Funerário, a \versal{SPT}rans e a Companhia de
Engenharia de Tráfego.

No cargo, era o responsável pela negociação dos sistemas de transportes,
incluindo as vans, sob o controle do \versal{PCC}. Saiu depois de várias decisões
intempestivas e desastrosas, culminando com o anúncio inesperado de que
iria rever todos os contratos de ônibus~ e de vans da prefeitura
(\url{https:/\allowbreak{}/\allowbreak{}goo.gl/\allowbreak{}tH9b\versal{VC})}, sem ao menos consultar o prefeito.

A proposta desgostou muitos setores, não o \versal{PCC}. Morais se tornou
suficientemente confiável para, fora do cargo, ser contratado como
advogado pelo \versal{PCC} para sua cooperativa de vans, a Transcooper
(\url{https:/\allowbreak{}/\allowbreak{}goo.gl/\allowbreak{}kbxGnw})

Em 2012, a convite do colega Michel Temer se filiou ao \versal{PMDB}. E, por
conta dessa aliança, em 2015 assumiu o posto de Secretário de Segurança
do Estado de São Paulo, mesmo posto que projetou seu mestre Temer. Saiu
criando problemas no governo: não conseguiu implementar sequer a meninas
dos olhos da Segurança, o sistema Detecta, de reconhecimento de atitudes
suspeitas, que a Prodesp deixou pronto e acabado para ser implementado
(\url{https:/\allowbreak{}/\allowbreak{}goo.gl/\allowbreak{}qQ2WeS}).

\section{Peça 3 -- a guerra contra o crime}

A guerra contra o crime organizado se dá em quatro frentes centrais:

\begin{enumerate}
\itemsep1pt\parskip0pt\parsep0pt
\item
  Na economia, na medida em que o desemprego e a falta de oportunidades
  são fatores de aliciamento dos jovens pelo crime.
\item
  Na periferia e na favela, onde o crime organizado fornece segurança à
  população.
\item
  Nos presídios, nos quais a ordem e a integridade são garantidas pelas
  organizações criminosas. Os massacres ocorrem quando há conflito entre
  elas.
\item
  No mundo empresarial e político, para identificação dos elos do crime
  com os sistemas formais de poder.
\end{enumerate}

Não se trata de tarefa trivial.

Em países menos atrasados, é desafio para ações interministeriais,
envolvendo educação, saúde, juventude, esportes, obras públicas etc.
Exige também integração com Secretarias estaduais e metropolitanas, com
a cooperação internacional, com a diplomacia. Exige capacidade de
trabalhar com dados, estatísticas, geo"-referenciamento.

De maneira solta, todos esses mecanismos existem. Mas sua coordenação
exige uma capacidade superior de gestão.

Como é o gestor Alexandre de Morais?

Em 2005, como presidente da Febem, o constitucionalista Morais ordenou o
maior processo de demissão em massa da história da instituição. Dois
anos depois, o \versal{STF} ordenou a readmissão de todos os demitidos. O~Estado
teve que arcar com uma conta de R\$ 32 milhões, suficiente para
construir 11 pequenas unidades, dentro do projeto de descentralização da
Febem (\url{https:/\allowbreak{}/\allowbreak{}goo.gl/\allowbreak{}\versal{JM}zBv9)}.

Como super"-secretário de Kassab, deixou a prefeitura com a fama de
anunciar planos que nunca eram implementados e que, muitas vezes, nem
planos eram: apenas ideias esparsas coladas com um tanto de retórica.

Depois, como Secretário de Segurança de Geraldo Alckmin não conseguiu
sequer implementar a principal peça de campanha: um sistema de
reconhecimento que já tinha sido desenvolvido pela Prodesp. Questionado
pela imprensa sobre a demora, limitou"-se a dizer, ao estilo Rolando"-Lero
que não concordava que as webcams ficassem na marginal, pois seriam
identificadas facilmente pelos motociclistas. A~Prodesp nunca foi
informada dessa ressalva.

\section{Peça 4 -- o Plano Nacional de Segurança Pública}

Das limitações gerenciais e profissionais de Alexandre de Morais nasceu
o Plano Nacional de Segurança Pública.

As três metas traçadas já indicam sua limitação. O~Plano define como
prioridades o combate ao homicídio, à violência contra a mulher e ao
trabalho diplomático com nações fronteiriças, visando conter o tráfico e
uma vaga racionalização e modernização do sistema carcerário.

Homicídios soltos e violência contra a mulher nunca foram de
responsabilidade do governo federal, e nem poderiam ser. Tratam"-se de
crimes locais, com motivação local e que exigem a atuação do poder
local. Como irá controlar, de Brasília, as fantasiosas patrulhas Maria
da Penha, com que pretende reduzir a violência contra a mulher? O que
essas patrulhas fariam?

A atuação do governo federal é no combate às organizações criminosas,
cujos tentáculos atingem vários estados e o exterior.

Mas esse desafio, Morais não quer encarar, pois significaria entrar no
terreno cinza que permeia as relações entre a economia formal, a
política e o crime.

Tome"-se o caso do Comendador Arcanjo, que dominava o jogo no Mato
Grosso. Algumas das grandes fortunas de soja foram construídas lavando
dinheiro de Arcanjo. Sua influência vai de cassinos na fronteira até
linhas de ônibus no \versal{ABC}. Recentemente conseguiu no Superior Tribunal de
Justiça (\versal{STJ}) a liberação de todos seus bens.

Como era também a influência de Carlinhos Cachoeira, suas ligações com a
Delta Engenharia, seus negócios com vários governadores de estado.

É do advogado do \versal{PCC} que se espera um papel similar ao dos Intocáveis?

Ora, entre advogados e clientes não existe a história da Muralha
Chinesa. Há sempre uma relação de total confiança e de abertura de todas
as informações do cliente, para que possa ser bem defendido.

Um Ministro da Justiça que manteve relações de confiança de tal ordem
com o \versal{PCC} irá conduzir o trabalho de combate ao \versal{PCC} e a outras
organizações criminosas?

No fim, o Plano Nacional de Segurança não passará disso: Morais
produzindo factoides e uma reunião de emergência para daqui a uma
semana; Michel Temer indo comer os pães de queijo na casa da presidente
do Supremo Carmen Lúcia; e a Ministra soltando uma frase de efeito de
seu repertório mineiro.

As atividades do \versal{PCC} são públicas e notórias: o próprio Alexandre de
Morais tem, no seu escritório, a relação das empresas controladas pela
quadrilha. Como eram notórias as atividades de Ivo Noal e Marechal, nos
tempos em que Temer era Secretário de Segurança. O~crime tem nome,
endereço, razão social.

Mesmo assim, seguem intocáveis. E, quando presos, transformam os
presídios em escritórios.
