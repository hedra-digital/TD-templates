\chapterspecial{25/\allowbreak{}08/\allowbreak{}2016 Xadrez do fim do ciclo Lava Jato}{}{}
 

\section{Peça 1 --- sobre as diversas nuances da mentira}

Dizia Santo Agostinho: ``Mente aquele que, pensando de determinado modo,
exprime algo diferente por palavras ou sinais''. ~Por esse prisma, a
Lava Jato mente costumeiramente. Seria um pecado.

Provavelmente, em seus cursos nos Estados Unidos, os bravos procuradores
deixaram se inebriar pelo pragmatismo das civilizações luteranas. Dizia
Lutero: ``Uma boa e sólida mentira para o bem comum e para a Igreja
inteira, uma mentira em caso de necessidade, uma mentira útil, uma
mentira para prestar serviço não seria contrária a Deus''.

Como ficaria, então, o desafio de encontrar a verdade? Só conseguindo
decifrar as intenções da mentira. Como dizia Golbery, a mentira é mais
rica que a verdade, porque permite extrair muitas informações adicionais
sobre os objetivos do mentiroso.

No ``Xadrez da delação que ficou parada no ar''
(\url{migre.me/\allowbreak{}uLf\versal{YT}}) mostramos um pouco da técnica de extrair
informações de notícias esparsas e de explorar as diversas nuances da
mentira. Tudo em torno da decisão da \versal{PGR} de não mais aceitar a delação
de Léo Pinheiro, da \versal{OAS}, que poderia implicar José Serra e Aécio Neves.

\section{Peça 2 -- a verdade e a mentira na Lava Jato}

A Lava Jato revelou a maior promiscuidade já havida entre uma operação e
a imprensa Em geral, tinha"-se um procurador ou delegado acumpliciado com
um repórter de polícia vazando informações. Agora se tem a força"-tarefa
em Curitiba, o pessoal da Procuradoria Geral da República em Brasília,
todos eles falando com repórteres de polícia, de política, setoristas,
enviados especiais.

Esse modelo horizontal cria blindagens aos vazamentos. Basta que a
informação esteja disponível para mais de um órgão, para começar a
vazar, sabendo que nem a Polícia Federal nem a \versal{PGR} abrirão
investigações.

Em vazamentos de tal abrangência, a única maneira de uniformizar o
discurso é através da verdade -- que comporta poucas versões -- não da
mentira. E~aí não é necessário nem as técnicas aprimoradas de
interrogatório para identificar a mentira.

\section{Peça 3 --- A mentira sobre o vazamento}

O primeiro ponto a se considerar é que a alegação do Procurador Geral
para justificar a suspensão do acordo de delação de Leo Pinheiro é uma
mentira.

Seu juiz, pare agora! Não chamei o \versal{PGR} de mentiroso. Refiro"-me a uma das
muitas justificativas filosóficas da mentira, segundo Santo Agostinho.

O que Janot disse sobre a imputação do vazamento ao Ministério Público:
``Não poderia ser o \versal{MPF} porque esse pretenso anexo jamais ingressou em
qualquer dependência do \versal{MP}''.

Ele mentiu contando uma verdade. De fato, não houve uma proposta oficial
de delação, muito menos um anexo oficial. Mas Léo Pinheiro mencionou o
episódio para os procuradores da Lava Jato. Para vazar, portanto,
bastaria ter a informação e uma relação de confiança com a revista.

Significa que foram os procuradores que vazaram? Não necessariamente.
Significa que o \versal{PGR} mentiu (filosoficamente), sem faltar com a verdade.

A questão é: por que (filosoficamente) mentiu?

\section{Peça 4 -- a mentira sobre a suspensão da delação}

A segunda mentira (filosófica) é sobre as razões da suspensão da
delação. Ela nos remete a um tema um pouco mais complexo, da valoração
do episódio.

Segundo procuradores da \versal{PGR}, a delação não será renegociada porque houve
quebra de confiança com o vazamento. Ora, \versal{GGN} já mostrou vazamentos em
17 delações anteriores. O~que o último e irrelevante vazamento teria de
diferente?

Segundo procuradores da \versal{PGR}, o episódio de Léo Pinheiro é único:
``Normalmente, quando houve vazamentos, a informação existia de fato.
Nesse caso, decidimos romper porque~foi vazado algo que o Ministério
Público Federal não tinha. Essa é a diferença''
(\url{migre.me/\allowbreak{}uLg\versal{JB})}, declararam ao \versal{G}1.

Mentiram, sem faltar com a verdade. De fato, é o único caso de vazamento
de notícia não confirmada. Mas, por definição, é o mais inofensivo dos
vazamentos. Se a informação não existe, nem deveria ser considerada.
Caso contrário, bastaria inventar uma mentira, colocar na capa
daVeja~para anular qualquer delação.

Portanto, mentiram (filosoficamente) ao tratar drasticamente o mais
irrelevante de todos os vazamentos e a fazer vista grossa aos vazamentos
relevantes.

Além disso, se a informação não existe formalmente, mas foi testemunhada
por todos os procuradores de Curitiba, qual a razão de imputá"-la aos
advogados de Léo Pinheiro?

Quando há dúvidas dessa natureza, a técnica de investigação sugere que
exponha os suspeitos e as informações ao teste do ``a quem interessa''.

Advogados da \versal{OAS}~-- segundo a \versal{PGR}, com o vazamento tentariam forçar o
acordo de delação. Não bate. Supõe"-se que a \versal{OAS} esteja sendo atendida
pelos melhores e mais caros advogados do país. Nem um leguleio cometeria
erro de cálculo dessa dimensão, de achar que o vazamento aceleraria a
aceitação da delação.

\versal{PGR}~-- segundo anteciparam diversos veículos, a delação teria alto teor
explosivo. Como tal, traz riscos políticos de monta à operação e ao \versal{PGR}.
Portanto, em tese, interessaria ao \versal{PGR} suspender a delação.

Dr. Janot taxou a capa de Veja de ``factoide informacional''. Qual o
nome que se dá à suspensão de uma investigação com base em um ``factoide
informacional''? Seria ``factoide processual''?

\section{Peça 5 -- o que está acontecendo na Lava Jato}

Vamos a um apanhado dos estilhaços de informações que povoaram a mídia
nos últimos dias, para tentar entender o desdobramento do factoide
informacional e do factoide processual.

Blog de Matheus Leitão, no \versal{G}1 de 24/\allowbreak{}08/\allowbreak{}2016~-- Informa que procuradores
que cuidam do caso de Gim Arguelo e outros, aguardam nova estratégia de
defesa no depoimento de Leo Pinheiro (que seria prestado ontem ao juiz
Sérgio Moro):``a expectativa agora é como Léo Pinheiro irá se comportar
no depoimento em Curitiba após a suspensão de sua colaboração.
Integrantes da Lava Jato acreditam que se ele der muitas informações é
porque acredita na renegociação da colaboração e na homologação pela
Justiça''.

A matéria fornece três informações preciosas:

Informação 1~-~ Até os procuradores de Brasília sabem que o que está em
jogo é``a nova estratégia da defesa''-- ou seja, uma reformulação na
proposta inicial de delação de Léo Pinheiro -- e não a suposta quebra de
confiança alegada por Janot.

Informação 2~-- Se aguardam as informações de mudanças na delação, é
porque consideram que Pinheiro tem informações relevantes para seu
processo. A~\versal{PGR} está abrindo mão de informações relevantes graças a um
``factoide informacional''.

Informação 3~-- Léo manteve"-se calado perante Moro. Sinal que ainda não
combinou com a \versal{PGR} a ``nova estratégia''. Troféu ``pedra da Roseta'' a
quem decifrar o teor da nova delação, se houver.

Coluna de Mônica Bergamo em 25/\allowbreak{}08/\allowbreak{}2016~-- ~``O governo de Michel Temer
acompanha com lupa a crise entre o Ministério Público Federal e o \versal{STF}
(Supremo Tribunal Federal). E~tem informações de que procuradores
tentaram investigar, além do ministro Dias Toffoli, também assessores e
familiares de outros dois magistrados da corte.

É provável que Gilmar Mendes esteja na mira dos procuradores da Lava
Jato.

A questão central é que a Lava Jato recorreu a um arsenal de infâmias
contra magistrados, cujo ponto alto foi o massacre a que foi submetido o
Ministro MarceloNavarro Ribeiro Dantas, do Superior Tribunal de Justiça
(\versal{STJ}).

Ribeiro entrou na delação do ex"-senador Delcídio do Amaral da mesma
maneira que pretendiam que Toffoli entrasse na de Léo Pinheiro:
certamente estimulado pelos procuradores Delcídio incluiu um caco
irrelevante sobre Marcelo.

Ex"-procurador e ex"-desembargador ficha limpa, garantista, unicamente
devido ao fato de ter dado um voto contrário à Lava Jato, Marcelo
Navarro Ribeiro Dantas foi massacrado com base na delação negociada de
Delcídio diretamente com a Procuradoria Geral da República. Na condição
de relator de um processo, ele poderia ter decidido o caso com voto
individual. Optou por leva"-lo ao pleno, onde seria apenas mais um voto.
Só por isso, seu nome foi incluído a fórceps na delação de Delcídio.

Esse histórico de abusos fragilizou a posição da Lava Jato, permitindo
que se tornasse alvo fácil de factoides verossímeis, como essa bobagem
da capa da~Veja.

Foi o que levou o Ministro João Otávio de Noronha, novo corregedor do
Conselho Nacional de Justiça, a declarar (\url{migre.me/\allowbreak{}uLiYr}):
``Para que o magistrado exerça a magistratura com plena liberdade, sem
medo da mídia, que se tornou um poder \redondo{[…]} Não pode um juiz se
curvar ao Ministério Público ou à Polícia Federal. O~magistrado se curva
às liberdades fundamentais e ao devido processo legal'', afirmou.

E o procurador curitibano, desancando Toffoli através da Folha, crente
que sabia jogar xadrez. Ou talvez soubesse. O~excesso de abusos contra o
inimigo comum serve agora para blindar os aliados comuns.

\section{Peça 6 -- o fim da ilusão dos procuradores}

Dia desses, confrontado com o link de um post meu -- no qual mostrava a
maneira como o \versal{MPF} se converteu em partido político -- um bravo
procurador goiano, desses que atuam corajosamente na linha de frente
contra o crime, rebateu dizendo que o artigo juntava um conjunto de
ilações para uma conclusão errada.

Quem atua na linha de frente não tem noção das sutilezas e sofisticações
dos jogos de poder, da arte de direcionar ou afrouxar a energia de um
órgão na direção pretendida. São massas"-de"-manobra que, com sua coragem
de encarar o bandido na ponta legitimam as jogadas políticas na cúpula.

Até agora, o artifício do inimigo comum -- o \versal{PT} e Lula -- garantia a
uniformidade da ação entre a cúpula e a base, a \versal{PGR} e os procuradores
que atuam na linha de frente. Livrar o país dessa corja era o ``bem
comum'', que justificava o endosso a todos os atos.

Não tinham a menor ideia que o processo de ascensão de um \versal{PGR} se faz no
convívio diuturno com as franjas do poder brasiliense, nos conchavos, na
identificação de onde estão a força e o poder para buscar sua proteção e
não ficar ao relento. São como limalhas de ferro expostos a um imã. Se o
poder não liga o imã, permanecem em suas funções republicanas. Se liga o
imã, imediatamente a limalha se alinhará em torno do poder. Se nos
tempos de Geraldo Brindeiro houvesse um partido ``republicano'' e
ingênuo como o \versal{PT}, sua atuação em nada seria diferente da de Janot. E~vice"-versa.

Depois do jogo consumado, resta o desabafo de um procurador, colhido
pelo Painel da Folha: ``Éramos lindos até o impeachment ser
irreversível. Agora que já nos usaram, dizem
chega''.\url{migre.me/\allowbreak{}uLhM0}

Bem"-vindo ao mundo real.
